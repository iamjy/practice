% -*-LaTeX-*-
% Converted automatically from troff to LaTeX
% by tr2latex (version 1.04, July 15, 1991)
% on Thu Dec 12 09:12:27 1991
% tr2latex was written by Kamal Al-Yahya at Stanford University
% (Kamal%Hanauma@SU-SCORE.ARPA)

\documentstyle[troffman,twoside]{article}
\begin{document}
%
% input file: pbmplus.1
%
\phead{PBMPLUS}{1L}{13 October 1993}{}{}

\shead{NAME}
pbmplus - enhanced portable bitmap toolkit

\shead{DESCRIPTION}
The {\it pbmplus} toolkit allows conversions between image files of
different format.  By means of using common intermediate formats, only
$2 \times N$ conversion filters are required to support $N$ distinct formats,
instead of the $N^2$ which would be required to convert directly between
any one format and any other.  The package also includes simple tools for
manipulating portable bitmaps.

The package consists of four upwardly compatible sections:
\begin{TPlist}{pbm}
\item[{pbm}]
Supports monochrome bitmaps (1 bit per pixel).
\item[{pgm}]
Supports grayscale images.  Reads either {\it pbm} or {\it pgm} formats
and writes {\it pgm} format.
\item[{ppm}]
Supports full-color images.  Reads either {\it pbm}, {\it pgm}, or {\it
ppm} formats, writes {\it ppm} format.
\item[{pnm}]
Supports content-independent manipulations on any of the three formats
listed above, as well as external formats having multiple types.  Reads
either {\it pbm}, {\it pgm}, or {\it ppm} formats, and generally writes
the same type as it read (whenever a {\it pnm} tool makes an exception
and ``promotes'' a file to a higher format, it informs the user).
\end{TPlist}

\shead{DESCRIPTION OF CONTENTS}
\begin{TPlist}{{\it cmuwmtopbm}}
\item[{{\it atktopbm}}]
convert Andrew Toolkit raster object to portable bitmap
\item[{{\it brushtopbm}}]
convert Xerox doodle brushes to portable bitmap
\item[{{\it cmuwmtopbm}}]
convert CMU window manager format to portable bitmap
\item[{{\it g3topbm}}]
convert Group 3 FAX to portable bitmap
\item[{{\it icontopbm}}]
convert Sun icon to portable bitmap
\item[{{\it gemtopbm}}]
convert GEM .img format to portable bitmap
\item[{{\it macptopbm}}]
convert MacPaint to portable bitmap
\item[{{\it mgrtopbm}}]
convert MGR format to portable bitmap
\item[{{\it pbmmerge}}]
merge wrapper routine
\item[{{\it pbmto10x}}]
convert portable bitmap to Gemini 10x printer graphics
\item[{{\it pbmtoascii}}]
convert portable bitmap to ASCII graphic form
\item[{{\it pbmtoatk}}]
convert portable bitmap to Andrew Toolkit raster object
\item[{{\it pbmtobbnbg}}]
convert portable bitmap to BBN BitGraph graphics
\item[{{\it pbmtocmuwm}}]
convert portable bitmap to CMU window manager format
\item[{{\it pbmtoepson}}]
convert portable bitmap to Epson printer graphics
\item[{{\it pbmtog3}}]
convert portable bitmap to Group 3 FAX
\item[{{\it pbmtogem}}]
convert portable bitmap into GEM .img file
\item[{{\it pbmtogo}}]
convert portable bitmap to GraphOn graphics
\item[{{\it pbmtoicon}}]
convert portable bitmap to Sun icon
\item[{{\it pbmtolj}}]
convert portable bitmap to HP LaserJet graphics
\item[{{\it pbmtomacp}}]
convert portable bitmap to MacPaint
\item[{{\it pbmtomgr}}]
convert portable bitmap to MGR format
\item[{{\it pbmtopi3}}]
convert portable bitmap to Atari Degas .pi3
\item[{{\it pbmtoplot}}]
convert portable bitmap into Unix plot(5) file
\item[{{\it pbmtoptx}}]
convert portable bitmap to Printronix graphics
\item[{{\it pbmtoxbm}}]
convert portable bitmap to X11 bitmap
\item[{{\it pbmtox10bm}}]
convert portable bitmap to X10 bitmap
\item[{{\it pbmtoybm}}]
convert portable bitmap into Bennet Yee ``face'' file
\item[{{\it pbmtozinc}}]
convert portable bitmap to Zinc Interface Library icon
\item[{{\it pbmlife}}]
apply Conway's rules of Life to a portable bitmap
\item[{{\it pbmmake}}]
create a blank bitmap of a specified size and color
\item[{{\it pbmmask}}]
create a mask bitmap from a regular bitmap
\item[{{\it pbmreduce}}]
reduce a portable bitmap N times, using Floyd-Steinberg
\item[{{\it pbmtext}}]
render text into a bitmap
\item[{{\it pbmupc}}]
create a Universal Product Code bitmap
\item[{{\it pi3topbm}}]
convert Atari Degas .pi3 to portable bitmap
\item[{{\it xbmtopbm}}]
convert X10 or X11 bitmap to portable bitmap
\item[{{\it ybmtopbm}}]
convert Bennet Yee ``face'' file into portable bitmap
\item[{{\it asciitopgm}}]
convert ASCII graphics into a portable graymap
\item[{{\it fstopgm}}]
convert Usenix FaceSaver$^{\rm tm}$ format to portable graymap
\item[{{\it hipstopgm}}]
convert HIPS format to portable graymap
\item[{{\it lispmtopgm}}]
convert a Lisp Machine bitmap file into pgm format
\item[{{\it pgmbentley}}]
Bentleyize a portable graymap
\item[{{\it pgmcrater}}]
create cratered terrain by fractal forgery
\item[{{\it pgmedge}}]
edge-detect a portable graymap
\item[{{\it pgmenhance}}]
edge-enhance a portable graymap
\item[{{\it pgmhist}}]
print a histogram of the values in a portable graymap
\item[{{\it pgmkernel}}]
generate a convolution kernel
\item[{{\it pgmmerge}}]
merge wrapper routine
\item[{{\it pgmnoise}}]
create a graymap made up of white noise
\item[{{\it pgmnorm}}]
normalize contrast in a portable graymap
\item[{{\it pgmoil}}]
turn a portable graymap into an oil painting
\item[{{\it pgmramp}}]
generate a grayscale ramp
\item[{{\it pgmtexture}}]
calculate textural features on a portable graymap
\item[{{\it pgmtofits}}]
convert portable graymap to FITS format
\item[{{\it pgmtofs}}]
convert portable graymap to Usenix FaceSaver$^{\rm tm}$ format
\item[{{\it pgmtolispm}}]
convert a portable graymap into Lisp Machine format
\item[{{\it pgmtopbm}}]
convert portable graymap to portable bitmap
\item[{{\it psidtopgm}}]
convert PostScript ``image'' data to portable graymap
\item[{{\it rawtopgm}}]
convert raw grayscale bytes to portable graymap
\item[{{\it giftoppm}}]
convert GIF to portable pixmap
\item[{{\it gouldtoppm}}]
convert Gould scanner file to portable pixmap
\item[{{\it ilbmtoppm}}]
convert IFF ILBM to portable pixmap
\item[{{\it imgtoppm}}]
convert Img-whatnot to portable pixmap
\item[{{\it mtvtoppm}}]
convert MTV ray-tracer output to portable pixmap
\item[{{\it pcxtoppm}}]
convert PC Paintbrush format to portable pixmap
\item[{{\it pgmtoppm}}]
colorize a portable graymap into a portable pixmap
\item[{{\it pi1toppm}}]
convert Atari Degas .pi1 to portable pixmap
\item[{{\it picttoppm}}]
convert Macintosh PICT to portable pixmap
\item[{{\it pjtoppm}}]
convert HP PaintJet file to portable pixmap
\item[{{\it ppmchange}}]
change all pixels of one color to another in a portable pixmap
\item[{{\it ppmdim}}]
dim a portable pixmap down to total blackness
\item[{{\it ppmdist}}]
simplistic grayscale assignment for machine generated, color images
\item[{{\it ppmdither}}]
ordered dither for color images
\item[{{\it ppmflash}}]
brighten a picture up to complete white-out
\item[{{\it ppmforge}}]
fractal forgeries of clouds, planets, and starry skies
\item[{{\it ppmhist}}]
print a histogram of a portable pixmap
\item[{{\it ppmmake}}]
create a pixmap of a specified size and color
\item[{{\it ppmmerge}}]
merge wrapper routine
\item[{{\it ppmmix}}]
blend together two portable pixmaps
\item[{{\it ppmpat}}]
create a pretty pixmap
\item[{{\it ppmquant}}]
quantize colors down to a specified number
\item[{{\it ppmquantall}}]
script to run ppmquant on a set of pixmaps
\item[{{\it ppmrelief}}]
run a Laplacian Relief filter on a portable pixmap
\item[{{\it ppmshift}}]
shift lines of a portable pixmap left or right by a random amount
\item[{{\it ppmspread}}]
displace a portable pixmap's pixels by a random amount
\item[{{\it ppmtoacad}}]
convert portable pixmap to AutoCAD database or slide
\item[{{\it ppmtoicr}}]
convert portable pixmap to NCSA ICR graphics
\item[{{\it ppmtoilbm}}]
convert portable pixmap to IFF ILBM
\item[{{\it ppmtomitsu}}]
convert a portable pixmap to a Mitsubishi S340-10 file
\item[{{\it ppmtopcx}}]
convert portable pixmap to PC Paintbrush format
\item[{{\it ppmtopgm}}]
convert portable pixmap to portable graymap
\item[{{\it ppmtopi1}}]
convert portable pixmap to Atari Degas .pi1
\item[{{\it ppmtopict}}]
convert portable pixmap to Macintosh PICT
\item[{{\it ppmtopj}}]
convert portable pixmap to HP PaintJet file
\item[{{\it ppmtopuzz}}]
convert portable pixmap to X11 ``puzzle'' file
\item[{{\it ppmtorgb3}}]
separate a portable pixmap into three portable graymaps
\item[{{\it ppmtosixel}}]
convert portable pixmap to DEC sixel format
\item[{{\it ppmtotga}}]
convert portable pixmap to TrueVision Targa file
\item[{{\it ppmtouil}}]
convert portable pixmap to Motif UIL icon file
\item[{{\it ppmtoxpm}}]
convert portable pixmap to XPM format
\item[{{\it ppmtoyuv}}]
convert portable pixmap to Abekas YUV format
\item[{{\it qrttoppm}}]
convert QRT ray-tracer output to portable pixmap
\item[{{\it rawtoppm}}]
convert raw RGB bytes to portable pixmap
\item[{{\it rgb3toppm}}]
combine three portable graymaps into one portable pixmap
\item[{{\it sldtoppm}}]
convert an AutoCAD slide file into a portable pixmap
\item[{{\it spctoppm}}]
convert Atari compressed Spectrum to portable pixmap
\item[{{\it sputoppm}}]
convert Atari uncompressed Spectrum to portable pixmap
\item[{{\it tgatoppm}}]
convert TrueVision Targa file to portable pixmap
\item[{{\it ximtoppm}}]
convert Xim to portable pixmap
\item[{{\it xvminitoppm}}]
convert a XV ``thumbnail'' picture to PPM
\item[{{\it xpmtoppm}}]
convert XPM format to portable pixmap
\item[{{\it yuvtoppm}}]
convert Abekas YUV format to portable pixmap
\item[{{\it anytopnm}}]
script to attempt to convert any format to P?M
\item[{{\it pnmalias}}]
antialias a portable anyumap.
\item[{{\it pnmarith}}]
perform arithmetic on two portable anymaps
\item[{{\it pnmcat}}]
concatenate portable anymaps
\item[{{\it pnmconvol}}]
general MxN convolution on a portable anymap
\item[{{\it pnmcrop}}]
crop all like-colored borders off a portable anymap
\item[{{\it pnmcut}}]
select a rectangular region from a portable anymap
\item[{{\it pnmdepth}}]
change the maxval in a portable anymap
\item[{{\it pnmenlarge}}]
enlarge a portable anymap N times
\item[{{\it pnmfile}}]
describe a portable anymap
\item[{{\it pnmflip}}]
perform one or more flip operations on a portable anymap
\item[{{\it pnmgamma}}]
perform gamma correction on a portable anymap
\item[{{\it pnmindex}}]
script to build a visual index of a bunch of anymaps
\item[{{\it pnminvert}}]
invert a portable anymap
\item[{{\it pnmmargin}}]
script to add a margin to a portable anymap
\item[{{\it pnmmerge}}]
merge wrapper routine
\item[{{\it pnmnoraw}}]
force a portable anymap into ASCII format
\item[{{\it pnmpaste}}]
paste a rectangle into a portable anymap
\item[{{\it pnmrotate}}]
rotate a portable anymap
\item[{{\it pnmscale}}]
scale a portable anymap
\item[{{\it pnmshear}}]
shear a portable anymap
\item[{{\it pnmsmooth}}]
script that uses pnmconvol to smooth a anymap
\item[{{\it pnmtile}}]
replicate a portable anymap into a specified size
\item[{{\it pnmtofits}}]
convert a portable anymap into FITS format
\item[{{\it pnmtogif}}]
convert portable anymap to GIF
\item[{{\it pnmtops}}]
convert portable anymap to PostScript
\item[{{\it pnmtorast}}]
convert portable anymap to Sun raster file
\item[{{\it pnmtotiff}}]
convert portable anymap to TIFF file
\item[{{\it pnmtoxwd}}]
convert portable anymap to X11 window dump
\item[{{\it fitstopnm}}]
fitstopnm - convert a FITS file into a portable anymap
\item[{{\it pstopnm}}]
script to convert PS to portable anymap with GhostScript
\item[{{\it rasttopnm}}]
convert Sun raster file to portable anymap
\item[{{\it tifftopnm}}]
convert TIFF file to portable anymap
\item[{{\it xwdtopnm}}]
convert X10 or X11 window dump to portable anymap
\end{TPlist}

\shead{SEE ALSO}
There are a number of related image-manipulation tools:
\begin{TPlist}{{\it Fuzzy Pixmap Manipulation}}
\item[{{\it IM Raster Toolkit}}]
A portable and efficient format toolkit.  The format supports pixels of
arbitrary channels, components, and bit precisions, while allowing
compression and machine byte-order independence.  Support for image
manipulation, digital halftoning, and format conversion.  Previously
distributed on tape c/o the University of Waterloo (an {\it ftp}
version is to appear later).  Author: Alan Paeth
(awpaeth@watcgl.uwaterloo.ca).
\item[{{\it Utah RLE Toolkit}}]
Conversion and manipulation package, similar to {\it pbmplus}.  Available
via {\it ftp} as {\it cs.utah.edu: pub/toolkit-2.0.tar.Z} and {\it
ucsd.edu: graphics\-/utah-raster-toolkit.\-tar.Z}.
\item[{{\it Fuzzy Pixmap Manipulation}}]
Conversion and manipulation package, similar to {\it pbmplus}.  Version
1.0 available via {\it ftp} as {\it
nl.cs.cmu.edu: /usr/mlm/ftp/fbm.tar.Z}, {\it
uunet.uu.net: pub/fbm.tar.Z}, and {\it ucsd.edu: graphics/fbm.tar.Z}.
Author: Michael Mauldin (mlm@nl.cs.cmu.edu).
\item[{{\it Img Software Set}}]
Reads and writes its own image format, displaying results on an X11
screen, and does some image manipulations.  Version 1.3 is available via
{\it ftp} as {\it ftp.x.org: contrib/img\_1.3.tar.Z}, and {\it
venera.isi.edu: pub/img\_1.3.tar.Z}, along with a large collection of
color images.  Author: Paul Raveling (raveling@venera.isi.edu).
\item[{{\it Xim}}]
Reads and writes its own image format, displays on an X11 screen, and
does some image manipulations.  Available in your nearest X11R4 source
tree as {\it contrib/clients/xim}.  A more recent version is available
via ftp from {\it video.mit.edu}.  It uses X11R4 and the OSF/Motif
toolkit to provide basic interactive image manipulation and reads/writes
GIF, xwd, xbm, tiff, rle, xim, and other formats.  Author: Philip R.
Thompson.
\item[{{\it xloadimage}}]
Reads in images in various formats and displays them on an X11 screen.
Available via {\it ftp} as {\it
ftp.x.org: contrib/xloadimage$\ast$}, and in your nearest {\it
comp.sources.\-x} archive.  Author: Jim Frost (madd@std.\-com).
\item[{{\it TIFF\ Software}}]
Nice portable library for reading and writing TIFF files, plus a few
tools for manipulating them and reading other formats.  Available via
{\it ftp} as {\it sgi.com: graphics/tiff/$\ast$.tar.Z} or {\it
uunet.uu.net: graphics/tiff.tar.Z}.  Author: Sam Leffler
(sam@sgi.com).
\item[{{\it ALV}}]
A Sun-specific image toolkit.  Version 2.0.6 posted to {\it
comp.sources.\-sun} on 11 December 1989.  Also available via email to {\it
alv-users-request @cs.bris.ac.uk}.
\item[{{\it popi}}]
An image manipulation language.  Version 2.1 posted to {\it
comp.\-sources.\-misc} on 12 December 1989.
\item[{{\it ImageMagick,}}]
X11 package for display and interactive manipulation of images.  Uses its
own format (MIFF), and includes some converters.  Available via {\it
ftp} as {\it ftp.x.org: contrib/ImageMagick.$\ast$.tar.Z}.
\item[{{\it Khoros}}]
A huge (\~{}100 meg) graphical development environment based on X11R4.
Components include a visual programming language, code generators for
extending the visual language and adding new application packages to the
system, an interactive user interface editor, an interactive image
display package, an extensive library of image and signal processing
routines, and 2D/3D plotting packages.  Available via {\it ftp} as
{\it pprg.unm.edu: pub/khoros/$\ast$}.
\item[{JPEG package}]
JPEG is a a standardized compression method for full-color and gray-scale
images of ``real-world'' scenes; this experimental package includes
programs to compress gif and ppm format files to JPEG format ({\it
cjpeg(1L)),} and to decompress them ({\it djpeg(1L)).} Available by {\it
ftp} as {\it uunet.uu.net: graphics/jpeg/jpegsrc.v4.tar.Z}.
\end{TPlist}

libpbm(3), libpgm(3), libpnm(3), libppm(3), pbm(5), pgm(5), pnm(5),
ppm(5), rasterfile(1)

\shead{AUTHOR}
Distribution of 10 December 1991.  \copyright 1989, 1991 by Jef Poskanzer.

Feedback and questions are welcome. Please send them to:
\begin{IPlist}
\IPitem{{}}
{\nofill
        jef@netcom.com
        jef@well.sf.ca.us
        apple!well!jef
\fill}
\end{IPlist}

When sending bug reports, always include the output from running any
pbmplus program with the -version flag, including descriptions of the
type of system you are on, the compiler you use, and whether you are
using Makefiles or Imakefiles.

When suggesting new formats or features, please include whatever
documentation you have, and a uuencoded sample.  The response time will
depend upon my schedule and the complexity of the task;  if you need it
right away, or it is a complicated job, you might consider paying me.

The Usenet newsgroup {\it alt.graphics.pixutils} is a forum for
discussion of image conversion and editing packages.  Posting queries
there may be better than mailing them to me, since it allows other people
to help provide answers.

Permission to use, copy, modify, and distribute this software and its
documentation for any purpose and without fee is hereby granted, provided
that the above copyright notice appear in all copies and that both that
copyright notice and this permission notice appear in supporting
documentation.  This software is provided ``as is'' without express or
implied warranty.  Thus, you may do what you want with this software.
Build it into your package, steal code from it, whatever.  Just be sure
to let people know where it came from.

% -*-LaTeX-*-
% Converted automatically from troff to LaTeX
% by tr2latex ($Revision: 2.2 $$Date: 1992/04/27 15:13:26 $ by C. Engel)
% on Fri Feb  4 14:35:11 1994
% tr2latex was written by Kamal Al-Yahya at Stanford University
% (Kamal%Hanauma@SU-SCORE.ARPA)
% and substantially enhanced by Christian Engel at RWTH Aachen
% (krischan@informatik.rwth-aachen.de).
%
% troff input file: asciitopgm.1

\newpage
%--------------------------------------------------
% start of input file: asciitopgm.1
%
\phead{asciitopgm}{1}{26 December 1994}{}{}

%.IX asciitopgm
\shead{NAME}
asciitopgm - convert ASCII graphics into a portable graymap
\shead{SYNOPSIS}
{\bf asciitopgm}
{\it [}{\rm -d}
{\rm divisor}{\it ]}
{\it height width}
{\rm [}{\it asciifile}{\rm ]}
\shead{DESCRIPTION}
Reads ASCII data as input.
Produces a portable graymap with pixel values which are an approximation
of the ``brightness'' of the ASCII characters,
assuming black-on-white printing.
In other words, a capital M is very dark, a period is ver light,
and a space is white.
Input lines which are fewer than
{\it width}
characters are automatically padded with spaces.
\par
The
{\it divisor}
argument is a floating-point number by which the output pixels are
divided; the default value is 1.0.
This can be used to adjust the brightness of the graymap:
for example, if the image is too dim, reduce the divisor.
\par
In keeping with (I believe) Fortran line-printer conventions,
input lines beginning with a + (plus) character are assumed
to ``overstrike'' the previous line, allowing a larger range of gray values.
\par
This tool contradicts the message in the
{\it pbmtoascii}
manual: ``Note that there is no asciitopbm tool - this
transformation is one-way.''
\shead{BUGS}
The table of ASCII-to-grey values is subject to interpretation,
and, of course, depends on the typeface intended for the input.
\shead{SEE ALSO}
pbmtoascii(1), pgm(5)
\shead{AUTHOR}
Wilson H. Bent. Jr. (whb@usc.edu)
%
% end of input file: asciitopgm.1
%--------------------------------------------------

% -*-LaTeX-*-
% Converted automatically from troff to LaTeX
% by tr2latex ($Revision: 2.2 $$Date: 1992/04/27 15:13:26 $ by C. Engel)
% on Fri Oct  8 16:07:54 1993
% tr2latex was written by Kamal Al-Yahya at Stanford University
% (Kamal%Hanauma@SU-SCORE.ARPA)
% and substantially enhanced by Christian Engel at RWTH Aachen
% (krischan@informatik.rwth-aachen.de).
%
% troff input file: atktopbm.1

\newpage
%--------------------------------------------------
% start of input file: atktopbm.1
%
\phead{atktopbm}{1}{26 September 1991}{}{}

%.IX atktopbm
\shead{NAME}
atktopbm - convert Andrew Toolkit raster object to portable bitmap
\shead{SYNOPSIS}
{\bf atktopbm}
{\rm [}{\it atkfile}{\rm ]}
\shead{DESCRIPTION}
Reads an Andrew Toolkit raster object as input.
%.IX "Andrew Toolkit raster object"
Produces a portable bitmap as output.
\shead{SEE ALSO}
pbmtoatk(1), pbm(5)
\shead{AUTHOR}
\copyright 1991 by Bill Janssen.
% Permission to use, copy, modify, and distribute this software and its
% documentation for any purpose and without fee is hereby granted, provided
% that the above copyright notice appear in all copies and that both that
% copyright notice and this permission notice appear in supporting
% documentation.  This software is provided "as is" without express or
% implied warranty.
%
% end of input file: atktopbm.1
%--------------------------------------------------
 
% -*-LaTeX-*-
% Converted automatically from troff to LaTeX
% by tr2latex ($Revision: 2.2 $$Date: 1992/04/27 15:13:26 $ by C. Engel)
% on Fri Oct  8 16:08:35 1993
% tr2latex was written by Kamal Al-Yahya at Stanford University
% (Kamal%Hanauma@SU-SCORE.ARPA)
% and substantially enhanced by Christian Engel at RWTH Aachen
% (krischan@informatik.rwth-aachen.de).
%
% troff input file: bioradtopgm.1

\newpage
%--------------------------------------------------
% start of input file: bioradtopgm.1
%
\phead{bioradtopgm}{1}{28 June 1993}{}{}

%.IX bioradtopgm
\shead{NAME}
bioradtopgm - convert a Biorad confocal file into a portable graymap
\shead{SYNOPSIS}
{\bf bioradtopgm}
{\rm [}{\bf -image\#}{\rm ]}
{\rm [}{\it imagedata}{\rm ]}
\shead{DESCRIPTION}
Reads a Biorad confocal file as input.
Produces a portable graymap as output.
If the resulting image is upside down, run it through
{\bf pnmflip\ -tb .}
%.IX pnmflip
\shead{OPTIONS}
\begin{TPlist}{{\bf -image\#}}
\item[{{\bf -image\#}}]
A Biorad image file may contain more than one image. With this flag,
you can specify which image to extract (only one at a time). The first
image in the file has number zero. If no
image number is supplied, only information about the image size and
the number of images in the input is printed out. No output is produced.
\end{TPlist}

\shead{BUGS}
A Biorad image may be in word format. If PbmPlus is not compiled with the
``BIGGRAYS'' flag, word files can not be converted. See the Makefile.
\shead{SEE ALSO}
pgm(5), pnmflip(1)
\shead{AUTHORS}
\copyright 1993 by Oliver Trepte (oliver@fysik4.kth.se).
\nwl
% Permission to use, copy, modify, and distribute this software and its
% documentation for any purpose and without fee is hereby granted, provided
% that the above copyright notice appear in all copies and that both that
% copyright notice and this permission notice appear in supporting
% documentation.  This software is provided "as is" without express or
% implied warranty.
%
% end of input file: bioradtopgm.1
%--------------------------------------------------
 
% -*-LaTeX-*-
% Converted automatically from troff to LaTeX
% by tr2latex ($Revision: 2.2 $$Date: 1992/04/27 15:13:26 $ by C. Engel)
% on Fri Oct  8 16:09:07 1993
% tr2latex was written by Kamal Al-Yahya at Stanford University
% (Kamal%Hanauma@SU-SCORE.ARPA)
% and substantially enhanced by Christian Engel at RWTH Aachen
% (krischan@informatik.rwth-aachen.de).
%
% troff input file: bmptoppm.1

\newpage
%--------------------------------------------------
% start of input file: bmptoppm.1
%
\phead{bmptoppm}{1}{26 Oct 1992}{}{}

%.IX bmptoppm
\shead{NAME}
bmptoppm -- convert a BMP file into a portable pixmap
\shead{SYNOPSIS}
{\bf bmptoppm}
{\rm [}{\it bmpfile}{\rm ]}
\shead{DESCRIPTION}
Reads a Microsoft Windows or OS/2 BMP file as input.
%.IX BMP
Produces a portable pixmap as output.
\shead{SEE ALSO}
ppmtobmp(1),
ppm(5)
\shead{AUTHOR}
\copyright 1992 by David W. Sanderson.
% Permission to use, copy, modify, and distribute this software and
% its documentation for any purpose and without fee is hereby granted,
% provided that the above copyright notice appear in all copies and
% that both that copyright notice and this permission notice appear in
% supporting documentation.  This software is provided "as is" without
% express or implied warranty.
%
% end of input file: bmptoppm.1
%--------------------------------------------------
 
% -*-LaTeX-*-
% Converted automatically from troff to LaTeX
% by tr2latex ($Revision: 2.2 $$Date: 1992/04/27 15:13:26 $ by C. Engel)
% on Fri Oct  8 16:07:54 1993
% tr2latex was written by Kamal Al-Yahya at Stanford University
% (Kamal%Hanauma@SU-SCORE.ARPA)
% and substantially enhanced by Christian Engel at RWTH Aachen
% (krischan@informatik.rwth-aachen.de).
%
% troff input file: brushtopbm.1

\newpage
%--------------------------------------------------
% start of input file: brushtopbm.1
%
\phead{brushtopbm}{1}{28 August 1988}{}{}

%.IX brushtopbm
\shead{NAME}
brushtopbm - convert a doodle brush file into a portable bitmap
\shead{SYNOPSIS}
{\bf brushtopbm}
{\rm [}{\it brushfile}{\rm ]}
\shead{DESCRIPTION}
Reads a Xerox doodle brush file as input.
%.IX "Xerox doodle brush format"
Produces a portable bitmap as output.
\par
Note that there is currently no pbmtobrush tool.
\shead{SEE ALSO}
pbm(5)
\shead{AUTHOR}
\copyright 1988 by Jef Poskanzer.
% Permission to use, copy, modify, and distribute this software and its
% documentation for any purpose and without fee is hereby granted, provided
% that the above copyright notice appear in all copies and that both that
% copyright notice and this permission notice appear in supporting
% documentation.  This software is provided "as is" without express or
% implied warranty.
%
% end of input file: brushtopbm.1
%--------------------------------------------------
 
% -*-LaTeX-*-
% Converted automatically from troff to LaTeX
% by tr2latex ($Revision: 2.2 $$Date: 1992/04/27 15:13:26 $ by C. Engel)
% on Fri Oct  8 16:07:55 1993
% tr2latex was written by Kamal Al-Yahya at Stanford University
% (Kamal%Hanauma@SU-SCORE.ARPA)
% and substantially enhanced by Christian Engel at RWTH Aachen
% (krischan@informatik.rwth-aachen.de).
%
% troff input file: cmuwmtopbm.1

\newpage
%--------------------------------------------------
% start of input file: cmuwmtopbm.1
%
\phead{cmuwmtopbm}{1}{15 April 1989}{}{}

%.IX cmuwmtopbm
\shead{NAME}
cmuwmtopbm - convert a CMU window manager bitmap into a portable bitmap
\shead{SYNOPSIS}
{\bf cmuwmtopbm}
{\rm [}{\it cmuwmfile}{\rm ]}
\shead{DESCRIPTION}
Reads a CMU window manager bitmap as input.
%.IX "CMU window manager bitmap"
Produces a portable bitmap as output.
\shead{SEE ALSO}
pbmtocmuwm(1), pbm(5)
\shead{AUTHOR}
\copyright 1989 by Jef Poskanzer.
% Permission to use, copy, modify, and distribute this software and its
% documentation for any purpose and without fee is hereby granted, provided
% that the above copyright notice appear in all copies and that both that
% copyright notice and this permission notice appear in supporting
% documentation.  This software is provided "as is" without express or
% implied warranty.
%
% end of input file: cmuwmtopbm.1
%--------------------------------------------------
 
% -*-LaTeX-*-
% Converted automatically from troff to LaTeX
% by tr2latex ($Revision: 2.2 $$Date: 1992/04/27 15:13:26 $ by C. Engel)
% on Mon Feb  7 08:49:58 1994
% tr2latex was written by Kamal Al-Yahya at Stanford University
% (Kamal%Hanauma@SU-SCORE.ARPA)
% and substantially enhanced by Christian Engel at RWTH Aachen
% (krischan@informatik.rwth-aachen.de).
%
% troff input file: fitstopnm.1

\newpage
%--------------------------------------------------
% start of input file: fitstopnm.1
%
\phead{fitstopnm}{1}{20 September 89}{}{}

\shead{NAME}
fitstopnm - convert a FITS file into a portable anymap
\shead{SYNOPSIS}
{\bf fitstopnm}
{\rm [}{\bf -image}
{\it N}{\rm ]}
{\rm [}{\bf -noraw}{\rm ]}
{\rm [}{\bf -scanmax}{\rm ]}
{\rm [}{\bf -printmax}{\rm ]}
{\rm [}{\bf -min}
{\it f}{\rm ]}
{\rm [}{\bf -max}
{\it f}{\rm ]}
{\rm [}{\it FITSfile}{\rm ]}
\shead{DESCRIPTION}
Reads a FITS file as input.
%.IX FITS
Produces a portable pixmap if the FITS file consists of 3 image planes
(NAXIS = 3 and NAXIS3 = 3), a portable graymap if the FITS file
consists of 2 image planes (NAXIS = 2), or whenever the 
{\bf --image}
flag is specified.
The results may need to be flipped top for bottom; if so, just
pipe the output through
{\bf pnmflip -tb.}
%.IX pnmflip
\shead{OPTIONS}
\par
The
{\bf -image}
option is for FITS files with three axes.
The assumption is that the third axis is for multiple images,
and this option lets you select which one you want.
\par
Flags 
{\bf -min}
and 
{\bf -max}
can be used to override the min and max values as read from the FITS
header or the image data if no DATAMIN and DATAMAX keywords are found.
Flag
{\bf -scanmax}
can be used to force the program to scan the data even when DATAMIN
and DATAMAX are found in the header. If 
{\bf -printmax}
is specified, the program will just print the min and max values and
quit.
Flag
{\bf -noraw}
can be used to force 
the program to produce an ASCII portable anymap.
\par
The program will tell what kind of anymap is writing.
All flags can be abbreviated to their shortest unique prefix.
\shead{REFERENCES}
FITS stands for Flexible Image Transport System.  A full description
can be found in Astronomy \& Astrophysics Supplement Series 44 (1981),
page 363.
\shead{SEE ALSO}
pnmtofits(1), pgm(5), pnmflip(1)
\shead{AUTHOR}
Copyright (C) 1989 by Jef Poskanzer, with modifications by 
Daniel Briggs (dbriggs@nrao.edu) and Alberto
Accomazzi (alberto@cfa.harvard.edu).
% Permission to use, copy, modify, and distribute this software and its
% documentation for any purpose and without fee is hereby granted, provided
% that the above copyright notice appear in all copies and that both that
% copyright notice and this permission notice appear in supporting
% documentation.  This software is provided "as is" without express or
% implied warranty.
%
% end of input file: fitstopnm.1
%--------------------------------------------------

% -*-LaTeX-*-
% Converted automatically from troff to LaTeX
% by tr2latex ($Revision: 2.2 $$Date: 1992/04/27 15:13:26 $ by C. Engel)
% on Fri Oct  8 16:08:24 1993
% tr2latex was written by Kamal Al-Yahya at Stanford University
% (Kamal%Hanauma@SU-SCORE.ARPA)
% and substantially enhanced by Christian Engel at RWTH Aachen
% (krischan@informatik.rwth-aachen.de).
%
% troff input file: fstopgm.1

\newpage
%--------------------------------------------------
% start of input file: fstopgm.1
%
\phead{fstopgm}{1}{06 April 89}{}{}

%.IX fstopgm
\shead{NAME}
fstopgm - convert a Usenix FaceSaver(tm) file into a portable graymap
\shead{SYNOPSIS}
{\bf fstopgm}
{\rm [}{\it fsfile}{\rm ]}
\shead{DESCRIPTION}
Reads a Usenix FaceSaver(tm) file as input.
%.IX FaceSaver
Produces a portable graymap as output.
\par
FaceSaver(tm) files sometimes have rectangular pixels.
While
{\it fstopgm}
won't re-scale them into square pixels for you,
it will give you the precise
{\it pnmscale}
command that will do the job.
Because of this, reading a FaceSaver(tm) image is a two-step process.
First you do:
\nofill
  fstopgm $>$ /dev/null
\fill
This will tell you whether you need to use
{\it pnmscale.}
Then use one of the following pipelines:
\nofill
  fstopgm $|$ pgmnorm
  fstopgm $|$ pnmscale -whatever $|$ pgmnorm
\fill
To go to PBM, you want something more like one of these:
\nofill
  fstopgm $|$ pnmenlarge 3 $|$ pgmnorm $|$ pgmtopbm
  fstopgm $|$ pnmenlarge 3 $|$ pnmscale $<$whatever$>$ $|$ pgmnorm $|$ pgmtopbm
\fill
You want to enlarge when going to a bitmap because otherwise you lose
information; but enlarging by more than 3 does not look good.
\par
FaceSaver is a registered trademark of Metron Computerware Ltd. of
Oakland, CA.
\shead{SEE ALSO}
pgmtofs(1), pgm(5), pgmnorm(1), pnmenlarge(1), pnmscale(1), pgmtopbm(1)
\shead{AUTHOR}
\copyright 1989 by Jef Poskanzer.
% Permission to use, copy, modify, and distribute this software and its
% documentation for any purpose and without fee is hereby granted, provided
% that the above copyright notice appear in all copies and that both that
% copyright notice and this permission notice appear in supporting
% documentation.  This software is provided "as is" without express or
% implied warranty.
%
% end of input file: fstopgm.1
%--------------------------------------------------
 
% -*-LaTeX-*-
% Converted automatically from troff to LaTeX
% by tr2latex ($Revision: 2.2 $$Date: 1992/04/27 15:13:26 $ by C. Engel)
% on Fri Oct  8 16:07:55 1993
% tr2latex was written by Kamal Al-Yahya at Stanford University
% (Kamal%Hanauma@SU-SCORE.ARPA)
% and substantially enhanced by Christian Engel at RWTH Aachen
% (krischan@informatik.rwth-aachen.de).
%
% troff input file: g3topbm.1

\newpage
%--------------------------------------------------
% start of input file: g3topbm.1
%
\phead{g3topbm}{1}{02 October 1989}{}{}

%.IX g3topbm
\shead{NAME}
g3topbm - convert a Group 3 fax file into a portable bitmap
\shead{SYNOPSIS}
{\bf g3topbm}
{\rm [}{\bf -kludge}{\rm ]}
{\rm [}{\bf -reversebits}{\rm ]}
{\rm [}{\bf -stretch}{\rm ]}
{\rm [}{\it g3file}{\rm ]}
\shead{DESCRIPTION}
Reads a Group 3 fax file as input.
%.IX "Group 3 fax"
%.IX fax
Produces a portable bitmap as output.
\shead{OPTIONS}
\begin{TPlist}{{\bf -kludge}}
\item[{{\bf -kludge}}]
Tells
{\it g3topbm}
to ignore the first few lines of the file;
sometimes fax files have some junk at the beginning.
\item[{{\bf -reversebits}}]
Tells
{\it g3topbm}
to interpret bits least-significant
first, instead of the default most-significant first.
Apparently some fax modems do it one way and others do it the other way.
If you get a whole bunch of ``bad code word'' messages, try using this
flag.
\item[{{\bf -stretch}}]
Tells
{\it g3topbm}
to stretch the image vertically by
duplicating each row.
This is for the low-quality transmission mode.
\end{TPlist}

\par
All flags can be abbreviated to their shortest unique prefix.
\shead{REFERENCES}
The standard for Group 3 fax is defined in CCITT Recommendation T.4.
\shead{BUGS}
Probably.
\shead{SEE ALSO}
pbmtog3(1), pbm(5)
\shead{AUTHOR}
\copyright 1989 by Paul Haeberli (paul@manray.sgi.com).
% Permission to use, copy, modify, and distribute this software and its
% documentation for any purpose and without fee is hereby granted, provided
% that the above copyright notice appear in all copies and that both that
% copyright notice and this permission notice appear in supporting
% documentation.  This software is provided "as is" without express or
% implied warranty.
%
% end of input file: g3topbm.1
%--------------------------------------------------
 
% -*-LaTeX-*-
% Converted automatically from troff to LaTeX
% by tr2latex ($Revision: 2.2 $$Date: 1992/04/27 15:13:26 $ by C. Engel)
% on Fri Oct  8 16:07:56 1993
% tr2latex was written by Kamal Al-Yahya at Stanford University
% (Kamal%Hanauma@SU-SCORE.ARPA)
% and substantially enhanced by Christian Engel at RWTH Aachen
% (krischan@informatik.rwth-aachen.de).
%
% troff input file: gemtopbm.1

\newpage
%--------------------------------------------------
% start of input file: gemtopbm.1
%
\phead{gemtopbm}{1}{3 December 1988}{}{}

%.IX gemtopbm
\shead{NAME}
gemtopbm - convert a GEM .img file into a portable bitmap
\shead{SYNOPSIS}
{\bf gemtopbm}
{\rm [}{\bf -d}{\rm ]}
{\it gemfile}
\shead{DESCRIPTION}
Reads a GEM .img file as input.
%.IX GEM
Produces a portable bitmap as output.
\shead{OPTIONS}
\begin{TPlist}{{\bf -d}}
\item[{{\bf -d}}]
Produce output describing the contents of the .img file.
\end{TPlist}

\shead{BUGS}
Does not support file containing more than one plane.
Can't read from standard input.
\shead{SEE ALSO}
pbmtogem(1), pbm(5)
\shead{AUTHOR}
\copyright 1988 Diomidis D. Spinellis (dds@cc.ic.ac.uk).
% Permission to use, copy, modify, and distribute this software and its
% documentation for any purpose and without fee is hereby granted,
% provided that the above copyright notice appear in all copies and that
% both that copyright notice and this permission notice appear in
% supporting documentation.
%
% end of input file: gemtopbm.1
%--------------------------------------------------
 
% -*-LaTeX-*-
% Converted automatically from troff to LaTeX
% by tr2latex ($Revision: 2.2 $$Date: 1992/04/27 15:13:26 $ by C. Engel)
% on Fri Oct  8 16:09:27 1993
% tr2latex was written by Kamal Al-Yahya at Stanford University
% (Kamal%Hanauma@SU-SCORE.ARPA)
% and substantially enhanced by Christian Engel at RWTH Aachen
% (krischan@informatik.rwth-aachen.de).
%
% troff input file: giftopnm.1

\newpage
%--------------------------------------------------
% start of input file: giftopnm.1
%
\phead{giftopnm}{1}{29 September 1993}{}{}

%.IX giftopnm
\shead{NAME}
giftopnm - convert a GIF file into a portable anymap
\shead{SYNOPSIS}
{\bf giftopnm}
{\rm [}{\bf -verbose}{\rm ]}
{\rm [}{\bf -comments}{\rm ]}
{\rm [}{\bf -image}
{\it N}{\rm ]}
{\rm [}{\it GIFfile}{\rm ]}
\shead{DESCRIPTION}
Reads a GIF file for input, and outputs portable anymap.
%.IX GIF
\shead{OPTIONS}
\begin{TPlist}{{\bf -verbose}}
\item[{{\bf -verbose}}]
Produces verbose output about the GIF file input.
\item[{{\bf -comments}}]
Only outputs GIF89 comment fields.
\item[{{\bf -image}}]
Output the specified gif image from the
input gif archive (where
{\it N}
is '1', '2', '20'...).
Normally there is only one image per file, so this option
is not needed.
\end{TPlist}

\par
All flags can be abbreviated to their shortest unique prefix.
\shead{BUGS}
This does not correctly handle the Plain Text Extension of the GIF89
standard, since I did not have any example input files containing them.
\shead{SEE ALSO}
ppmtogif(1), ppm(5)
\shead{AUTHOR}
\copyright 1993 by David Koblas (koblas@netcom.com)
% Permission to use, copy, modify, and distribute this software and its
% documentation for any purpose and without fee is hereby granted,
% provided that the above copyright notice appear in all copies and that
% both that copyright notice and this permission notice appear in
% supporting documentation.
%
% end of input file: giftopnm.1
%--------------------------------------------------
 
% -*-LaTeX-*-
% Converted automatically from troff to LaTeX
% by tr2latex ($Revision: 2.2 $$Date: 1992/04/27 15:13:26 $ by C. Engel)
% on Fri Oct  8 16:08:40 1993
% tr2latex was written by Kamal Al-Yahya at Stanford University
% (Kamal%Hanauma@SU-SCORE.ARPA)
% and substantially enhanced by Christian Engel at RWTH Aachen
% (krischan@informatik.rwth-aachen.de).
%
% troff input file: gouldtoppm.1

\newpage
%--------------------------------------------------
% start of input file: gouldtoppm.1
%
\phead{gouldtoppm}{1}{20 May 1990}{}{}

%.IX gouldtoppm
\shead{NAME}
gouldtoppm - convert Gould scanner file into a portable pixmap
\shead{SYNOPSIS}
{\bf gouldtoppm}
{\rm [}{\it gouldfile}{\rm ]}
\shead{DESCRIPTION}
Reads a file produced by the Gould scanner as input.
%.IX "Gould scanner"
Produces a portable pixmap as output.
\shead{SEE ALSO}
ppm(5)
\shead{AUTHOR}
\copyright 1990 by Stephen Paul Lesniewski.
% Permission to use, copy, modify, and distribute this software and its
% documentation for any purpose and without fee is hereby granted, provided
% that the above copyright notice appear in all copies and that both that
% copyright notice and this permission notice appear in supporting
% documentation.  This software is provided "as is" without express or
% implied warranty.
%
% end of input file: gouldtoppm.1
%--------------------------------------------------
 
% -*-LaTeX-*-
% Converted automatically from troff to LaTeX
% by tr2latex ($Revision: 2.2 $$Date: 1992/04/27 15:13:26 $ by C. Engel)
% on Fri Oct  8 16:08:25 1993
% tr2latex was written by Kamal Al-Yahya at Stanford University
% (Kamal%Hanauma@SU-SCORE.ARPA)
% and substantially enhanced by Christian Engel at RWTH Aachen
% (krischan@informatik.rwth-aachen.de).
%
% troff input file: hipstopgm.1

\newpage
%--------------------------------------------------
% start of input file: hipstopgm.1
%
\phead{hipstopgm}{1}{24 August 89}{}{}

%.IX hipstopgm
\shead{NAME}
hipstopgm - convert a HIPS file into a portable graymap
\shead{SYNOPSIS}
{\bf hipstopgm}
{\rm [}{\it hipsfile}{\rm ]}
\shead{DESCRIPTION}
Reads a HIPS file as input.
%.IX HIPS
Produces a portable graymap as output.
\par
If the HIPS file contains more than one frame in sequence, hipstopgm
will concatenate all the frames vertically.
\par
HIPS is a format developed at the Human Information Processing
Laboratory, NYU.
\shead{SEE ALSO}
pgm(5)
\shead{AUTHOR}
\copyright 1989 by Jef Poskanzer.
% Permission to use, copy, modify, and distribute this software and its
% documentation for any purpose and without fee is hereby granted, provided
% that the above copyright notice appear in all copies and that both that
% copyright notice and this permission notice appear in supporting
% documentation.  This software is provided "as is" without express or
% implied warranty.
%
% end of input file: hipstopgm.1
%--------------------------------------------------
 
% -*-LaTeX-*-
% Converted automatically from troff to LaTeX
% by tr2latex ($Revision: 2.2 $$Date: 1992/04/27 15:13:26 $ by C. Engel)
% on Fri Oct  8 16:07:57 1993
% tr2latex was written by Kamal Al-Yahya at Stanford University
% (Kamal%Hanauma@SU-SCORE.ARPA)
% and substantially enhanced by Christian Engel at RWTH Aachen
% (krischan@informatik.rwth-aachen.de).
%
% troff input file: icontopbm.1

\newpage
%--------------------------------------------------
% start of input file: icontopbm.1
%
\phead{icontopbm}{1}{31 August 1988}{}{}

%.IX icontopbm
\shead{NAME}
icontopbm - convert a Sun icon into a portable bitmap
\shead{SYNOPSIS}
{\bf icontopbm}
{\rm [}{\it iconfile}{\rm ]}
\shead{DESCRIPTION}
Reads a Sun icon as input.
%.IX Sun
%.IX "Sun icon format"
Produces a portable bitmap as output.
\shead{SEE ALSO}
pbmtoicon(1), pbm(5)
\shead{AUTHOR}
\copyright 1988 by Jef Poskanzer.
% Permission to use, copy, modify, and distribute this software and its
% documentation for any purpose and without fee is hereby granted, provided
% that the above copyright notice appear in all copies and that both that
% copyright notice and this permission notice appear in supporting
% documentation.  This software is provided "as is" without express or
% implied warranty.
%
% end of input file: icontopbm.1
%--------------------------------------------------
 
% -*-LaTeX-*-
% Converted automatically from troff to LaTeX
% by tr2latex ($Revision: 2.2 $$Date: 1992/04/27 15:13:26 $ by C. Engel)
% on Fri Oct  8 16:08:41 1993
% tr2latex was written by Kamal Al-Yahya at Stanford University
% (Kamal%Hanauma@SU-SCORE.ARPA)
% and substantially enhanced by Christian Engel at RWTH Aachen
% (krischan@informatik.rwth-aachen.de).
%
% troff input file: ilbmtoppm.1

\newpage
%--------------------------------------------------
% start of input file: ilbmtoppm.1
%
\phead{ilbmtoppm}{1}{20 June 1993}{}{}

%.IX ilbmtoppm
\shead{NAME}
ilbmtoppm - convert an ILBM file into a portable pixmap
\shead{SYNOPSIS}
{\bf ilbmtoppm}
{\rm [}{\bf -verbose}{\rm ]}
{\rm [}{\it ILBMfile}{\rm ]}
\shead{DESCRIPTION}
Reads an IFF ILBM file as input.
%.IX "IFF"
%.IX "ILBM"
Produces a portable pixmap as output.
Supported ILBM types are:
\begin{TPlist}{Normal ILBMs with 1-16 planes.}
\item[{Normal ILBMs with 1-16 planes.}]
\item[{Amiga Extra-Halfbrite (EHB)}]
\item[{Amiga Hold-and-Modify (HAM) with 3-16 planes.}]
%.IX "HAM"
\item[{24 bit.}]
\item[{Color map (BMHD + CMAP chunk only, nPlanes = 0).}]
\item[{Unofficial direct color.}]
1-16 planes for each color component.
\item[{Chunks used:}]
BMHD, CMAP, CAMG (only HAM \& EHB flags used), BODY
unofficial DCOL chunk to identify direct color ILBM
\item[{Chunks ignored:}]
GRAB, DEST, SPRT, CRNG, CCRT, CLUT, DPPV, DRNG, EPSF
\item[{Other chunks (ignored but displayed in verbose mode):}]
NAME, AUTH, (c), ANNO, DPI
\item[{Unknown chunks are skipped.}]
\end{TPlist}

\shead{OPTIONS}
\begin{TPlist}{{\bf -verbose}}
\item[{{\bf -verbose}}]
Give some informaton about the ILBM file.
\end{TPlist}

\shead{BUGS}
Probably.
\shead{REFERENCES}
Amiga ROM Kernel Reference Manual - Devices (3rd Ed.)
Addison Wesley, ISBN 0--201--56775--X
\shead{SEE ALSO}
ppm(5), ppmtoilbm(1)
\shead{AUTHORS}
\copyright 1989 by Jef Poskanzer.
\nwl
Modified June 1993 by Ingo Wilken (Ingo.Wilken@informatik.uni-oldenburg.de)
% Permission to use, copy, modify, and distribute this software and its
% documentation for any purpose and without fee is hereby granted, provided
% that the above copyright notice appear in all copies and that both that
% copyright notice and this permission notice appear in supporting
% documentation.  This software is provided "as is" without express or
% implied warranty.
%
% end of input file: ilbmtoppm.1
%--------------------------------------------------
 
% -*-LaTeX-*-
% Converted automatically from troff to LaTeX
% by tr2latex ($Revision: 2.2 $$Date: 1992/04/27 15:13:26 $ by C. Engel)
% on Fri Oct  8 16:08:41 1993
% tr2latex was written by Kamal Al-Yahya at Stanford University
% (Kamal%Hanauma@SU-SCORE.ARPA)
% and substantially enhanced by Christian Engel at RWTH Aachen
% (krischan@informatik.rwth-aachen.de).
%
% troff input file: imgtoppm.1

\newpage
%--------------------------------------------------
% start of input file: imgtoppm.1
%
\phead{imgtoppm}{1}{05 September 1989}{}{}

%.IX imgtoppm
\shead{NAME}
imgtoppm - convert an Img-whatnot file into a portable pixmap
\shead{SYNOPSIS}
{\bf imgtoppm}
{\rm [}{\it imgfile}{\rm ]}
\shead{DESCRIPTION}
Reads an Img-whatnot file as input.
Produces a portable pixmap as output.
The Img-whatnot toolkit is available for FTP on venera.isi.edu,
along with numerous images in this format.
%.IX Img-whatnot
\shead{SEE ALSO}
ppm(5)
\shead{AUTHOR}
Based on a simple conversion program posted to comp.graphics by Ed Falk.

\copyright 1989 by Jef Poskanzer.
% Permission to use, copy, modify, and distribute this software and its
% documentation for any purpose and without fee is hereby granted, provided
% that the above copyright notice appear in all copies and that both that
% copyright notice and this permission notice appear in supporting
% documentation.  This software is provided "as is" without express or
% implied warranty.
%
% end of input file: imgtoppm.1
%--------------------------------------------------
 
% -*-LaTeX-*-
% Converted automatically from troff to LaTeX
% by tr2latex ($Revision: 2.2 $$Date: 1992/04/27 15:13:26 $ by C. Engel)
% on Fri Oct  8 16:08:19 1993
% tr2latex was written by Kamal Al-Yahya at Stanford University
% (Kamal%Hanauma@SU-SCORE.ARPA)
% and substantially enhanced by Christian Engel at RWTH Aachen
% (krischan@informatik.rwth-aachen.de).
%
% troff input file: libpbm.3

\newpage
%--------------------------------------------------
% start of input file: libpbm.3
%
\phead{libpbm}{3}{}{}{}

\shead{NAME}
libpbm - functions to support portable bitmap programs
\shead{SYNOPSIS}
\def\Ss{\par\vspace{1.0\baselineskip}%.ft CW
\nofill
}
\def\Se{\fill
%.ft P
\par\vspace{1.0\baselineskip}}
\Ss
\#include $<$pbm.h$>$
cc ... libpbm.a
\Se
\shead{DESCRIPTION - PACKAGE-WIDE ROUTINES}
%.SS KEYWORD MATCHING
\Ss
int pm\_keymatch( char* str, char* keyword, int minchars )
\Se
Does a case-insensitive match of
{\bf str}
against
{\bf keyword}{\rm .}
{\bf str}
can be a leading sunstring of
{\bf keyword}{\rm ,}
but at least
{\bf minchars}
must be present.
%.SS LOG BASE TWO
\Ss
int pm\_maxvaltobits( int maxval )
int pm\_bitstomaxval( int bits )
\Se
Convert between a maxval and the minimum number of bits required
to hold it.
%.SS MESSAGES AND ERRORS
\Ss
void pm\_message( char* fmt, ... )
\Se
{\bf printf()}
style routine to write an informational message.
\Ss
void pm\_error( char* fmt, ... )
\Se
{\bf printf()}
style routine to write an error message and abort.
\Ss
void pm\_usage( char* usage )
\Se
Write a usage message.
The string should indicate what arguments are to be provided to the program.
%.SS GENERIC FILE MANAGEMENT
\Ss
FILE* pm\_openr( char* name )
\Se
Open the given file for reading, with appropriate error checking.
A filename of ``-'' is taken as equivalent to stdin.
\Ss
FILE* pm\_openw( char* name )
\Se
Open the given file for writing, with appropriate error checking.
\Ss
void pm\_close( FILE* fp )
\Se
Close the file descriptor, with appropriate error checking.
%.SS ENDIAN I/O
\Ss
int pm\_readbigshort( FILE* in, short* sP )
int pm\_writebigshort( FILE* out, short s )
int pm\_readbiglong( FILE* in, long* lP )
int pm\_writebiglong( FILE* out, long l )
int pm\_readlittleshort( FILE* in, short* sP )
int pm\_writelittleshort( FILE* out, short s )
int pm\_readlittlelong( FILE* in, long* lP )
int pm\_writelittlelong( FILE* out, long l )
\Se
Routines to read and write short and long ints in either big- or
little-endian byte order.
\shead{DESCRIPTION - PBM-SPECIFIC ROUTINES}
%.SS TYPES AND CONSTANTS
\Ss
typedef ... bit;
\#define PBM\_WHITE ...
\#define PBM\_BLACK ...
\Se
each
{\bf bit}
should contain only the values of
{\bf PBM\_WHITE}
or
{\bf PBM\_BLACK}{\rm .}
\Ss
\#define PBM\_FORMAT ...
\#define RPBM\_FORMAT ...
\#define PBM\_TYPE PBM\_FORMAT
\#define PBM\_FORMAT\_TYPE(f) ...
\Se
For distinguishing different file formats and types.
%.SS INITIALIZATION
\Ss
void pbm\_init( int* argcP, char* argv[] )
\Se
All PBM programs must call this routine.
%.SS MEMORY MANAGEMENT
\Ss
bit** pbm\_allocarray( int cols, int rows )
\Se
Allocate an array of bits.
\Ss
bit* pbm\_allocrow( int cols )
\Se
Allocate a row of the given number of bits.
\Ss
void pbm\_freearray( bit** bits, int rows )
\Se
Free the array allocated with
{\bf pbm\_allocarray()}
containing the given number
of rows.
\Ss
void pbm\_freerow( bit* bitrow )
\Se
Free a row of bits.
%.SS READING FILES
\Ss
void pbm\_readpbminit( FILE* fp, int* colsP, int* rowsP, int* formatP )
\Se
Read the header from a PBM file, filling in the rows, cols and format
variables.
\Ss
void pbm\_readpbmrow( FILE* fp, bit* bitrow, int cols, int format )
\Se
Read a row of bits into the bitrow array.
Format and cols were filled in by
{\bf pbm\_readpbminit()}{\rm .}
\Ss
bit** pbm\_readpbm( FILE* fp, int* colsP, int* rowsP )
\Se
Read an entire bitmap file into memory, returning the allocated array and
filling in the rows and cols variables.
This function combines
{\bf pbm\_readpbminit()}{\rm ,}
{\bf pbm\_allocarray()}
and
{\bf pbm\_readpbmrow()}{\rm .}
\Ss
char* pm\_read\_unknown\_size( FILE* fp, long* nread )
\Se
Read an entire file or input stream of unknown size to a buffer.
Allocate memory more memory as needed. The calling routine has
to free the allocated buffer with
{\bf free()}{\rm .}
{\bf pm\_read\_unknown\_size()}
returns a pointer to the allocated buffer. The
{\bf nread}
argument returns the number of bytes read.
%.SS WRITING FILES
\Ss
void pbm\_writepbminit( FILE* fp, int cols, int rows, int forceplain )
\Se
Write the header for a portable bitmap file.
The forceplain flag forces a plain-format file to be written, as opposed
to a raw-format one.
\Ss
void pbm\_writepbmrow( FILE* fp, bit* bitrow, int cols, int forceplain )
\Se
Write a row from a portable bitmap.
\Ss
void pbm\_writepbm( FILE* fp, bit** bits, int cols, int rows, int forceplain )
\Se
Write the header and all data for a portable bitmap.
This function combines
{\bf pbm\_writepbminit()}
and
{\bf pbm\_writepbmrow()}{\rm .}
\shead{SEE ALSO}
libpgm(3), libppm(3), libpnm(3)
\shead{AUTHOR}
\copyright 1989, 1991 by Tony Hansen and Jef Poskanzer.
% Permission to use, copy, modify, and distribute this software and its
% documentation for any purpose and without fee is hereby granted, provided
% that the above copyright notice appear in all copies and that both that
% copyright notice and this permission notice appear in supporting
% documentation.  This software is provided "as is" without express or
% implied warranty.
%
% end of input file: libpbm.3
%--------------------------------------------------
 
% -*-LaTeX-*-
% Converted automatically from troff to LaTeX
% by tr2latex ($Revision: 2.2 $$Date: 1992/04/27 15:13:26 $ by C. Engel)
% on Fri Oct  8 16:08:37 1993
% tr2latex was written by Kamal Al-Yahya at Stanford University
% (Kamal%Hanauma@SU-SCORE.ARPA)
% and substantially enhanced by Christian Engel at RWTH Aachen
% (krischan@informatik.rwth-aachen.de).
%
% troff input file: libpgm.3

\newpage
%--------------------------------------------------
% start of input file: libpgm.3
%
\phead{libpgm}{3}{}{}{}

\shead{NAME}
libpgm - functions to support portable graymap programs
\shead{SYNOPSIS}
\def\Ss{\par\vspace{1.0\baselineskip}%.ft CW
\nofill
}
\def\Se{\fill
%.ft P
\par\vspace{1.0\baselineskip}}
\Ss
\#include $<$pgm.h$>$
cc ... libpgm.a libpbm.a
\Se
\shead{DESCRIPTION}
%.SS TYPES AND CONSTANTS
\Ss
typedef ... gray;
\#define PGM\_MAXMAXVAL ...
extern gray pgm\_pbmmaxval;
\Se
Each
{\bf gray}
should contain only the values between
{\bf 0}
and
{\bf PGM\_MAXMAXVAL}{\rm .}
{\bf pgm\_pbmmaxval}
is the maxval used when a PGM program reads a PBM file.
Normally it is 1; however, for some programs, a larger value gives better
results.
\Ss
\#define PGM\_FORMAT ...
\#define RPGM\_FORMAT ...
\#define PGM\_TYPE PGM\_FORMAT
int PGM\_FORMAT\_TYPE( int format )
\Se
For distinguishing different file formats and types.
%.SS INITIALIZATION
\Ss
void pgm\_init( int* argcP, char* argv[] )
\Se
All PGM programs must call this routine.
%.SS MEMORY MANAGEMENT
\Ss
gray** pgm\_allocarray( int cols, int rows )
\Se
Allocate an array of grays.
\Ss
gray* pgm\_allocrow( int cols )
\Se
Allocate a row of the given number of grays.
\Ss
void pgm\_freearray( gray** grays, int rows )
\Se
Free the array allocated with
{\bf pgm\_allocarray()}
containing the given number
of rows.
\Ss
void pgm\_freerow( gray* grayrow )
\Se
Free a row of grays.
%.SS READING FILES
\Ss
void pgm\_readpgminit( FILE* fp, int* colsP, int* rowsP, gray* maxvalP, int* formatP )
\Se
Read the header from a PGM file, filling in the rows, cols, maxval and format
variables.
\Ss
void pgm\_readpgmrow( FILE* fp, gray* grayrow, int cols, gray maxval, int format )
\Se
Read a row of grays into the grayrow array.
Format, cols, and maxval were filled in by
{\bf pgm\_readpgminit()}{\rm .}
\Ss
gray** pgm\_readpgm( FILE* fp, int* colsP, int* rowsP, gray* maxvalP )
\Se
Read an entire graymap file into memory, returning the allocated array and
filling in the rows, cols and maxval variables.
This function combines
{\bf pgm\_readpgminit()}{\rm ,}
{\bf pgm\_allocarray()}
and
{\bf pgm\_readpgmrow()}{\rm .}
%.SS WRITING FILES
\Ss
void pgm\_writepgminit( FILE* fp, int cols, int rows, gray maxval, int forceplain )
\Se
Write the header for a portable graymap file.
The forceplain flag forces a plain-format file to be written, as opposed
to a raw-format one.
\Ss
void pgm\_writepgmrow( FILE* fp, gray* grayrow, int cols, gray maxval, int forceplain )
\Se
Write a row from a portable graymap.
\Ss
void pgm\_writepgm( FILE* fp, gray** grays, int cols, int rows, gray maxval, int forceplain )
\Se
Write the header and all data for a portable graymap.
This function combines
{\bf pgm\_writepgminit()}
and
{\bf pgm\_writepgmrow()}{\rm .}
\shead{SEE ALSO}
libpbm(3), libppm(3), libpnm(3)
\shead{AUTHOR}
\copyright 1989, 1991 by Tony Hansen and Jef Poskanzer.
% Permission to use, copy, modify, and distribute this software and its
% documentation for any purpose and without fee is hereby granted, provided
% that the above copyright notice appear in all copies and that both that
% copyright notice and this permission notice appear in supporting
% documentation.  This software is provided "as is" without express or
% implied warranty.
%
% end of input file: libpgm.3
%--------------------------------------------------
 
% -*-LaTeX-*-
% Converted automatically from troff to LaTeX
% by tr2latex ($Revision: 2.2 $$Date: 1992/04/27 15:13:26 $ by C. Engel)
% on Fri Oct  8 16:09:34 1993
% tr2latex was written by Kamal Al-Yahya at Stanford University
% (Kamal%Hanauma@SU-SCORE.ARPA)
% and substantially enhanced by Christian Engel at RWTH Aachen
% (krischan@informatik.rwth-aachen.de).
%
% troff input file: libpnm.3

\newpage
%--------------------------------------------------
% start of input file: libpnm.3
%
\phead{libpnm}{3}{}{}{}

\shead{NAME}
libpnm - functions to support portable anymap programs
\shead{SYNOPSIS}
\def\Ss{\par\vspace{1.0\baselineskip}%.ft CW
\nofill
}
\def\Se{\fill
%.ft P
\par\vspace{1.0\baselineskip}}
\Ss
\#include $<$pnm.h$>$
cc ... libpnm.a libppm.a libpgm.a libpbm.a
\Se
\shead{DESCRIPTION}
%.SS TYPES AND CONSTANTS
\Ss
typedef ... xel;
typedef ... xelval;
\#define PNM\_MAXMAXVAL ...
extern xelval pnm\_pbmmaxval;
\Se
Each
{\bf xel}
contains three
{\bf xelval}{\rm s,}
each of which should contain only the values between
{\bf 0}
and
{\bf PNM\_MAXMAXVAL}{\rm .}
{\bf pnm\_pbmmaxval}
is the maxval used when a PNM program reads a PBM file.
Normally it is 1; however, for some programs, a larger value gives better
results.
%.SS XEL MANIPULATIONS
\Ss
xelval PNM\_GET1( xel x )
\Se
This macro extracts a single value from an xel, when you know it's
from a PBM or PGM file.
When it's from a PPM file, use
{\bf PPM\_GETR()}{\rm ,}
{\bf PPM\_GETG()}{\rm ,}
and
{\bf PPM\_GETB()}{\rm .}
\Ss
void PNM\_ASSIGN1( xel x, xelval v )
\Se
This macro assigns a single value to an xel, when you know it's
from a PBM or PGM file.
When it's from a PPM file, use
{\bf PPM\_ASSIGN()}{\rm .}
\Ss
int PNM\_EQUAL( xel x, xel y )
\Se
This macro checks two xels for equality.
\Ss
int PNM\_FORMAT\_TYPE( int format )
\Se
For distinguishing different file types.
%.SS INITIALIZATION
\Ss
void pnm\_init( int* argcP, char* argv[] )
\Se
All PNM programs must call this routine.
%.SS MEMORY MANAGEMENT
\Ss
xel** pnm\_allocarray( int cols, int rows )
\Se
Allocate an array of xels.
\Ss
xel* pnm\_allocrow( int cols )
\Se
Allocate a row of the given number of xels.
\Ss
void pnm\_freearray( xel** xels, int rows )
\Se
Free the array allocated with
{\bf pnm\_allocarray()}
containing the given number
of rows.
\Ss
void pnm\_freerow( xel* xelrow )
\Se
Free a row of xels.
%.SS READING FILES
\Ss
void pnm\_readpnminit( FILE* fp, int* colsP, int* rowsP, xelval* maxvalP, int* formatP )
\Se
Read the header from a PNM file, filling in the rows, cols, maxval and format
variables.
\Ss
void pnm\_readpnmrow( FILE* fp, xel* xelrow, int cols, xelval maxval, int format )
\Se
Read a row of xels into the xelrow array.
Format, cols, and maxval were filled in by
{\bf pnm\_readpnminit()}{\rm .}
\Ss
xel** pnm\_readpnm( FILE* fp, int* colsP, int* rowsP, xelval* maxvalP, int* formatP )
\Se
Read an entire anymap file into memory, returning the allocated array and
filling in the rows, cols, maxval, and format variables.
This function combines
{\bf pnm\_readpnminit()}{\rm ,}
{\bf pnm\_allocarray()}
and
{\bf pnm\_readpnmrow()}{\rm .}
Unlike the equivalent functions in PBM, PGM, and PPM, it returns the format
so you can tell what type the file is.
%.SS WRITING FILES
\Ss
void pnm\_writepnminit( FILE* fp, int cols, int rows, xelval maxval, int format, int forceplain )
\Se
Write the header for a portable anymap file.
Unlike the equivalent functions in PBM, PGM, and PPM, you have to specify
the output type.
The forceplain flag forces a plain-format file to be written, as opposed
to a raw-format one.
\Ss
void pnm\_writepnmrow( FILE* fp, xel* xelrow, int cols, xelval maxval, int format, int forceplain )
\Se
Write a row from a portable anymap.
\Ss
void pnm\_writepnm( FILE* fp, xel** xels, int cols, int rows, xelval maxval, int format, int forceplain )
\Se
Write the header and all data for a portable anymap.
This function combines
{\bf pnm\_writepnminit()}
and
{\bf pnm\_writepnmrow()}{\rm .}
%.SS FORMAT PROMOTION
\Ss
void pnm\_promoteformatrow( xel* xelrow, int cols, xelval maxval, int format, xelval newmaxval, int newformat )
\Se
Promote a row of xels from one maxval and format to a new set.
Used when combining multiple anymaps of different types - just
take the max of the maxvals and the max of the formats, and
promote them all to that.
\Ss
void pnm\_promoteformat( xel** xels, int cols, int rows, xelval maxval, int format, xelval newmaxval, int newformat )
\Se
Promote an entire anymap.
%.SS XEL MANIPULATION
\Ss
xel pnm\_whitexel( xelval maxval, int format )
xel pnm\_blackxel( xelval maxval, int format )
\Se
Return a white or black xel for the given maxval and format.
\Ss
void pnm\_invertxel( xel* x, xelval maxval, int format )
\Se
Invert an xel.
\Ss
xel pnm\_backgroundxelrow( xel* xelrow, int cols, xelval maxval, int format )
\Se
Figure out an appropriate background xel based on this row.
\Ss
xel pnm\_backgroundxel( xel** xels, int cols, int rows, xelval maxval, int format )
\Se
Figure out a background xel based on an entire anymap.
This can do a slightly better job than
{\bf pnm\_backgroundxelrow()}{\rm .}
\shead{SEE ALSO}
pbm(3), pgm(3), ppm(3)
\shead{AUTHOR}
\copyright 1989, 1991 by Tony Hansen and Jef Poskanzer.
% Permission to use, copy, modify, and distribute this software and its
% documentation for any purpose and without fee is hereby granted, provided
% that the above copyright notice appear in all copies and that both that
% copyright notice and this permission notice appear in supporting
% documentation.  This software is provided "as is" without express or
% implied warranty.
%
% end of input file: libpnm.3
%--------------------------------------------------
 
% -*-LaTeX-*-
% Converted automatically from troff to LaTeX
% by tr2latex ($Revision: 2.2 $$Date: 1992/04/27 15:13:26 $ by C. Engel)
% on Fri Oct  8 16:09:12 1993
% tr2latex was written by Kamal Al-Yahya at Stanford University
% (Kamal%Hanauma@SU-SCORE.ARPA)
% and substantially enhanced by Christian Engel at RWTH Aachen
% (krischan@informatik.rwth-aachen.de).
%
% troff input file: libppm.3

\newpage
%--------------------------------------------------
% start of input file: libppm.3
%
\phead{libppm}{3}{}{}{}

\shead{NAME}
libppm - functions to support portable pixmap programs
\shead{SYNOPSIS}
\def\Ss{\par\vspace{1.0\baselineskip}%.ft CW
\nofill
}
\def\Se{\fill
%.ft P
\par\vspace{1.0\baselineskip}}
\Ss
\#include $<$ppm.h$>$
cc ... libppm.a libpgm.a libpbm.a
\Se
\shead{DESCRIPTION}
%.SS TYPES AND CONSTANTS
\Ss
typedef ... pixel;
typedef ... pixval;
\#define PPM\_MAXMAXVAL ...
extern pixval ppm\_pbmmaxval;
\Se
Each
{\bf pixel}
contains three
{\bf pixval}{\rm s,}
each of which should contain only the values between
{\bf 0}
and
{\bf PPM\_MAXMAXVAL}{\rm .}
{\bf ppm\_pbmmaxval}
is the maxval used when a PPM program reads a PBM file.
Normally it is 1; however, for some programs, a larger value gives better
results.
\Ss
\#define PPM\_FORMAT ...
\#define RPPM\_FORMAT ...
\#define PPM\_TYPE PPM\_FORMAT
int PPM\_FORMAT\_TYPE( int format )
\Se
For distinguishing different file formats and types.
\Ss
pixval PPM\_GETR( pixel p )
pixval PPM\_GETG( pixel p )
pixval PPM\_GETB( pixel p )
\Se
These three macros retrieve the red, green or blue value from the given
pixel.
\Ss
void PPM\_ASSIGN( pixel p, pixval red, pixval grn, pixval blu )
\Se
This macro assigns the given red, green and blue values to the pixel.
\Ss
int PPM\_EQUAL( pixel p, pixel q )
\Se
This macro checks two pixels for equality.
\Ss
void PPM\_DEPTH( pixel newp, pixel p, pixval oldmaxval, pixval newmaxval )
\Se
This macro scales the colors of pixel
{\bf p}
according the old and new maximum values and assigns the new values to
{\bf newp}{\rm .}
It is intended to make writing ppmtowhatever easier.
\Ss
float PPM\_LUMIN( pixel p )
\Se
This macro determines the luminance of the pixel
{\bf p}{\rm .}
%.SS MEMORY MANAGEMENT
\Ss
pixel** ppm\_allocarray( int cols, int rows )
\Se
Allocate an array of pixels.
\Ss
pixel* ppm\_allocrow( int cols )
\Se
Allocate a row of the given number of pixels.
\Ss
void ppm\_freearray( pixel** pixels, int rows )
\Se
Free the array allocated with
{\bf ppm\_allocarray()}
containing the given number
of rows.
\Ss
void pbm\_freerow( pixel* pixelrow )
\Se
Free a row of pixels.
%.SS READING PBM FILES
\Ss
void ppm\_readppminit( FILE* fp, int* colsP, int* rowsP, pixval* maxvalP, int* formatP )
\Se
Read the header from a PPM file, filling in the rows, cols, maxval and format
variables.
\Ss
void ppm\_readppmrow( FILE* fp, pixel* pixelrow, int cols, pixval maxval, int format )
\Se
Read a row of pixels into the pixelrow array.
Format, cols, and maxval were filled in by
{\bf ppm\_readppminit()}{\rm .}
\Ss
pixel** ppm\_readppm( FILE* fp, int* colsP, int* rowsP, pixval* maxvalP )
\Se
Read an entire pixmap file into memory, returning the allocated array and
filling in the rows, cols and maxval variables.
This function combines
{\bf ppm\_readppminit()}{\rm ,}
{\bf ppm\_allocarray()}
and
{\bf ppm\_readppmrow()}{\rm .}
%.SS WRITING FILES
\Ss
void ppm\_writeppminit( FILE* fp, int cols, int rows, pixval maxval, int forceplain )
\Se
Write the header for a portable pixmap file.
The forceplain flag forces a plain-format file to be written, as opposed
to a raw-format one.
\Ss
void ppm\_writeppmrow( FILE* fp, pixel* pixelrow, int cols, pixval maxval, int forceplain )
\Se
Write a row from a portable pixmap.
\Ss
void ppm\_writeppm( FILE* fp, pixel** pixels, int cols, int rows, pixval maxval, int forceplain )
\Se
Write the header and all data for a portable pixmap.
This function combines
{\bf ppm\_writeppminit()}
and
{\bf ppm\_writeppmrow()}{\rm .}
%.SS COLOR NAMES
\Ss
pixel ppm\_parsecolor( char* colorname, pixval maxval )
\Se
Parses an ASCII color name into a pixel.
The color can be specified in three ways.  One, as a name, assuming
that a pointer to an X11-style color names file was compiled in.  Two,
as an X11-style hexadecimal number: \#rgb, \#rrggbb, \#rrrgggbbb, or
\#rrrrggggbbbb.  Three, as a triplet of decimal floating point numbers
separated by commas: r.r,g.g,b.b.
\Ss
char* ppm\_colorname( pixel* colorP, pixval maxval, int hexok )
\Se
Returns a pointer to a string describing the given color.
If the X11 color names file is available and the color appears in
it, that name is returned.
Otherwise, if the hexok flag is true then a hexadecimal colorspec
is returned; if hexok is false and the X11 color names file is
available, then the closest matching color is returned;
otherwise, it's an error.
\shead{SEE ALSO}
pbm(3), pgm(3)
\shead{AUTHOR}
\copyright 1989, 1991 by Tony Hansen and Jef Poskanzer.
% Permission to use, copy, modify, and distribute this software and its
% documentation for any purpose and without fee is hereby granted, provided
% that the above copyright notice appear in all copies and that both that
% copyright notice and this permission notice appear in supporting
% documentation.  This software is provided "as is" without express or
% implied warranty.
%
% end of input file: libppm.3
%--------------------------------------------------
 
% -*-LaTeX-*-
% Converted automatically from troff to LaTeX
% by tr2latex ($Revision: 2.2 $$Date: 1992/04/27 15:13:26 $ by C. Engel)
% on Fri Oct  8 16:08:25 1993
% tr2latex was written by Kamal Al-Yahya at Stanford University
% (Kamal%Hanauma@SU-SCORE.ARPA)
% and substantially enhanced by Christian Engel at RWTH Aachen
% (krischan@informatik.rwth-aachen.de).
%
% troff input file: lispmtopgm.1

\newpage
%--------------------------------------------------
% start of input file: lispmtopgm.1
%
\phead{lispmtopgm}{1}{06 March 1990}{}{}

%.IX lispmtopgm
\shead{NAME}
lispmtopgm - convert a Lisp Machine bitmap file into pgm format
\shead{SYNOPSIS}
{\bf lispmtopgm}
{\rm [}{\it lispmfile}{\rm ]}
\shead{DESCRIPTION}
Reads a Lisp Machine bitmap as input.
%.IX "Lisp Machine bitmap"
Produces a portable graymap as output.
\par
This is the file format written by the tv:write-bit-array-file function on
TI Explorer and Symbolics lisp machines.
\par
Multi-plane bitmaps on lisp machines are color; but the lispm image file
format does not include a color map, so we must treat it as a graymap 
instead.  This is unfortunate.
\shead{SEE ALSO}
pgmtolispm(1), pgm(5)
\shead{BUGS}
The Lispm bitmap file format is a bit quirky;  Usually the image in the file
has its width rounded up to the next higher multiple of 32, but not always.
If the width is not a multiple of 32, we don't deal with it properly, but 
because of the Lispm microcode, such arrays are probably not image data 
anyway.
\par
Also, the lispm code for saving bitmaps has a bug, in that if you are writing a
bitmap which is not mod32 across, the file may be up to 7 bits too short!  They
round down instead of up, and we don't handle this bug gracefully.
\par
No color.
\shead{AUTHOR}
\copyright 1991 by Jamie Zawinski and Jef Poskanzer.
% Permission to use, copy, modify, and distribute this software and its
% documentation for any purpose and without fee is hereby granted, provided
% that the above copyright notice appear in all copies and that both that
% copyright notice and this permission notice appear in supporting
% documentation.  This software is provided "as is" without express or
% implied warranty.
%
% end of input file: lispmtopgm.1
%--------------------------------------------------
 
% -*-LaTeX-*-
% Converted automatically from troff to LaTeX
% by tr2latex ($Revision: 2.2 $$Date: 1992/04/27 15:13:26 $ by C. Engel)
% on Fri Oct  8 16:07:57 1993
% tr2latex was written by Kamal Al-Yahya at Stanford University
% (Kamal%Hanauma@SU-SCORE.ARPA)
% and substantially enhanced by Christian Engel at RWTH Aachen
% (krischan@informatik.rwth-aachen.de).
%
% troff input file: macptopbm.1

\newpage
%--------------------------------------------------
% start of input file: macptopbm.1
%
\phead{macptopbm}{1}{29 March 1989}{}{}

%.IX macptopbm
\shead{NAME}
macptopbm - convert a MacPaint file into a portable bitmap
\shead{SYNOPSIS}
{\bf macptopbm}
{\rm [}{\bf -extraskip}
{\it N}{\rm ]}
{\rm [}{\it macpfile}{\rm ]}
\shead{DESCRIPTION}
Reads a MacPaint file as input.
%.IX MacPaint
%.IX Macintosh
Produces a portable bitmap as output.
\shead{OPTIONS}
\begin{TPlist}{{\bf -extraskip}}
\item[{{\bf -extraskip}}]
This flag is to get around a problem with some methods
of transferring files from the Mac world to the Unix world.
Most of these methods leave the Mac files alone, but a few of
them add the ``finderinfo'' data onto the front of the Unix file.
This means an extra 128 bytes to skip over when reading the file.
The symptom to watch for is that the resulting PBM file looks shifted
to one side.
If you get this, try
{\bf -extraskip}
128, and if that still doesn't look right try another value.
\end{TPlist}

\par
All flags can be abbreviated to their shortest unique prefix.
\shead{SEE ALSO}
picttoppm(1), pbmtomacp(1), pbm(5)
\shead{AUTHOR}
\copyright 1988 by Jef Poskanzer.
% Permission to use, copy, modify, and distribute this software and its
% documentation for any purpose and without fee is hereby granted, provided
% that the above copyright notice appear in all copies and that both that
% copyright notice and this permission notice appear in supporting
% documentation.  This software is provided "as is" without express or
% implied warranty.
The MacPaint-reading code is \copyright 1987 by Patrick J. Naughton
(naughton@wind.sun.com).
% Permission to use, copy, modify, and distribute this software and its
% documentation for any purpose and without fee is hereby granted,
% provided that the above copyright notice appear in all copies and that
% both that copyright notice and this permission notice appear in
% supporting documentation. 
%
% end of input file: macptopbm.1
%--------------------------------------------------
 
% -*-LaTeX-*-
% Converted automatically from troff to LaTeX
% by tr2latex ($Revision: 2.2 $$Date: 1992/04/27 15:13:26 $ by C. Engel)
% on Fri Oct  8 16:07:58 1993
% tr2latex was written by Kamal Al-Yahya at Stanford University
% (Kamal%Hanauma@SU-SCORE.ARPA)
% and substantially enhanced by Christian Engel at RWTH Aachen
% (krischan@informatik.rwth-aachen.de).
%
% troff input file: mgrtopbm.1

\newpage
%--------------------------------------------------
% start of input file: mgrtopbm.1
%
\phead{mgrtopbm}{1}{24 January 1989}{}{}

%.IX mgrtopbm
\shead{NAME}
mgrtopbm - convert a MGR bitmap into a portable bitmap
\shead{SYNOPSIS}
{\bf mgrtopbm}
{\rm [}{\it mgrfile}{\rm ]}
\shead{DESCRIPTION}
Reads a MGR bitmap as input.
%.IX MGR
Produces a portable bitmap as output.
\shead{SEE ALSO}
pbmtomgr(1), pbm(5)
\shead{AUTHOR}
\copyright 1989 by Jef Poskanzer.
% Permission to use, copy, modify, and distribute this software and its
% documentation for any purpose and without fee is hereby granted, provided
% that the above copyright notice appear in all copies and that both that
% copyright notice and this permission notice appear in supporting
% documentation.  This software is provided "as is" without express or
% implied warranty.
%
% end of input file: mgrtopbm.1
%--------------------------------------------------
 
% -*-LaTeX-*-
% Converted automatically from troff to LaTeX
% by tr2latex ($Revision: 2.2 $$Date: 1992/04/27 15:13:26 $ by C. Engel)
% on Fri Oct  8 16:08:42 1993
% tr2latex was written by Kamal Al-Yahya at Stanford University
% (Kamal%Hanauma@SU-SCORE.ARPA)
% and substantially enhanced by Christian Engel at RWTH Aachen
% (krischan@informatik.rwth-aachen.de).
%
% troff input file: mtvtoppm.1

\newpage
%--------------------------------------------------
% start of input file: mtvtoppm.1
%
\phead{mtvtoppm}{1}{02 February 1989}{}{}

%.IX mtvtoppm
\shead{NAME}
mtvtoppm - convert output from the MTV or PRT ray tracers into a portable pixmap
\shead{SYNOPSIS}
{\bf mtvtoppm}
{\rm [}{\it mtvfile}{\rm ]}
\shead{DESCRIPTION}
Reads an input file from Mark VanDeWettering's MTV ray tracer.
%.IX "MTV raytracer"
Produces a portable pixmap as output.
\par
The PRT raytracer also produces this format.
%.IX "PRT raytracer"
\shead{SEE ALSO}
ppm(5)
\shead{AUTHOR}
\copyright 1989 by Jef Poskanzer.
% Permission to use, copy, modify, and distribute this software and its
% documentation for any purpose and without fee is hereby granted, provided
% that the above copyright notice appear in all copies and that both that
% copyright notice and this permission notice appear in supporting
% documentation.  This software is provided "as is" without express or
% implied warranty.
%
% end of input file: mtvtoppm.1
%--------------------------------------------------
 
% -*-LaTeX-*-
% Converted automatically from troff to LaTeX
% by tr2latex ($Revision: 2.2 $$Date: 1992/04/27 15:13:26 $ by C. Engel)
% on Fri Oct  8 16:08:20 1993
% tr2latex was written by Kamal Al-Yahya at Stanford University
% (Kamal%Hanauma@SU-SCORE.ARPA)
% and substantially enhanced by Christian Engel at RWTH Aachen
% (krischan@informatik.rwth-aachen.de).
%
% troff input file: pbm.5

\newpage
%--------------------------------------------------
% start of input file: pbm.5
%
\phead{pbm}{5}{27 September 1991}{}{}

\shead{NAME}
pbm - portable bitmap file format
\shead{DESCRIPTION}
The portable bitmap format is a lowest common denominator monochrome
file format.
%.IX "PBM file format"
It was originally designed to make it reasonable to mail bitmaps
between different types of machines using the typical stupid network
mailers we have today.
Now it serves as the common language of a large family of bitmap
conversion filters.
The definition is as follows:
\begin{IPlist}
\IPitem{{-}}
A ``magic number'' for identifying the file type.
A pbm file's magic number is the two characters ``P1''.
%.IX "magic numbers"
\IPitem{{-}}
Whitespace (blanks, TABs, CRs, LFs).
\IPitem{{-}}
A width, formatted as ASCII characters in decimal.
\IPitem{{-}}
Whitespace.
\IPitem{{-}}
A height, again in ASCII decimal.
\IPitem{{-}}
Whitespace.
\IPitem{{-}}
Width * height bits, each either '1' or '0', starting at the top-left
corner of the bitmap, proceeding in normal English reading order.
\IPitem{{-}}
The character '1' means black, '0' means white.
\IPitem{{-}}
Whitespace in the bits section is ignored.
\IPitem{{-}}
Characters from a ``\#'' to the next end-of-line are ignored (comments).
\IPitem{{-}}
No line should be longer than 70 characters.
\end{IPlist}

\par
Here is an example of a small bitmap in this format:
\nofill
P1
\# feep.pbm
24 7
0 0 0 0 0 0 0 0 0 0 0 0 0 0 0 0 0 0 0 0 0 0 0 0
0 1 1 1 1 0 0 1 1 1 1 0 0 1 1 1 1 0 0 1 1 1 1 0
0 1 0 0 0 0 0 1 0 0 0 0 0 1 0 0 0 0 0 1 0 0 1 0
0 1 1 1 0 0 0 1 1 1 0 0 0 1 1 1 0 0 0 1 1 1 1 0
0 1 0 0 0 0 0 1 0 0 0 0 0 1 0 0 0 0 0 1 0 0 0 0
0 1 0 0 0 0 0 1 1 1 1 0 0 1 1 1 1 0 0 1 0 0 0 0
0 0 0 0 0 0 0 0 0 0 0 0 0 0 0 0 0 0 0 0 0 0 0 0
\fill
\par
Programs that read this format should be as lenient as possible,
accepting anything that looks remotely like a bitmap.
\par
There is also a variant on the format, available
by setting the RAWBITS option at compile time.  This variant is
%.IX RAWBITS
different in the following ways:
\begin{IPlist}
\IPitem{{-}}
The ``magic number'' is ``P4'' instead of ``P1''.
\IPitem{{-}}
The bits are stored eight per byte, high bit first low bit last.
\IPitem{{-}}
No whitespace is allowed in the bits section, and only a single character
of whitespace (typically a newline) is allowed after the height.
\IPitem{{-}}
The files are eight times smaller and many times faster to read and write.
\end{IPlist}

\shead{SEE ALSO}
atktopbm(1), brushtopbm(1), cmuwmtopbm(1), g3topbm(1),
gemtopbm(1), icontopbm(1),
macptopbm(1), mgrtopbm(1), pi3topbm(1), xbmtopbm(1),
ybmtopbm(1),
pbmto10x(1), pnmtoascii(1), pbmtoatk(1), pbmtobbnbg(1),
pbmtocmuwm(1), pbmtoepson(1),
pbmtog3(1), pbmtogem(1), pbmtogo(1), pbmtoicon(1), pbmtolj(1),
pbmtomacp(1), pbmtomgr(1), pbmtopi3(1), pbmtoplot(1), pbmtoptx(1),
pbmtox10bm(1), pbmtoxbm(1), pbmtoybm(1),
pbmtozinc(1),
pbmlife(1), pbmmake(1), pbmmask(1), pbmreduce(1),
pbmtext(1), pbmupc(1),
pnm(5), pgm(5), ppm(5)
\shead{AUTHOR}
\copyright 1989, 1991 by Jef Poskanzer.
% Permission to use, copy, modify, and distribute this software and its
% documentation for any purpose and without fee is hereby granted, provided
% that the above copyright notice appear in all copies and that both that
% copyright notice and this permission notice appear in supporting
% documentation.  This software is provided "as is" without express or
% implied warranty.
%
% end of input file: pbm.5
%--------------------------------------------------
 
% -*-LaTeX-*-
% Converted automatically from troff to LaTeX
% by tr2latex ($Revision: 2.2 $$Date: 1992/04/27 15:13:26 $ by C. Engel)
% on Fri Oct  8 16:08:16 1993
% tr2latex was written by Kamal Al-Yahya at Stanford University
% (Kamal%Hanauma@SU-SCORE.ARPA)
% and substantially enhanced by Christian Engel at RWTH Aachen
% (krischan@informatik.rwth-aachen.de).
%
% troff input file: pbmclean.1

\newpage
%--------------------------------------------------
% start of input file: pbmclean.1
%
\phead{pbmclean}{12 Dec 1990}{}{}{}

\shead{NAME}
pbmclean - flip isolated pixels in portable bitmap
\shead{SYNOPSIS}
pbmclean [-connect] [pbmfile]
\shead{DESCRIPTION}
Reads a portable bitmap as input. Outputs a portable bitmap with every 
pixel which has less than %
\it connect %
\rm identical neighbours inverted.
Pbmclean can be used to clean up ``snow'' on bitmap images.
\shead{SEE ALSO}
pbm(5)
\shead{AUTHOR}
\copyright 1990 by Angus Duggan.
\copyright 1989 by Jef Poskanzer.

Permission to use, copy, modify, and distribute this software and its
documentation for any purpose and without fee is hereby granted, provided
that the above copyright notice appear in all copies and that both that
copyright notice and this permission notice appear in supporting
documentation.  This software is provided "as is" without express or
implied warranty.
%
% end of input file: pbmclean.1
%--------------------------------------------------
 
% -*-LaTeX-*-
% Converted automatically from troff to LaTeX
% by tr2latex ($Revision: 2.2 $$Date: 1992/04/27 15:13:26 $ by C. Engel)
% on Fri Oct  8 16:07:58 1993
% tr2latex was written by Kamal Al-Yahya at Stanford University
% (Kamal%Hanauma@SU-SCORE.ARPA)
% and substantially enhanced by Christian Engel at RWTH Aachen
% (krischan@informatik.rwth-aachen.de).
%
% troff input file: pbmlife.1

\newpage
%--------------------------------------------------
% start of input file: pbmlife.1
%
\phead{pbmlife}{1}{21 February 1991}{}{}

%.IX pbmlife
\shead{NAME}
pbmlife - apply Conway's rules of Life to a portable bitmap
\shead{SYNOPSIS}
{\bf pbmlife}
{\rm [}{\it pbmfile}{\rm ]}
\shead{DESCRIPTION}
Reads a portable bitmap as input.
Applies the rules of Life to it for one generation,
%.IX Life
and produces a portable bitmap as output.
\par
A white pixel in the image is interpreted as a live beastie, and a
black pixel as an empty space.
\shead{SEE ALSO}
pbm(5)
\shead{AUTHOR}
\copyright 1988, 1991 by Jef Poskanzer.
% Permission to use, copy, modify, and distribute this software and its
% documentation for any purpose and without fee is hereby granted, provided
% that the above copyright notice appear in all copies and that both that
% copyright notice and this permission notice appear in supporting
% documentation.  This software is provided "as is" without express or
% implied warranty.
%
% end of input file: pbmlife.1
%--------------------------------------------------
 
% -*-LaTeX-*-
% Converted automatically from troff to LaTeX
% by tr2latex ($Revision: 2.2 $$Date: 1992/04/27 15:13:26 $ by C. Engel)
% on Fri Oct  8 16:07:59 1993
% tr2latex was written by Kamal Al-Yahya at Stanford University
% (Kamal%Hanauma@SU-SCORE.ARPA)
% and substantially enhanced by Christian Engel at RWTH Aachen
% (krischan@informatik.rwth-aachen.de).
%
% troff input file: pbmmake.1

\newpage
%--------------------------------------------------
% start of input file: pbmmake.1
%
\phead{pbmmake}{1}{22 February 1989}{}{}

%.IX pbmmake
\shead{NAME}
pbmmake - create a blank bitmap of a specified size
\shead{SYNOPSIS}
{\bf pbmmake}
{\rm [}{\bf -white}{\rm $|$}{\bf -black}{\rm $|$}{\bf -gray}
{\rm ]}
{\it width height}
\shead{DESCRIPTION}
Produces a portable bitmap of the specified width and height.
%.IX "generating bitmaps"
The color defaults to white.
\shead{OPTIONS}
\par
In addition to the usual
{\bf -white}
and
{\bf -black}{\rm ,}
this program implements
{\bf -gray}{\rm .}
This gives a simple 50\% gray pattern with 1's and 0's alternating.
\par
All flags can be abbreviated to their shortest unique prefix.
\shead{SEE ALSO}
pbm(5), ppmmake(1)
\shead{AUTHOR}
\copyright 1989 by Jef Poskanzer.
% Permission to use, copy, modify, and distribute this software and its
% documentation for any purpose and without fee is hereby granted, provided
% that the above copyright notice appear in all copies and that both that
% copyright notice and this permission notice appear in supporting
% documentation.  This software is provided "as is" without express or
% implied warranty.
%
% end of input file: pbmmake.1
%--------------------------------------------------
 
% -*-LaTeX-*-
% Converted automatically from troff to LaTeX
% by tr2latex ($Revision: 2.2 $$Date: 1992/04/27 15:13:26 $ by C. Engel)
% on Fri Oct  8 16:08:00 1993
% tr2latex was written by Kamal Al-Yahya at Stanford University
% (Kamal%Hanauma@SU-SCORE.ARPA)
% and substantially enhanced by Christian Engel at RWTH Aachen
% (krischan@informatik.rwth-aachen.de).
%
% troff input file: pbmmask.1

\newpage
%--------------------------------------------------
% start of input file: pbmmask.1
%
\phead{pbmmask}{1}{08 August 1989}{}{}

%.IX pbmmask
\shead{NAME}
pbmmask - create a mask bitmap from a regular bitmap
\shead{SYNOPSIS}
{\bf pbmmask}
{\rm [}{\bf -expand}{\rm ]}
{\rm [}{\it pbmfile}{\rm ]}
\shead{DESCRIPTION}
Reads a portable bitmap as input.
Creates a corresponding mask bitmap and writes it out.
\par
The color to be interpreted as ``background'' is determined automatically.
Regardless of which color is background, the mask will be white where
the background is and black where the figure is.
\par
This lets you do a masked paste like this, for objects with a black background:
\nofill
    pbmmask obj $>$ objmask
    pnmpaste $<$ dest -and objmask $<$x$>$ $<$y$>$ $|$ pnmpaste -or obj $<$x$>$ $<$y$>$
\fill
%.IX pnmpaste
For objects with a white background, you can either invert them or
add a step:
\nofill
    pbmmask obj $>$ objmask
    pnminvert objmask $|$ pnmpaste -and obj 0 0 $>$ blackback
    pnmpaste $<$ dest -and objmask $<$x$>$ $<$y$>$ $|$ pnmpaste -or blackback $<$x$>$ $<$y$>$
\fill
%.IX pnminvert
Note that this three-step version works for objects with black backgrounds
too, if you don't care about the wasted time.
\par
You can also use masks with graymaps and pixmaps, using the
{\it pnmarith}
tool.  For instance:
\nofill
    ppmtopgm obj.ppm $|$ pgmtopbm -threshold $|$ pbmmask $>$ objmask.pbm
    pnmarith -multiply dest.ppm objmask.pbm $>$ t1.ppm
    pnminvert objmask.pbm $|$ pnmarith -multiply obj.ppm - $>$ t2.ppm
    pnmarith -add t1.ppm t2.ppm
\fill
%.IX pnmarith
An interesting variation on this is to pipe the mask through the
{\it pnmsmooth}
%.IX pnmsmooth
script before using it.  This makes the boundary between the two images less
sharp.
%.OPTIONS
\begin{TPlist}{{\bf -expand}}
\item[{{\bf -expand}}]
Expands the mask by one pixel out from the image.
This is useful if you want a little white border around your image.
(A better solution might be to turn the
{\it pbmlife}
tool into a general cellular automaton tool...)
\end{TPlist}

\shead{SEE ALSO}
pnmpaste(1), pnminvert(1), pbm(5), pnmarith(1), pnmsmooth(1)
\shead{AUTHOR}
\copyright 1988 by Jef Poskanzer.
% Permission to use, copy, modify, and distribute this software and its
% documentation for any purpose and without fee is hereby granted, provided
% that the above copyright notice appear in all copies and that both that
% copyright notice and this permission notice appear in supporting
% documentation.  This software is provided "as is" without express or
% implied warranty.
%
% end of input file: pbmmask.1
%--------------------------------------------------
 
% -*-LaTeX-*-
% Converted automatically from troff to LaTeX
% by tr2latex ($Revision: 2.2 $$Date: 1992/04/27 15:13:26 $ by C. Engel)
% on Fri Oct  8 16:08:16 1993
% tr2latex was written by Kamal Al-Yahya at Stanford University
% (Kamal%Hanauma@SU-SCORE.ARPA)
% and substantially enhanced by Christian Engel at RWTH Aachen
% (krischan@informatik.rwth-aachen.de).
%
% troff input file: pbmpscale.1

\newpage
%--------------------------------------------------
% start of input file: pbmpscale.1
%
\phead{pbmpscale}{12 Dec 1990}{}{}{}

\shead{NAME}
pbmpscale - enlarge a portable bitmap with edge smoothing
\shead{SYNOPSIS}
pbmpscale N [ pbmfile ]
\shead{DESCRIPTION}
Reads a portable bitmap as input, and outputs a portable bitmap
enlarged N times. Enlargement is done by pixel replication,
with some additional smoothing of corners and edges.
\shead{SEE ALSO}
pnmenlarge(1), ppmscale(1), pbm(5)
\shead{AUTHOR}
\copyright 1990 by Angus Duggan.
\copyright 1989 by Jef Poskanzer.

Permission to use, copy, modify, and distribute this software and its
documentation for any purpose and without fee is hereby granted, provided
that the above copyright notice appear in all copies and that both that
copyright notice and this permission notice appear in supporting
documentation.  This software is provided "as is" without express or
implied warranty.
\shead{NOTES}
pbmpscale works best for enlargements of 2. Enlargements greater than 2
should be done by as many enlargements of 2 as possible, followed by an
enlargement by the remaining factor.
%
% end of input file: pbmpscale.1
%--------------------------------------------------
 
% -*-LaTeX-*-
% Converted automatically from troff to LaTeX
% by tr2latex ($Revision: 2.2 $$Date: 1992/04/27 15:13:26 $ by C. Engel)
% on Fri Oct  8 16:08:00 1993
% tr2latex was written by Kamal Al-Yahya at Stanford University
% (Kamal%Hanauma@SU-SCORE.ARPA)
% and substantially enhanced by Christian Engel at RWTH Aachen
% (krischan@informatik.rwth-aachen.de).
%
% troff input file: pbmreduce.1

\newpage
%--------------------------------------------------
% start of input file: pbmreduce.1
%
\phead{pbmreduce}{1}{02 August 1989}{}{}

%.IX pbmreduce
\shead{NAME}
pbmreduce - read a portable bitmap and reduce it N times
\shead{SYNOPSIS}
{\bf pbmreduce}
{\rm [}{\bf -floyd}{\rm $|$}{\bf -fs}{\rm $|$}{\bf -threshold}
{\rm ]}
{\rm [}{\bf -value}
{\it val}{\rm ]}
{\it N}
{\rm [}{\it pbmfile}{\rm ]}
\shead{DESCRIPTION}
Reads a portable bitmap as input.
Reduces it by a factor of
{\it N}{\rm ,}
and produces a portable bitmap as output.
%.IX shrinking
\par
{\it pbmreduce}
duplicates a lot of the functionality of
{\it pgmtopbm;}
%.IX pgmtopbm
you could do something like
{\bf pnmscale $|$ pgmtopbm,}
%.IX pnmscale
but
{\it pbmreduce}
is a lot faster.
\par
{\it pbmreduce}
can be used to ``re-halftone'' an image.
%.IX halftoning
Let's say you have a scanner that only produces black\&white, not
grayscale, and it does a terrible job of halftoning (most b\&w scanners
fit this description).
One way to fix the halftoning is to scan at the highest possible
resolution, say 300 dpi, and then reduce by a factor of three or
so using
{\it pbmreduce}{\rm .}
You can even correct the brightness of an image, by using the
{\bf -value}
flag.
\shead{OPTIONS}
\par
By default, the halftoning after the reduction is done via
boustrophedonic Floyd-Steinberg error diffusion; however, the
%.IX Floyd-Steinberg
%.IX "error diffusion"
{\bf -threshold}
flag can be used to specify simple thresholding.  This gives better
%.IX thresholding
results when reducing line drawings.
\par
The
{\bf -value}
flag alters the thresholding value for all quantizations.
It should be a real number between 0 and 1.
Above 0.5 means darker images; below 0.5 means lighter.
\par
All flags can be abbreviated to their shortest unique prefix.
\shead{SEE ALSO}
pnmenlarge(1), pnmscale(1), pgmtopbm(1), pbm(5)
\shead{AUTHOR}
\copyright 1988 by Jef Poskanzer.
% Permission to use, copy, modify, and distribute this software and its
% documentation for any purpose and without fee is hereby granted, provided
% that the above copyright notice appear in all copies and that both that
% copyright notice and this permission notice appear in supporting
% documentation.  This software is provided "as is" without express or
% implied warranty.
%
% end of input file: pbmreduce.1
%--------------------------------------------------
 
% -*-LaTeX-*-
% Converted automatically from troff to LaTeX
% by tr2latex ($Revision: 2.2 $$Date: 1992/04/27 15:13:26 $ by C. Engel)
% on Fri Oct  8 16:08:01 1993
% tr2latex was written by Kamal Al-Yahya at Stanford University
% (Kamal%Hanauma@SU-SCORE.ARPA)
% and substantially enhanced by Christian Engel at RWTH Aachen
% (krischan@informatik.rwth-aachen.de).
%
% troff input file: pbmtext.1

\newpage
%--------------------------------------------------
% start of input file: pbmtext.1
%
\phead{pbmtext}{1}{5 January 1991}{}{}

%.IX pbmtext
\shead{NAME}
pbmtext - render text into a bitmap
\shead{SYNOPSIS}
{\bf pbmtext}
{\rm [}{\bf -font}
{\it fontfile}{\rm ]}
{\rm [}{\it text}{\rm ]}
\shead{DESCRIPTION}
\par
Takes the specified text, either a single line from the command
line or multiple lines from standard input, and renders it
into a bitmap.
%.IX text
\shead{OPTIONS}
\par
By default, pbmtext uses a built-in font.
You can also specify your own font with the
{\bf -font}
flag.
The
{\it fontfile}
is a pbm file, created in a very specific way.
In your window system of choice, display the following text
in the desired (fixed-width) font:
\nofill

    M ",/\^{}\_[`jpqy$|$ M

    /  !"\#\$\%\&'()*+ /
    $<$ ,-./01234567 $<$
    $>$ 89:;$<$=$>$?@ABC $>$
    @ DEFGHIJKLMNO @
    \_ PQRSTUVWXYZ[ \_
    \{ \bs ]\^{}\_`abcdefg \{
    \} hijklmnopqrs \}
    \~{} tuvwxyz\{$|$\}\~{}  \~{}

    M ",/\^{}\_[`jpqy$|$ M

\fill
Do a screen grab or window dump of that text, using for instance
{\it xwd}{\rm ,}
{\it xgrabsc}{\rm ,}
or
{\it screendump}{\rm .}
Convert the result into a pbm file.
If necessary, use
{\it pnmcut}
to remove everything except the text.
Finally, run it through
{\it pnmcrop}
%.IX pnmcrop
to make sure the edges are right up against the text.
{\it pbmtext}
can figure out the sizes and spacings from that.
\shead{SEE ALSO}
pbm(5), pnmcut(1), pnmcrop(1)
\shead{AUTHOR}
\copyright 1991 by Jef Poskanzer.
% Permission to use, copy, modify, and distribute this software and its
% documentation for any purpose and without fee is hereby granted, provided
% that the above copyright notice appear in all copies and that both that
% copyright notice and this permission notice appear in supporting
% documentation.  This software is provided "as is" without express or
% implied warranty.
%
% end of input file: pbmtext.1
%--------------------------------------------------
 
% -*-LaTeX-*-
% Converted automatically from troff to LaTeX
% by tr2latex ($Revision: 2.2 $$Date: 1992/04/27 15:13:26 $ by C. Engel)
% on Fri Oct  8 16:08:02 1993
% tr2latex was written by Kamal Al-Yahya at Stanford University
% (Kamal%Hanauma@SU-SCORE.ARPA)
% and substantially enhanced by Christian Engel at RWTH Aachen
% (krischan@informatik.rwth-aachen.de).
%
% troff input file: pbmto10x.1

\newpage
%--------------------------------------------------
% start of input file: pbmto10x.1
%
\phead{pbmto10x}{1}{1 January 1990}{}{}

%.IX pbmto10x
\shead{NAME}
pbmto10x - convert a portable bitmap into Gemini 10X printer graphics
\shead{SYNOPSIS}
{\bf pbmto10x}
{\rm [}{\bf -h}{\rm ]}
{\rm [}{\it pbmfile}{\rm ]}
\shead{DESCRIPTION}
Reads a portable bitmap as input.
Produces a file of Gemini 10X printer graphics as output.
%.IX "Gemini 10X printer graphics"
The 10x's printer codes are alleged to be similar to the Epson codes.
%.IX Epson
\par
Note that there is no 10xtopbm tool - this transformation is one way.
\shead{OPTIONS}
\par
The resolution is normally 60H by 72V.
If the
{\bf -h}
flag is specified, resolution is 120H by 144V.
You may find it useful to rotate landscape images before printing.
\shead{SEE ALSO}
pbm(5)
\shead{AUTHOR}
\copyright 1990 by Ken Yap.
% Permission to use, copy, modify, and distribute this software and its
% documentation for any purpose and without fee is hereby granted, provided
% that the above copyright notice appear in all copies and that both that
% copyright notice and this permission notice appear in supporting
% documentation.  This software is provided "as is" without express or
% implied warranty.
%
% end of input file: pbmto10x.1
%--------------------------------------------------
 
% -*-LaTeX-*-
% Converted automatically from troff to LaTeX
% by tr2latex ($Revision: 2.2 $$Date: 1992/04/27 15:13:26 $ by C. Engel)
% on Fri Oct  8 16:08:02 1993
% tr2latex was written by Kamal Al-Yahya at Stanford University
% (Kamal%Hanauma@SU-SCORE.ARPA)
% and substantially enhanced by Christian Engel at RWTH Aachen
% (krischan@informatik.rwth-aachen.de).
%
% troff input file: pbmtoascii.1

\newpage
%--------------------------------------------------
% start of input file: pbmtoascii.1
%
\phead{pbmtoascii}{1}{20 March 1992}{}{}

\shead{NAME}
pbmtoascii - convert a portable bitmap into ASCII graphics
\shead{SYNOPSIS}
{\bf pbmtoascii}
{\rm [}{\bf -1x2}{\rm $|$}{\bf -2x4}{\rm ]}
{\rm [}{\it pbmfile}{\rm ]}
\shead{DESCRIPTION}
Reads a portable bitmap as input.
Produces a somewhat crude ASCII graphic as output.
\par
Note that there is no asciitopbm tool - this transformation is one-way.
\shead{OPTIONS}
The
{\bf -1x2}
and
{\bf -2x4}
flags give you two alternate ways for the bits to get mapped to characters.
With
{\bf 1x2}{\rm ,}
the default, each character represents a group of 1 bit across by 2 bits down.
With
{\bf -2x4}{\rm ,}
each character represents 2 bits across by 4 bits down.
With the 1x2 mode you can see the individual bits, so it's useful for
previewing small bitmaps on a non-graphics terminal.
The 2x4 mode lets you display larger bitmaps on a standard 80-column display,
but it obscures bit-level details.
2x4 mode is also good for displaying
graymaps - ``pnmscale -width 158 $|$ pgmnorm $|$ pgmtopbm -thresh''
should give good results.
\shead{SEE ALSO}
pbm(5)
\shead{AUTHOR}
\copyright 1988, 1992 by Jef Poskanzer.
% Permission to use, copy, modify, and distribute this software and its
% documentation for any purpose and without fee is hereby granted, provided
% that the above copyright notice appear in all copies and that both that
% copyright notice and this permission notice appear in supporting
% documentation.  This software is provided "as is" without express or
% implied warranty.
%
% end of input file: pbmtoascii.1
%--------------------------------------------------
 
% -*-LaTeX-*-
% Converted automatically from troff to LaTeX
% by tr2latex ($Revision: 2.2 $$Date: 1992/04/27 15:13:26 $ by C. Engel)
% on Fri Oct  8 16:08:03 1993
% tr2latex was written by Kamal Al-Yahya at Stanford University
% (Kamal%Hanauma@SU-SCORE.ARPA)
% and substantially enhanced by Christian Engel at RWTH Aachen
% (krischan@informatik.rwth-aachen.de).
%
% troff input file: pbmtoatk.1

\newpage
%--------------------------------------------------
% start of input file: pbmtoatk.1
%
\phead{pbmtoatk}{1}{26 September 1991}{}{}

%.IX pbmtoatk
\shead{NAME}
pbmtoatk - convert portable bitmap to Andrew Toolkit raster object
\shead{SYNOPSIS}
{\bf pbmtoatk}
{\rm [}{\it pbmfile}{\rm ]}
\shead{DESCRIPTION}
Reads a portable bitmap as input.
Produces a Andrew Toolkit raster object as output.
%.IX "Andrew Toolkit raster object"
\shead{SEE ALSO}
atktopbm(1), pbm(5)
\shead{AUTHOR}
\copyright 1991 by Bill Janssen.
% Permission to use, copy, modify, and distribute this software and its
% documentation for any purpose and without fee is hereby granted, provided
% that the above copyright notice appear in all copies and that both that
% copyright notice and this permission notice appear in supporting
% documentation.  This software is provided "as is" without express or
% implied warranty.
%
% end of input file: pbmtoatk.1
%--------------------------------------------------
 
% -*-LaTeX-*-
% Converted automatically from troff to LaTeX
% by tr2latex ($Revision: 2.2 $$Date: 1992/04/27 15:13:26 $ by C. Engel)
% on Fri Oct  8 16:08:03 1993
% tr2latex was written by Kamal Al-Yahya at Stanford University
% (Kamal%Hanauma@SU-SCORE.ARPA)
% and substantially enhanced by Christian Engel at RWTH Aachen
% (krischan@informatik.rwth-aachen.de).
%
% troff input file: pbmtobbnbg.1

\newpage
%--------------------------------------------------
% start of input file: pbmtobbnbg.1
%
\phead{pbmtobg}{1}{16 May 1989}{}{}

%.IX pbmtobbnbg
\shead{NAME}
pbmtobg - convert a portable bitmap into BitGraph graphics
\shead{SYNOPSIS}
{\bf pbmtobg}
{\rm [}{\it rasterop}{\rm ]}
{\rm [}{\it x}
{\it y}{\rm ]}
$<$
{\it pbmfile}
\shead{DESCRIPTION}
Reads a portable bitmap as input.
Produces BBN BitGraph terminal Display Pixel Data (DPD) sequence as output.
%.IX "BBN BitGraph"
\par
The rasterop can be specified on the command line.  If this is omitted, 3
(replace) will be used.  A position in (x,y) coordinates can also be
specified.  If both are given, the rasterop comes first.  The portable bitmap
is always taken from the standard input.
\par
Note that there is no bgtopbm tool.
\shead{SEE ALSO}
pbm(5)
\shead{AUTHOR}
Copyright 1989 by Mike Parker.
% Permission to use, copy, modify, and distribute this software and its
% documentation for any purpose and without fee is hereby granted, provided
% that the above copyright notice appear in all copies and that both that
% copyright notice and this permission notice appear in supporting
% documentation.  This software is provided "as is" without express or
% implied warranty.
%
% end of input file: pbmtobbnbg.1
%--------------------------------------------------
 
% -*-LaTeX-*-
% Converted automatically from troff to LaTeX
% by tr2latex ($Revision: 2.2 $$Date: 1992/04/27 15:13:26 $ by C. Engel)
% on Fri Oct  8 16:08:04 1993
% tr2latex was written by Kamal Al-Yahya at Stanford University
% (Kamal%Hanauma@SU-SCORE.ARPA)
% and substantially enhanced by Christian Engel at RWTH Aachen
% (krischan@informatik.rwth-aachen.de).
%
% troff input file: pbmtocmuwm.1

\newpage
%--------------------------------------------------
% start of input file: pbmtocmuwm.1
%
\phead{pbmtocmuwm}{1}{15 April 1989}{}{}

%.IX pbmtocmuwm
\shead{NAME}
pbmtocmuwm - convert a portable bitmap into a CMU window manager bitmap
\shead{SYNOPSIS}
{\bf pbmtocmuwm}
{\rm [}{\it pbmfile}{\rm ]}
\shead{DESCRIPTION}
Reads a portable bitmap as input.
Produces a CMU window manager bitmap as output.
%.IX "CMU window manager bitmap"
\shead{SEE ALSO}
cmuwmtopbm(1), pbm(5)
\shead{AUTHOR}
\copyright 1989 by Jef Poskanzer.
% Permission to use, copy, modify, and distribute this software and its
% documentation for any purpose and without fee is hereby granted, provided
% that the above copyright notice appear in all copies and that both that
% copyright notice and this permission notice appear in supporting
% documentation.  This software is provided "as is" without express or
% implied warranty.
%
% end of input file: pbmtocmuwm.1
%--------------------------------------------------
 
% -*-LaTeX-*-
% Converted automatically from troff to LaTeX
% by tr2latex ($Revision: 2.2 $$Date: 1992/04/27 15:13:26 $ by C. Engel)
% on Fri Oct  8 16:08:17 1993
% tr2latex was written by Kamal Al-Yahya at Stanford University
% (Kamal%Hanauma@SU-SCORE.ARPA)
% and substantially enhanced by Christian Engel at RWTH Aachen
% (krischan@informatik.rwth-aachen.de).
%
% troff input file: pbmtoepsi.1

\newpage
%--------------------------------------------------
% start of input file: pbmtoepsi.1
%
\phead{pbmtoepsi}{1}{1992}{}{}

%.IX pbmtoepsi
\shead{NAME}
pbmtoepsi - convert a portable bitmap into an encapsulated PostScript
style preview bitmap
\shead{SYNOPSIS}
{\bf pbmtoepsi}
{\rm [}{\bf -bbonly}{\rm ]}
{\rm [}{\it pbmfile}{\rm ]}
\shead{DESCRIPTION}
Reads a portable bitmap as input.
Produce an encapsulated Postscript style bitmap as output. The output
is not a stand alone postscript file, it is only a preview bitmap,
which can be included in an encapsulated PostScript file.
Note that there is no epsitopbm tool - this transformation is one way.

This utility is a part of the pstoepsi tool by Doug Crabill
(dgc@cs.purdue.edu).
\shead{OPTIONS}
\begin{TPlist}{{\bf -bbonly}}
\item[{{\bf -bbonly}}]
Only create a boundary box, don't fill it with the image.
\end{TPlist}

\shead{SEE ALSO}
pbm(5), pnmtops(1), psidtopgm(1)
\shead{AUTHOR}
\copyright 1988 Jef Poskanzer
\nwl
modified by Doug Crabill 1992.
% Permission to use, copy, modify, and distribute this software and its
% documentation for any purpose and without fee is hereby granted, provided
% that the above copyright notice appear in all copies and that both that
% copyright notice and this permission notice appear in supporting
% documentation.  This software is provided "as is" without express or
% implied warranty.
%
% end of input file: pbmtoepsi.1
%--------------------------------------------------
 
% -*-LaTeX-*-
% Converted automatically from troff to LaTeX
% by tr2latex ($Revision: 2.2 $$Date: 1992/04/27 15:13:26 $ by C. Engel)
% on Fri Oct  8 16:08:04 1993
% tr2latex was written by Kamal Al-Yahya at Stanford University
% (Kamal%Hanauma@SU-SCORE.ARPA)
% and substantially enhanced by Christian Engel at RWTH Aachen
% (krischan@informatik.rwth-aachen.de).
%
% troff input file: pbmtoepson.1

\newpage
%--------------------------------------------------
% start of input file: pbmtoepson.1
%
\phead{pbmtoepson}{1}{4 January 1991}{}{}

%.IX pbmtoepson
\shead{NAME}
pbmtoepson - convert a portable bitmap into Epson printer graphics
\shead{SYNOPSIS}
{\bf pbmtoepson}
{\rm [}{\it pbmfile}{\rm ]}
\shead{DESCRIPTION}
Reads a portable bitmap as input.
Produces a file of Epson printer graphics as output.
%.IX Epson
\par
Note that there is no epsontopbm tool - this transformation is one way.
\shead{SEE ALSO}
pbm(5)
\shead{AUTHOR}
\copyright 1991 by John Tiller (tiller@galois.msfc.nasa.gov) and Jef Poskanzer.
% Permission to use, copy, modify, and distribute this software and its
% documentation for any purpose and without fee is hereby granted, provided
% that the above copyright notice appear in all copies and that both that
% copyright notice and this permission notice appear in supporting
% documentation.  This software is provided "as is" without express or
% implied warranty.
%
% end of input file: pbmtoepson.1
%--------------------------------------------------
 
% -*-LaTeX-*-
% Converted automatically from troff to LaTeX
% by tr2latex ($Revision: 2.2 $$Date: 1992/04/27 15:13:26 $ by C. Engel)
% on Fri Oct  8 16:08:05 1993
% tr2latex was written by Kamal Al-Yahya at Stanford University
% (Kamal%Hanauma@SU-SCORE.ARPA)
% and substantially enhanced by Christian Engel at RWTH Aachen
% (krischan@informatik.rwth-aachen.de).
%
% troff input file: pbmtog3.1

\newpage
%--------------------------------------------------
% start of input file: pbmtog3.1
%
\phead{pbmtog3}{1}{02 October 1989}{}{}

%.IX pbmtog3
\shead{NAME}
pbmtog3 - convert a portable bitmap into a Group 3 fax file
\shead{SYNOPSIS}
{\bf pbmtog3}
{\rm [}{\it pbmfile}{\rm ]}
\shead{DESCRIPTION}
Reads a portable bitmap as output.
Produces a Group 3 fax file as input.
%.IX "Group 3 fax"
%.IX fax
\shead{REFERENCES}
The standard for Group 3 fax is defined in CCITT Recommendation T.4.
\shead{BUGS}
Probably.
\shead{SEE ALSO}
g3topbm(1), pbm(5)
\shead{AUTHOR}
\copyright 1989 by Paul Haeberli (paul@manray.sgi.com).
% Permission to use, copy, modify, and distribute this software and its
% documentation for any purpose and without fee is hereby granted, provided
% that the above copyright notice appear in all copies and that both that
% copyright notice and this permission notice appear in supporting
% documentation.  This software is provided "as is" without express or
% implied warranty.
%
% end of input file: pbmtog3.1
%--------------------------------------------------
 
% -*-LaTeX-*-
% Converted automatically from troff to LaTeX
% by tr2latex ($Revision: 2.2 $$Date: 1992/04/27 15:13:26 $ by C. Engel)
% on Fri Oct  8 16:08:05 1993
% tr2latex was written by Kamal Al-Yahya at Stanford University
% (Kamal%Hanauma@SU-SCORE.ARPA)
% and substantially enhanced by Christian Engel at RWTH Aachen
% (krischan@informatik.rwth-aachen.de).
%
% troff input file: pbmtogem.1

\newpage
%--------------------------------------------------
% start of input file: pbmtogem.1
%
\phead{pbmtogem}{1}{11 March 1990}{}{}

%.IX pbmtogem
\shead{NAME}
pbmtogem - convert a portable bitmap into a GEM .img file
\shead{SYNOPSIS}
{\bf pbmtogem}
{\rm [}{\it pbmfile}{\rm ]}
\shead{DESCRIPTION}
Reads a portable bitmap as input.
Produces a GEM .img file as output.
%.IX GEM
\shead{BUGS}
It does not support compression of the data.
\shead{SEE ALSO}
gemtopbm(1), pbm(5)
\shead{AUTHOR}
\copyright 1988 by David Beckemeyer (bdt!david) and Jef Poskanzer.
% Permission to use, copy, modify, and distribute this software and its
% documentation for any purpose and without fee is hereby granted,
% provided that the above copyright notice appear in all copies and that
% both that copyright notice and this permission notice appear in
% supporting documentation.
%
% end of input file: pbmtogem.1
%--------------------------------------------------
 
% -*-LaTeX-*-
% Converted automatically from troff to LaTeX
% by tr2latex ($Revision: 2.2 $$Date: 1992/04/27 15:13:26 $ by C. Engel)
% on Fri Oct  8 16:08:06 1993
% tr2latex was written by Kamal Al-Yahya at Stanford University
% (Kamal%Hanauma@SU-SCORE.ARPA)
% and substantially enhanced by Christian Engel at RWTH Aachen
% (krischan@informatik.rwth-aachen.de).
%
% troff input file: pbmtogo.1

\newpage
%--------------------------------------------------
% start of input file: pbmtogo.1
%
\phead{pbmtogo}{1}{24 November 1989}{}{}

%.IX pbmtogo
\shead{NAME}
pbmtogo - convert a portable bitmap into compressed GraphOn graphics
\shead{SYNOPSIS}
{\bf pbmtogo}
{\rm [}{\it pbmfile}{\rm ]}
\shead{DESCRIPTION}
Reads a portable bitmap as input.
Produces 2D compressed GraphOn graphics as output.
%.IX GraphOn
Be sure to set up your GraphOn with the following modes: 8 bits / no parity;
obeys no XON/XOFF; NULs are accepted.  These are all on the Comm menu.
Also, remember to turn off tty post processing.
Note that there is no gotopbm tool.
\shead{SEE ALSO}
pbm(5)
\shead{AUTHOR}
\copyright 1988, 1989 by Jef Poskanzer, Michael Haberler, and Bo Thide'.
% Permission to use, copy, modify, and distribute this software and its
% documentation for any purpose and without fee is hereby granted, provided
% that the above copyright notice appear in all copies and that both that
% copyright notice and this permission notice appear in supporting
% documentation.  This software is provided "as is" without express or
% implied warranty.
%
% end of input file: pbmtogo.1
%--------------------------------------------------
 
% -*-LaTeX-*-
% Converted automatically from troff to LaTeX
% by tr2latex ($Revision: 2.2 $$Date: 1992/04/27 15:13:26 $ by C. Engel)
% on Fri Oct  8 16:08:06 1993
% tr2latex was written by Kamal Al-Yahya at Stanford University
% (Kamal%Hanauma@SU-SCORE.ARPA)
% and substantially enhanced by Christian Engel at RWTH Aachen
% (krischan@informatik.rwth-aachen.de).
%
% troff input file: pbmtoicon.1

\newpage
%--------------------------------------------------
% start of input file: pbmtoicon.1
%
\phead{pbmtoicon}{1}{31 August 1988}{}{}

%.IX pbmtoicon
\shead{NAME}
pbmtoicon - convert a portable bitmap into a Sun icon
\shead{SYNOPSIS}
{\bf pbmtoicon}
{\rm [}{\it pbmfile}{\rm ]}
\shead{DESCRIPTION}
Reads a portable bitmap as input.
Produces a Sun icon as output.
%.IX Sun
%.IX "Sun icon format"
\shead{SEE ALSO}
icontopbm(1), pbm(5)
\shead{AUTHOR}
\copyright 1988 by Jef Poskanzer.
% Permission to use, copy, modify, and distribute this software and its
% documentation for any purpose and without fee is hereby granted, provided
% that the above copyright notice appear in all copies and that both that
% copyright notice and this permission notice appear in supporting
% documentation.  This software is provided "as is" without express or
% implied warranty.
%
% end of input file: pbmtoicon.1
%--------------------------------------------------
 
% -*-LaTeX-*-
% Converted automatically from troff to LaTeX
% by tr2latex ($Revision: 2.2 $$Date: 1992/04/27 15:13:26 $ by C. Engel)
% on Fri Oct  8 16:08:07 1993
% tr2latex was written by Kamal Al-Yahya at Stanford University
% (Kamal%Hanauma@SU-SCORE.ARPA)
% and substantially enhanced by Christian Engel at RWTH Aachen
% (krischan@informatik.rwth-aachen.de).
%
% troff input file: pbmtolj.1

\newpage
%--------------------------------------------------
% start of input file: pbmtolj.1
%
\phead{pbmtolj}{1}{29 August 1988}{}{}

%.IX pbmtolj
\shead{NAME}
pbmtolj - convert a portable bitmap into HP LaserJet format
\shead{SYNOPSIS}
{\bf pbmtolj}
{\rm [}{\bf -resolution}
{\it N}{\rm ]}
{\rm [}{\it pbmfile}{\rm ]}
\shead{DESCRIPTION}
Reads a portable bitmap as input.
Produces HP LaserJet data as output.
%.IX "HP LaserJet"
\par
Note that there is no ljtopbm tool.
\shead{OPTIONS}
\begin{TPlist}{{\bf -resolution}}
\item[{{\bf -resolution}}]
Specifies the resolution of the output device, in dpi.
Typical values are 75, 100, 150, 300.
The default is 75.
\end{TPlist}

\par
All flags can be abbreviated to their shortest unique prefix.
\shead{SEE ALSO}
pbm(5)
\shead{AUTHOR}
\copyright 1988 by Jef Poskanzer and Michael Haberler.
% Permission to use, copy, modify, and distribute this software and its
% documentation for any purpose and without fee is hereby granted, provided
% that the above copyright notice appear in all copies and that both that
% copyright notice and this permission notice appear in supporting
% documentation.  This software is provided "as is" without express or
% implied warranty.
%
% end of input file: pbmtolj.1
%--------------------------------------------------
 
% -*-LaTeX-*-
% Converted automatically from troff to LaTeX
% by tr2latex ($Revision: 2.2 $$Date: 1992/04/27 15:13:26 $ by C. Engel)
% on Fri Oct  8 16:08:15 1993
% tr2latex was written by Kamal Al-Yahya at Stanford University
% (Kamal%Hanauma@SU-SCORE.ARPA)
% and substantially enhanced by Christian Engel at RWTH Aachen
% (krischan@informatik.rwth-aachen.de).
%
% troff input file: pbmtoln03.1

\newpage
%--------------------------------------------------
% start of input file: pbmtoln03.1
%
\phead{pbmtoln03}{1}{7 May 1993}{}{}

%.IX pbmtoln03
\shead{NAME}
pbmtoln03 - convert protable bitmap to DEC LN03+ Sixel output
\shead{SYNOPSIS}
{\bf pbmtoln03}
{\rm [}{\bf -rltbf}{\rm ]}
{\it pbmfile}
\shead{DESCRIPTION}
Reads a portable bitmap as input.
%.IX "DEC LN03+ Sixel"
Produces a DEC LN03+ Sixel output file.
\shead{OPTIONS}
\begin{TPlist}{{\bf -l nn}}
\item[{{\bf -l nn}}]
Use ``nn'' as value for left margin (default 0).
\item[{{\bf -r nn}}]
Use ``nn'' as value for right margin (default 2400).
\item[{{\bf -t nn}}]
Use ``nn'' as value for top margin (default 0).
\item[{{\bf -b nn}}]
Use ``nn'' as value for bottom margin (default 3400).
\item[{{\bf -f nn}}]
Use ``nn'' as value for form length (default 3400).
\end{TPlist}

\shead{SEE ALSO}
pbm(5)
\shead{AUTHOR}
Tim Cook, 26 Feb 1992
%
% end of input file: pbmtoln03.1
%--------------------------------------------------
 
% -*-LaTeX-*-
% Converted automatically from troff to LaTeX
% by tr2latex ($Revision: 2.2 $$Date: 1992/04/27 15:13:26 $ by C. Engel)
% on Fri Oct  8 16:08:18 1993
% tr2latex was written by Kamal Al-Yahya at Stanford University
% (Kamal%Hanauma@SU-SCORE.ARPA)
% and substantially enhanced by Christian Engel at RWTH Aachen
% (krischan@informatik.rwth-aachen.de).
%
% troff input file: pbmtolps.1

\newpage
%--------------------------------------------------
% start of input file: pbmtolps.1
%
\phead{pbmtolps}{12 Dec 1990}{}{}{}

\shead{NAME}
pbmtolps - convert portable bitmap to PostScript
\shead{SYNOPSIS}
pbmtolps [ -dpi n ] [ pbmfile ]
\shead{DESCRIPTION}
Reads a portable bitmap as input, and outputs PostScript.
The output Postscript uses lines instead of the image operator to
generate a (device dependent) picture which will be imaged
much faster.
\par
The Postscript path length is constrained to be less that 1000
points so that no limits are overrun on the Apple Laserwriter
and (presumably) no other printers.
\shead{SEE ALSO}
pgmtops(1), ppmtops(1), pbm(5)
\shead{AUTHOR}
\copyright George Phillips (phillips@cs.ubc.ca).
%
% end of input file: pbmtolps.1
%--------------------------------------------------
 
% -*-LaTeX-*-
% Converted automatically from troff to LaTeX
% by tr2latex ($Revision: 2.2 $$Date: 1992/04/27 15:13:26 $ by C. Engel)
% on Fri Oct  8 16:08:08 1993
% tr2latex was written by Kamal Al-Yahya at Stanford University
% (Kamal%Hanauma@SU-SCORE.ARPA)
% and substantially enhanced by Christian Engel at RWTH Aachen
% (krischan@informatik.rwth-aachen.de).
%
% troff input file: pbmtomacp.1

\newpage
%--------------------------------------------------
% start of input file: pbmtomacp.1
%
\phead{pbmtomacp}{1}{31 August 1988}{}{}

%.IX pbmtomacp
\shead{NAME}
pbmtomacp - convert a portable bitmap into a MacPaint file
\shead{SYNOPSIS}
{\bf pbmtomacp}
{\rm [}{\bf -l}
{\it left}{\rm ]}
{\rm [}{\bf -r}
{\it right}{\rm ]}
{\rm [}{\bf -b}
{\it bottom}{\rm ]}
{\rm [}{\bf -t}
{\it top}{\rm ]}
{\rm [}{\it pbmfile}{\rm ]}
\shead{DESCRIPTION}
Reads a portable bitmap as input.
If no input-file is given, standard input is assumed.
Produces a MacPaint file as output.
%.IX MacPaint
%.IX Macintosh
\par
The generated file is only the data fork of a picture.
You will need a program such as
{\it mcvert}
to generate a Macbinary or a BinHex file that contains the necessary
information to identify the file as a PNTG file to MacOS.
\shead{OPTIONS}
\par
Left, right, bottom \& top let you define a square into the pbm file,
that must be converted.
Default is the whole file.
If the file is too large for a MacPaint-file, the bitmap is cut to fit
from ( left, top ).
\shead{BUGS}
The source code contains comments in a language other than English.
\shead{SEE ALSO}
ppmtopict(1), macptopbm(1), pbm(5), mcvert(1)
\shead{AUTHOR}
\copyright 1988 by Douwe van der Schaaf ($\ldots$!mcvax!uvapsy!vdschaaf).
% Permission to use, copy, modify, and distribute this software and its
% documentation for any purpose and without fee is hereby granted, provided
% that the above copyright notice appear in all copies and that both that
% copyright notice and this permission notice appear in supporting
% documentation.  This software is provided "as is" without express or
% implied warranty.
%
% end of input file: pbmtomacp.1
%--------------------------------------------------
 
% -*-LaTeX-*-
% Converted automatically from troff to LaTeX
% by tr2latex ($Revision: 2.2 $$Date: 1992/04/27 15:13:26 $ by C. Engel)
% on Fri Oct  8 16:08:08 1993
% tr2latex was written by Kamal Al-Yahya at Stanford University
% (Kamal%Hanauma@SU-SCORE.ARPA)
% and substantially enhanced by Christian Engel at RWTH Aachen
% (krischan@informatik.rwth-aachen.de).
%
% troff input file: pbmtomgr.1

\newpage
%--------------------------------------------------
% start of input file: pbmtomgr.1
%
\phead{pbmtomgr}{1}{24 January 1989}{}{}

%.IX pbmtomgr
\shead{NAME}
pbmtomgr - convert a portable bitmap into a MGR bitmap
\shead{SYNOPSIS}
{\bf pbmtomgr}
{\rm [}{\it pbmfile}{\rm ]}
\shead{DESCRIPTION}
Reads a portable bitmap as input.
Produces a MGR bitmap as output.
%.IX MGR
\shead{SEE ALSO}
mgrtopbm(1), pbm(5)
\shead{AUTHOR}
\copyright 1989 by Jef Poskanzer.
% Permission to use, copy, modify, and distribute this software and its
% documentation for any purpose and without fee is hereby granted, provided
% that the above copyright notice appear in all copies and that both that
% copyright notice and this permission notice appear in supporting
% documentation.  This software is provided "as is" without express or
% implied warranty.
%
% end of input file: pbmtomgr.1
%--------------------------------------------------
 
% -*-LaTeX-*-
% Converted automatically from troff to LaTeX
% by tr2latex ($Revision: 2.2 $$Date: 1992/04/27 15:13:26 $ by C. Engel)
% on Fri Oct  8 16:08:36 1993
% tr2latex was written by Kamal Al-Yahya at Stanford University
% (Kamal%Hanauma@SU-SCORE.ARPA)
% and substantially enhanced by Christian Engel at RWTH Aachen
% (krischan@informatik.rwth-aachen.de).
%
% troff input file: pbmtopgm.1

\newpage
%--------------------------------------------------
% start of input file: pbmtopgm.1
%
\phead{pbmtopgm}{12 Dec 1990}{}{}{}

\shead{NAME}
pbmtopgm - convert portable bitmap to portable graymap by averaging areas
\shead{SYNOPSIS}
pbmtopgm $<$width$>$ $<$height$>$ [pbmfile]
\shead{DESCRIPTION}
Reads a portable bitmap as input. Outputs a portable graymap created by
averaging the number of pixels within a sample area of
%
\it width %
\rm by %
\it height %
\rm around each point. Pbmtopgm is similar to a
special case of ppmconvol. A ppmsmooth step may be needed after pbmtopgm.
\par
Pbmtopgm has the effect of anti-aliasing bitmaps which contain distinct
line features.
\shead{SEE ALSO}
pbm(5)
\shead{AUTHOR}
\copyright 1990 by Angus Duggan.
\copyright 1989 by Jef Poskanzer.

Permission to use, copy, modify, and distribute this software and its
documentation for any purpose and without fee is hereby granted, provided
that the above copyright notice appear in all copies and that both that
copyright notice and this permission notice appear in supporting
documentation.  This software is provided "as is" without express or
implied warranty.
\shead{NOTES}
Pbmtopgm works best with odd sample width and heights.
%
% end of input file: pbmtopgm.1
%--------------------------------------------------
 
% -*-LaTeX-*-
% Converted automatically from troff to LaTeX
% by tr2latex ($Revision: 2.2 $$Date: 1992/04/27 15:13:26 $ by C. Engel)
% on Fri Oct  8 16:08:09 1993
% tr2latex was written by Kamal Al-Yahya at Stanford University
% (Kamal%Hanauma@SU-SCORE.ARPA)
% and substantially enhanced by Christian Engel at RWTH Aachen
% (krischan@informatik.rwth-aachen.de).
%
% troff input file: pbmtopi3.1

\newpage
%--------------------------------------------------
% start of input file: pbmtopi3.1
%
\phead{pbmtopi3}{1}{11 March 1990}{}{}

%.IX pbmtopi3
\shead{NAME}
pbmtopi3 - convert a portable bitmap into an Atari Degas .pi3 file 
\shead{SYNOPSIS}
{\bf pbmtopi3}
{\rm [}{\it pbmfile}{\rm ]}
\shead{DESCRIPTION}
Reads a portable bitmap as input.
Produces an Atari Degas .pi3 file as output.
%.IX Atari
%.IX "Degas .pi3"
\shead{SEE ALSO}
pi3topbm(1), pbm(5), ppmtopi1(1), pi1toppm(1)
\shead{AUTHOR}
\copyright 1988 by David Beckemeyer (bdt!david) and Jef Poskanzer.
% Permission to use, copy, modify, and distribute this software and its
% documentation for any purpose and without fee is hereby granted,
% provided that the above copyright notice appear in all copies and that
% both that copyright notice and this permission notice appear in
% supporting documentation.
%
% end of input file: pbmtopi3.1
%--------------------------------------------------
 
% -*-LaTeX-*-
% Converted automatically from troff to LaTeX
% by tr2latex ($Revision: 2.2 $$Date: 1992/04/27 15:13:26 $ by C. Engel)
% on Fri Oct  8 16:08:18 1993
% tr2latex was written by Kamal Al-Yahya at Stanford University
% (Kamal%Hanauma@SU-SCORE.ARPA)
% and substantially enhanced by Christian Engel at RWTH Aachen
% (krischan@informatik.rwth-aachen.de).
%
% troff input file: pbmtopk.1

\newpage
%--------------------------------------------------
% start of input file: pbmtopk.1
%
\phead{pbmtopk}{1}{6 August 1990}{}{}

\shead{NAME}
pbmtopk - convert a portable bitmap into a packed (PK) format font
\shead{SYNOPSIS}
pbmtopk pkfile[.pk] tfmfile[.tfm] resolution [-s designsize] [-p num param...]
[-C codingscheme] [-F family] [-f optfile] [-c num]
[-W width] [-H height] [-D depth]
[-I ital] [-h horiz] [-v vert] [-x xoff] [-y yoff] [pbmfile]...
\shead{DESCRIPTION}
Reads portable bitmaps as input, and produces a packed (PK) font file and a
TFM (TeX font metric) file as output. The resolution parameter indicates the
resolution of the font, in dots per inch. If the filename ``-'' is used for any
of the filenames, the standard input stream (or standard output where
appropriate) will be used.
\shead{OPTIONS}
\begin{IPlist}
\IPitem{{-s\ designsize}}
Sets the design size of the font, in TeX's points (72.27pt to the inch). The
default design size is 1. The TFM parameters are given as multiples of the
design size.
\IPitem{{-p\ num\ param...}}
Sets the first num font parameters for the font. The first seven parameters
are the slant,
interword spacing, interword space stretchability, interword space
shrinkability, x-height, quad width, and post-sentence extra space of the
font. Math and symbol fonts may have more parameters; see The TeXbook for a
list of these. Reasonable default values are chosen for parameters which are
not specified.
\IPitem{{-C\ codingscheme}}
Sets the coding scheme comment in the TFM file.
\IPitem{{-F\ family}}
Sets the font family comment in the TFM file.
\IPitem{{-f\ optfile}}
Reads the file optfile, which should contain a lines of the form:
\par\vspace{1.0\baselineskip}
\nofill
\raggedright
   filename xoff yoff horiz vert width height depth ital
\fill
%.ad
\par\vspace{1.0\baselineskip}
The pbm files specified by the filename parameters are inserted consecutively
in the font with the specified attributes. If any of the attributes are
omitted, or replaced with ``*'', a default value will be calculated from the
size of the bitmap. The settings of the -W, -H, -D, -I, -h, -v, -x, and -y
options do not affected characters created in this way.
The character number can be changed by including a line starting with
``='',
followed by the new number.
Lines beginning with
``\%'' or ``\#'' are ignored.
\IPitem{{-c\ num}}
Sets the character number of the next bitmap encountered to num.
\IPitem{{-W\ width}}
Sets the TFM width of the next character to width (in design size multiples).
\IPitem{{-H\ height}}
Sets the TFM height of the next character to height (in design size multiples).
\IPitem{{-D\ depth}}
Sets the TFM depth of the next character to depth (in design size multiples).
\IPitem{{-I\ ital}}
Sets the italic correction of the next character to
ital (in design size multiples).
\IPitem{{-h\ horiz}}
Sets the horizontal escapement of the next character to horiz (in pixels).
\IPitem{{-v\ vert}}
Sets the vertical escapement of the next character to vert (in pixels).
\IPitem{{-x\ xoff}}
Sets the horizontal offset of the next character to xoff (in pixels).
\IPitem{{-y\ yoff}}
Sets the vertical offset of the next character to yoff (in pixels, from the
top row).
\end{IPlist}

\shead{SEE ALSO}
pktopbm(1), pbm(5)
\shead{AUTHOR}
Adapted from Tom Rokicki's pxtopk by Angus Duggan (ajcd@dcs.ed.ac.uk).

%
% end of input file: pbmtopk.1
%--------------------------------------------------
 
% -*-LaTeX-*-
% Converted automatically from troff to LaTeX
% by tr2latex ($Revision: 2.2 $$Date: 1992/04/27 15:13:26 $ by C. Engel)
% on Fri Oct  8 16:08:09 1993
% tr2latex was written by Kamal Al-Yahya at Stanford University
% (Kamal%Hanauma@SU-SCORE.ARPA)
% and substantially enhanced by Christian Engel at RWTH Aachen
% (krischan@informatik.rwth-aachen.de).
%
% troff input file: pbmtoplot.1

\newpage
%--------------------------------------------------
% start of input file: pbmtoplot.1
%
\phead{pbmtoplot}{1}{1 September 1990}{}{}

%.IX pbmtoplot
\shead{NAME}
pbmtoplot - convert a portable bitmap into a Unix plot(5) file
\shead{SYNOPSIS}
{\bf pbmtoplot}
{\rm [}{\it pbmfile}{\rm ]}
\shead{DESCRIPTION}
Reads a portable bitmap as input.
Produces a Unix
{\it plot}
file.
%.IX plot
\par
Note that there is no plottopbm tool - this transformation is one-way.
\shead{SEE ALSO}
pbm(5), plot(5)
\shead{AUTHOR}
\copyright 1990 by Arthur David Olson.
% Permission to use, copy, modify, and distribute this software and its
% documentation for any purpose and without fee is hereby granted, provided
% that the above copyright notice appear in all copies and that both that
% copyright notice and this permission notice appear in supporting
% documentation.  This software is provided "as is" without express or
% implied warranty.
%
% end of input file: pbmtoplot.1
%--------------------------------------------------
 
% -*-LaTeX-*-
% Converted automatically from troff to LaTeX
% by tr2latex ($Revision: 2.2 $$Date: 1992/04/27 15:13:26 $ by C. Engel)
% on Fri Oct  8 16:08:10 1993
% tr2latex was written by Kamal Al-Yahya at Stanford University
% (Kamal%Hanauma@SU-SCORE.ARPA)
% and substantially enhanced by Christian Engel at RWTH Aachen
% (krischan@informatik.rwth-aachen.de).
%
% troff input file: pbmtoptx.1

\newpage
%--------------------------------------------------
% start of input file: pbmtoptx.1
%
\phead{pbmtoptx}{1}{31 August 1988}{}{}

%.IX pbmtoptx
\shead{NAME}
pbmtoptx - convert a portable bitmap into Printronix printer graphics
\shead{SYNOPSIS}
{\bf pbmtoptx}
{\rm [}{\it pbmfile}{\rm ]}
\shead{DESCRIPTION}
Reads a portable bitmap as input.
Produces a file of Printronix printer graphics as output.
%.IX Printronix
\par
Note that there is no ptxtopbm tool - this transformation is one way.
\shead{SEE ALSO}
pbm(5)
\shead{AUTHOR}
\copyright 1988 by Jef Poskanzer.
% Permission to use, copy, modify, and distribute this software and its
% documentation for any purpose and without fee is hereby granted, provided
% that the above copyright notice appear in all copies and that both that
% copyright notice and this permission notice appear in supporting
% documentation.  This software is provided "as is" without express or
% implied warranty.
%
% end of input file: pbmtoptx.1
%--------------------------------------------------
 
% -*-LaTeX-*-
% Converted automatically from troff to LaTeX
% by tr2latex ($Revision: 2.2 $$Date: 1992/04/27 15:13:26 $ by C. Engel)
% on Fri Oct  8 16:08:10 1993
% tr2latex was written by Kamal Al-Yahya at Stanford University
% (Kamal%Hanauma@SU-SCORE.ARPA)
% and substantially enhanced by Christian Engel at RWTH Aachen
% (krischan@informatik.rwth-aachen.de).
%
% troff input file: pbmtox10bm.1

\newpage
%--------------------------------------------------
% start of input file: pbmtox10bm.1
%
\phead{pbmtox10bm}{1}{31 August 1988}{}{}

%.IX pbmtox10bm
\shead{NAME}
pbmtox10bm - convert a portable bitmap into an X10 bitmap
\shead{SYNOPSIS}
{\bf pbmtox10bm}
{\rm [}{\it pbmfile}{\rm ]}
\shead{DESCRIPTION}
Reads a portable bitmap as input.
Produces an X10 bitmap as output.
This older format is maintained for compatibility.
%.IX "X bitmap"
%.IX "X window system"
\par
Note that there is no x10bmtopbm tool, because
{\it xbmtopbm}
can read both X11 and X10 bitmaps.
\shead{SEE ALSO}
pbmtoxbm(1), xbmtopbm(1), pbm(5)
\shead{AUTHOR}
\copyright 1988 by Jef Poskanzer.
% Permission to use, copy, modify, and distribute this software and its
% documentation for any purpose and without fee is hereby granted, provided
% that the above copyright notice appear in all copies and that both that
% copyright notice and this permission notice appear in supporting
% documentation.  This software is provided "as is" without express or
% implied warranty.
%
% end of input file: pbmtox10bm.1
%--------------------------------------------------
 
% -*-LaTeX-*-
% Converted automatically from troff to LaTeX
% by tr2latex ($Revision: 2.2 $$Date: 1992/04/27 15:13:26 $ by C. Engel)
% on Fri Oct  8 16:08:11 1993
% tr2latex was written by Kamal Al-Yahya at Stanford University
% (Kamal%Hanauma@SU-SCORE.ARPA)
% and substantially enhanced by Christian Engel at RWTH Aachen
% (krischan@informatik.rwth-aachen.de).
%
% troff input file: pbmtoxbm.1

\newpage
%--------------------------------------------------
% start of input file: pbmtoxbm.1
%
\phead{pbmtoxbm}{1}{31 August 1988}{}{}

%.IX pbmtoxbm
\shead{NAME}
pbmtoxbm - convert a portable bitmap into an X11 bitmap
\shead{SYNOPSIS}
{\bf pbmtoxbm}
{\rm [}{\it pbmfile}{\rm ]}
\shead{DESCRIPTION}
Reads a portable bitmap as input.
Produces an X11 bitmap as output.
%.IX "X bitmap"
%.IX "X window system"
\shead{SEE ALSO}
pbmtox10bm(1), xbmtopbm(1), pbm(5)
\shead{AUTHOR}
\copyright 1988 by Jef Poskanzer.
% Permission to use, copy, modify, and distribute this software and its
% documentation for any purpose and without fee is hereby granted, provided
% that the above copyright notice appear in all copies and that both that
% copyright notice and this permission notice appear in supporting
% documentation.  This software is provided "as is" without express or
% implied warranty.
%
% end of input file: pbmtoxbm.1
%--------------------------------------------------
 
% -*-LaTeX-*-
% Converted automatically from troff to LaTeX
% by tr2latex ($Revision: 2.2 $$Date: 1992/04/27 15:13:26 $ by C. Engel)
% on Fri Oct  8 16:08:12 1993
% tr2latex was written by Kamal Al-Yahya at Stanford University
% (Kamal%Hanauma@SU-SCORE.ARPA)
% and substantially enhanced by Christian Engel at RWTH Aachen
% (krischan@informatik.rwth-aachen.de).
%
% troff input file: pbmtoybm.1

\newpage
%--------------------------------------------------
% start of input file: pbmtoybm.1
%
\phead{pbmtoybm}{1}{06 March 1990}{}{}

%.IX pbmtoybm
\shead{NAME}
pgmtoybm - convert a portable bitmap into a Bennet Yee ``face'' file
\shead{SYNOPSIS}
{\bf pbmtoybm}
{\rm [}{\it pbmfile}{\rm ]}
\shead{DESCRIPTION}
Reads a portable bitmap as input.
Produces as output a file acceptable to the
{\it face}
and
{\it xbm}
programs by Bennet Yee (bsy+@cs.cmu.edu).
%.IX face
\shead{SEE ALSO}
ybmtopbm(1), pbm(5), face(1), face(5), xbm(1)
\shead{AUTHOR}
\copyright 1991 by Jamie Zawinski and Jef Poskanzer.
% Permission to use, copy, modify, and distribute this software and its
% documentation for any purpose and without fee is hereby granted, provided
% that the above copyright notice appear in all copies and that both that
% copyright notice and this permission notice appear in supporting
% documentation.  This software is provided "as is" without express or
% implied warranty.
%
% end of input file: pbmtoybm.1
%--------------------------------------------------
 
% -*-LaTeX-*-
% Converted automatically from troff to LaTeX
% by tr2latex ($Revision: 2.2 $$Date: 1992/04/27 15:13:26 $ by C. Engel)
% on Fri Oct  8 16:08:12 1993
% tr2latex was written by Kamal Al-Yahya at Stanford University
% (Kamal%Hanauma@SU-SCORE.ARPA)
% and substantially enhanced by Christian Engel at RWTH Aachen
% (krischan@informatik.rwth-aachen.de).
%
% troff input file: pbmtozinc.1

\newpage
%--------------------------------------------------
% start of input file: pbmtozinc.1
%
\phead{pbmtozinc}{l}{02 November 1990}{}{}

%.IX pbmtozinc
\shead{NAME}
pbmtozinc - convert a portable bitmap into a Zinc bitmap
\shead{SYNOPSIS}
{\bf pbmtozinc}
{\rm [}{\it pbmfile}{\rm ]}
\shead{DESCRIPTION}
Reads a portable bitmap as input.
Produces a bitmap in the format used by the Zinc Interface Library
(ZIL) Version 1.0 as output.
%.IX "Zinc Interface Library"
\shead{SEE ALSO}
pbm(5)
\shead{AUTHOR}
\copyright 1988 by James Darrell McCauley (jdm5548@diamond.tamu.edu)
and Jef Poskanzer.
% Permission to use, copy, modify, and distribute this software and its
% documentation for any purpose and without fee is hereby granted, provided
% that the above copyright notice appear in all copies and that both that
% copyright notice and this permission notice appear in supporting
% documentation.  This software is provided "as is" without express or
% implied warranty.
%
% Zinc and Zinc Interface Library are trademarks of
% Zinc Software Inc., Pleasant Grove, Utah.
%
% end of input file: pbmtozinc.1
%--------------------------------------------------
 
% -*-LaTeX-*-
% Converted automatically from troff to LaTeX
% by tr2latex ($Revision: 2.2 $$Date: 1992/04/27 15:13:26 $ by C. Engel)
% on Fri Oct  8 16:08:13 1993
% tr2latex was written by Kamal Al-Yahya at Stanford University
% (Kamal%Hanauma@SU-SCORE.ARPA)
% and substantially enhanced by Christian Engel at RWTH Aachen
% (krischan@informatik.rwth-aachen.de).
%
% troff input file: pbmupc.1

\newpage
%--------------------------------------------------
% start of input file: pbmupc.1
%
\phead{pbmupc}{1}{14 March 1989}{}{}

%.IX pbmupc
\shead{NAME}
pbmupc - create a Universal Product Code bitmap
\shead{SYNOPSIS}
{\bf pbmupc}
{\rm [}{\bf -s1}{\rm $|$}{\bf -s2}{\rm ]}
{\it type manufac product}
\shead{DESCRIPTION}
Generates a Universal Product Code symbol.
%.IX "Universal Product Code"
The three arguments are: a one digit product type, a five digit
manufacturer code, and a five digit product code.
For example, ``0 72890 00011'' is the code for Heineken.
%.IX Heineken
\par
As presently configured,
{\it pbmupc}
produces a bitmap 230 bits wide and 175 bits high.
The size can be altered by changing the defines at the beginning of
the program, or by running the output through
{\it pnmenlarge}
or
{\it pnmscale}{\rm .}
\shead{OPTIONS}
\par
The
{\bf -s1}
and
{\bf -s2}
flags select the style of UPC to generate.
The default,
{\bf -s1}{\rm ,}
looks more or less like this:
\nofill
 $|$$|$$|$$|$$|$$|$$|$$|$$|$$|$$|$$|$$|$$|$$|$$|$
 $|$$|$$|$$|$$|$$|$$|$$|$$|$$|$$|$$|$$|$$|$$|$$|$
 $|$$|$$|$$|$$|$$|$$|$$|$$|$$|$$|$$|$$|$$|$$|$$|$
 $|$$|$$|$$|$$|$$|$$|$$|$$|$$|$$|$$|$$|$$|$$|$$|$
0$|$$|$12345$|$$|$67890$|$$|$5
\fill
The other style,
{\bf -s2}{\rm ,}
puts the product type digit higher up, and
doesn't display the checksum digit:
\nofill
 $|$$|$$|$$|$$|$$|$$|$$|$$|$$|$$|$$|$$|$$|$$|$$|$
 $|$$|$$|$$|$$|$$|$$|$$|$$|$$|$$|$$|$$|$$|$$|$$|$
0$|$$|$$|$$|$$|$$|$$|$$|$$|$$|$$|$$|$$|$$|$$|$$|$
 $|$$|$$|$$|$$|$$|$$|$$|$$|$$|$$|$$|$$|$$|$$|$$|$
 $|$$|$12345$|$$|$67890$|$$|$
\fill
\shead{SEE ALSO}
pbm(5)
\shead{AUTHOR}
\copyright 1989 by Jef Poskanzer.
% Permission to use, copy, modify, and distribute this software and its
% documentation for any purpose and without fee is hereby granted, provided
% that the above copyright notice appear in all copies and that both that
% copyright notice and this permission notice appear in supporting
% documentation.  This software is provided "as is" without express or
% implied warranty.
%
% end of input file: pbmupc.1
%--------------------------------------------------
 
% -*-LaTeX-*-
% Converted automatically from troff to LaTeX
% by tr2latex ($Revision: 2.2 $$Date: 1992/04/27 15:13:26 $ by C. Engel)
% on Fri Oct  8 16:08:42 1993
% tr2latex was written by Kamal Al-Yahya at Stanford University
% (Kamal%Hanauma@SU-SCORE.ARPA)
% and substantially enhanced by Christian Engel at RWTH Aachen
% (krischan@informatik.rwth-aachen.de).
%
% troff input file: pcxtoppm.1

\newpage
%--------------------------------------------------
% start of input file: pcxtoppm.1
%
\phead{pcxtoppm}{1}{9 April 1990}{}{}

%.IX pcxtoppm
\shead{NAME}
pcxtoppm - convert a PCX file into a portable pixmap
\shead{SYNOPSIS}
{\bf pcxtoppm}
{\rm [}{\it pcxfile}{\rm ]}
\shead{DESCRIPTION}
Reads a PCX file as input.
%.IX PCX
Produces a portable pixmap as output.
\shead{SEE ALSO}
ppmtopcx(1), ppm(5)
\shead{AUTHOR}
\copyright 1990 by Michael Davidson.
% Permission to use, copy, modify, and distribute this software and its
% documentation for any purpose and without fee is hereby granted, provided
% that the above copyright notice appear in all copies and that both that
% copyright notice and this permission notice appear in supporting
% documentation.  This software is provided "as is" without express or
% implied warranty.
%
% end of input file: pcxtoppm.1
%--------------------------------------------------
 
% -*-LaTeX-*-
% Converted automatically from troff to LaTeX
% by tr2latex ($Revision: 2.2 $$Date: 1992/04/27 15:13:26 $ by C. Engel)
% on Fri Oct  8 16:08:38 1993
% tr2latex was written by Kamal Al-Yahya at Stanford University
% (Kamal%Hanauma@SU-SCORE.ARPA)
% and substantially enhanced by Christian Engel at RWTH Aachen
% (krischan@informatik.rwth-aachen.de).
%
% troff input file: pgm.5

\newpage
%--------------------------------------------------
% start of input file: pgm.5
%
\phead{pgm}{5}{12 November 1991}{}{}

\shead{NAME}
pgm - portable graymap file format
\shead{DESCRIPTION}
The portable graymap format is a lowest common denominator grayscale
file format.
%.IX "PGM file format"
The definition is as follows:
\begin{IPlist}
\IPitem{{-}}
A ``magic number'' for identifying the file type.
A pgm file's magic number is the two characters ``P2''.
%.IX "magic numbers"
\IPitem{{-}}
Whitespace (blanks, TABs, CRs, LFs).
\IPitem{{-}}
A width, formatted as ASCII characters in decimal.
\IPitem{{-}}
Whitespace.
\IPitem{{-}}
A height, again in ASCII decimal.
\IPitem{{-}}
Whitespace.
\IPitem{{-}}
The maximum gray value, again in ASCII decimal.
\IPitem{{-}}
Whitespace.
\IPitem{{-}}
Width * height gray values, each in ASCII decimal, between 0 and the specified
maximum value, separated by whitespace, starting at the top-left
corner of the graymap, proceeding in normal English reading order.
A value of 0 means black, and the maximum value means white.
\IPitem{{-}}
Characters from a ``\#'' to the next end-of-line are ignored (comments).
\IPitem{{-}}
No line should be longer than 70 characters.
\end{IPlist}

\par
Here is an example of a small graymap in this format:
\nofill
P2
\# feep.pgm
24 7
15
0  0  0  0  0  0  0  0  0  0  0  0  0  0  0  0  0  0  0  0  0  0  0  0
0  3  3  3  3  0  0  7  7  7  7  0  0 11 11 11 11  0  0 15 15 15 15  0
0  3  0  0  0  0  0  7  0  0  0  0  0 11  0  0  0  0  0 15  0  0 15  0
0  3  3  3  0  0  0  7  7  7  0  0  0 11 11 11  0  0  0 15 15 15 15  0
0  3  0  0  0  0  0  7  0  0  0  0  0 11  0  0  0  0  0 15  0  0  0  0
0  3  0  0  0  0  0  7  7  7  7  0  0 11 11 11 11  0  0 15  0  0  0  0
0  0  0  0  0  0  0  0  0  0  0  0  0  0  0  0  0  0  0  0  0  0  0  0
\fill
\par
Programs that read this format should be as lenient as possible,
accepting anything that looks remotely like a graymap.
\par
There is also a variant on the format, available
by setting the RAWBITS option at compile time.  This variant is
different in the following ways:
%.IX RAWBITS
\begin{IPlist}
\IPitem{{-}}
The ``magic number'' is ``P5'' instead of ``P2''.
\IPitem{{-}}
The gray values are stored as plain bytes, instead of ASCII decimal.
\IPitem{{-}}
No whitespace is allowed in the grays section, and only a single character
of whitespace (typically a newline) is allowed after the maxval.
\IPitem{{-}}
The files are smaller and many times faster to read and write.
\end{IPlist}

\par
Note that this raw format can only be used for maxvals less than
or equal to 255.
If you use the
{\it pgm}
library and try to write a file with a larger maxval,
it will automatically fall back on the slower but more general plain
format.
\shead{SEE ALSO}
fitstopgm(1), fstopgm(1), hipstopgm(1), lispmtopgm(1), psidtopgm(1),
rawtopgm(1),
pgmbentley(1), pgmcrater(1), pgmedge(1), pgmenhance(1), pgmhist(1), pgmnorm(1),
pgmoil(1), pgmramp(1), pgmtexture(1),
pgmtofits(1), pgmtofs(1), pgmtolispm(1), pgmtopbm(1),
pnm(5), pbm(5), ppm(5)
\shead{AUTHOR}
\copyright 1989, 1991 by Jef Poskanzer.
% Permission to use, copy, modify, and distribute this software and its
% documentation for any purpose and without fee is hereby granted, provided
% that the above copyright notice appear in all copies and that both that
% copyright notice and this permission notice appear in supporting
% documentation.  This software is provided "as is" without express or
% implied warranty.
%
% end of input file: pgm.5
%--------------------------------------------------
 
% -*-LaTeX-*-
% Converted automatically from troff to LaTeX
% by tr2latex ($Revision: 2.2 $$Date: 1992/04/27 15:13:26 $ by C. Engel)
% on Fri Oct  8 16:08:26 1993
% tr2latex was written by Kamal Al-Yahya at Stanford University
% (Kamal%Hanauma@SU-SCORE.ARPA)
% and substantially enhanced by Christian Engel at RWTH Aachen
% (krischan@informatik.rwth-aachen.de).
%
% troff input file: pgmbentley.1

\newpage
%--------------------------------------------------
% start of input file: pgmbentley.1
%
\phead{pgmbentley}{1}{11 January 1991}{}{}

%.IX pgmbentley
\shead{NAME}
pgmbentley - Bentleyize a portable graymap
\shead{SYNOPSIS}
{\bf pgmbentley}
{\rm [}{\it pgmfile}{\rm ]}
\shead{DESCRIPTION}
Reads a portable graymap as input.
Performs The Bentley Effect, and writes a portable graymap as output.
%.IX "Bentley Effect"
\par
The Bentley Effect is described in ``Beyond Photography'' by Holzmann,
chapter 4, photo 4.
It's a vertical smearing based on brightness.
\shead{SEE ALSO}
pgmoil(1), ppmrelief(1), pgm(5)
\shead{AUTHOR}
\copyright 1990 by Wilson Bent (whb@hoh-2.att.com).
% Permission to use, copy, modify, and distribute this software and its
% documentation for any purpose and without fee is hereby granted, provided
% that the above copyright notice appear in all copies and that both that
% copyright notice and this permission notice appear in supporting
% documentation.  This software is provided "as is" without express or
% implied warranty.
%
% end of input file: pgmbentley.1
%--------------------------------------------------
 
% -*-LaTeX-*-
% Converted automatically from troff to LaTeX
% by tr2latex ($Revision: 2.2 $$Date: 1992/04/27 15:13:26 $ by C. Engel)
% on Fri Oct  8 16:08:27 1993
% tr2latex was written by Kamal Al-Yahya at Stanford University
% (Kamal%Hanauma@SU-SCORE.ARPA)
% and substantially enhanced by Christian Engel at RWTH Aachen
% (krischan@informatik.rwth-aachen.de).
%
% troff input file: pgmcrater.1

\newpage
%--------------------------------------------------
% start of input file: pgmcrater.1
%
\phead{pgmcrater}{1}{15 October 1991}{}{}

%.IX pgmcrater
%.IX fractals
%.IX craters
\shead{NAME}
pgmcrater - create cratered terrain by fractal forgery
\shead{SYNOPSIS}
\raggedright
{\bf pgmcrater}
'ti 15
{\rm [}{\bf -number}
{\it n}{\rm ]}
{\rm [}{\bf -height}{\rm $|$}{\bf -ysize}
{\it s}{\rm ]}
{\rm [}{\bf -width}{\rm $|$}{\bf -xsize}
{\it s}{\rm ]}
{\rm [}{\bf -gamma}
{\it g}{\rm ]}
%.ad
\shead{DESCRIPTION}
{\bf pgmcrater}
creates a portable graymap which mimics cratered terrain.  The graymap
is created by simulating the impact of a given number of craters with
random position and size, then rendering the resulting terrain
elevations based on a light source shining from one side of the
screen.  The size distribution of the craters is based on a power law
which results in many more small craters than large ones.  The number
of craters of a given size varies as the reciprocal of the area as
described on pages 31 and 32 of Peitgen and Saupe[1]; cratered bodies
in the Solar System are observed to obey this relationship.  The
formula used to obtain crater radii governed by this law from a
uniformly distributed pseudorandom sequence was developed by Rudy
Rucker.
\par
High resolution images with large numbers of craters often benefit
from being piped through
{\bf pnmsmooth}{\rm .}
The averaging performed by this process eliminates some of the jagged
pixels and lends a mellow ``telescopic image'' feel to the overall
picture.
\shead{OPTIONS}
\begin{TPlist}{{\bf -number}{\it \ n}
}
\item[{{\bf -number}{\it \ n}
}]
Causes
{\it n}
craters to be generated.  If no
{\bf -number}
specification is given, 50000 craters will be generated.  Don't expect
to see them all!  For every large crater there are many, many more
tiny ones which tend simply to erode the landscape.  In general, the
more craters you specify the more realistic the result; ideally you
want the entire terrain to have been extensively turned over again and
again by cratering.  High resolution images containing five to ten
million craters are stunning but take quite a while to create.
\item[{{\bf -height}{\it \ height}
}]
Sets the height of the generated image to
{\it height}
pixels.  The default height is 256 pixels.
\item[{{\bf -width}{\it \ width}
}]
Sets the width of the generated image to
{\it width}
pixels.  The default width is 256 pixels.
\item[{{\bf -xsize}{\it \ width}
}]
Sets the width of the generated image to
{\it width}
pixels.  The default width is 256 pixels.
\item[{{\bf -ysize}{\it \ height}
}]
Sets the height of the generated image to
{\it height}
pixels.  The default height is 256 pixels.
\item[{{\bf -gamma}{\it \ factor}
}]
The specified
{\it factor}
is used to gamma correct the graymap in the same manner as performed
by
{\bf pnmgamma}{\rm .}
The default value is 1.0, which results in a medium contrast image.
Values larger than 1 lighten the image and reduce contrast, while
values less than 1 darken the image, increasing contrast.
\end{TPlist}

\par
All flags can be abbreviated to their shortest unique prefix.
\shead{BUGS}
The
{\bf -gamma}
option isn't really necessary since you can achieve the same
effect by piping the output from
{\bf pgmcrater}
through
{\bf pnmgamma}{\rm .}
However,
{\bf pgmcrater}
performs an internal gamma map anyway in the process of rendering the
elevation array into a graymap, so there's no additional overhead in
allowing a user-specified gamma.
\par
Real craters have two distinct morphologies.
{\bf pgmcrater}
simulates only small craters, which are hemispherical in shape
(regardless of the incidence angle of the impacting body, as long as the
velocity is sufficiently high).  Large craters, such as Copernicus and
Tycho on the Moon, have a ``walled plain'' shape with a cross-section more
like:
\nofill
%.ne 2
%.cs R 18
                /\bs                             /\bs 
\nwl
          \_\_\_\_\_/  \bs \_\_\_\_\_\_\_\_\_\_\_\_/\bs \_\_\_\_\_\_\_\_\_\_\_\_/  \bs \_\_\_\_\_
%.cs R
\fill
%.ss 12
Larger craters should really use this profile, including the central
peak, and totally obliterate the pre-existing terrain.
\shead{SEE ALSO}
{\bf pgm}{\rm (5),}
{\bf pnmgamma}{\rm (1),}
{\bf pnmsmooth}{\rm (1)}
\begin{TPlist}{[1]}
\item[{[1]}]
Peitgen, H.-O., and Saupe, D. eds., The Science Of Fractal Images,
New York: Springer Verlag, 1988.
%.ne 10
\end{TPlist}

\shead{AUTHOR}
\ind{1\parindent}{\nofill
    John Walker
    Autodesk SA
    Avenue des Champs-Montants 14b
    CH-2074 MARIN
    Suisse/Schweiz/Svizzera/Svizra/Switzerland
\fill}
\begin{TPlist}{Usenet:}
\item[{Usenet:}]
kelvin@Autodesk.com
\item[{Fax:}]
038/33 88 15
\item[{Voice:}]
038/33 76 33
\end{TPlist}

\par
Permission to use, copy, modify, and distribute this software and its
documentation for any purpose and without fee is hereby granted,
without any conditions or restrictions.  This software is provided ``as
is'' without express or implied warranty.
\par
{\bf PLUGWARE!}
If you like this kind of stuff, you may also enjoy ``James Gleick's
Chaos--The Software'' for MS-DOS, available for \$59.95 from your
local software store or directly from Autodesk, Inc., Attn: Science
Series, 2320 Marinship Way, Sausalito, CA 94965, USA.  Telephone:
(800) 688-2344 toll-free or, outside the U.S. (415) 332-2344 Ext
4886.  Fax: (415) 289-4718.  ``Chaos--The Software'' includes a more
comprehensive fractal forgery generator which creates
three-dimensional landscapes as well as clouds and planets, plus five
more modules which explore other aspects of Chaos.  The user guide of
more than 200 pages includes an introduction by James Gleick and
detailed explanations by Rudy Rucker of the mathematics and algorithms
used by each program.
%
% end of input file: pgmcrater.1
%--------------------------------------------------
 
% -*-LaTeX-*-
% Converted automatically from troff to LaTeX
% by tr2latex ($Revision: 2.2 $$Date: 1992/04/27 15:13:26 $ by C. Engel)
% on Fri Oct  8 16:08:27 1993
% tr2latex was written by Kamal Al-Yahya at Stanford University
% (Kamal%Hanauma@SU-SCORE.ARPA)
% and substantially enhanced by Christian Engel at RWTH Aachen
% (krischan@informatik.rwth-aachen.de).
%
% troff input file: pgmedge.1

\newpage
%--------------------------------------------------
% start of input file: pgmedge.1
%
\phead{pgmedge}{1}{04 February 1990}{}{}

%.IX pgmedge
\shead{NAME}
pgmedge - edge-detect a portable graymap
\shead{SYNOPSIS}
{\bf pgmedge}
{\rm [}{\it pgmfile}{\rm ]}
\shead{DESCRIPTION}
Reads a portable graymap as input.
Outlines the edges, and writes a portable graymap as output.
%.IX "edge detection"
Piping the result through
{\bf pgmtopbm -threshold}
and playing with the
threshold value will give a bitmap of the edges.
%.IX thresholding
\par
The edge detection technique used is to take the Pythagorean sum of
two Sobel gradient operators at 90 degrees to each other.
For more details see ``Digital Image Processing'' by Gonzalez and Wintz,
chapter 7.
\shead{SEE ALSO}
pgmenhance(1), pgmtopbm(1), pgm(5), pbm(5)
\shead{AUTHOR}
\copyright 1991 by Jef Poskanzer.
% Permission to use, copy, modify, and distribute this software and its
% documentation for any purpose and without fee is hereby granted, provided
% that the above copyright notice appear in all copies and that both that
% copyright notice and this permission notice appear in supporting
% documentation.  This software is provided "as is" without express or
% implied warranty.
%
% end of input file: pgmedge.1
%--------------------------------------------------
 
% -*-LaTeX-*-
% Converted automatically from troff to LaTeX
% by tr2latex ($Revision: 2.2 $$Date: 1992/04/27 15:13:26 $ by C. Engel)
% on Fri Oct  8 16:08:28 1993
% tr2latex was written by Kamal Al-Yahya at Stanford University
% (Kamal%Hanauma@SU-SCORE.ARPA)
% and substantially enhanced by Christian Engel at RWTH Aachen
% (krischan@informatik.rwth-aachen.de).
%
% troff input file: pgmenhance.1

\newpage
%--------------------------------------------------
% start of input file: pgmenhance.1
%
\phead{pgmenhance}{1}{13 January 1989}{}{}

%.IX pgmenhance
\shead{NAME}
pgmenhance - edge-enhance a portable graymap
\shead{SYNOPSIS}
{\bf pgmenhance}
{\rm [}{\it -N}{\rm ]}
{\rm [}{\it pgmfile}{\rm ]}
\shead{DESCRIPTION}
Reads a portable graymap as input.
Enhances the edges, and writes a portable graymap as output.
%.IX "edge enhancement"
\par
The edge enhancing technique is taken from Philip R. Thompson's ``xim''
program, which in turn took it from section 6 of ``Digital Halftones by
Dot Diffusion'', D. E. Knuth, ACM Transaction on Graphics Vol. 6, No. 4,
October 1987, which in turn got it from two 1976 papers by J. F. Jarvis
{\it et. al.}
\shead{OPTIONS}
\par
The optional
{\it -N}
flag should be a digit from 1 to 9.
1 is the lowest level of enhancement, 9 is the highest,
The default is 9.
\shead{SEE ALSO}
pgmedge(1), pgm(5), pbm(5)
\shead{AUTHOR}
\copyright 1989 by Jef Poskanzer.
% Permission to use, copy, modify, and distribute this software and its
% documentation for any purpose and without fee is hereby granted, provided
% that the above copyright notice appear in all copies and that both that
% copyright notice and this permission notice appear in supporting
% documentation.  This software is provided "as is" without express or
% implied warranty.
%
% end of input file: pgmenhance.1
%--------------------------------------------------
 
% -*-LaTeX-*-
% Converted automatically from troff to LaTeX
% by tr2latex ($Revision: 2.2 $$Date: 1992/04/27 15:13:26 $ by C. Engel)
% on Fri Oct  8 16:08:29 1993
% tr2latex was written by Kamal Al-Yahya at Stanford University
% (Kamal%Hanauma@SU-SCORE.ARPA)
% and substantially enhanced by Christian Engel at RWTH Aachen
% (krischan@informatik.rwth-aachen.de).
%
% troff input file: pgmhist.1

\newpage
%--------------------------------------------------
% start of input file: pgmhist.1
%
\phead{pgmhist}{1}{28 February 1989}{}{}

%.IX pgmhist
\shead{NAME}
pgmhist - print a histogram of the values in a portable graymap
\shead{SYNOPSIS}
{\bf pgmhist}
{\rm [}{\it pgmfile}{\rm ]}
\shead{DESCRIPTION}
Reads a portable graymap as input.
Prints a histogram of the gray values.
\shead{SEE ALSO}
pgmnorm(1), pgm(5), ppmhist(1)
\shead{AUTHOR}
\copyright 1989 by Jef Poskanzer.
% Permission to use, copy, modify, and distribute this software and its
% documentation for any purpose and without fee is hereby granted, provided
% that the above copyright notice appear in all copies and that both that
% copyright notice and this permission notice appear in supporting
% documentation.  This software is provided "as is" without express or
% implied warranty.
%
% end of input file: pgmhist.1
%--------------------------------------------------
 
% -*-LaTeX-*-
% Converted automatically from troff to LaTeX
% by tr2latex ($Revision: 2.2 $$Date: 1992/04/27 15:13:26 $ by C. Engel)
% on Fri Feb  4 14:35:48 1994
% tr2latex was written by Kamal Al-Yahya at Stanford University
% (Kamal%Hanauma@SU-SCORE.ARPA)
% and substantially enhanced by Christian Engel at RWTH Aachen
% (krischan@informatik.rwth-aachen.de).
%
% troff input file: pgmkernel.1

\newpage
%--------------------------------------------------
% start of input file: pgmkernel.1
%
\phead{pgmkernel}{1}{10 December 1992}{}{}

%.IX pgmkernel
\shead{NAME}
pgmkernel - generate a convolution kernel
\shead{SYNOPSIS}
{\bf pgmkernel} [{\bf --weight}{\it w}] {\it width} [{\it height}]
\shead{DESCRIPTION}
Generates a portable graymap array of size
{\it width} x {\it height} (or {\it width} x {\it width} if {\it height}
is not specified) to be used as a convolution file by
{\bf pnmconvol}{\rm .}
The data in the convolution array K are computed according to the
formula:
\par
K(i,j) = 1 / ( 1 + w * sqrt((i-width/2)\^{}2 + (j-height/2)\^{}2)) 
\par
where 
{\it w}
is a coefficient specified via the 
{\it --weight}
flag, and
{\it width}
and 
{\it height}
are the X and Y filter sizes.
\par
The output PGM file is always written out in ASCII format.
\shead{OPTIONS}
The optional 
{\it -weight}
flag should be a real number greater than -1.
The default value is 6.0.
\shead{BUGS}
The computation time is proportional to 
{\it width}
* 
{\it height}{\rm .}
This increases rapidly with the increase of the kernel size.  
A better approach could be using a FFT in these cases.
\shead{SEE ALSO}
pnmconvol(1), pnmsmooth(1)
\shead{AUTHOR}
Alberto Accomazzi (alberto@cfa.harvard.edu).
%
% end of input file: pgmkernel.1
%--------------------------------------------------

% -*-LaTeX-*-
% Converted automatically from troff to LaTeX
% by tr2latex ($Revision: 2.2 $$Date: 1992/04/27 15:13:26 $ by C. Engel)
% on Mon Nov 29 13:21:08 1993
% tr2latex was written by Kamal Al-Yahya at Stanford University
% (Kamal%Hanauma@SU-SCORE.ARPA)
% and substantially enhanced by Christian Engel at RWTH Aachen
% (krischan@informatik.rwth-aachen.de).
%
% troff input file: pgmnoise.1

\newpage
%--------------------------------------------------
% start of input file: pgmnoise.1
%
\phead{pgmnoise}{1}{16 November 1993}{}{}

%.IX pgmnoise
\shead{NAME}
pgmnoise - create a graymap made up of white noise
\shead{SYNOPSIS}
{\bf pgmnoise}
{\it width height}
\shead{DESCRIPTION}
Creates a portable graymap that is made up of random pixels with
gray values in the range of 0 to PGM\_MAXMAXVAL (depends on the compilation,
either 255 or 65535). The graymap has a size of width * height pixels.
\shead{SEE ALSO}
pgm(5)
\shead{AUTHOR}
\copyright 1993 by Frank Neumann.
% Permission to use, copy, modify, and distribute this software and its
% documentation for any purpose and without fee is hereby granted, provided
% that the above copyright notice appear in all copies and that both that
% copyright notice and this permission notice appear in supporting
% documentation.  This software is provided "as is" without express or
% implied warranty.
%
% end of input file: pgmnoise.1
%--------------------------------------------------

% -*-LaTeX-*-
% Converted automatically from troff to LaTeX
% by tr2latex ($Revision: 2.2 $$Date: 1992/04/27 15:13:26 $ by C. Engel)
% on Fri Oct  8 16:08:29 1993
% tr2latex was written by Kamal Al-Yahya at Stanford University
% (Kamal%Hanauma@SU-SCORE.ARPA)
% and substantially enhanced by Christian Engel at RWTH Aachen
% (krischan@informatik.rwth-aachen.de).
%
% troff input file: pgmnorm.1

\newpage
%--------------------------------------------------
% start of input file: pgmnorm.1
%
\phead{pgmnorm}{1}{28 February 1989}{}{}

%.IX pgmnorm
\shead{NAME}
pgmnorm - normalize the contrast in a portable graymap
\shead{SYNOPSIS}
{\bf pgmnorm}
{\rm [}{\bf -bpercent}
{\it N}
$|$
{\bf -bvalue}
{\it N}{\rm ]}
{\rm [}{\bf -wpercent}
{\it N}
$|$
{\bf -wvalue}
{\it N}{\rm ]}
{\rm [}{\it pgmfile}{\rm ]}
\shead{DESCRIPTION}
Reads a portable graymap as input.
Normalizes the contrast by forcing the lightest pixels to white, the
%.IX "contrast normalization"
darkest pixels to black, and linearly rescaling the ones in between;
and produces a portable graymap as output.
\shead{OPTIONS}
\par
By default, the darkest 2 percent of all pixels are mapped to black, and
the lightest 1 percent are mapped to white.
You can override these percentages by using the
{\bf -bpercent}
and
{\bf -wpercent}
flags,
or you can specify the exact pixel values to be mapped by using the
{\bf -bvalue}
and
{\bf -wvalue}
flags.
Appropriate numbers for the flags can be gotten from the
{\it pgmhist}
tool.
%.IX pgmhist
If you just want to enhance the contrast, then choose values at elbows in the
histogram; {\it e.g.}, if value 29 represents 3\% of the image but value 30
represents 20\%, choose 30 for
{\it bvalue}{\rm .}
If you want to lighten the
image, then set
{\it bvalue}
to 0 and just fiddle with
{\it wvalue}{\rm ;}
similarly, to darken the image, set
{\it wvalue}
to maxval and play with
{\it bvalue}{\rm .}
\par
All flags can be abbreviated to their shortest unique prefix.
\shead{SEE ALSO}
pgmhist(1), pgm(5)
\shead{AUTHOR}
Partially based on the fbnorm filter in Michael Mauldin's ``Fuzzy
Pixmap''
package.

\copyright 1989 by Jef Poskanzer.
% Permission to use, copy, modify, and distribute this software and its
% documentation for any purpose and without fee is hereby granted, provided
% that the above copyright notice appear in all copies and that both that
% copyright notice and this permission notice appear in supporting
% documentation.  This software is provided "as is" without express or
% implied warranty.
%
% end of input file: pgmnorm.1
%--------------------------------------------------
 
% -*-LaTeX-*-
% Converted automatically from troff to LaTeX
% by tr2latex ($Revision: 2.2 $$Date: 1992/04/27 15:13:26 $ by C. Engel)
% on Fri Oct  8 16:08:30 1993
% tr2latex was written by Kamal Al-Yahya at Stanford University
% (Kamal%Hanauma@SU-SCORE.ARPA)
% and substantially enhanced by Christian Engel at RWTH Aachen
% (krischan@informatik.rwth-aachen.de).
%
% troff input file: pgmoil.1

\newpage
%--------------------------------------------------
% start of input file: pgmoil.1
%
\phead{pgmoil}{1}{11 January 1991}{}{}

%.IX pgmoil
\shead{NAME}
pgmoil - turn a portable graymap into an oil painting
\shead{SYNOPSIS}
{\bf pgmoil}
{\rm [}{\bf -n}
{\it N}{\rm ]}
{\rm [}{\it pgmfile}{\rm ]}
\shead{DESCRIPTION}
Reads a portable graymap as input.
Does an ``oil transfer'', and writes a portable graymap as output.
%.IX "oil transfer"
\par
The oil transfer is described in ``Beyond Photography'' by Holzmann,
chapter 4, photo 7.
It's a sort of localized smearing.
\shead{OPTIONS}
\par
The optional
{\bf -n}
flag controls the size of the area smeared.  The default value is 3.
\shead{BUGS}
Takes a long time to run.
\shead{SEE ALSO}
pgmbentley(1), ppmrelief(1), pgm(5)
\shead{AUTHOR}
\copyright 1990 by Wilson Bent (whb@hoh-2.att.com).
% Permission to use, copy, modify, and distribute this software and its
% documentation for any purpose and without fee is hereby granted, provided
% that the above copyright notice appear in all copies and that both that
% copyright notice and this permission notice appear in supporting
% documentation.  This software is provided "as is" without express or
% implied warranty.
%
% end of input file: pgmoil.1
%--------------------------------------------------
 
% -*-LaTeX-*-
% Converted automatically from troff to LaTeX
% by tr2latex ($Revision: 2.2 $$Date: 1992/04/27 15:13:26 $ by C. Engel)
% on Fri Oct  8 16:08:30 1993
% tr2latex was written by Kamal Al-Yahya at Stanford University
% (Kamal%Hanauma@SU-SCORE.ARPA)
% and substantially enhanced by Christian Engel at RWTH Aachen
% (krischan@informatik.rwth-aachen.de).
%
% troff input file: pgmramp.1

\newpage
%--------------------------------------------------
% start of input file: pgmramp.1
%
\phead{pgmramp}{1}{24 November 1989}{}{}

%.IX pgmramp
\shead{NAME}
pgmramp - generate a grayscale ramp
\shead{SYNOPSIS}
{\bf pgmramp}
{\bf -lr}{\rm $|$}{\bf -tb}
$|$
{\bf -rectangle}{\rm $|$}{\bf -ellipse}
{\it width height}
\shead{DESCRIPTION}
Generates a graymap of the specified size containing a black-to-white ramp.
%.IX "generating graymaps"
These ramps are useful for multiplying with other images, using the
{\it pnmarith}
tool.
%.IX pnmarith
\shead{OPTIONS}
\begin{TPlist}{{\bf -lr}}
\item[{{\bf -lr}}]
A left to right ramp.
\item[{{\bf -tb}}]
A top to bottom ramp.
\item[{{\bf -rectangle}}]
A rectangular ramp.
\item[{{\bf -ellipse}}]
An elliptical ramp.
\end{TPlist}

\par
All flags can be abbreviated to their shortest unique prefix.
\shead{SEE ALSO}
pnmarith(1), pgm(5)
\shead{AUTHOR}
\copyright 1989 by Jef Poskanzer.
% Permission to use, copy, modify, and distribute this software and its
% documentation for any purpose and without fee is hereby granted, provided
% that the above copyright notice appear in all copies and that both that
% copyright notice and this permission notice appear in supporting
% documentation.  This software is provided "as is" without express or
% implied warranty.
%
% end of input file: pgmramp.1
%--------------------------------------------------
 
% -*-LaTeX-*-
% Converted automatically from troff to LaTeX
% by tr2latex ($Revision: 2.2 $$Date: 1992/04/27 15:13:26 $ by C. Engel)
% on Fri Oct  8 16:08:35 1993
% tr2latex was written by Kamal Al-Yahya at Stanford University
% (Kamal%Hanauma@SU-SCORE.ARPA)
% and substantially enhanced by Christian Engel at RWTH Aachen
% (krischan@informatik.rwth-aachen.de).
%
% troff input file: pgmtexture.1

\newpage
%--------------------------------------------------
% start of input file: pgmtexture.1
%
\phead{pgmtexture}{1}{22 Aug 1991}{}{}

%.IX pgmtexture
\shead{NAME}
pgmtexture - calculate textural features on a portable graymap
\shead{SYNOPSIS}
{\bf pgmtexture}
{\rm [}{\bf -d}
{\it d}{\rm ]}
{\rm [}{\it pgmfile}{\rm ]}
\shead{DESCRIPTION}
Reads a portable graymap as input.  Calculates textural features
based on spatial dependence matrices at 0, 45, 90, and 135 degrees for
a distance 
{\it d}
(default = 1).
Textural features include:
\begin{IPlist}
\IPitem{{}}
(1) Angular Second Moment,
\nwl
(2) Contrast,
\nwl
(3) Correlation,
\nwl
(4) Variance,          
\nwl
(5) Inverse Difference Moment,
\nwl
(6) Sum Average,
\nwl
(7) Sum Variance,
\nwl
(8) Sum Entropy,
\nwl
(9) Entropy,
\nwl
(10) Difference Variance,
\nwl
(11) Difference Entropy,
\nwl
(12, 13) Information Measures of Correlation, and
\nwl
(14) Maximal Correlation Coefficient.
\end{IPlist}

\par
Algorithm taken from:
\nwl
Haralick, R.M., K. Shanmugam, and I. Dinstein. 1973. Textural features
for image classification.  
{\it IEEE Transactions on Systems, Man,}
{\it and Cybertinetics,}
SMC-3(6):610-621.
\shead{BUGS}
The program can run incredibly slow for large images (larger than 64 x 64)
and command line options are limited.
The method for finding (14) the maximal correlation coefficient, which
requires finding the second largest eigenvalue of a matrix Q, does not
always converge.
\shead{REFERENCES}
{\it IEEE Transactions on Systems, Man,}
{\it and Cybertinetics,}
SMC-3(6):610-621.
\shead{SEE ALSO}
pgm(5), pnmcut(1)
\shead{AUTHOR}
\copyright 1991 by Texas Agricultural Experiment Station, employer for
hire of James Darrell McCauley. 
% Permission to use, copy, modify, and distribute this software and its
% documentation for any purpose and without fee is hereby granted, provided
% that the above copyright notice appear in all copies and that both that
% copyright notice and this permission notice appear in supporting
% documentation.  This software is provided "as is" without express or
% implied warranty.
%
% THE TEXAS AGRICULTURAL EXPERIMENT STATION (TAES) AND THE TEXAS A&M
% UNIVERSITY SYSTEM (TAMUS) MAKE NO EXPRESS OR IMPLIED WARRANTIES
% (INCLUDING BY WAY OF EXAMPLE, MERCHANTABILITY) WITH RESPECT TO ANY
% ITEM, AND SHALL NOT BE LIABLE FOR ANY DIRECT, INDIRECT, INCIDENTAL
% OR CONSEQUENTAL DAMAGES ARISING OUT OF THE POSESSION OR USE OF
% ANY SUCH ITEM. LICENSEE AND/OR USER AGREES TO INDEMNIFY AND HOLD
% TAES AND TAMUS HARMLESS FROM ANY CLAIMS ARISING OUT OF THE USE OR
% POSSESSION OF SUCH ITEMS.
%
% end of input file: pgmtexture.1
%--------------------------------------------------
 
% -*-LaTeX-*-
% Converted automatically from troff to LaTeX
% by tr2latex ($Revision: 2.2 $$Date: 1992/04/27 15:13:26 $ by C. Engel)
% on Fri Oct  8 16:08:31 1993
% tr2latex was written by Kamal Al-Yahya at Stanford University
% (Kamal%Hanauma@SU-SCORE.ARPA)
% and substantially enhanced by Christian Engel at RWTH Aachen
% (krischan@informatik.rwth-aachen.de).
%
% troff input file: pgmtofits.1

\newpage
%--------------------------------------------------
% start of input file: pgmtofits.1
%
\phead{pgmtofits}{1}{20 September 1989}{}{}

%.IX pgmtofits
\shead{NAME}
pgmtofits - convert a portable graymap into FITS format
\shead{SYNOPSIS}
{\bf pgmtofits}
{\rm [}{\it pgmfile}{\rm ]}
\shead{DESCRIPTION}
Reads a portable graymap as input.
Produces a FITS file as output.
%.IX FITS
\par
FITS stands for Flexible Image Transport System.  A full description
can be found in Astronomy \& Astrophysics Supplement Series 44 (1981),
page 363.
\shead{SEE ALSO}
fitstopgm(1), pgm(5)
\shead{AUTHOR}
\copyright 1989 by Wilson H. Bent (whb@hoh-2.att.com).
% Permission to use, copy, modify, and distribute this software and its
% documentation for any purpose and without fee is hereby granted, provided
% that the above copyright notice appear in all copies and that both that
% copyright notice and this permission notice appear in supporting
% documentation.  This software is provided "as is" without express or
% implied warranty.
%
% end of input file: pgmtofits.1
%--------------------------------------------------
 
% -*-LaTeX-*-
% Converted automatically from troff to LaTeX
% by tr2latex ($Revision: 2.2 $$Date: 1992/04/27 15:13:26 $ by C. Engel)
% on Fri Oct  8 16:08:31 1993
% tr2latex was written by Kamal Al-Yahya at Stanford University
% (Kamal%Hanauma@SU-SCORE.ARPA)
% and substantially enhanced by Christian Engel at RWTH Aachen
% (krischan@informatik.rwth-aachen.de).
%
% troff input file: pgmtofs.1

\newpage
%--------------------------------------------------
% start of input file: pgmtofs.1
%
\phead{pgmtofs}{1}{18 May 1990}{}{}

%.IX pgmtofs
\shead{NAME}
pgmtofs - convert portable graymap to Usenix FaceSaver(tm) format
\shead{SYNOPSIS}
{\bf pgmtofs}
{\rm [}{\it pgmfile}{\rm ]}
\shead{DESCRIPTION}
Reads a portable graymap as input.
Produces Usenix FaceSaver(tm) format as output.
%.IX FaceSaver
\par
FaceSaver is a registered trademark of Metron Computerware Ltd. of
Oakland, CA.
\shead{SEE ALSO}
fstopgm(1), pgm(5)
\shead{AUTHOR}
\copyright 1991 by Jef Poskanzer.
% Permission to use, copy, modify, and distribute this software and its
% documentation for any purpose and without fee is hereby granted, provided
% that the above copyright notice appear in all copies and that both that
% copyright notice and this permission notice appear in supporting
% documentation.  This software is provided "as is" without express or
% implied warranty.
%
% end of input file: pgmtofs.1
%--------------------------------------------------
 
% -*-LaTeX-*-
% Converted automatically from troff to LaTeX
% by tr2latex ($Revision: 2.2 $$Date: 1992/04/27 15:13:26 $ by C. Engel)
% on Fri Oct  8 16:08:32 1993
% tr2latex was written by Kamal Al-Yahya at Stanford University
% (Kamal%Hanauma@SU-SCORE.ARPA)
% and substantially enhanced by Christian Engel at RWTH Aachen
% (krischan@informatik.rwth-aachen.de).
%
% troff input file: pgmtolispm.1

\newpage
%--------------------------------------------------
% start of input file: pgmtolispm.1
%
\phead{pgmtolispm}{1}{06 March 1990}{}{}

%.IX pgmtolispm
\shead{NAME}
pgmtolispm - convert a portable graymap into Lisp Machine format
\shead{SYNOPSIS}
{\bf pgmtolispm}
{\rm [}{\it pgmfile}{\rm ]}
\shead{DESCRIPTION}
Reads a portable graymap as input.
Produces a Lisp Machine bitmap as output.
%.IX "Lisp Machine bitmap"
\par
This is the file format read by the tv:read-bit-array-file function on
TI Explorer and Symbolics lisp machines.
\par
Given a pgm (instead of a pbm) a multi-plane image will be output.
This is probably not useful unless you have a color lisp machine.
\par
Multi-plane bitmaps on lisp machines are color; but the lispm image file
format does not include a color map, so we must treat it as a graymap 
instead.  This is unfortunate.
\shead{SEE ALSO}
lispmtopgm(1), pgm(5)
\shead{BUGS}
Output width is always rounded up to the nearest multiple of 32; this might 
not always be what you want, but it probably is (arrays which are not 
modulo 32 cannot be passed to the Lispm BITBLT function, and thus cannot 
easily be displayed on the screen).
\par
No color.
\shead{AUTHOR}
\copyright 1991 by Jamie Zawinski and Jef Poskanzer.
% Permission to use, copy, modify, and distribute this software and its
% documentation for any purpose and without fee is hereby granted, provided
% that the above copyright notice appear in all copies and that both that
% copyright notice and this permission notice appear in supporting
% documentation.  This software is provided "as is" without express or
% implied warranty.
%
% end of input file: pgmtolispm.1
%--------------------------------------------------
 
% -*-LaTeX-*-
% Converted automatically from troff to LaTeX
% by tr2latex ($Revision: 2.2 $$Date: 1992/04/27 15:13:26 $ by C. Engel)
% on Fri Oct  8 16:08:33 1993
% tr2latex was written by Kamal Al-Yahya at Stanford University
% (Kamal%Hanauma@SU-SCORE.ARPA)
% and substantially enhanced by Christian Engel at RWTH Aachen
% (krischan@informatik.rwth-aachen.de).
%
% troff input file: pgmtopbm.1

\newpage
%--------------------------------------------------
% start of input file: pgmtopbm.1
%
\phead{pgmtopbm}{1}{26 July 1988}{}{}

%.IX pgmtopbm
\shead{NAME}
pgmtopbm - convert a portable graymap into a portable bitmap
\shead{SYNOPSIS}
{\bf pgmtopbm}
{\rm [}{\bf -floyd}{\rm $|$}{\bf -fs}{\rm $|$}{\bf -threshold}
{\rm $|$}{\bf -dither8}{\rm $|$}{\bf -d8}{\rm $|$}{\bf -cluster3}
{\rm $|$}{\bf -c3}{\rm $|$}{\bf -cluster4}{\rm $|$}{\bf -c4}
{\rm $|$}{\bf -cluster8}{\rm $|$}{\bf -c8}{\rm ]}
{\rm [}{\bf -value}
{\it val}{\rm ]}
{\rm [}{\it pgmfile}{\rm ]}
\shead{DESCRIPTION}
Reads a portable graymap as input.
Produces a portable bitmap as output.
%.IX halftoning
\par
Note that there is no pbmtopgm converter, because any pgm program can
read pbm files automagically.
\shead{OPTIONS}
\par
The default quantization method is boustrophedonic Floyd-Steinberg error
diffusion
{\rm (}{\bf -floyd}
or
{\bf -fs}{\rm ).}
%.IX Floyd-Steinberg
%.IX "error diffusion"
Also available are simple thresholding
{\rm (}{\bf -threshold}{\rm );}
%.IX thresholding
Bayer's ordered dither
{\rm (}{\bf -dither8}{\rm )}
with a 16x16 matrix; and three different sizes of 45-degree clustered-dot dither
{\rm (}{\bf -cluster3}{\rm ,}
{\bf -cluster4}{\rm ,}
{\bf -cluster8}{\rm ).}
%.IX dithering
\par
Floyd-Steinberg will almost always give the best looking results; however,
looking good is not always what you want.
For instance, thresholding can be used in a pipeline with the
{\it pnmconvol}
%.IX pnmconvol
tool, for tasks like edge and peak detection.
And clustered-dot dithering gives a newspaper-ish look, a useful special effect.
\par
The
{\bf -value}
flag alters the thresholding value for Floyd-Steinberg and
simple thresholding.
It should be a real number between 0 and 1.
Above 0.5 means darker images; below 0.5 means lighter.
\par
All flags can be abbreviated to their shortest unique prefix.
\shead{REFERENCES}
The only reference you need for this stuff is ``Digital Halftoning'' by
Robert Ulichney, MIT Press, ISBN 0--262--21009--6.
\shead{SEE ALSO}
pbmreduce(1), pgm(5), pbm(5), pnmconvol(1)
\shead{AUTHOR}
\copyright 1989 by Jef Poskanzer.
% Permission to use, copy, modify, and distribute this software and its
% documentation for any purpose and without fee is hereby granted, provided
% that the above copyright notice appear in all copies and that both that
% copyright notice and this permission notice appear in supporting
% documentation.  This software is provided "as is" without express or
% implied warranty.
%
% end of input file: pgmtopbm.1
%--------------------------------------------------
 
% -*-LaTeX-*-
% Converted automatically from troff to LaTeX
% by tr2latex ($Revision: 2.2 $$Date: 1992/04/27 15:13:26 $ by C. Engel)
% on Fri Oct  8 16:08:43 1993
% tr2latex was written by Kamal Al-Yahya at Stanford University
% (Kamal%Hanauma@SU-SCORE.ARPA)
% and substantially enhanced by Christian Engel at RWTH Aachen
% (krischan@informatik.rwth-aachen.de).
%
% troff input file: pgmtoppm.1

\newpage
%--------------------------------------------------
% start of input file: pgmtoppm.1
%
\phead{pgmtoppm}{1}{11 January 1991}{}{}

%.IX pgmtoppm
\shead{NAME}
pgmtoppm - colorize a portable graymap into a portable pixmap
\shead{SYNOPSIS}
{\bf pgmtoppm}
{\it colorspec}
{\rm [}{\it pgmfile}{\rm ]}
\nwl
{\bf pgmtoppm}
{\it colorspec1}{\bf -}{\it colorspec2}
{\rm [}{\it pgmfile}{\rm ]}
\nwl
{\bf pgmtoppm -map}
{\it mapfile}
{\rm [}{\it pgmfile}{\rm ]}
\shead{DESCRIPTION}
Reads a portable graymap as input.
Colorizes it by multiplying the the gray values by specified color or colors,
and produces a portable pixmap as output.
%.IX colorization
\par
If only one color is specified, black in the pgm file stays black and
white in the pgm file turns into the specified color in the ppm file.
If two colors (separated by a dash) are specified, then black gets mapped
to the first color and white gets mapped to the second.
\par
The color can be specified in five ways:
%.IX "specifying colors"
\begin{TPlist}{o}
\item[{o}]
A name, assuming
that a pointer to an X11-style color names file was compiled in.
\item[{o}]
An X11-style hexadecimal specifier: rgb:r/g/b, where r g and b are
each 1- to 4-digit hexadecimal numbers.
\item[{o}]
An X11-style decimal specifier: rgbi:r/g/b, where r g and b are
floating point numbers between 0 and 1.
\item[{o}]
For backwards compatibility, an old-X11-style hexadecimal
number: \#rgb, \#rrggbb, \#rrrgggbbb, or \#rrrrggggbbbb.
\item[{o}]
For backwards compatibility, a triplet of numbers
separated by commas: r,g,b, where r g and b are
floating point numbers between 0 and 1.
(This style was added before MIT came up with the similar rgbi style.)
\end{TPlist}

\par
Also, the
{\bf -map}
flag lets you specify an entire colormap to be used.
The mapfile is just a
{\it ppm}
file; it can be any shape, all that matters
is the colors in it and their order.
In this case, black gets mapped into the first
color in the map file, and white gets mapped to the last.
\shead{SEE ALSO}
rgb3toppm(1), ppmtopgm(1), ppmtorgb3(1), ppm(5), pgm(5)
\shead{AUTHOR}
\copyright 1991 by Jef Poskanzer.
% Permission to use, copy, modify, and distribute this software and its
% documentation for any purpose and without fee is hereby granted, provided
% that the above copyright notice appear in all copies and that both that
% copyright notice and this permission notice appear in supporting
% documentation.  This software is provided "as is" without express or
% implied warranty.
%
% end of input file: pgmtoppm.1
%--------------------------------------------------
 
% -*-LaTeX-*-
% Converted automatically from troff to LaTeX
% by tr2latex ($Revision: 2.2 $$Date: 1992/04/27 15:13:26 $ by C. Engel)
% on Fri Oct  8 16:08:43 1993
% tr2latex was written by Kamal Al-Yahya at Stanford University
% (Kamal%Hanauma@SU-SCORE.ARPA)
% and substantially enhanced by Christian Engel at RWTH Aachen
% (krischan@informatik.rwth-aachen.de).
%
% troff input file: pi1toppm.1

\newpage
%--------------------------------------------------
% start of input file: pi1toppm.1
%
\phead{pi1toppm}{1}{19 July 1990}{}{}

%.IX pi1toppm
\shead{NAME}
pi1toppm - convert an Atari Degas .pi1 into a portable pixmap
\shead{SYNOPSIS}
{\bf pi1toppm}
{\rm [}{\it pi1file}{\rm ]}
\shead{DESCRIPTION}
Reads an Atari Degas .pi1 file as input.
%.IX Atari
%.IX "Degas .pi1"
Produces a portable pixmap as output.
\shead{SEE ALSO}
ppmtopi1(1), ppm(5), pi3topbm(1), pbmtopi3(1)
\shead{AUTHOR}
\copyright 1991 by Steve Belczyk (seb3@gte.com) and Jef Poskanzer.
% Permission to use, copy, modify and distribute this software and its
% documentation for any purpose and without fee is hereby granted, provided
% that the above copyright notice appear in all copies and that both that
% copyright notice and this permission notice appear in supporting
% documentation.  This software is provided "as is" without express or
% implied warranty.
%
% end of input file: pi1toppm.1
%--------------------------------------------------
 
% -*-LaTeX-*-
% Converted automatically from troff to LaTeX
% by tr2latex ($Revision: 2.2 $$Date: 1992/04/27 15:13:26 $ by C. Engel)
% on Fri Oct  8 16:08:13 1993
% tr2latex was written by Kamal Al-Yahya at Stanford University
% (Kamal%Hanauma@SU-SCORE.ARPA)
% and substantially enhanced by Christian Engel at RWTH Aachen
% (krischan@informatik.rwth-aachen.de).
%
% troff input file: pi3topbm.1

\newpage
%--------------------------------------------------
% start of input file: pi3topbm.1
%
\phead{pi3topbm}{1}{11 March 1990}{}{}

%.IX pi3topbm
\shead{NAME}
pi3topbm - convert an Atari Degas .pi3 file into a portable bitmap
\shead{SYNOPSIS}
{\bf pi3topbm}
{\rm [}{\it pi3file}{\rm ]}
\shead{DESCRIPTION}
Reads an Atari Degas .pi3 file as input.
%.IX Atari
%.IX "Degas .pi3"
Produces a portable bitmap as output.
\shead{SEE ALSO}
pbmtopi3(1), pbm(5), pi1toppm(1), ppmtopi1(1)
\shead{AUTHOR}
\copyright 1988 by David Beckemeyer (bdt!david) and Diomidis D. Spinellis.
% Permission to use, copy, modify, and distribute this software and its
% documentation for any purpose and without fee is hereby granted,
% provided that the above copyright notice appear in all copies and that
% both that copyright notice and this permission notice appear in
% supporting documentation.
%
% end of input file: pi3topbm.1
%--------------------------------------------------
 
% -*-LaTeX-*-
% Converted automatically from troff to LaTeX
% by tr2latex ($Revision: 2.2 $$Date: 1992/04/27 15:13:26 $ by C. Engel)
% on Fri Oct  8 16:08:44 1993
% tr2latex was written by Kamal Al-Yahya at Stanford University
% (Kamal%Hanauma@SU-SCORE.ARPA)
% and substantially enhanced by Christian Engel at RWTH Aachen
% (krischan@informatik.rwth-aachen.de).
%
% troff input file: picttoppm.1

\newpage
%--------------------------------------------------
% start of input file: picttoppm.1
%
\phead{picttoppm}{1}{29 November 1991}{}{}

%.IX picttoppm
\shead{NAME}
picttoppm - convert a Macintosh PICT file into a portable pixmap
\shead{SYNOPSIS}
{\bf picttoppm}
{\rm [}{\bf -verbose}{\rm ]}
{\rm [}{\bf -fullres}{\rm ]}
{\rm [}{\bf -noheader}{\rm ]}
{\rm [}{\it pictfile}{\rm ]}
\shead{DESCRIPTION}
Reads a PICT file (version 1 or 2) and outputs a portable pixmap.
%.IX PICT
%.IX Macintosh
Useful as the first step in converting a scanned image to something
that can be displayed on Unix.
\shead{OPTIONS}
\begin{TPlist}{{\bf --fullres}}
\item[{{\bf --fullres}}]
Force any images in the PICT file to be output with at least their
full resolution.  A PICT file may indicate that a contained
image is to be scaled down before output.  This option forces images
to retain their sizes and prevent information loss.
\item[{{\bf --noheader}}]
Do not skip the 512 byte header that is present on all PICT files.
This is useful when you have PICT data that was not stored in
the data fork of a PICT file.
\item[{{\bf --verbose}}]
Turns on verbose mode which prints a 
a whole bunch of information that only
{\it picttoppm}
hackers really care about.
\end{TPlist}

\shead{BUGS}
The PICT file format is a general drawing format.
{\it picttoppm}
only supports a small subset of its operations but is still very useful for
files produced by scanning software.  In particular, text added to a
scanned image will be silently ignored.
\shead{SEE ALSO}
Inside Macintosh volume 5,
ppmtopict(1),
ppm(5)
\shead{AUTHOR}
\copyright 1989 George Phillips (phillips@cs.ubc.ca).
% Permission is granted to freely distribute this program in whole or in
% part provided you don't make money off it, you don't pretend that you
% wrote it and that this notice accompanies the code.
%
% George Phillips <phillips@cs.ubc.ca>
% Department of Computer Science
% University of British Columbia
%
% end of input file: picttoppm.1
%--------------------------------------------------
 
% -*-LaTeX-*-
% Converted automatically from troff to LaTeX
% by tr2latex ($Revision: 2.2 $$Date: 1992/04/27 15:13:26 $ by C. Engel)
% on Fri Oct  8 16:08:44 1993
% tr2latex was written by Kamal Al-Yahya at Stanford University
% (Kamal%Hanauma@SU-SCORE.ARPA)
% and substantially enhanced by Christian Engel at RWTH Aachen
% (krischan@informatik.rwth-aachen.de).
%
% troff input file: pjtoppm.1

\newpage
%--------------------------------------------------
% start of input file: pjtoppm.1
%
\phead{pjtoppm}{1}{14 July 1991}{}{}

%.IX pjtoppm
\shead{NAME}
pjtoppm - convert an HP PaintJet file to a portable pixmap
\shead{SYNOPSIS}
{\bf pjtoppm}
{\rm [}{\it paintjet}{\rm ]}
\shead{DESCRIPTION}
Reads an HP PaintJet file as input and converts it into a portable pixmap.
This was a quick hack to save some trees, and it only handles a small
subset of the paintjet commands.
In particular, it will only handle 
enough commands to convert most raster image files.
\shead{REFERENCES}
HP PaintJet XL Color Graphics Printer User's Guide
\shead{SEE ALSO}
ppmtopj(1)
\shead{AUTHOR}
\copyright 1991 by Christos Zoulas.
% Permission to use, copy, modify, and distribute this software and its
% documentation for any purpose and without fee is hereby granted, provided
% that the above copyright notice appear in all copies and that both that
% copyright notice and this permission notice appear in supporting
% documentation.  This software is provided "as is" without express or
% implied warranty.
%
% end of input file: pjtoppm.1
%--------------------------------------------------
 
% -*-LaTeX-*-
% Converted automatically from troff to LaTeX
% by tr2latex ($Revision: 2.2 $$Date: 1992/04/27 15:13:26 $ by C. Engel)
% on Fri Oct  8 16:08:19 1993
% tr2latex was written by Kamal Al-Yahya at Stanford University
% (Kamal%Hanauma@SU-SCORE.ARPA)
% and substantially enhanced by Christian Engel at RWTH Aachen
% (krischan@informatik.rwth-aachen.de).
%
% troff input file: pktopbm.1

\newpage
%--------------------------------------------------
% start of input file: pktopbm.1
%
\phead{pktopbm}{1}{6 August 1990}{}{}

\shead{NAME}
pktopbm - convert packed (PK) format font into portable bitmap(s)
\shead{SYNOPSIS}
pktopbm pkfile[.pk] [-c num] pbmfile ...
\shead{DESCRIPTION}
Reads a packed (PK) font file as input, and produces portable bitmaps as
output. If the filename ``-'' is used for any
of the filenames, the standard input stream (or standard output where
appropriate) will be used.
\shead{OPTIONS}
\begin{IPlist}
\IPitem{{-c\ num}}
Sets the character number of the next bitmap written to num.
\end{IPlist}

\shead{SEE ALSO}
pbmtopk(1), pbm(5)
\shead{AUTHOR}
Adapted from Tom Rokicki's pxtopk by Angus Duggan (ajcd@dcs.ed.ac.uk.

%
% end of input file: pktopbm.1
%--------------------------------------------------
 
% -*-LaTeX-*-
% Converted automatically from troff to LaTeX
% by tr2latex ($Revision: 2.2 $$Date: 1992/04/27 15:13:26 $ by C. Engel)
% on Fri Oct  8 16:09:35 1993
% tr2latex was written by Kamal Al-Yahya at Stanford University
% (Kamal%Hanauma@SU-SCORE.ARPA)
% and substantially enhanced by Christian Engel at RWTH Aachen
% (krischan@informatik.rwth-aachen.de).
%
% troff input file: pnm.5

\newpage
%--------------------------------------------------
% start of input file: pnm.5
%
\phead{pnm}{5}{27 September 1991}{}{}

\shead{NAME}
pnm - portable anymap file format
\shead{DESCRIPTION}
The
{\it pnm}
programs operate on portable bitmaps, graymaps, and pixmaps, produced by the
{\it pbm, pgm,}
and
{\it ppm}
segments.  There is no file format associated with
{\it pnm}
itself.
\shead{SEE ALSO}
anytopnm(1), rasttopnm(1), tifftopnm(1), xwdtopnm(1),
pnmtops(1), pnmtorast(1), pnmtotiff(1), pnmtoxwd(1),
pnmarith(1), pnmcat(1), pnmconvol(1), pnmcrop(1), pnmcut(1),
pnmdepth(1), pnmenlarge(1), pnmfile(1), pnmflip(1), pnmgamma(1),
pnmindex(1), pnminvert(1), pnmmargin(1), pnmnoraw(1), pnmpaste(1),
pnmrotate(1), pnmscale(1), pnmshear(1), pnmsmooth(1), pnmtile(1),
ppm(5), pgm(5), pbm(5)
\shead{AUTHOR}
\copyright 1989, 1991 by Jef Poskanzer.
% Permission to use, copy, modify, and distribute this software and its
% documentation for any purpose and without fee is hereby granted, provided
% that the above copyright notice appear in all copies and that both that
% copyright notice and this permission notice appear in supporting
% documentation.  This software is provided "as is" without express or
% implied warranty.
%
% end of input file: pnm.5
%--------------------------------------------------
 
% -*-LaTeX-*-
% Converted automatically from troff to LaTeX
% by tr2latex ($Revision: 2.2 $$Date: 1992/04/27 15:13:26 $ by C. Engel)
% on Mon Feb  7 08:47:49 1994
% tr2latex was written by Kamal Al-Yahya at Stanford University
% (Kamal%Hanauma@SU-SCORE.ARPA)
% and substantially enhanced by Christian Engel at RWTH Aachen
% (krischan@informatik.rwth-aachen.de).
%
% troff input file: pnmalias.1

\newpage
%--------------------------------------------------
% start of input file: pnmalias.1
%
\phead{pnmalias}{1}{30 April 1992}{}{}

\shead{NAME}
pnmalias - antialias a portable anyumap.
\shead{SYNOPSIS}
{\bf pnmalias}
{\rm [}{\bf -bgcolor}
{\it color}{\rm ]}
{\rm [}{\bf -fgcolor}
{\it color}{\rm ]}
{\rm [}{\bf -bonly}{\rm ]}
{\rm [}{\bf -fonly}{\rm ]}
{\rm [}{\bf -balias}{\rm ]}
{\rm [}{\bf -falias}{\rm ]}
{\rm [}{\bf -weight}
{\it w}{\rm ]}
{\rm [}{\it pnmfile}{\rm ]}
\shead{DESCRIPTION}
Reads a portable anymap as input, and applies anti-aliasing to background and
foreground pixels.
If the input file is a portable bitmap, the 
output anti-aliased image is promoted to a graymap, and a message is printed
informing the user of the change in format.
\shead{OPTIONS}
\par
{\bf --bgcolor}
{\it colorb,}
{\bf --fgcolor}
{\it colorf}
\ind{1\parindent}set the background color to 
{\it colorb,}
and the foreground to color to
{\it colorf.}
Pixels with these values will be anti-aliased. by default,
the background color is taken to be black, and foreground color
is assumed to be white.  
The colors can be specified in five ways:
\begin{TPlist}{{\bf o}}
\item[{{\bf o}}]
A name, assuming
that a pointer to an X11-style color names file was compiled in.
\item[{{\bf o}}]
An X11-style hexadecimal specifier: rgb:r/g/b, where r g and b are
each 1- to 4-digit hexadecimal numbers.
\item[{{\bf o}}]
An X11-style decimal specifier: rgbi:r/g/b, where r g and b are
floating point numbers between 0 and 1.
\item[{{\bf o}}]
For backwards compatibility, an old-X11-style hexadecimal
number: \#rgb, \#rrggbb, \#rrrgggbbb, or \#rrrrggggbbbb.
\item[{{\bf o}}]
For backwards compatibility, a triplet of numbers
separated by commas: r,g,b, where r g and b are
floating point numbers between 0 and 1.
(This style was added before MIT came up with the similar rgbi style.)
\end{TPlist}
\par\noindent
Note that even when dealing with graymaps, background and foreground
colors need to be specified in the fashion described above.
In this case, background and foreground pixel values are taken to be the
value of the red component for the given color.
\ind{1\parindent}
\par
{\bf --bonly}{\rm ,}
{\bf --fonly}
\ind{1\parindent}Apply anti-aliasing only to background 
{\rm (}{\bf --bonly}{\rm ),}
or foreground
{\rm (}{\bf --fonly}{\rm )}
pixels.
\ind{1\parindent}
\par
{\bf --balias}{\rm ,}
{\bf --falias}
\ind{1\parindent}Apply anti-aliasing to all pixels surrounding background
{\rm (}{\bf --balias}{\rm ),}
or foreground
{\rm (}{\bf --falias}{\rm )}
pixels.  By default, anti-aliasing takes place only among neighboring
background and foreground pixels.
\ind{1\parindent}
\par
{\bf --weight}
{\it w}
\ind{1\parindent}Use 
{\it w}
as the central weight for the aliasing filter.
{\it W}
must be a real number in the range
0 $<$ 
{\it w}
$<$ 1.
The lower the value of 
{\it w}
is, the ``blurrier'' the output image is.  The default is w = 1/3.
\shead{SEE ALSO}
pbmtext(1), pnmsmooth(1), pnm(5)
\shead{AUTHOR}
Copyright (C) 1992 by Alberto Accomazzi, Smithsonian Astrophysical Observatory.
% Permission to use, copy, modify, and distribute this software and its
% documentation for any purpose and without fee is hereby granted, provided
% that the above copyright notice appear in all copies and that both that
% copyright notice and this permission notice appear in supporting
% documentation.  This software is provided "as is" without express or
% implied warranty.
%
% end of input file: pnmalias.1
%--------------------------------------------------

% -*-LaTeX-*-
% Converted automatically from troff to LaTeX
% by tr2latex ($Revision: 2.2 $$Date: 1992/04/27 15:13:26 $ by C. Engel)
% on Fri Oct  8 16:09:17 1993
% tr2latex was written by Kamal Al-Yahya at Stanford University
% (Kamal%Hanauma@SU-SCORE.ARPA)
% and substantially enhanced by Christian Engel at RWTH Aachen
% (krischan@informatik.rwth-aachen.de).
%
% troff input file: pnmarith.1

\newpage
%--------------------------------------------------
% start of input file: pnmarith.1
%
\phead{pnmarith}{1}{26 August 1993}{}{}

%.IX pnmarith
\shead{NAME}
pnmarith - perform arithmetic on two portable anymaps
\shead{SYNOPSIS}
{\bf pnmarith}
{\bf -add}{\rm $|$}{\bf -subtract}{\rm $|$}{\bf -multiply}{\rm $|$}{\bf -difference}
{\it pnmfile1 pnmfile2}
\shead{DESCRIPTION}
Reads two portable anymaps as input.
Performs the specified arithmetic operation,
and produces a portable anymap as output.
The two input anymaps must be the same width and height.
\par
The arithmetic is performed between corresponding pixels in the two
anymaps, as if maxval was 1.0, black was 0.0, and a linear scale in between.
Results that fall outside of [0..1) are truncated.
\par
The operator
{\it -difference}
calculates the absolute value of
{\it pnmarith -subtract pnmfile1 pnmfile2,}
{\it i.e.}, no truncation is done.
\par
All flags can be abbreviated to their shortest unique prefix.
\shead{SEE ALSO}
pbmmask(1), pnmpaste(1), pnminvert(1), pnm(5)
\shead{AUTHOR}
\copyright 1989, 1991 by Jef Poskanzer..
Lightly modified by Marcel Wijkstra (wijkstra@fwi.uva.nl).
% Permission to use, copy, modify, and distribute this software and its
% documentation for any purpose and without fee is hereby granted, provided
% that the above copyright notice appear in all copies and that both that
% copyright notice and this permission notice appear in supporting
% documentation.  This software is provided "as is" without express or
% implied warranty.
%
% end of input file: pnmarith.1
%--------------------------------------------------
 
% -*-LaTeX-*-
% Converted automatically from troff to LaTeX
% by tr2latex ($Revision: 2.2 $$Date: 1992/04/27 15:13:26 $ by C. Engel)
% on Fri Oct  8 16:09:17 1993
% tr2latex was written by Kamal Al-Yahya at Stanford University
% (Kamal%Hanauma@SU-SCORE.ARPA)
% and substantially enhanced by Christian Engel at RWTH Aachen
% (krischan@informatik.rwth-aachen.de).
%
% troff input file: pnmcat.1

\newpage
%--------------------------------------------------
% start of input file: pnmcat.1
%
\phead{pnmcat}{1}{12 March 1989}{}{}

%.IX pnmcat
\shead{NAME}
pnmcat - concatenate portable anymaps
\shead{SYNOPSIS}
{\bf pnmcat}
{\rm [}{\bf -white}{\rm $|$}{\bf -black}{\rm ]}
{\bf -leftright}{\rm $|$}{\bf -lr}
{\rm [}{\bf -jtop}{\rm $|$}{\bf -jbottom}{\rm ]}
{\it pnmfile pnmfile}
{\rm ...}
\nwl
{\bf pnmcat}
{\rm [}{\bf -white}{\rm $|$}{\bf -black}{\rm ]}
{\bf -topbottom}{\rm $|$}{\bf -tb}
{\rm [}{\bf -jleft}{\rm $|$}{\bf -jright}{\rm ]}
{\it pnmfile pnmfile}
{\rm ...}
\shead{DESCRIPTION}
Reads portable anymaps as input.
Concatenates them either left to right or top to bottom, and produces a
portable anymap as output.
%.IX concatenation
\shead{OPTIONS}
\par
If the anymaps are not all the same height (left-right) or width (top-bottom),
the smaller ones have to be justified with the largest.
By default, they get centered, but you can specify one side or the other
with one of the -j* flags.
So,
{\bf -topbottom -jleft}
would stack the anymaps on top of each other, flush with the left edge.
\par
The
{\bf -white}
and
{\bf -black}
flags specify what color to use to fill in the extra space
when doing this justification.
If neither is specified, the program makes a guess.
\par
All flags can be abbreviated to their shortest unique prefix.
\shead{SEE ALSO}
pnm(5)
\shead{AUTHOR}
\copyright 1989 by Jef Poskanzer.
% Permission to use, copy, modify, and distribute this software and its
% documentation for any purpose and without fee is hereby granted, provided
% that the above copyright notice appear in all copies and that both that
% copyright notice and this permission notice appear in supporting
% documentation.  This software is provided "as is" without express or
% implied warranty.
%
% end of input file: pnmcat.1
%--------------------------------------------------
 
% -*-LaTeX-*-
% Converted automatically from troff to LaTeX
% by tr2latex ($Revision: 2.2 $$Date: 1992/04/27 15:13:26 $ by C. Engel)
% on Fri Oct  8 16:09:31 1993
% tr2latex was written by Kamal Al-Yahya at Stanford University
% (Kamal%Hanauma@SU-SCORE.ARPA)
% and substantially enhanced by Christian Engel at RWTH Aachen
% (krischan@informatik.rwth-aachen.de).
%
% troff input file: pnmcomp.1

\newpage
%--------------------------------------------------
% start of input file: pnmcomp.1
%
\phead{pnmcomp}{1}{21 February 1989}{}{}

%.IX pnmcomp
\shead{NAME}
pnmcomp - composite two portable anymap files together
\shead{SYNOPSIS}
{\bf pnmcomp}
{\rm [}{\it -invert}{\rm ]}
{\rm [}{\it -xoff}{\rm N}{\it ]}
{\rm [}{\it -yoff}{\rm N}{\it ]}
{\rm [}{\it -alpha}{\rm pgmfile}{\it ]}
{\rm overlay}
{\rm [}{\it pnm-input}{\rm ]}
{\rm [}{\it pnm-output}{\rm ]}
\shead{DESCRIPTION}
Reads in a portable any map image and put a overlay upon it, with optional
alpha mask.  The 
{\it -alpha pgmfile}
allows you to also add an alpha mask file to the compositing process, the
range of max and min can be swapped by using the
{\it -invert}
option.
The
{\it -xoff}
and
{\it -yoff}
arguments can be negative, allowing you to shift the overlay off the
top corner of the screen.
\shead{SEE ALSO}
pnm(5)
\shead{AUTHOR}
\copyright 1992 by David Koblas (koblas@mips.com).
% Permission to use, copy, modify, and distribute this software and its
% documentation for any purpose and without fee is hereby granted, provided
% that the above copyright notice appear in all copies and that both that
% copyright notice and this permission notice appear in supporting
% documentation.  This software is provided "as is" without express or
% implied warranty.
%
% end of input file: pnmcomp.1
%--------------------------------------------------
 
% -*-LaTeX-*-
% Converted automatically from troff to LaTeX
% by tr2latex ($Revision: 2.2 $$Date: 1992/04/27 15:13:26 $ by C. Engel)
% on Fri Oct  8 16:09:18 1993
% tr2latex was written by Kamal Al-Yahya at Stanford University
% (Kamal%Hanauma@SU-SCORE.ARPA)
% and substantially enhanced by Christian Engel at RWTH Aachen
% (krischan@informatik.rwth-aachen.de).
%
% troff input file: pnmconvol.1

\newpage
%--------------------------------------------------
% start of input file: pnmconvol.1
%
\phead{pnmconvol}{1}{13 January 1991}{}{}

%.IX pnmconvol
\shead{NAME}
pnmconvol - general MxN convolution on a portable anymap
\shead{SYNOPSIS}
{\bf pnmconvol}
{\it convolutionfile}
{\rm [}{\it pnmfile}{\rm ]}
\shead{DESCRIPTION}
Reads two portable anymaps as input.
Convolves the second using the first,
and writes a portable anymap as output.
%.IX convolution
\par
Convolution means replacing each pixel with a weighted average of the
nearby pixels.  The weights and the area to average are determined by
the convolution matrix.
The unsigned numbers in the convolution file are offset by -maxval/2 to
make signed numbers, and then normalized, so the actual values in the
convolution file are only relative.
\par
Here is a sample convolution file;
it does a simple average of the nine immediate neighbors, resulting
in a smoothed image:
\nofill
    P2
    3 3
    18
    10 10 10
    10 10 10
    10 10 10
\fill
\par
To see how this works, do the above-mentioned offset: 10 - 18/2 gives 1.
The possible range of values is from 0 to 18, and after the offset
that's -9 to 9.  The normalization step makes the range -1 to 1, and
the values get scaled correspondingly so they become 1/9 - exactly what
you want.
The equivalent matrix for 5x5 smoothing would have maxval 50 and be
filled with 26.
\par
The convolution file will usually be a graymap,
so that the same convolution gets applied to each color component.
However, if you want to use a pixmap and do a different convolution to
different colors, you can certainly do that.
\shead{SEE ALSO}
pnmsmooth(1), pnm(5)
\shead{AUTHOR}
\copyright 1989, 1991 by Jef Poskanzer.
% Permission to use, copy, modify, and distribute this software and its
% documentation for any purpose and without fee is hereby granted, provided
% that the above copyright notice appear in all copies and that both that
% copyright notice and this permission notice appear in supporting
% documentation.  This software is provided "as is" without express or
% implied warranty.
%
% end of input file: pnmconvol.1
%--------------------------------------------------
 
% -*-LaTeX-*-
% Converted automatically from troff to LaTeX
% by tr2latex ($Revision: 2.2 $$Date: 1992/04/27 15:13:26 $ by C. Engel)
% on Fri Oct  8 16:09:18 1993
% tr2latex was written by Kamal Al-Yahya at Stanford University
% (Kamal%Hanauma@SU-SCORE.ARPA)
% and substantially enhanced by Christian Engel at RWTH Aachen
% (krischan@informatik.rwth-aachen.de).
%
% troff input file: pnmcrop.1

\newpage
%--------------------------------------------------
% start of input file: pnmcrop.1
%
\phead{pnmcrop}{1}{25 February 1989}{}{}

%.IX pnmcrop
\shead{NAME}
pnmcrop - crop a portable anymap
\shead{SYNOPSIS}
{\bf pnmcrop}
{\rm [}{\bf -white}{\rm $|$}{\bf -black}{\rm ]}
{\rm [}{\bf -left}{\rm ]}
{\rm [}{\bf -right}{\rm ]}
{\rm [}{\bf -top}{\rm ]}
{\rm [}{\bf -bottom}{\rm ]}
{\rm [}{\it pnmfile}{\rm ]}
\shead{DESCRIPTION}
Reads a portable anymap as input.
Removes edges that are the background color,
and produces a portable anymap as output.
%.IX cropping
\shead{OPTIONS}
\par
By default, it makes a guess as to what the background color is.
You can override the default with the
{\bf -white}
and
{\bf -black}
flags.
\par
The options
{\bf -left, -right, -top}
and
{\bf -bottom}
restrict cropping to the sides specified. The default is to crop all sides of
the image.
\par
All flags can be abbreviated to their shortest unique prefix.
\shead{SEE ALSO}
pnmcut(1), pnm(5)
\shead{AUTHOR}
\copyright 1989 by Jef Poskanzer.
% Permission to use, copy, modify, and distribute this software and its
% documentation for any purpose and without fee is hereby granted, provided
% that the above copyright notice appear in all copies and that both that
% copyright notice and this permission notice appear in supporting
% documentation.  This software is provided "as is" without express or
% implied warranty.
%
% end of input file: pnmcrop.1
%--------------------------------------------------
 
% -*-LaTeX-*-
% Converted automatically from troff to LaTeX
% by tr2latex ($Revision: 2.2 $$Date: 1992/04/27 15:13:26 $ by C. Engel)
% on Fri Oct  8 16:09:19 1993
% tr2latex was written by Kamal Al-Yahya at Stanford University
% (Kamal%Hanauma@SU-SCORE.ARPA)
% and substantially enhanced by Christian Engel at RWTH Aachen
% (krischan@informatik.rwth-aachen.de).
%
% troff input file: pnmcut.1

\newpage
%--------------------------------------------------
% start of input file: pnmcut.1
%
\phead{pnmcut}{1}{21 February 1989}{}{}

%.IX pnmcut
\shead{NAME}
pnmcut - cut a rectangle out of a portable anymap
\shead{SYNOPSIS}
{\bf pnmcut}
{\it x y width height}
{\rm [}{\it pnmfile}{\rm ]}
\shead{DESCRIPTION}
Reads a portable anymap as input.
Extracts the specified rectangle,
and produces a portable anymap as output.
%.IX cut
The
{\it x}
and
{\it y}
can be negative, in which case they are interpreted
relative to the right and bottom of the anymap, respectively.
\shead{SEE ALSO}
pnm(5)
\shead{AUTHOR}
\copyright 1989 by Jef Poskanzer.
% Permission to use, copy, modify, and distribute this software and its
% documentation for any purpose and without fee is hereby granted, provided
% that the above copyright notice appear in all copies and that both that
% copyright notice and this permission notice appear in supporting
% documentation.  This software is provided "as is" without express or
% implied warranty.
%
% end of input file: pnmcut.1
%--------------------------------------------------
 
% -*-LaTeX-*-
% Converted automatically from troff to LaTeX
% by tr2latex ($Revision: 2.2 $$Date: 1992/04/27 15:13:26 $ by C. Engel)
% on Fri Oct  8 16:09:20 1993
% tr2latex was written by Kamal Al-Yahya at Stanford University
% (Kamal%Hanauma@SU-SCORE.ARPA)
% and substantially enhanced by Christian Engel at RWTH Aachen
% (krischan@informatik.rwth-aachen.de).
%
% troff input file: pnmdepth.1

\newpage
%--------------------------------------------------
% start of input file: pnmdepth.1
%
\phead{pnmdepth}{1}{12 January 1991}{}{}

%.IX pnmdepth
\shead{NAME}
pnmdepth - change the maxval in a portable anymap
\shead{SYNOPSIS}
{\bf pnmdepth}
{\it newmaxval}
{\rm [}{\it pnmfile}{\rm ]}
\shead{DESCRIPTION}
Reads a portable anymap as input.
Scales all the pixel values, and writes out the image with the new maxval.
Scaling the colors down to a smaller maxval will result in some loss
of information.
\par
Be careful of off-by-one errors when choosing the new maxval.
For instance, if you want the color values to be five bits wide,
use a maxval of 31, not 32.
\shead{SEE ALSO}
pnm(5), ppmquant(1), ppmdither(1)
\shead{AUTHOR}
\copyright 1989, 1991 by Jef Poskanzer.
% Permission to use, copy, modify, and distribute this software and its
% documentation for any purpose and without fee is hereby granted, provided
% that the above copyright notice appear in all copies and that both that
% copyright notice and this permission notice appear in supporting
% documentation.  This software is provided "as is" without express or
% implied warranty.
%
% end of input file: pnmdepth.1
%--------------------------------------------------
 
% -*-LaTeX-*-
% Converted automatically from troff to LaTeX
% by tr2latex ($Revision: 2.2 $$Date: 1992/04/27 15:13:26 $ by C. Engel)
% on Fri Oct  8 16:09:20 1993
% tr2latex was written by Kamal Al-Yahya at Stanford University
% (Kamal%Hanauma@SU-SCORE.ARPA)
% and substantially enhanced by Christian Engel at RWTH Aachen
% (krischan@informatik.rwth-aachen.de).
%
% troff input file: pnmenlarge.1

\newpage
%--------------------------------------------------
% start of input file: pnmenlarge.1
%
\phead{pnmenlarge}{1}{26 February 1989}{}{}

%.IX pnmenlarge
\shead{NAME}
pnmenlarge - read a portable anymap and enlarge it N times
\shead{SYNOPSIS}
{\bf pnmenlarge}
{\it N}
{\rm [}{\it pnmfile}{\rm ]}
\shead{DESCRIPTION}
Reads a portable anymap as input.
Replicates its pixels
{\it N}
times, and produces a portable anymap as output.
%.IX enlarging
\par
{\it pnmenlarge}
can only enlarge by integer factors.
The slower but more general
{\it pnmscale}
%.IX pnmscale
can enlarge or reduce by arbitrary
factors, and
{\it pbmreduce}
%.IX pbmreduce
can reduce by integer factors, but only for bitmaps.
\par
If you enlarge by a factor of 3 or more, you should probably add a
{\it pnmsmooth}
%.IX pnmsmooth
step; otherwise, you can see the original pixels in the resulting image.
\shead{SEE ALSO}
pbmreduce(1), pnmscale(1), pnmsmooth(1), pnm(5)
\shead{AUTHOR}
\copyright 1989 by Jef Poskanzer.
% Permission to use, copy, modify, and distribute this software and its
% documentation for any purpose and without fee is hereby granted, provided
% that the above copyright notice appear in all copies and that both that
% copyright notice and this permission notice appear in supporting
% documentation.  This software is provided "as is" without express or
% implied warranty.
%
% end of input file: pnmenlarge.1
%--------------------------------------------------
 
% -*-LaTeX-*-
% Converted automatically from troff to LaTeX
% by tr2latex ($Revision: 2.2 $$Date: 1992/04/27 15:13:26 $ by C. Engel)
% on Fri Oct  8 16:09:21 1993
% tr2latex was written by Kamal Al-Yahya at Stanford University
% (Kamal%Hanauma@SU-SCORE.ARPA)
% and substantially enhanced by Christian Engel at RWTH Aachen
% (krischan@informatik.rwth-aachen.de).
%
% troff input file: pnmfile.1

\newpage
%--------------------------------------------------
% start of input file: pnmfile.1
%
\phead{pnmfile}{1}{9 January 1991}{}{}

%.IX pnmfile
\shead{NAME}
pnmfile - describe a portable anymap
\shead{SYNOPSIS}
{\bf pnmfile}
{\rm [}{\it pnmfile}{\rm ]}
{\rm ...}
\shead{DESCRIPTION}
Reads one or more portable anymaps as input.
Writes out short descriptions of the image type, size, etc.
This is mostly for use in shell scripts, so the format is not
particularly pretty.
\shead{SEE ALSO}
pnm(5), file(1)
\shead{AUTHOR}
\copyright 1991 by Jef Poskanzer.
% Permission to use, copy, modify, and distribute this software and its
% documentation for any purpose and without fee is hereby granted, provided
% that the above copyright notice appear in all copies and that both that
% copyright notice and this permission notice appear in supporting
% documentation.  This software is provided "as is" without express or
% implied warranty.
%
% end of input file: pnmfile.1
%--------------------------------------------------
 
% -*-LaTeX-*-
% Converted automatically from troff to LaTeX
% by tr2latex ($Revision: 2.2 $$Date: 1992/04/27 15:13:26 $ by C. Engel)
% on Fri Oct  8 16:09:21 1993
% tr2latex was written by Kamal Al-Yahya at Stanford University
% (Kamal%Hanauma@SU-SCORE.ARPA)
% and substantially enhanced by Christian Engel at RWTH Aachen
% (krischan@informatik.rwth-aachen.de).
%
% troff input file: pnmflip.1

\newpage
%--------------------------------------------------
% start of input file: pnmflip.1
%
\phead{pnmflip}{1}{25 July 1989}{}{}

%.IX pnmflip
\shead{NAME}
pnmflip - perform one or more flip operations on a portable anymap
\shead{SYNOPSIS}
{\bf pnmflip}
{\rm [}{\bf -leftright}{\rm $|$}{\bf -lr}{\rm ]}
{\rm [}{\bf -topbottom}{\rm $|$}{\bf -tb}{\rm ]}
{\rm [}{\bf -transpose}{\rm $|$}{\bf -xy}{\rm ]}
{\rm [}{\bf -rotate90}{\rm $|$}{\bf -r90}{\rm $|$}{\bf -ccw}
{\rm ]}
{\rm [}{\bf -rotate270}{\rm $|$}{\bf -r270}{\rm $|$}{\bf -cw}
{\rm ]}
{\rm [}{\bf -rotate180}{\rm $|$}{\bf -r180}{\rm ]}
{\rm [}{\it pnmfile}{\rm ]}
\shead{DESCRIPTION}
Reads a portable anymap as input.
Performs one or more flip operations, in the order specified, and
writes out a portable anymap.
%.IX rotation
%.IX reflection
%.IX transposition
\shead{OPTIONS}
\par
The flip operations available are: left for right
{\rm (}{\bf -leftright}
or
{\bf -lr}{\rm );}
top for bottom
{\rm (}{\bf -topbottom}
or
{\bf -tb}{\rm );}
and transposition
{\rm (}{\bf -transpose}
or
{\bf -xy}{\rm ).}
In addition, some canned concatenations are available:
{\bf -rotate90}
or
{\bf -ccw}
is equivalent to
{\bf -transpose}
{\bf -topbottom}{\rm ;}
{\bf -rotate270}
or
{\bf -cw}
is equivalent to
{\bf -transpose}
{\bf -leftright}{\rm ;}
and
{\bf -rotate180}
is equivalent to
{\bf -leftright}
{\bf -topbottom}{\rm .}
\par
All flags can be abbreviated to their shortest unique prefix.
\shead{SEE ALSO}
pnmrotate(1), pnm(5)
\shead{AUTHOR}
\copyright 1989 by Jef Poskanzer.
% Permission to use, copy, modify, and distribute this software and its
% documentation for any purpose and without fee is hereby granted, provided
% that the above copyright notice appear in all copies and that both that
% copyright notice and this permission notice appear in supporting
% documentation.  This software is provided "as is" without express or
% implied warranty.
%
% end of input file: pnmflip.1
%--------------------------------------------------
 
% -*-LaTeX-*-
% Converted automatically from troff to LaTeX
% by tr2latex ($Revision: 2.2 $$Date: 1992/04/27 15:13:26 $ by C. Engel)
% on Fri Oct  8 16:09:29 1993
% tr2latex was written by Kamal Al-Yahya at Stanford University
% (Kamal%Hanauma@SU-SCORE.ARPA)
% and substantially enhanced by Christian Engel at RWTH Aachen
% (krischan@informatik.rwth-aachen.de).
%
% troff input file: pnmgamma.1

\newpage
%--------------------------------------------------
% start of input file: pnmgamma.1
%
\phead{pnmgamma}{1}{12 January 1991}{}{}

%.IX pnmgamma
\shead{NAME}
pnmgamma - perform gamma correction on a portable anymap
\shead{SYNOPSIS}
{\bf pnmgamma}
{\it value}
{\rm [}{\it pnmfile}{\rm ]}
\nwl
{\bf pnmgamma}
{\it redvalue greenvalue bluevalue}
{\rm [}{\it pnmfile}{\rm ]}
\shead{DESCRIPTION}
Reads a portable anymap as input.
Performs gamma correction,
and produces a portable anymap as output.
%.IX "gamma correction"
\par
The arguments specify what gamma value(s) to use.
A value of 1.0 leaves the image alone, less than one darkens it,
and greater than one lightens it.
\shead{SEE ALSO}
pnm(5)
\shead{AUTHOR}
\copyright 1991 by Bill Davidson and Jef Poskanzer.
% Permission to use, copy, modify, and distribute this software and its
% documentation for any purpose and without fee is hereby granted, provided
% that the above copyright notice appear in all copies and that both that
% copyright notice and this permission notice appear in supporting
% documentation.  This software is provided "as is" without express or
% implied warranty.
%
% end of input file: pnmgamma.1
%--------------------------------------------------
 
% -*-LaTeX-*-
% Converted automatically from troff to LaTeX
% by tr2latex ($Revision: 2.2 $$Date: 1992/04/27 15:13:26 $ by C. Engel)
% on Fri Oct  8 16:09:22 1993
% tr2latex was written by Kamal Al-Yahya at Stanford University
% (Kamal%Hanauma@SU-SCORE.ARPA)
% and substantially enhanced by Christian Engel at RWTH Aachen
% (krischan@informatik.rwth-aachen.de).
%
% troff input file: pnminvert.1

\newpage
%--------------------------------------------------
% start of input file: pnminvert.1
%
\phead{pnminvert}{1}{08 August 1989}{}{}

%.IX pnminvert
\shead{NAME}
pnminvert - invert a portable anymap
\shead{SYNOPSIS}
{\bf pnminvert}
{\rm [}{\it pnmfile}{\rm ]}
\shead{DESCRIPTION}
Reads a portable anymap as input.
Inverts it black for white and produces a portable anymap as output.
\shead{SEE ALSO}
pnm(5)
\shead{AUTHOR}
\copyright 1989 by Jef Poskanzer.
% Permission to use, copy, modify, and distribute this software and its
% documentation for any purpose and without fee is hereby granted, provided
% that the above copyright notice appear in all copies and that both that
% copyright notice and this permission notice appear in supporting
% documentation.  This software is provided "as is" without express or
% implied warranty.
%
% end of input file: pnminvert.1
%--------------------------------------------------
 
% -*-LaTeX-*-
% Converted automatically from troff to LaTeX
% by tr2latex ($Revision: 2.2 $$Date: 1992/04/27 15:13:26 $ by C. Engel)
% on Fri Oct  8 16:09:32 1993
% tr2latex was written by Kamal Al-Yahya at Stanford University
% (Kamal%Hanauma@SU-SCORE.ARPA)
% and substantially enhanced by Christian Engel at RWTH Aachen
% (krischan@informatik.rwth-aachen.de).
%
% troff input file: pnmnlfilt.1

\newpage
%--------------------------------------------------
% start of input file: pnmnlfilt.1
%
\phead{pnmnlfilt}{1}{5 February 1993}{}{}

%.IX pnmnlfilt
\shead{NAME}
pnmnlfilt - non-linear filters: smooth, alpha trim mean, optimal
estimation smoothing, edge enhancement.
\shead{SYNOPSIS}
{\bf pnmnlfilt}
{\rm alpha}
{\rm radius}
{\rm [}{\it pnmfile}{\rm ]}
\shead{DESCRIPTION}
%.IX smoothing
%.IX dithering
%.IX alpha trim
%.IX mean filter
%.IX median filter
%.IX optimal estimation
This is something of a swiss army knife filter. It has 3 distinct operating
modes. In all of the modes each pixel in the image is examined and processed
according to it and its surrounding pixels values. Rather than using the
9 pixels in a 3x3 block, 7 hexagonal area samples are taken, the size of
the hexagons being controlled by the radius parameter. A radius value of
0.3333 means that the 7 hexagons exactly fit into the center pixel ({\it i.e.},
there will be no filtering effect). A radius value of 1.0 means that
the 7 hexagons exactly fit a 3x3 pixel array.
\shead{Alpha trimmed mean filter.	(0.0 $<$= alpha $<$= 0.5)}
\par
The value of the center pixel will be
replaced by the mean of the 7 hexagon values, but the 7 values are
sorted by size and the top and bottom alpha portion of the 7 are
excluded from the mean.  This implies that an alpha value of 0.0 gives
the same sort of output as a normal convolution ({\it i.e.}, averaging or
smoothing filter), where radius will determine the ``strength'' of the
filter. A good value to start from for subtle filtering is alpha = 0.0, radius = 0.55
For a more blatant effect, try alpha 0.0 and radius 1.0
\par
An alpha value of 0.5 will cause the median value of the
7 hexagons to be used to replace the center pixel value. This sort
of filter is good for eliminating ``pop'' or single pixel noise from
an image without spreading the noise out or smudging features on
the image. Judicious use of the radius parameter will fine tune the
filtering. Intermediate values of alpha give effects somewhere
between smoothing and ``pop'' noise reduction. For subtle filtering
try starting with values of alpha = 0.4, radius = 0.6  For a more blatant
effect try alpha = 0.5, radius = 1.0
\shead{Optimal estimation smoothing. (1.0 $<$= alpha $<$= 2.0)}
\par
This type of filter applies a smoothing filter adaptively over the image.
For each pixel the variance of the surrounding hexagon values is calculated,
and the amount of smoothing is made inversely proportional to it. The idea
is that if the variance is small then it is due to noise in the image, while
if the variance is large, it is because of ``wanted'' image features. As usual
the radius parameter controls the effective radius, but it probably advisable to
leave the radius between 0.8 and 1.0 for the variance calculation to be meaningful.
The alpha parameter sets the noise threshold, over which less smoothing will be done.
This means that small values of alpha will give the most subtle filtering effect,
while large values will tend to smooth all parts of the image. You could start
with values like alpha = 1.2, radius = 1.0 and try increasing or decreasing the
alpha parameter to get the desired effect. This type of filter is best for
filtering out dithering noise in both bitmap and color images.
\shead{Edge enhancement. (-0.1 $>$= alpha $>$= -0.9)}
\par
This is the opposite type of filter to the smoothing filter. It enhances
edges. The alpha parameter controls the amount of edge enhancement, from
subtle (-0.1) to blatant (-0.9). The radius parameter controls the effective
radius as usual, but useful values are between 0.5 and 0.9. Try starting
with values of alpha = 0.3, radius = 0.8
\shead{Combination use.}
\par
The various modes of 
{\bf pnmnlfilt}
can be used one after the other to get the desired result. For instance to
turn a monochrome dithered image into a grayscale image you could try
one or two passes of the smoothing filter, followed by a pass of the optimal estimation
filter, then some subtle edge enhancement. Note that using edge enhancement is
only likely to be useful after one of the non-linear filters (alpha trimmed mean
or optimal estimation filter), as edge enhancement is the direct opposite of
smoothing.
\par
For reducing color quantization noise in images ({\it i.e.}, turning .gif files back into
24 bit files) you could try a pass of the optimal estimation filter
(alpha 1.2, radius 1.0), a pass of the median filter (alpha 0.5, radius 0.55),
and possibly a pass of the edge enhancement filter.
Several passes of the optimal estimation filter with declining alpha
values are more effective than a single pass with a large alpha value.
As usual, there is a tradeoff between filtering effectiveness and loosing
detail. Experimentation is encouraged.
\shead{References:}
\par
The alpha-trimmed mean filter is 
based on the description in IEEE CG\&A May 1990 
Page 23 by Mark E. Lee and Richard A. Redner,
and has been enhanced to allow continuous alpha adjustment.
\par
The optimal estimation filter is taken from an article ``Converting Dithered
Images Back to Gray Scale'' by Allen Stenger, Dr Dobb's Journal, November
1992, and this article references ``Digital Image Enhancement and Noise Filtering by
Use of Local Statistics'', Jong-Sen Lee, IEEE Transactions on Pattern Analysis and
Machine Intelligence, March 1980.
\par
The edge enhancement details are from pgmenhance(1),
which is taken from Philip R. Thompson's ``xim''
program, which in turn took it from section 6 of ``Digital Halftones by
Dot Diffusion'', D. E. Knuth, ACM Transaction on Graphics Vol. 6, No. 4,
October 1987, which in turn got it from two 1976 papers by J. F. Jarvis
{\it et. al.}
\shead{SEE ALSO}
pgmenhance(1), pnmconvol(1), pnm(5)
\shead{BUGS}
Integers and tables may overflow if PPM\_MAXMAXVAL is greater than 255.
\shead{AUTHOR}
Graeme W. Gill    graeme@labtam.oz.au
%
% end of input file: pnmnlfilt.1
%--------------------------------------------------
 
% -*-LaTeX-*-
% Converted automatically from troff to LaTeX
% by tr2latex ($Revision: 2.2 $$Date: 1992/04/27 15:13:26 $ by C. Engel)
% on Fri Oct  8 16:09:23 1993
% tr2latex was written by Kamal Al-Yahya at Stanford University
% (Kamal%Hanauma@SU-SCORE.ARPA)
% and substantially enhanced by Christian Engel at RWTH Aachen
% (krischan@informatik.rwth-aachen.de).
%
% troff input file: pnmnoraw.1

\newpage
%--------------------------------------------------
% start of input file: pnmnoraw.1
%
\phead{pnmnoraw}{1}{8 January 1991}{}{}

%.IX pnmnoraw
\shead{NAME}
pnmnoraw - force a portable anymap into plain format
\shead{SYNOPSIS}
{\bf pnmnoraw}
{\rm [}{\it pnmfile}{\rm ]}
\shead{DESCRIPTION}
Reads a portable anymap as input.
Writes it out in plain (non-raw) format.
This is fairly useless if you haven't defined the PBMPLUS\_RAWBITS
compile-time option.
%.IX RAWBITS
\shead{SEE ALSO}
pnm(5)
\shead{AUTHOR}
\copyright 1991 by Jef Poskanzer.
% Permission to use, copy, modify, and distribute this software and its
% documentation for any purpose and without fee is hereby granted, provided
% that the above copyright notice appear in all copies and that both that
% copyright notice and this permission notice appear in supporting
% documentation.  This software is provided "as is" without express or
% implied warranty.
%
% end of input file: pnmnoraw.1
%--------------------------------------------------
 
% -*-LaTeX-*-
% Converted automatically from troff to LaTeX
% by tr2latex ($Revision: 2.2 $$Date: 1992/04/27 15:13:26 $ by C. Engel)
% on Fri Oct  8 16:09:33 1993
% tr2latex was written by Kamal Al-Yahya at Stanford University
% (Kamal%Hanauma@SU-SCORE.ARPA)
% and substantially enhanced by Christian Engel at RWTH Aachen
% (krischan@informatik.rwth-aachen.de).
%
% troff input file: pnmpad.1

\newpage
%--------------------------------------------------
% start of input file: pnmpad.1
%
\phead{pnmpad}{12 Dec 1990}{}{}{}

\shead{NAME}
pnmpad - add borders to portable anymap
\shead{SYNOPSIS}
pnmpad [-white$|$-black] [-l\#] [-r\#] [-t\#] [-b\#] [pnmfile]
\shead{DESCRIPTION}
Reads a portable anymap as input. Outputs a portable anymap with extra
borders of the sizes specified. The colour of the borders can be set to
black or white (default black).

\shead{SEE ALSO}
pbmmake(1), pnmpaste(1), pbm(5)
\shead{AUTHOR}
\copyright 1990 by Angus Duggan.
\copyright 1989 by Jef Poskanzer.

Permission to use, copy, modify, and distribute this software and its
documentation for any purpose and without fee is hereby granted, provided
that the above copyright notice appear in all copies and that both that
copyright notice and this permission notice appear in supporting
documentation.  This software is provided "as is" without express or
implied warranty.
%
% end of input file: pnmpad.1
%--------------------------------------------------
 
% -*-LaTeX-*-
% Converted automatically from troff to LaTeX
% by tr2latex ($Revision: 2.2 $$Date: 1992/04/27 15:13:26 $ by C. Engel)
% on Fri Oct  8 16:09:23 1993
% tr2latex was written by Kamal Al-Yahya at Stanford University
% (Kamal%Hanauma@SU-SCORE.ARPA)
% and substantially enhanced by Christian Engel at RWTH Aachen
% (krischan@informatik.rwth-aachen.de).
%
% troff input file: pnmpaste.1

\newpage
%--------------------------------------------------
% start of input file: pnmpaste.1
%
\phead{pnmpaste}{1}{21 February 1991}{}{}

%.IX pnmpaste
\shead{NAME}
pnmpaste - paste a rectangle into a portable anymap
\shead{SYNOPSIS}
{\bf pnmpaste}
{\rm [}{\bf -replace}{\rm $|$}{\bf -or}{\rm $|$}{\bf -and}
{\rm $|$}{\bf -xor}{\rm ]}
{\it frompnmfile x y}
{\rm [}{\it intopnmfile}{\rm ]}
\shead{DESCRIPTION}
Reads two portable anymaps as input.
Inserts the first anymap into the second at the specified location,
and produces a portable anymap the same size as the second as output.
%.IX paste
If the second anymap is not specified, it is read from stdin.
The
{\it x}
and
{\it y}
can be negative, in which case they are interpreted
relative to the right and bottom of the anymap, respectively.
\par
This tool is most useful in combination with
{\it pnmcut}{\rm .}
%.IX pnmcut
For instance, if you want to edit a small segment of a large
image, and your image editor cannot edit the
large image, you can cut out the segment you are interested in,
edit it, and then paste it back in.
\par
Another useful companion tool is
{\it pbmmask}{\rm .}
%.IX pnmmask
%.OPTIONS
\par
The optional flag specifies the operation to use when doing the paste.
The default is
{\bf -replace}{\rm .}
The other, logical operations are only allowed if both input images
are bitmaps.
%.IX "logical operations"
These operations act as if white is TRUE and black is FALSE.
\par
All flags can be abbreviated to their shortest unique prefix.
\shead{SEE ALSO}
pnmcut(1), pnminvert(1), pnmarith(1), pnm(5), pbmmask(1)
\shead{AUTHOR}
\copyright 1989, 1991 by Jef Poskanzer.
% Permission to use, copy, modify, and distribute this software and its
% documentation for any purpose and without fee is hereby granted, provided
% that the above copyright notice appear in all copies and that both that
% copyright notice and this permission notice appear in supporting
% documentation.  This software is provided "as is" without express or
% implied warranty.
%
% end of input file: pnmpaste.1
%--------------------------------------------------
 
% -*-LaTeX-*-
% Converted automatically from troff to LaTeX
% by tr2latex ($Revision: 2.2 $$Date: 1992/04/27 15:13:26 $ by C. Engel)
% on Fri Oct  8 16:09:29 1993
% tr2latex was written by Kamal Al-Yahya at Stanford University
% (Kamal%Hanauma@SU-SCORE.ARPA)
% and substantially enhanced by Christian Engel at RWTH Aachen
% (krischan@informatik.rwth-aachen.de).
%
% troff input file: pnmrotate.1

\newpage
%--------------------------------------------------
% start of input file: pnmrotate.1
%
\phead{pnmrotate}{1}{12 January 1991}{}{}

%.IX pnmrotate
\shead{NAME}
pnmrotate - rotate a portable anymap by some angle
\shead{SYNOPSIS}
{\bf pnmrotate}
{\rm [}{\bf -noantialias}{\rm ]}
{\it angle}
{\rm [}{\it pnmfile}{\rm ]}
\shead{DESCRIPTION}
Reads a portable anymap as input.
Rotates it by the specified angle
and produces a portable anymap as output.
%.IX rotation
If the input file is in color, the output will be too,
otherwise it will be grayscale.
The angle is in degrees (floating point), measured counter-clockwise.
It can be negative, but it should be between -90 and 90.
Also, for rotations greater than 45 degrees you may get better results
if you first use
{\it pnmflip}
%.IX pnmflip
to do a 90 degree rotation and then
{\it pnmrotate}
less than 45 degrees back the other direction
\par
The rotation algorithm is Alan Paeth's three-shear method.
Each shear is implemented by looping over the source pixels and distributing
fractions to each of the destination pixels.
This has an ``anti-aliasing'' effect - it avoids jagged edges and similar
artifacts.
%.IX anti-aliasing
However, it also means that the original colors or gray levels in the image
are modified.
If you need to keep precisely the same set of colors, you can use the
{\bf -noantialias}
flag.  This does the shearing by moving pixels without changing their values.
If you want anti-aliasing and don't care about the precise colors, but
still need a limited *number* of colors, you can run the result through
{\it ppmquant}{\rm .}
%.IX ppmquant
\par
All flags can be abbreviated to their shortest unique prefix.
\shead{REFERENCES}
``A Fast Algorithm for General Raster Rotation'' by Alan Paeth,
Graphics Interface '86, pp. 77-81.
\shead{SEE ALSO}
pnmshear(1), pnmflip(1), pnm(5), ppmquant(1)
\shead{AUTHOR}
\copyright 1989, 1991 by Jef Poskanzer.
% Permission to use, copy, modify, and distribute this software and its
% documentation for any purpose and without fee is hereby granted, provided
% that the above copyright notice appear in all copies and that both that
% copyright notice and this permission notice appear in supporting
% documentation.  This software is provided "as is" without express or
% implied warranty.
%
% end of input file: pnmrotate.1
%--------------------------------------------------
 
% -*-LaTeX-*-
% Converted automatically from troff to LaTeX
% by tr2latex ($Revision: 2.2 $$Date: 1992/04/27 15:13:26 $ by C. Engel)
% on Fri Oct  8 16:09:24 1993
% tr2latex was written by Kamal Al-Yahya at Stanford University
% (Kamal%Hanauma@SU-SCORE.ARPA)
% and substantially enhanced by Christian Engel at RWTH Aachen
% (krischan@informatik.rwth-aachen.de).
%
% troff input file: pnmscale.1

\newpage
%--------------------------------------------------
% start of input file: pnmscale.1
%
\phead{pnmscale}{1}{12 January 1991}{}{}

%.IX pnmscale
\shead{NAME}
pnmscale - scale a portable anymap
\shead{SYNOPSIS}
{\bf pnmscale}
{\it s}
{\rm [}{\it pnmfile}{\rm ]}
\nwl
{\bf pnmscale}
{\bf -xsize}{\rm $|$}{\bf -width}{\rm $|$}{\bf -ysize}{\rm $|$}
{\bf -height}
{\it s}
{\rm [}{\it pnmfile}{\rm ]}
\nwl
{\bf pnmscale}
{\bf -xscale}{\rm $|$}{\bf -yscale}
{\it s}
{\rm [}{\it pnmfile}{\rm ]}
\nwl
{\bf pnmscale}
{\bf -xscale}{\rm $|$}{\bf -xsize}{\rm $|$}{\bf -width}
{\it s}
{\bf -yscale}{\rm $|$}{\bf -ysize}{\rm $|$}{\bf -height}
{\it s}
{\rm [}{\it pnmfile}{\rm ]}
\nwl
{\bf pnmscale -xysize}
{\it x y}
{\rm [}{\it pnmfile}{\rm ]}
\shead{DESCRIPTION}
Reads a portable anymap as input.
Scales it by the specified factor or factors and produces a portable
anymap as output.
%.IX shrinking
%.IX enlarging
If the input file is in color, the output will be too,
otherwise it will be grayscale.
You can both enlarge (scale factor $>$ 1) and reduce (scale factor $<$ 1).
\par
You can specify one dimension as a pixel size, and the other dimension
will be scaled correspondingly.
\par
You can specify one dimension as a scale, and the other dimension
will not be scaled.
\par
You can specify different sizes or scales for each axis.
\par
Or, you can use the special
{\bf -xysize}
flag, which fits the image into
the specified size without changing the aspect ratio.
\par
All flags can be abbreviated to their shortest unique prefix.
\par
If you enlarge by a factor of 3 or more, you should probably add a
{\it pnmsmooth}
%.IX pnmsmooth
step; otherwise, you can see the original pixels in the resulting image.
\shead{SEE ALSO}
pbmreduce(1), pnmenlarge(1), pnmsmooth(1), pnm(5)
\shead{AUTHOR}
\copyright 1989, 1991 by Jef Poskanzer.
% Permission to use, copy, modify, and distribute this software and its
% documentation for any purpose and without fee is hereby granted, provided
% that the above copyright notice appear in all copies and that both that
% copyright notice and this permission notice appear in supporting
% documentation.  This software is provided "as is" without express or
% implied warranty.
%
% end of input file: pnmscale.1
%--------------------------------------------------
 
% -*-LaTeX-*-
% Converted automatically from troff to LaTeX
% by tr2latex ($Revision: 2.2 $$Date: 1992/04/27 15:13:26 $ by C. Engel)
% on Fri Oct  8 16:09:30 1993
% tr2latex was written by Kamal Al-Yahya at Stanford University
% (Kamal%Hanauma@SU-SCORE.ARPA)
% and substantially enhanced by Christian Engel at RWTH Aachen
% (krischan@informatik.rwth-aachen.de).
%
% troff input file: pnmshear.1

\newpage
%--------------------------------------------------
% start of input file: pnmshear.1
%
\phead{pnmshear}{1}{12 January 1991}{}{}

%.IX pnmshear
\shead{NAME}
pnmshear - shear a portable anymap by some angle
\shead{SYNOPSIS}
{\bf pnmshear}
{\rm [}{\bf -noantialias}{\rm ]}
{\it angle}
{\rm [}{\it pnmfile}{\rm ]}
\shead{DESCRIPTION}
Reads a portable anymap as input.
Shears it by the specified angle and produces a portable
anymap as output.
%.IX shearing
If the input file is in color, the output will be too,
otherwise it will be grayscale.
The angle is in degrees (floating point), and measures this:
\nofill
    +-------+  +-------+
    $|$       $|$  $|$\bs        \bs 
    $|$  OLD  $|$  $|$ \bs   NEW  \bs 
    $|$       $|$  $|$an\bs        \bs 
    +-------+  $|$gle+-------+
\fill
If the angle is negative, it shears the other way:
\nofill
    +-------+  $|$-an+-------+
    $|$       $|$  $|$gl/       /
    $|$  OLD  $|$  $|$e/  NEW  /
    $|$       $|$  $|$/       /
    +-------+  +-------+
\fill
The angle should not get too close to 90 or -90, or the resulting
anymap will be unreasonably wide.
\par
The shearing is implemented by looping over the source pixels and distributing
fractions to each of the destination pixels.
This has an ``anti-aliasing'' effect - it avoids jagged edges and similar
artifacts.
%.IX anti-aliasing
However, it also means that the original colors or gray levels in the image
are modified.
If you need to keep precisely the same set of colors, you can use
the
{\bf -noantialias}
flag.  This does the shearing by moving pixels without changing their values.
If you want anti-aliasing and don't care about the precise colors, but
still need a limited *number* of colors, you can run the result through
{\it ppmquant}{\rm .}
%.IX ppmquant
\par
All flags can be abbreviated to their shortest unique prefix.
\shead{SEE ALSO}
pnmrotate(1), pnmflip(1), pnm(5), ppmquant(1)
\shead{AUTHOR}
\copyright 1989, 1991 by Jef Poskanzer.
% Permission to use, copy, modify, and distribute this software and its
% documentation for any purpose and without fee is hereby granted, provided
% that the above copyright notice appear in all copies and that both that
% copyright notice and this permission notice appear in supporting
% documentation.  This software is provided "as is" without express or
% implied warranty.
%
% end of input file: pnmshear.1
%--------------------------------------------------
 
% -*-LaTeX-*-
% Converted automatically from troff to LaTeX
% by tr2latex ($Revision: 2.2 $$Date: 1992/04/27 15:13:26 $ by C. Engel)
% on Fri Oct  8 16:09:24 1993
% tr2latex was written by Kamal Al-Yahya at Stanford University
% (Kamal%Hanauma@SU-SCORE.ARPA)
% and substantially enhanced by Christian Engel at RWTH Aachen
% (krischan@informatik.rwth-aachen.de).
%
% troff input file: pnmtile.1

\newpage
%--------------------------------------------------
% start of input file: pnmtile.1
%
\phead{pnmtile}{1}{13 May 1989}{}{}

%.IX pnmtile
\shead{NAME}
pnmtile - replicate a portable anymap into a specified size
\shead{SYNOPSIS}
{\bf pnmtile}
{\it width height}
{\rm [}{\it pnmfile}{\rm ]}
\shead{DESCRIPTION}
Reads a portable anymap as input.
Replicates it until it is the specified size,
and produces a portable anymap as output.
%.IX tiling
\shead{SEE ALSO}
pnm(5)
\shead{AUTHOR}
\copyright 1989 by Jef Poskanzer.
% Permission to use, copy, modify, and distribute this software and its
% documentation for any purpose and without fee is hereby granted, provided
% that the above copyright notice appear in all copies and that both that
% copyright notice and this permission notice appear in supporting
% documentation.  This software is provided "as is" without express or
% implied warranty.
%
% end of input file: pnmtile.1
%--------------------------------------------------
 
% -*-LaTeX-*-
% Converted automatically from troff to LaTeX
% by tr2latex ($Revision: 2.2 $$Date: 1992/04/27 15:13:26 $ by C. Engel)
% on Mon Feb  7 08:49:06 1994
% tr2latex was written by Kamal Al-Yahya at Stanford University
% (Kamal%Hanauma@SU-SCORE.ARPA)
% and substantially enhanced by Christian Engel at RWTH Aachen
% (krischan@informatik.rwth-aachen.de).
%
% troff input file: pnmtofits.1

\newpage
%--------------------------------------------------
% start of input file: pnmtofits.1
%
\phead{pnmtofits}{1}{5 Dec 1992}{}{}
\shead{NAME}
pnmtofits - convert a portable anymap into FITS format
\shead{SYNOPSIS}
{\bf pnmtofits}
{\rm [}{\bf --max}
{\it f}{\rm ]}
{\rm [}{\bf --min}
{\it f}{\rm ]}
{\rm [}{\it pnmfile}{\rm ]}
\shead{DESCRIPTION}
Reads a portable anymap as input.
Produces a FITS (Flexible Image Transport System) file as output.
The resolution of the output file is either 8 bits/pixel,
or 16 bits/pixel, depending on the value of maxval in the input file.
If the input file is a portable bitmap or a portable graymap, the output file
consists of a single plane image (NAXIS = 2). If instead the input file is
a portable pixmap, the output file will consist of a three-plane image
(NAXIS = 3, NAXIS3 = 3).
A full description of the FITS format
can be found in Astronomy \& Astrophysics Supplement Series 44 (1981), page 363.
\shead{OPTIONS}
\par
Flags 
{\bf --min}
and 
{\bf --max}
can be used to set DATAMAX, DATAMIN, BSCALE and BZERO in the FITS
header, but do not cause the data to be rescaled.
\shead{SEE ALSO}
fitstopnm(1), pgm(5)
\shead{AUTHOR}
Copyright (C) 1989 by Wilson H. Bent (whb@hoh-2.att.com), with
modifications
by Alberto Accomazzi (alberto@cfa.harvard.edu).
% Permission to use, copy, modify, and distribute this software and its
% documentation for any purpose and without fee is hereby granted, provided
% that the above copyright notice appear in all copies and that both that
% copyright notice and this permission notice appear in supporting
% documentation.  This software is provided "as is" without express or
% implied warranty.
%
% end of input file: pnmtofits.1
%--------------------------------------------------

% -*-LaTeX-*-
% Converted automatically from troff to LaTeX
% by tr2latex ($Revision: 2.2 $$Date: 1992/04/27 15:13:26 $ by C. Engel)
% on Fri Oct  8 16:09:25 1993
% tr2latex was written by Kamal Al-Yahya at Stanford University
% (Kamal%Hanauma@SU-SCORE.ARPA)
% and substantially enhanced by Christian Engel at RWTH Aachen
% (krischan@informatik.rwth-aachen.de).
%
% troff input file: pnmtops.1

\newpage
%--------------------------------------------------
% start of input file: pnmtops.1
%
\phead{pnmtops}{1}{26 October 1991}{}{}

%.IX pnmtops
\shead{NAME}
pnmtops - convert portable anymap to PostScript
\shead{SYNOPSIS}
{\bf pnmtops}
{\rm [}{\bf -scale}
{\it s}{\rm ]}
{\rm [}{\bf -turn}{\rm $|$}{\bf -noturn}{\rm ]}
{\rm [}{\bf -rle}{\rm $|$}{\bf -runlength}{\rm ]}
{\rm [}{\bf -dpi}
{\it n}{\rm ]}
{\rm [}{\bf -width}
{\it n}{\rm ]}
{\rm [}{\bf -height}
{\it n}{\rm ]}
{\rm [}{\it pnmfile}{\rm ]}
\shead{DESCRIPTION}
Reads a portable anymap as input.
Produces Encapsulated PostScript as output.
%.IX PostScript
\par
If the input file is in color (PPM), a color PostScript file gets
written.
Some PostScript interpreters can't handle color PostScript.
If you have one of these you will need to run your image through
{\it ppmtopgm}
first.
\par
Note that there is no pstopnm
tool - this transformation is one-way, because a pstopnm tool would
be a full-fledged PostScript interpreter, which is beyond the scope
of this package.
However, see the
{\it psidtopgm}
tool, which can read grayscale non-runlength PostScript image data.
Also, if you're willing to install the fairly large GhostScript package,
it comes with a pstoppm script.
\shead{OPTIONS}
\par
The
{\bf -scale}
flag controls the scale of the result.  The default scale is 1,
which on a 300 dpi printer such as the Apple LaserWriter makes
the output look about the same size as the input would if it was displayed
on a typical 72 dpi screen.
To get one PNM pixel per 300 dpi printer pixel, use ``-scale 0.25''.
\par
The
{\bf -turn}
and
{\bf -noturn}
flags control whether the image gets turned 90 degrees.
Normally, if an image is wider than it is tall, it gets turned
automatically to better fit the page.
If the
{\bf -turn}
flag is specified, it will be turned no matter what its shape; and if the
{\bf -noturn}
flag is specified, it will
{\it not}
be turned no matter what its shape.
\par
The
{\bf -rle}
or
{\bf -runlength}
flag specifies run-length compression.  This may save
time if the host-to-printer link is slow; but normally the printer's processing
time dominates, so
{\bf -rle}
makes things slower.
\par
The
{\bf -dpi}
flag lets you specify the dots per inch of your output device.
The default is 300 dpi.
In theory PostScript is device-independent and you don't have to
worry about this, but in practice its raster rendering can have
unsightly bands if the device pixels and the image pixels aren't
in sync.
\par
The
{\bf -width}
and
{\bf -height}
flags let you specify the size of the page.
The default is 8.5 inches by 11 inches.
\par
All flags can be abbreviated to their shortest unique prefix.
\shead{SEE ALSO}
pnm(5), psidtopgm(1)
\shead{AUTHOR}
\copyright 1989, 1991 by Jef Poskanzer.
% Permission to use, copy, modify, and distribute this software and its
% documentation for any purpose and without fee is hereby granted, provided
% that the above copyright notice appear in all copies and that both that
% copyright notice and this permission notice appear in supporting
% documentation.  This software is provided "as is" without express or
% implied warranty.
%
% end of input file: pnmtops.1
%--------------------------------------------------
 
% -*-LaTeX-*-
% Converted automatically from troff to LaTeX
% by tr2latex ($Revision: 2.2 $$Date: 1992/04/27 15:13:26 $ by C. Engel)
% on Fri Oct  8 16:09:26 1993
% tr2latex was written by Kamal Al-Yahya at Stanford University
% (Kamal%Hanauma@SU-SCORE.ARPA)
% and substantially enhanced by Christian Engel at RWTH Aachen
% (krischan@informatik.rwth-aachen.de).
%
% troff input file: pnmtorast.1

\newpage
%--------------------------------------------------
% start of input file: pnmtorast.1
%
\phead{pnmtorast}{1}{12 January 1991}{}{}

%.IX pnmtorast
\shead{NAME}
pnmtorast - convert a portable pixmap into a Sun rasterfile
\shead{SYNOPSIS}
{\bf pnmtorast}
{\rm [}{\bf -standard}{\rm $|$}{\bf -rle}{\rm ]}
{\rm [}{\it pnmfile}{\rm ]}
\shead{DESCRIPTION}
Reads a portable pixmap as input.
Produces a Sun rasterfile as output.
%.IX Sun
%.IX rasterfile
\par
Color values in Sun rasterfiles are eight bits wide, so
{\it pnmtorast}
will automatically scale colors to have a maxval of 255.
An extra
{\it pnmdepth}
step is not necessary.
\shead{OPTIONS}
\par
The
{\bf -standard}
flag forces the result to be in RT\_STANDARD form; the
{\bf -rle}
flag, RT\_BYTE\_ENCODED, which is smaller but, well, less standard.
The default is
{\bf -rle}{\rm .}
\par
All flags can be abbreviated to their shortest unique prefix.
\shead{SEE ALSO}
rasttopnm(1), pnm(5)
\shead{AUTHOR}
\copyright 1989, 1991 by Jef Poskanzer.
% Permission to use, copy, modify, and distribute this software and its
% documentation for any purpose and without fee is hereby granted, provided
% that the above copyright notice appear in all copies and that both that
% copyright notice and this permission notice appear in supporting
% documentation.  This software is provided "as is" without express or
% implied warranty.
%
% end of input file: pnmtorast.1
%--------------------------------------------------
 
% -*-LaTeX-*-
% Converted automatically from troff to LaTeX
% by tr2latex ($Revision: 2.2 $$Date: 1992/04/27 15:13:26 $ by C. Engel)
% on Fri Oct  8 16:09:31 1993
% tr2latex was written by Kamal Al-Yahya at Stanford University
% (Kamal%Hanauma@SU-SCORE.ARPA)
% and substantially enhanced by Christian Engel at RWTH Aachen
% (krischan@informatik.rwth-aachen.de).
%
% troff input file: pnmtosir.1

\newpage
%--------------------------------------------------
% start of input file: pnmtosir.1
%
\phead{pnmtosir}{1}{20 March 1991}{}{}

\shead{NAME}
pnmtosir - convert a portable anymap into a Solitaire format
\shead{SYNOPSIS}
{\bf pnmtosir}
{\rm [}{\it pnmfile}{\rm ]}
\shead{DESCRIPTION}
Reads a portable anymap as input.
Produces a Solitaire Image Recorder format.
\par
pnmtosir produces an MGI TYPE 17 file for
{\it pbm}
and
{\it pgm}
files.
For
{\it ppm}{\rm ,}
it writes a MGI TYPE 11 file.
\shead{SEE ALSO}
sirtopnm(1), pnm(5)
\shead{BUGS}

\shead{AUTHOR}
\copyright 1991 by Marvin Landis.
% Permission to use, copy, modify, and distribute this software and its
% documentation for any purpose and without fee is hereby granted, provided
% that the above copyright notice appear in all copies and that both that
% copyright notice and this permission notice appear in supporting
% documentation.  This software is provided "as is" without express or
% implied warranty.

%
% end of input file: pnmtosir.1
%--------------------------------------------------
 
% -*-LaTeX-*-
% Converted automatically from troff to LaTeX
% by tr2latex ($Revision: 2.2 $$Date: 1992/04/27 15:13:26 $ by C. Engel)
% on Fri Oct  8 16:09:36 1993
% tr2latex was written by Kamal Al-Yahya at Stanford University
% (Kamal%Hanauma@SU-SCORE.ARPA)
% and substantially enhanced by Christian Engel at RWTH Aachen
% (krischan@informatik.rwth-aachen.de).
%
% troff input file: pnmtotiff.1

\newpage
%--------------------------------------------------
% start of input file: pnmtotiff.1
%
\phead{pnmtotiff}{1}{13 January 1991}{}{}

%.IX pnmtotiff
\shead{NAME}
pnmtotiff - convert a a portable anymap into a TIFF file
\shead{SYNOPSIS}
{\bf pnmtotiff}
{\rm [}{\bf -none}{\rm $|$}{\bf -packbits}{\rm $|$}
{\bf -lzw}{\rm $|$}{\bf -g3}{\rm $|$}{\bf -g4}{\rm ]}
{\rm [}{\bf -2d}{\rm ]}
{\rm [}{\bf -fill}{\rm ]}
{\rm [}{\bf -predictor}
{\it n}{\rm ]}
{\rm [}{\bf -msb2lsb}{\rm $|$}{\bf -lsb2msb}{\rm ]}
{\rm [}{\bf -rowsperstrip}
{\it n}{\rm ]}
{\rm [}{\it pnmfile}{\rm ]}
\shead{DESCRIPTION}
Reads a portable anymap as input.
Produces a TIFF file as output.
%.IX TIFF
\shead{OPTIONS}
\par
By default,
{\it pnmtotiff}
creates a TIFF file with LZW compression.
This is your best bet most of the time.
However, some TIFF readers can't deal with it.
If you want to try another compression scheme or tweak some of the
other even more obscure output options, there are a number of
flags to play with.
\par
The
{\bf -none}{\rm ,}
{\bf -packbits}{\rm ,}
{\bf -lzw}{\rm ,}
{\bf -g3}{\rm ,}
and
{\bf -g4}
options are used to override the default and set the compression
scheme used in creating the output file.  The CCITT Group 3 and Group
4 compression algorithms can only be used with bilevel data.  The
{\bf -2d}
and
{\bf -fill}
options are meaningful only with Group 3 compression:
{\bf -2d}
requests 2-dimensional encoding, while
{\bf -fill}
requests that each encoded scanline be zero-filled to a byte boundry.
The
{\bf -predictor}
option is only meaningful with LZW compression: a predictor value of 2
causes each scanline of the output image to undergo horizontal
differencing before it is encoded; a value of 1 forces each scanline
to be encoded without differencing.
\rm
By default,
{\it pnmtotiff}
creates a TIFF file with msb-to-lsb fill order.
The
{\bf -msb2lsb}
and
{\bf -lsb2msb}
options are used to override the default and set the fill order used
in creating the file.
\rm
The
{\bf -rowsperstrip}
option can be used to set the number of rows (scanlines) in each
strip of data in the output file.  By default, the output file has
the number of rows per strip set to a value that will ensure each
strip is no more than 8 kilobytes long.
\shead{BUGS}
This program is not self-contained.  To use it you must fetch the
TIFF Software package listed in the OTHER.SYSTEMS file and configure
PBMPLUS to use libtiff.  See PBMPLUS's Makefile for details on this
configuration.
\shead{SEE ALSO}
tifftopnm(1), pnm(5)
\shead{AUTHOR}
Derived by Jef Poskanzer from ras2tiff.c, which is
\copyright 1990 by Sun Microsystems, Inc.\hfil\break
Author: Patrick J. Naughton (naughton@wind.sun.com).
% Permission to use, copy, modify, and distribute this software and its
% documentation for any purpose and without fee is hereby granted,
% provided that the above copyright notice appear in all copies and that
% both that copyright notice and this permission notice appear in
% supporting documentation.
% 
% This file is provided AS IS with no warranties of any kind.  The author
% shall have no liability with respect to the infringement of copyrights,
% trade secrets or any patents by this file or any part thereof.  In no
% event will the author be liable for any lost revenue or profits or
% other special, indirect and consequential damages.
%
% end of input file: pnmtotiff.1
%--------------------------------------------------
 
% -*-LaTeX-*-
% Converted automatically from troff to LaTeX
% by tr2latex ($Revision: 2.2 $$Date: 1992/04/27 15:13:26 $ by C. Engel)
% on Fri Oct  8 16:09:26 1993
% tr2latex was written by Kamal Al-Yahya at Stanford University
% (Kamal%Hanauma@SU-SCORE.ARPA)
% and substantially enhanced by Christian Engel at RWTH Aachen
% (krischan@informatik.rwth-aachen.de).
%
% troff input file: pnmtoxwd.1

\newpage
%--------------------------------------------------
% start of input file: pnmtoxwd.1
%
\phead{pnmtoxwd}{1}{24 September 1991}{}{}

%.IX pnmtoxwd
\shead{NAME}
pnmtoxwd - convert a portable anymap into an X11 window dump
\shead{SYNOPSIS}
{\bf pnmtoxwd}
{\rm [}{\bf -pseudodepth}
{\it n}{\rm ]}
{\rm [}{\bf -directcolor}{\rm ]}
{\rm [}{\it pnmfile}{\rm ]}
\shead{DESCRIPTION}
Reads a portable anymap as input.
Produces an X11 window dump as output.
%.IX XWD
%.IX "X window system"
This window dump can be displayed using the xwud tool.
\par
Normally, pnmtoxwd produces a StaticGray dump file for
{\it pbm}
and
{\it pgm}
files.
For
{\it ppm}{\rm ,}
it writes a PseudoColor dump file if there are up
to 256 colors in the input, and a DirectColor dump file otherwise.
The
{\bf -directcolor}
flag can be used to force a DirectColor dump.
And the
{\bf -pseudodepth}
flag can be used to change the depth of PseudoColor dumps from the default
of 8 bits / 256 colors.
\shead{SEE ALSO}
xwdtopnm(1), pnm(5), xwud(1)
\shead{AUTHOR}
\copyright 1989, 1991 by Jef Poskanzer.
% Permission to use, copy, modify, and distribute this software and its
% documentation for any purpose and without fee is hereby granted, provided
% that the above copyright notice appear in all copies and that both that
% copyright notice and this permission notice appear in supporting
% documentation.  This software is provided "as is" without express or
% implied warranty.
%
% end of input file: pnmtoxwd.1
%--------------------------------------------------
 
% -*-LaTeX-*-
% Converted automatically from troff to LaTeX
% by tr2latex ($Revision: 2.2 $$Date: 1992/04/27 15:13:26 $ by C. Engel)
% on Fri Oct  8 16:09:13 1993
% tr2latex was written by Kamal Al-Yahya at Stanford University
% (Kamal%Hanauma@SU-SCORE.ARPA)
% and substantially enhanced by Christian Engel at RWTH Aachen
% (krischan@informatik.rwth-aachen.de).
%
% troff input file: ppm.5

\newpage
%--------------------------------------------------
% start of input file: ppm.5
%
\phead{ppm}{5}{27 September 1991}{}{}

\shead{NAME}
ppm - portable pixmap file format
\shead{DESCRIPTION}
The portable pixmap format is a lowest common denominator color image
file format.
%.IX "PPM file format"
The definition is as follows:
\begin{IPlist}
\IPitem{{-}}
A ``magic number'' for identifying the file type.
A ppm file's magic number is the two characters ``P3''.
%.IX "magic numbers"
\IPitem{{-}}
Whitespace (blanks, TABs, CRs, LFs).
\IPitem{{-}}
A width, formatted as ASCII characters in decimal.
\IPitem{{-}}
Whitespace.
\IPitem{{-}}
A height, again in ASCII decimal.
\IPitem{{-}}
Whitespace.
\IPitem{{-}}
The maximum color-component value, again in ASCII decimal.
\IPitem{{-}}
Whitespace.
\IPitem{{-}}
Width * height pixels, each three ASCII decimal values between 0 and the
specified maximum value, starting at the top-left
corner of the pixmap, proceeding in normal English reading order.
The three values for each pixel represent red, green, and blue, respectively;
a value of 0 means that color is off, and the maximum value means that color
is maxxed out.
\IPitem{{-}}
Characters from a ``\#'' to the next end-of-line are ignored (comments).
\IPitem{{-}}
No line should be longer than 70 characters.
\end{IPlist}

\par
Here is an example of a small pixmap in this format:
\nofill
P3
\# feep.ppm
4 4
15
 0  0  0    0  0  0    0  0  0   15  0 15
 0  0  0    0 15  7    0  0  0    0  0  0
 0  0  0    0  0  0    0 15  7    0  0  0
15  0 15    0  0  0    0  0  0    0  0  0
\fill
\par
Programs that read this format should be as lenient as possible,
accepting anything that looks remotely like a pixmap.
\par
There is also a variant on the format, available
by setting the RAWBITS option at compile time.  This variant is
different in the following ways:
%.IX RAWBITS
\begin{IPlist}
\IPitem{{-}}
The ``magic number'' is ``P6'' instead of ``P3''.
\IPitem{{-}}
The pixel values are stored as plain bytes, instead of ASCII decimal.
\IPitem{{-}}
Whitespace is not allowed in the pixels area, and only a single character
of whitespace (typically a newline) is allowed after the maxval.
\IPitem{{-}}
The files are smaller and many times faster to read and write.
\end{IPlist}

\par
Note that this raw format can only be used for maxvals less than
or equal to 255.
If you use the
{\it ppm}
library and try to write a file with a larger maxval,
it will automatically fall back on the slower but more general plain
format.
\shead{SEE ALSO}
giftoppm(1), gouldtoppm(1), ilbmtoppm(1), imgtoppm(1), mtvtoppm(1),
pcxtoppm(1), pgmtoppm(1), pi1toppm(1), picttoppm(1), pjtoppm(1), qrttoppm(1),
rawtoppm(1), rgb3toppm(1), sldtoppm(1), spctoppm(1), sputoppm(1), tgatoppm(1),
ximtoppm(1), xpmtoppm(1), yuvtoppm(1),
ppmtoacad(1), ppmtogif(1), ppmtoicr(1), ppmtoilbm(1), ppmtopcx(1), ppmtopgm(1),
ppmtopi1(1), ppmtopict(1), ppmtopj(1), ppmtopuzz(1), ppmtorgb3(1),
ppmtosixel(1), ppmtotga(1), ppmtouil(1), ppmtoxpm(1), ppmtoyuv(1),
ppmdither(1), ppmforge(1), ppmhist(1), ppmmake(1), ppmpat(1), ppmquant(1),
ppmquantall(1), ppmrelief(1),
pnm(5), pgm(5), pbm(5)
\shead{AUTHOR}
\copyright 1989, 1991 by Jef Poskanzer.
% Permission to use, copy, modify, and distribute this software and its
% documentation for any purpose and without fee is hereby granted, provided
% that the above copyright notice appear in all copies and that both that
% copyright notice and this permission notice appear in supporting
% documentation.  This software is provided "as is" without express or
% implied warranty.
%
% end of input file: ppm.5
%--------------------------------------------------
 
% -*-LaTeX-*-
% Converted automatically from troff to LaTeX
% by tr2latex ($Revision: 2.2 $$Date: 1992/04/27 15:13:26 $ by C. Engel)
% on Fri Oct  8 16:09:08 1993
% tr2latex was written by Kamal Al-Yahya at Stanford University
% (Kamal%Hanauma@SU-SCORE.ARPA)
% and substantially enhanced by Christian Engel at RWTH Aachen
% (krischan@informatik.rwth-aachen.de).
%
% troff input file: ppmbrighten.1

\newpage
%--------------------------------------------------
% start of input file: ppmbrighten.1
%
\phead{ppmbrighten}{1}{20 Nov 1990}{}{}

\shead{NAME}
ppmbrighten - change an images Saturation and Value from an HSV map
\shead{SYNOPSIS}
ppmbrighten [-n] [-s $<$+- saturation$>$] [-v $<$+- value$>$] $<$ppmfile$>$
\shead{DESCRIPTION}
Reads a portable pixmap as input.
Converts the image from RGB space to HSV space and changes
the Value by $<$+- value$>$ as a percentage.
Likewise with the Saturation.
Doubling the Value would involve
\par\vspace{1.0\baselineskip}
ppmbrighten -v 100
\par\vspace{1.0\baselineskip}
to add 100 percent to the Value.
\par
The 'n' option normalizes the Value to exist between 0 and 1
(normalized).
\shead{SEE ALSO}
pgmnorm(1), ppm(5)
\shead{AUTHOR}
\copyright 1990 by Brian Moffet.
\copyright 1989 by Jef Poskanzer.

Permission to use, copy, modify, and distribute this software and its
documentation for any purpose and without fee is hereby granted, provided
that the above copyright notice appear in all copies and that both that
copyright notice and this permission notice appear in supporting
documentation.  This software is provided "as is" without express or
implied warranty.
\shead{NOTES}
This program does not change the number of colors.
%
% end of input file: ppmbrighten.1
%--------------------------------------------------
 
% -*-LaTeX-*-
% Converted automatically from troff to LaTeX
% by tr2latex ($Revision: 2.2 $$Date: 1992/04/27 15:13:26 $ by C. Engel)
% on Mon Feb  7 08:50:59 1994
% tr2latex was written by Kamal Al-Yahya at Stanford University
% (Kamal%Hanauma@SU-SCORE.ARPA)
% and substantially enhanced by Christian Engel at RWTH Aachen
% (krischan@informatik.rwth-aachen.de).
%
% troff input file: ppmchange.1

\newpage
%--------------------------------------------------
% start of input file: ppmchange.1
%
\phead{ppmchange}{1}{3 December 1993}{}{}

\shead{NAME}
ppmchange - change all pixels of one color to another in a portable pixmap
\shead{SYNOPSIS}
{\bf ppmchange}
{\it oldcolor newcolor [...]}
{\rm [}{\it ppmfile}{\rm ]}
\shead{DESCRIPTION}
Reads a portable pixmap as input.
Changes all pixels of 
{\it oldcolor}
to 
{\it newcolor}{\rm ,}
leaving all others unchanged.
Up to 256 colors may be replaced by specifying couples of colors on
the command line.  
\par
The colors can be specified in five ways:
\begin{TPlist}{o}
\item[{o}]
A name, assuming
that a pointer to an X11-style color names file was compiled in.
\item[{o}]
An X11-style hexadecimal specifier: rgb:r/g/b, where r g and b are
each 1- to 4-digit hexadecimal numbers.
\item[{o}]
An X11-style decimal specifier: rgbi:r/g/b, where r g and b are
floating point numbers between 0 and 1.
\item[{o}]
For backwards compatibility, an old-X11-style hexadecimal
number: \#rgb, \#rrggbb, \#rrrgggbbb, or \#rrrrggggbbbb.
\item[{o}]
For backwards compatibility, a triplet of numbers
separated by commas: r,g,b, where r g and b are
floating point numbers between 0 and 1.
(This style was added before MIT came up with the similar rgbi style.)
\end{TPlist}
\shead{SEE ALSO}
pgmtoppm(1), ppm(5)
\shead{AUTHOR}
Wilson H. Bent. Jr. (whb@usc.edu)
with modifications by Alberto Accomazzi (alberto@cfa.harvard.edu)
%
% end of input file: ppmchange.1
%--------------------------------------------------

% -*-LaTeX-*-
% Converted automatically from troff to LaTeX
% by tr2latex ($Revision: 2.2 $$Date: 1992/04/27 15:13:26 $ by C. Engel)
% on Mon Nov 29 13:33:23 1993
% tr2latex was written by Kamal Al-Yahya at Stanford University
% (Kamal%Hanauma@SU-SCORE.ARPA)
% and substantially enhanced by Christian Engel at RWTH Aachen
% (krischan@informatik.rwth-aachen.de).
%
% troff input file: ppmdim.1

\newpage
%--------------------------------------------------
% start of input file: ppmdim.1
%
\phead{ppmdim}{1}{16 November 1993}{}{}

%.IX ppmdim
\shead{NAME}
ppmdim - dim a portable pixmap down to total blackness
\shead{SYNOPSIS}
ppmdim
{\it dimfactor}
{\rm [}{\it ppmfile}{\rm ]}
\shead{DESCRIPTION}
Reads a portable pixmap as input. Diminishes its brightness by
the specified dimfactor down to total blackness.
The dimfactor may be in the range from 0.0 (total blackness,
deep night, nada, null, nothing) to 1.0 (original picture's
brightness).
\par
As
{\it pnmgamma}
does not do the brightness correction in the way I
wanted it, this small program was written.
\par
ppmdim is similar to
{\it ppmbrighten}
, but not exactly the same.
\shead{SEE ALSO}
ppm(5), ppmflash(1), pnmgamma(1), ppmbrighten(1)
\shead{AUTHOR}
Copyright (C) 1993 by Frank Neumann
% Permission to use, copy, modify, and distribute this software and its
% documentation for any purpose and without fee is hereby granted, provided
% that the above copyright notice appear in all copies and that both that
% copyright notice and this permission notice appear in supporting
% documentation.  This software is provided "as is" without express or
% implied warranty.
%
% end of input file: ppmdim.1
%--------------------------------------------------

% -*-LaTeX-*-
% Converted automatically from troff to LaTeX
% by tr2latex ($Revision: 2.2 $$Date: 1992/04/27 15:13:26 $ by C. Engel)
% on Fri Oct  8 16:09:08 1993
% tr2latex was written by Kamal Al-Yahya at Stanford University
% (Kamal%Hanauma@SU-SCORE.ARPA)
% and substantially enhanced by Christian Engel at RWTH Aachen
% (krischan@informatik.rwth-aachen.de).
%
% troff input file: ppmdist.1

\newpage
%--------------------------------------------------
% start of input file: ppmdist.1
%
\phead{ppmdist}{1}{22 July 1992}{}{}

%.IX ppmdist
\shead{NAME}
ppmdist - simplistic grayscale assignment for machine generated, color images
\shead{SYNOPSIS}
{\bf ppmdist}
{\rm [}{\bf -intensity}{\rm $|$}{\bf -frequency}{\rm ]}
{\rm [}{\it ppmfile}{\rm ]}
\shead{DESCRIPTION}
Reads a portable pixmap as input, performs a simplistic grayscale
assignment intended for use with grayscale or bitmap printers.

Often conversion from ppm to pgm will yield an image with contrast too
low for good printer output.  The program maximizes contrast between
the gray levels output.

A ppm input of n colors is read, and a pgm of n gray levels is written.
The gray levels take on the values 0..n-1, while maxval takes on n-1.

The mapping from color to stepped grayscale can be performed in order
of input pixel intensity, or input pixel frequency (number of repetitions).
\shead{OPTIONS}
\begin{TPlist}{}
\item[{}]
{\bf -frequency}
Sort input colors by the number of times a color appears in the input,
before mapping to evenly distributed graylevels of output.
{\bf -intensity}
Sort input colors by their grayscale intensity, before mapping to evenly
distributed graylevels of output.  This is the default.
\end{TPlist}

\shead{BUGS}
Helpful only for images with a very small number of colors.
Perhaps should have been an option to ppmtopgm(1).
\shead{SEE ALSO}
ppmtopgm(1), ppmhist(1), ppm(5)
\shead{AUTHOR}
\copyright 1993 by Dan Stromberg.
% Permission to use, copy, modify, and distribute this software and its
% documentation for any purpose and without fee is hereby granted, provided
% that the above copyright notice appear in all copies and that both that
% copyright notice and this permission notice appear in supporting
% documentation.  This software is provided "as is" without express or
% implied warranty.
%
% end of input file: ppmdist.1
%--------------------------------------------------
 
% -*-LaTeX-*-
% Converted automatically from troff to LaTeX
% by tr2latex ($Revision: 2.2 $$Date: 1992/04/27 15:13:26 $ by C. Engel)
% on Fri Oct  8 16:08:45 1993
% tr2latex was written by Kamal Al-Yahya at Stanford University
% (Kamal%Hanauma@SU-SCORE.ARPA)
% and substantially enhanced by Christian Engel at RWTH Aachen
% (krischan@informatik.rwth-aachen.de).
%
% troff input file: ppmdither.1

\newpage
%--------------------------------------------------
% start of input file: ppmdither.1
%
\phead{ppmdither}{1}{14 July 1991}{}{}

%.IX ppmdither
\shead{NAME}
ppmdither - ordered dither for color images
\shead{SYNOPSIS}
{\bf ppmdither}
{\rm [}{\bf -dim}
{\it dimension}{\rm ]}
{\rm [}{\bf -red}
{\it shades}{\rm ]}
{\rm [}{\bf -green}
{\it shades}{\rm ]}
{\rm [}{\bf -blue}
{\it shades}{\rm ]}
{\rm [}{\it ppmfile}{\rm ]}
\shead{DESCRIPTION}
Reads a portable pixmap as input, and applies dithering to it to reduce
the number of colors used down to the specified number of shades for
each primary.
The default number of shades is red=5, green=9, blue=5, for
a total of 225 colors.
To convert the image to a binary rgb format 
suitable for color printers, use -red 2 -green 2 -blue 2.
The maximum
number of colors that can be used is 256 and can be computed as the 
product of the number of red, green and blue shades.
\shead{OPTIONS}
\begin{TPlist}{{\bf -dim}{\it \ dimension}
}
\item[{{\bf -dim}{\it \ dimension}
}]
The size of the dithering matrix.
Must be a power of 2.
\item[{{\bf -red}{\it \ shades}
}]
The number of red shades to be used; minimum of 2.
\item[{{\bf -green}{\it \ shades}
}]
The number of green shades to be used; minimum of 2.
\item[{{\bf -blue}{\it \ shades}
}]
The number of blue shades to be used; minimum of 2.
\end{TPlist}

\shead{SEE ALSO}
pnmdepth(1), ppmquant(1), ppm(5)
\shead{AUTHOR}
\copyright 1991 by Christos Zoulas.
% Permission to use, copy, modify, and distribute this software and its
% documentation for any purpose and without fee is hereby granted, provided
% that the above copyright notice appear in all copies and that both that
% copyright notice and this permission notice appear in supporting
% documentation.  This software is provided "as is" without express or
% implied warranty.
%
% end of input file: ppmdither.1
%--------------------------------------------------
 
% -*-LaTeX-*-
% Converted automatically from troff to LaTeX
% by tr2latex ($Revision: 2.2 $$Date: 1992/04/27 15:13:26 $ by C. Engel)
% on Mon Nov 29 13:33:29 1993
% tr2latex was written by Kamal Al-Yahya at Stanford University
% (Kamal%Hanauma@SU-SCORE.ARPA)
% and substantially enhanced by Christian Engel at RWTH Aachen
% (krischan@informatik.rwth-aachen.de).
%
% troff input file: ppmflash.1

\newpage
%--------------------------------------------------
% start of input file: ppmflash.1
%
\phead{ppmflash}{1}{16 November 1993}{}{}

%.IX ppmflash
\shead{NAME}
ppmflash - brighten a picture up to complete white-out
\shead{SYNOPSIS}
ppmflash 
{\it flashfactor}
{\rm [}{\it ppmfile}{\rm ]}
\shead{DESCRIPTION}
Reads a portable pixmap as input. Increases its brightness by
the specified flashfactor up to a total white-out image.
The flashfactor may be in the range from 0.0 (original picture's
brightness) to 1.0 (full white-out, The Second After).
\par
As
{\it pnmgamma}
does not do the brightness correction in the way I
wanted it, this small program was written.
\par
This program is similar to
{\it ppmbrighten}
, but not exactly the same.
\shead{SEE ALSO}
ppm(5), ppmdim(1), pnmgamma(1), ppmbrighten(1)
\shead{AUTHOR}
Copyright (C) 1993 by Frank Neumann
% Permission to use, copy, modify, and distribute this software and its
% documentation for any purpose and without fee is hereby granted, provided
% that the above copyright notice appear in all copies and that both that
% copyright notice and this permission notice appear in supporting
% documentation.  This software is provided "as is" without express or
% implied warranty.
%
% end of input file: ppmflash.1
%--------------------------------------------------

% -*-LaTeX-*-
% Converted automatically from troff to LaTeX
% by tr2latex ($Revision: 2.2 $$Date: 1992/04/27 15:13:26 $ by C. Engel)
% on Fri Oct  8 16:08:46 1993
% tr2latex was written by Kamal Al-Yahya at Stanford University
% (Kamal%Hanauma@SU-SCORE.ARPA)
% and substantially enhanced by Christian Engel at RWTH Aachen
% (krischan@informatik.rwth-aachen.de).
%
% troff input file: ppmforge.1

\newpage
%--------------------------------------------------
% start of input file: ppmforge.1
%
\phead{ppmforge}{1}{25 October 1991}{}{}

%.IX ppmforge
%.IX fractals
%.IX clouds
%.IX planets
%.IX stars
\shead{NAME}
ppmforge - fractal forgeries of clouds, planets, and starry skies
\shead{SYNOPSIS}
\raggedright
%.nh
{\bf ppmforge}
{\rm [}{\bf -clouds}{\rm ]}
'in +9n
{\rm [}{\bf -night}{\rm ]}
{\rm [}{\bf -dimension}
{\it dimen}{\rm ]}
{\rm [}{\bf -hour}
{\it hour}{\rm ]}
{\rm [}{\bf -inclination$|$-tilt}
{\it angle}{\rm ]}
{\rm [}{\bf -mesh}
{\it size}{\rm ]}
{\rm [}{\bf -power}
{\it factor}{\rm ]}
{\rm [}{\bf -glaciers}
{\it level}{\rm ]}
{\rm [}{\bf -ice}
{\it level}{\rm ]}
{\rm [}{\bf -saturation}
{\it sat}{\rm ]}
{\rm [}{\bf -seed}
{\it seed}{\rm ]}
{\rm [}{\bf -stars}
{\it fraction}{\rm ]}
{\rm [}{\bf -xsize$|$-width}
{\it width}{\rm ]}
{\rm [}{\bf -ysize$|$-height}
{\it height}{\rm ]}
\ind{-4.5em}
%.ad
%.hy
\shead{DESCRIPTION}
{\bf ppmforge}
generates three kinds of ``random fractal forgeries,'' the term coined
by Richard F. Voss of the IBM Thomas J. Watson Research Center for
seemingly realistic pictures of natural objects generated by simple
algorithms embodying randomness and fractal self-similarity.  The
techniques used by
{\bf ppmforge}
are essentially those
given by Voss[1], particularly the technique of spectral synthesis
explained in more detail by Dietmar Saupe[2].
\par
The program generates two varieties of pictures: planets and clouds,
which are just different renderings of data generated in an identical
manner, illustrating the unity of the fractal structure of these very
different objects.  A third type of picture, a starry sky, is
synthesised directly from pseudorandom numbers.
\par
The generation of planets or clouds begins with the preparation of an
array of random data in the frequency domain.  The size of this
array, the ``mesh size,'' can be set with the
{\bf -mesh}
option; the larger the mesh the more realistic the pictures but the
calculation time and memory requirement increases as the square of the
mesh size.  The fractal dimension, which you can specify with the
{\bf -dimension}
option, determines the roughness of the terrain on the planet or the
scale of detail in the clouds.  As the fractal dimension is increased,
more high frequency components are added into the random mesh.
\par
Once the mesh is generated, an inverse two dimensional Fourier
transform is performed upon it.  This converts the original random
frequency domain data into spatial amplitudes.  We scale the real
components that result from the Fourier transform into numbers from 0
to 1 associated with each point on the mesh.  You can further
modify this number by applying a ``power law scale'' to it with the
{\bf -power}
option.   Unity scale
leaves the numbers unmodified; a power scale of 0.5 takes the square
root of the numbers in the mesh, while a power scale of 3 replaces the
numbers in the mesh with their cubes.  Power law scaling is best
envisioned by thinking of the data as representing the elevation of
terrain; powers less than 1 yield landscapes with vertical scarps that
look like glacially-carved valleys; powers greater than one make
fairy-castle spires (which require large mesh sizes and high
resolution for best results).
\par
After these calculations, we have a array of the specified size
containing numbers that range from 0 to 1.  The pixmaps are generated as
follows:
\begin{TPlist}{{\bf Clouds}}
\item[{{\bf Clouds}}]
A colour map is created that ranges from pure blue to white by
increasing admixture (desaturation) of blue with white.  Numbers less
than 0.5 are coloured blue, numbers between 0.5 and 1.0 are coloured
with corresponding levels of white, with 1.0 being pure white.
\item[{{\bf Planet}}]
The mesh is projected onto a sphere.  Values less than 0.5 are treated
as water and values between 0.5 and 1.0 as land.  The water areas are
coloured based upon the water depth, and land based on its elevation.
The random depth data are used to create clouds over the oceans.  An
atmosphere approximately like the Earth's is simulated; its light
absorption is calculated to create a blue cast around the limb of the
planet.  A function that rises from 0 to 1 based on latitude is
modulated by the local elevation to generate polar ice caps--high
altitude terrain carries glaciers farther from the pole.  Based on the
position of the star with respect to the observer, the apparent colour
of each pixel of the planet is calculated by ray-tracing from the star
to the planet to the observer and applying a lighting model that sums
ambient light and diffuse reflection (for most planets ambient light
is zero, as their primary star is the only source of illumination).
Additional random data are used to generate stars around the planet.
\item[{{\bf Night}}]
A sequence of pseudorandom numbers is used to generate stars with a
user specified density.
\end{TPlist}

\par
Cloud pictures always contain 256 or fewer colours and may be
displayed on most colour mapped devices without further processing.
Planet pictures often contain tens of thousands of colours which
must be compressed with
{\bf ppmquant}
or
{\bf ppmdither}
before encoding in a colour mapped format.  If the display resolution is
high enough,
{\bf ppmdither}
generally produces better looking planets.
{\bf ppmquant}
tends to create discrete colour bands, particularly in the oceans,
which are unrealistic and distracting.  The number of colours in starry
sky pictures generated with the
{\bf -night}
option depends on the value specified for
{\bf -saturation}{\rm .}
Small values limit the colour temperature distribution of the stars
and reduce the number of colours in the image.
If the
{\bf -saturation}
is set to 0, none of the stars will be coloured and the resulting
image will never contain more than 256 colours.
Night sky pictures with many different star colours often look
best when colour compressed by
{\bf pnmdepth}
rather than
{\bf ppmquant}
or
{\bf ppmdither}{\rm .}
Try
{\it newmaxval}
settings of 63, 31, or 15 with
{\bf pnmdepth}
to reduce the number of colours in the picture to 256 or fewer.
\shead{OPTIONS}
\begin{TPlist}{{\bf -clouds}}
\item[{{\bf -clouds}}]
Generate clouds.  A pixmap of fractal clouds is generated.  Selecting clouds
sets the default for fractal dimension to 2.15 and power scale factor
to 0.75.
\item[{{\bf -dimension}{\it \ dimen}
}]
Sets the fractal dimension to the specified
{\it dimen}{\rm ,}
which may be any floating point value between 0 and 3.  Higher fractal
dimensions create more ``chaotic'' images, which require higher
resolution output and a larger FFT mesh size to look good.  If no
dimension is specified, 2.4 is used when generating planets and 2.15
for clouds.
\item[{{\bf -glaciers}{\it \ level}
}]
The floating point
{\it level}
setting controls the extent to which terrain elevation causes ice to
appear at lower latitudes.  The default value of 0.75 makes the polar
caps extend toward the equator across high terrain and forms glaciers
in the highest mountains, as on Earth.  Higher values make ice sheets
that cover more and more of the land surface, simulating planets in the
midst of an ice age.  Lower values tend to be boring, resulting in
unrealistic geometrically-precise ice cap boundaries.
\item[{{\bf -hour}{\it \ hour}
}]
When generating a planet,
{\it hour}
is used as the ``hour angle at the central meridian.''  If you specify
{\bf -hour\ 12}{\rm ,}
for example, the planet will be fully illuminated, corresponding to
high noon at the longitude at the centre of the screen.  You can
specify any floating point value between 0 and 24 for
{\it hour}{\rm ,}
but values which place most of the planet in darkness (0 to 4 and 20
to 24) result in crescents which, while pretty, don't give you many
illuminated pixels for the amount of computing that's required.  If no
{\bf -hour}
option is specified, a random hour angle is chosen, biased so that
only 25\% of the images generated will be crescents.
\item[{{\bf -ice}{\it \ level}
}]
Sets the extent of the polar ice caps to the given floating point
{\it level}{\rm .}
The default level of 0.4 produces ice caps similar to those of the Earth.
Smaller values reduce the amount of ice, while larger
{\bf -ice}
settings create more prominent ice caps.  Sufficiently large values,
such as 100 or more, in conjunction with small settings for
{\bf -glaciers}
(try 0.1) create ``ice balls'' like Europa.
\item[{{\bf -inclination$|$-tilt}{\it \ angle}
}]
The inclination angle of the planet with regard to its primary star is
set to
{\it angle}{\rm ,}
which can be any floating point value from -90 to 90.  The inclination
angle can be thought of as specifying, in degrees, the ``season'' the
planet is presently experiencing or, more precisely, the latitude at
which the star transits the zenith at local noon.  If 0, the planet
is at equinox; the star is directly overhead at the equator.
Positive values represent summer in the northern hemisphere, negative
values summer in the southern hemisphere.  The Earth's inclination
angle, for example, is about 23.5 at the June solstice, 0 at the
equinoxes in March and September, and -23.5 at the December solstice.
If no inclination angle is specified, a random value between -21.6 and
21.6 degrees is chosen.
\item[{{\bf -mesh}{\it \ size}
}]
A mesh of
{\it size}{\rm \ by\ }{\it size}
will be used for the fast Fourier transform (FFT).  Note that memory
requirements and computation speed increase as the square of
{\it size}{\rm ;}
if you double the mesh size, the program will use four times the
memory and run four times as long.  The default mesh is 256x256, which
produces reasonably good looking pictures while using half a megabyte
for the 256x256 array of single precision complex numbers
required by the FFT.  On machines with limited memory capacity, you
may have to reduce the mesh size to avoid running out of RAM.
Increasing the mesh size produces better looking pictures; the
difference becomes particularly noticeable when generating high
resolution images with relatively high fractal dimensions (between 2.2
and 3).
\item[{{\bf -night}}]
A starry sky is generated.  The stars are created by the same algorithm
used for the stars that surround planet pictures, but the output
consists exclusively of stars.
\item[{{\bf -power}{\it \ factor}
}]
Sets the ``power factor'' used to scale elevations synthesised from
the FFT to
{\it factor}{\rm ,}
which can be any floating point number greater than zero.  If no
factor is specified a default of 1.2 is used if a planet is being
generated, or 0.75 if clouds are selected by the
{\bf -clouds}
option.  The result of the FFT image synthesis is an array of elevation
values between 0 and 1.  A non-unity power factor exponentiates each
of these elevations to the specified power.  For example, a power
factor of 2 squares each value, while a power factor of 0.5 replaces
each with its square root.  (Note that exponentiating values between 0
and 1 yields values that remain within that range.)  Power factors
less than 1 emphasise large-scale elevation changes at the expense of
small variations.  Power factors greater than 1 increase the roughness
of the terrain and, like high fractal dimensions, may require a larger
FFT mesh size and/or higher screen resolution to look good.
\item[{{\bf -saturation}{\it \ sat}
}]
Controls the degree of colour saturation of the stars that surround planet
pictures and fill starry skies created with the
{\bf -night}
option.  The default value of 125 creates stars which resemble the sky
as seen by the human eye from Earth's surface.  Stars are dim; only
the brightest activate the cones in the human retina, causing colour
to be perceived.  Higher values of
{\it sat}
approximate the appearance of stars from Earth orbit, where better
dark adaptation, absence of skyglow, and the concentration of light
from a given star onto a smaller area of the retina thanks to the lack
of atmospheric turbulence enhances the perception of colour.  Values
greater than 250 create ``science fiction'' skies that, while pretty,
don't occur in this universe.
\item[{\ }]
Thanks to the inverse square law combined with Nature's love of
mediocrity, there are many, many dim stars for every bright one.
This population relationship is accurately reflected in the skies
created by
{\bf ppmforge}{\rm .}
Dim, low mass stars live much longer than bright massive stars,
consequently there are many reddish stars for every blue giant.  This
relationship is preserved by
{\bf ppmforge}{\rm .}
You can reverse the proportion, simulating the sky as seen in a starburst
galaxy, by specifying a negative
{\it sat}
value.
\item[{{\bf -seed}{\it \ num}
}]
Sets the seed for the random number generator to the integer
{\it num}{\rm .}
The seed used to create each picture is displayed on standard output (unless
suppressed with the
{\bf -quiet}
option).  Pictures generated with the same seed will be identical.  If no
{\bf -seed}
is specified, a random seed derived from the date and time will be
chosen.  Specifying an explicit seed allows you to re-render a picture
you particularly like at a higher resolution or with different viewing
parameters.
\item[{{\bf -stars}{\it \ fraction}
}]
Specifies the percentage of pixels, in tenths of a percent, which will
appear as stars, either surrounding a planet or filling the entire
frame if
{\bf -night}
is specified.  The default
{\it fraction}
is 100.
\item[{{\bf -xsize$|$-width}{\it \ width}
}]
Sets the width of the generated image to
{\it width}
pixels.  The default width is 256 pixels.  Images must be at least as
wide as they are high; if a width less than the height is specified,
it will be increased to equal the height.  If you must have a long
skinny pixmap, make a square one with
{\bf ppmforge}{\rm ,}
then use
{\bf pnmcut}
to extract a portion of the shape and size you require.
\item[{{\bf -ysize$|$-height}{\it \ height}
}]
Sets the height of the generated image to
{\it height}
pixels.  The default height is 256 pixels.  If the height specified
exceeds the width, the width will be increased to equal the height.
\end{TPlist}

\par
All flags can be abbreviated to their shortest unique prefix.
\shead{BUGS}
\par
The algorithms require the output pixmap to be at least as wide as it
is high, and the width to be an even number of pixels.  These
constraints are enforced by increasing the size of the requested
pixmap if necessary.
\par
You may have to reduce the FFT mesh size on machines with 16 bit
integers and segmented pointer architectures.
\shead{SEE ALSO}
{\bf pnmcut}{\rm (1),}
{\bf pnmdepth}{\rm (1),}
{\bf ppmdither}{\rm (1),}
{\bf ppmquant}{\rm (1),}
{\bf ppm}{\rm (5)}
\begin{TPlist}{[1] }
\item[{[1] }]
Voss, Richard F., ``Random Fractal Forgeries,'' in Earnshaw
{\it et. al.}, Fundamental Algorithms for Computer Graphics, Berlin:
Springer-Verlag, 1985.
\item[{[2]}]
Peitgen, H.-O., and Saupe, D. eds., The Science Of Fractal Images,
New York: Springer Verlag, 1988.
%.ne 10
\end{TPlist}

\shead{AUTHOR}
\ind{1\parindent}{\nofill
    John Walker
    Autodesk SA
    Avenue des Champs-Montants 14b
    CH-2074 MARIN
    Suisse/Schweiz/Svizzera/Svizra/Switzerland
\fill}
\begin{TPlist}{Usenet:}
\item[{Usenet:}]
kelvin@Autodesk.com
\item[{Fax:}]
038/33 88 15
\item[{Voice:}]
038/33 76 33
\end{TPlist}

\par
Permission to use, copy, modify, and distribute this software and its
documentation for any purpose and without fee is hereby granted,
without any conditions or restrictions.  This software is provided ``as
is'' without express or implied warranty.
\par
{\bf PLUGWARE!}
If you like this kind of stuff, you may also enjoy ``James Gleick's
Chaos--The Software'' for MS-DOS, available for \$59.95 from your
local software store or directly from Autodesk, Inc., Attn: Science
Series, 2320 Marinship Way, Sausalito, CA 94965, USA.  Telephone:
(800) 688-2344 toll-free or, outside the U.S. (415) 332-2344 Ext
4886.  Fax: (415) 289-4718.  ``Chaos--The Software'' includes a more
comprehensive fractal forgery generator which creates
three-dimensional landscapes as well as clouds and planets, plus five
more modules which explore other aspects of Chaos.  The user guide of
more than 200 pages includes an introduction by James Gleick and
detailed explanations by Rudy Rucker of the mathematics and algorithms
used by each program.
%
% end of input file: ppmforge.1
%--------------------------------------------------
 
% -*-LaTeX-*-
% Converted automatically from troff to LaTeX
% by tr2latex ($Revision: 2.2 $$Date: 1992/04/27 15:13:26 $ by C. Engel)
% on Fri Oct  8 16:08:47 1993
% tr2latex was written by Kamal Al-Yahya at Stanford University
% (Kamal%Hanauma@SU-SCORE.ARPA)
% and substantially enhanced by Christian Engel at RWTH Aachen
% (krischan@informatik.rwth-aachen.de).
%
% troff input file: ppmhist.1

\newpage
%--------------------------------------------------
% start of input file: ppmhist.1
%
\phead{ppmhist}{1}{03 April 1989}{}{}

%.IX ppmhist
\shead{NAME}
ppmhist - print a histogram of a portable pixmap
\shead{SYNOPSIS}
{\bf ppmhist}
{\rm [}{\it ppmfile}{\rm ]}
\shead{DESCRIPTION}
Reads a portable pixmap as input.
Generates a histogram of the colors in the pixmap.
\shead{SEE ALSO}
ppm(5), pgmhist(1)
\shead{AUTHOR}
\copyright 1989 by Jef Poskanzer.
% Permission to use, copy, modify, and distribute this software and its
% documentation for any purpose and without fee is hereby granted, provided
% that the above copyright notice appear in all copies and that both that
% copyright notice and this permission notice appear in supporting
% documentation.  This software is provided "as is" without express or
% implied warranty.
%
% end of input file: ppmhist.1
%--------------------------------------------------
 
% -*-LaTeX-*-
% Converted automatically from troff to LaTeX
% by tr2latex ($Revision: 2.2 $$Date: 1992/04/27 15:13:26 $ by C. Engel)
% on Fri Oct  8 16:08:48 1993
% tr2latex was written by Kamal Al-Yahya at Stanford University
% (Kamal%Hanauma@SU-SCORE.ARPA)
% and substantially enhanced by Christian Engel at RWTH Aachen
% (krischan@informatik.rwth-aachen.de).
%
% troff input file: ppmmake.1

\newpage
%--------------------------------------------------
% start of input file: ppmmake.1
%
\phead{ppmmake}{1}{24 September 1991}{}{}

%.IX ppmmake
\shead{NAME}
ppmmake - create a pixmap of a specified size and color
\shead{SYNOPSIS}
{\bf ppmmake}
{\it color width height}
\shead{DESCRIPTION}
Produces a portable pixmap of the specified color, width, and height.
%.IX "generating pixmaps"
\par
The color can be specified in five ways:
%.IX "specifying colors"
\begin{TPlist}{o}
\item[{o}]
A name, assuming
that a pointer to an X11-style color names file was compiled in.
\item[{o}]
An X11-style hexadecimal specifier: rgb:r/g/b, where r g and b are
each 1- to 4-digit hexadecimal numbers.
\item[{o}]
An X11-style decimal specifier: rgbi:r/g/b, where r g and b are
floating point numbers between 0 and 1.
\item[{o}]
For backwards compatibility, an old-X11-style hexadecimal
number: \#rgb, \#rrggbb, \#rrrgggbbb, or \#rrrrggggbbbb.
\item[{o}]
For backwards compatibility, a triplet of numbers
separated by commas: r,g,b, where r g and b are
floating point numbers between 0 and 1.
(This style was added before MIT came up with the similar rgbi style.)
\end{TPlist}

\shead{SEE ALSO}
ppm(5), pbmmake(1)
\shead{AUTHOR}
\copyright 1991 by Jef Poskanzer.
% Permission to use, copy, modify, and distribute this software and its
% documentation for any purpose and without fee is hereby granted, provided
% that the above copyright notice appear in all copies and that both that
% copyright notice and this permission notice appear in supporting
% documentation.  This software is provided "as is" without express or
% implied warranty.
%
% end of input file: ppmmake.1
%--------------------------------------------------
 
% -*-LaTeX-*-
% Converted automatically from troff to LaTeX
% by tr2latex ($Revision: 2.2 $$Date: 1992/04/27 15:13:26 $ by C. Engel)
% on Mon Nov 29 13:23:43 1993
% tr2latex was written by Kamal Al-Yahya at Stanford University
% (Kamal%Hanauma@SU-SCORE.ARPA)
% and substantially enhanced by Christian Engel at RWTH Aachen
% (krischan@informatik.rwth-aachen.de).
%
% troff input file: ppmmix.1

\newpage
%--------------------------------------------------
% start of input file: ppmmix.1
%
\phead{ppmmix}{1}{16 November 1993}{}{}

%.IX ppmmix
\shead{NAME}
ppmmix - blend together two portable pixmaps
\shead{SYNOPSIS}
ppmmix 
{\it fadefactor}
{\it ppmfile1 ppmfile2}
\shead{DESCRIPTION}
Reads two portable pixmaps as input. Mixes them together using the
specified fade factor. The fade factor may be in the range from 0.0
(only ppmfile1's image data) to 1.0 (only ppmfile2's image data).
Anything in between gains a smooth blend between the two images.
\par
The two pixmaps must have the same size.
\shead{SEE ALSO}
ppm(5)
\shead{AUTHOR}
\copyright 1993 by Frank Neumann.
% Permission to use, , modify, and distribute this software and its
% documentation for any purpose and without fee is hereby granted, provided
% that the above copyright notice appear in all copies and that both that
% copyright notice and this permission notice appear in supporting
% documentation.  This software is provided "as is" without express or
% implied warranty.
%
% end of input file: ppmmix.1
%--------------------------------------------------

% -*-LaTeX-*-
% Converted automatically from troff to LaTeX
% by tr2latex ($Revision: 2.2 $$Date: 1992/04/27 15:13:26 $ by C. Engel)
% on Fri Oct  8 16:09:06 1993
% tr2latex was written by Kamal Al-Yahya at Stanford University
% (Kamal%Hanauma@SU-SCORE.ARPA)
% and substantially enhanced by Christian Engel at RWTH Aachen
% (krischan@informatik.rwth-aachen.de).
%
% troff input file: ppmpat.1

\newpage
%--------------------------------------------------
% start of input file: ppmpat.1
%
\phead{ppmpat}{1}{04 September 1989}{}{}

%.IX ppmpat
\shead{NAME}
ppmpat - make a pretty pixmap
\shead{SYNOPSIS}
{\bf ppmpat}
{\bf -gingham2}{\rm $|$}{\bf -g2}{\rm $|$}{\bf -gingham3}{\rm $|$}
{\bf -g3}{\rm $|$}{\bf -madras}{\rm $|$}{\bf -tartan}{\rm $|$}
{\bf -poles}{\rm $|$}{\bf -squig}{\rm $|$}{\bf -camo}{\rm $|$}
{\bf -anticamo}
{\it width height}
\shead{DESCRIPTION}
Produces a portable pixmap of the specified width and height,
with a pattern in it.
%.IX "generating pixmaps"
\par
This program is mainly to demonstrate use of the ppmdraw routines, a
simple but powerful drawing library.
See the ppmdraw.h include file for more info on using these routines.
Still, some of the patterns can be rather pretty.
If you have a color workstation, something like
{\bf ppmpat\ -squig\ 300\ 300 $|$ ppmquant\ 128}
should generate a nice background.
\shead{OPTIONS}
\par
The different flags specify various different pattern types:
\begin{TPlist}{{\bf -gingham2}}
\item[{{\bf -gingham2}}]
A gingham check pattern.  Can be tiled.
\item[{{\bf -gingham3}}]
A slightly more complicated gingham.  Can be tiled.
\item[{{\bf -madras}}]
A madras plaid.  Can be tiled.
\item[{{\bf -tartan}}]
A tartan plaid.  Can be tiled.
\item[{{\bf -poles}}]
Color gradients centered on randomly-placed poles.
May need to be run through
{\it ppmquant}{\rm .}
\item[{{\bf -squig}}]
Squiggley tubular pattern.  Can be tiled.
May need to be run through
{\it ppmquant}{\rm .}
\item[{{\bf -camo}}]
Camouflage pattern.
May need to be run through
{\it ppmquant}{\rm .}
\item[{{\bf -anticamo}}]
Anti-camouflage pattern - like -camo, but ultra-bright colors.
May need to be run through
{\it ppmquant}{\rm .}
\end{TPlist}

\par
All flags can be abbreviated to their shortest unique prefix.
\shead{REFERENCES}
Some of the patterns are from ``Designer's Guide to Color 3'' by Jeanne Allen.
\shead{SEE ALSO}
pnmtile(1), ppmquant(1), ppm(5)
\shead{AUTHOR}
\copyright 1989 by Jef Poskanzer.
% Permission to use, copy, modify, and distribute this software and its
% documentation for any purpose and without fee is hereby granted, provided
% that the above copyright notice appear in all copies and that both that
% copyright notice and this permission notice appear in supporting
% documentation.  This software is provided "as is" without express or
% implied warranty.
%
% end of input file: ppmpat.1
%--------------------------------------------------
 
% -*-LaTeX-*-
% Converted automatically from troff to LaTeX
% by tr2latex ($Revision: 2.2 $$Date: 1992/04/27 15:13:26 $ by C. Engel)
% on Fri Oct  8 16:08:48 1993
% tr2latex was written by Kamal Al-Yahya at Stanford University
% (Kamal%Hanauma@SU-SCORE.ARPA)
% and substantially enhanced by Christian Engel at RWTH Aachen
% (krischan@informatik.rwth-aachen.de).
%
% troff input file: ppmquant.1

\newpage
%--------------------------------------------------
% start of input file: ppmquant.1
%
\phead{ppmquant}{1}{12 January 1991}{}{}

%.IX ppmquant
\shead{NAME}
ppmquant - quantize the colors in a portable pixmap down to a specified number

\shead{SYNOPSIS}
{\bf ppmquant}
{\rm [}{\bf -floyd}{\rm $|$}{\bf -fs}{\rm ]}
{\it ncolors}
{\rm [}{\it ppmfile}{\rm ]}
\nwl
{\bf ppmquant}
{\rm [}{\bf -floyd}{\rm $|$}{\bf -fs}{\rm ]}
{\bf -map}
{\it mapfile}
{\rm [}{\it ppmfile}{\rm ]}

\shead{DESCRIPTION}
Reads a portable pixmap as input.
Chooses
{\it ncolors}
colors to best represent the image, maps the existing colors
to the new ones, and writes a portable pixmap as output.
%.IX "colormap reduction"
\par
The quantization method is Heckbert's ``median cut''.
%.IX "median cut"
\par
Alternately, you can skip the color-choosing step by
specifying your own set of colors with the
{\bf -map}
flag.  The
{\it mapfile}
is just a
{\it ppm}
file; it can be any shape, all that matters is the colors in it.
For instance, to quantize down to the 8-color IBM TTL color set, you
might use:
{\tt\nofill
    P3
    8 1
    255
      0   0   0
    255   0   0
      0 255   0
      0   0 255
    255 255   0
    255   0 255
      0 255 255
    255 255 255
\fill}

If you want to quantize one pixmap to use the colors in another one,
just use the second one as the mapfile.
You don't have to reduce it down to only one pixel of each color,
just use it as is.
\par
The
{\bf -floyd}{\rm /}{\bf -fs}
flag enables a Floyd-Steinberg error diffusion step.
%.IX Floyd-Steinberg
%.IX "error diffusion"
Floyd-Steinberg gives vastly better results on images where the unmodified
quantization has banding or other artifacts, especially when going to a
small number of colors such as the above IBM set.
However, it does take substantially more CPU time, so the default is off.
\par
All flags can be abbreviated to their shortest unique prefix.
\shead{REFERENCES}
``Color Image Quantization for Frame Buffer Display'' by Paul Heckbert,
SIGGRAPH '82 Proceedings, page 297.
\shead{SEE ALSO}
ppmquantall(1), pnmdepth(1), ppmdither(1), ppm(5)
\shead{AUTHOR}
\copyright 1989, 1991 by Jef Poskanzer.
% Permission to use, copy, modify, and distribute this software and its
% documentation for any purpose and without fee is hereby granted, provided
% that the above copyright notice appear in all copies and that both that
% copyright notice and this permission notice appear in supporting
% documentation.  This software is provided "as is" without express or
% implied warranty.
%
% end of input file: ppmquant.1
%--------------------------------------------------
 
% -*-LaTeX-*-
% Converted automatically from troff to LaTeX
% by tr2latex ($Revision: 2.2 $$Date: 1992/04/27 15:13:26 $ by C. Engel)
% on Mon Nov 29 13:33:41 1993
% tr2latex was written by Kamal Al-Yahya at Stanford University
% (Kamal%Hanauma@SU-SCORE.ARPA)
% and substantially enhanced by Christian Engel at RWTH Aachen
% (krischan@informatik.rwth-aachen.de).
%
% troff input file: ppmquantall.1

\newpage
%--------------------------------------------------
% start of input file: ppmquantall.1
%
\phead{ppmquantall}{1}{27 July 1990}{}{}

%.IX ppmquantall
\shead{NAME}
ppmquantall - run ppmquant on a bunch of files all at once, so they share a common colormap
\shead{SYNOPSIS}
{\bf ppmquantall}
{\it ncolors ppmfile}
{\rm ...}
\shead{DESCRIPTION}
Takes a bunch of portable pixmap as input.
Chooses
{\it ncolors}
colors to best represent all of the images, maps the
existing colors to the new ones, and
{\bf overwrites the input files}
with the new quantized versions.
%.IX "colormap reduction"
\par
Verbose explanation: Let's say you've got a dozen pixmaps that you want
to display on the screen all at the same time.  Your screen can only
display 256 different colors, but the pixmaps have a total of a thousand
or so different colors.  For a single pixmap you solve this problem with
{\it ppmquant}{\rm ;}
%.IX ppmquant
this script solves it for multiple pixmaps.  All it does is
concatenate them together into one big pixmap, run
{\it ppmquant}
on that, and then split it up into little pixmaps again.
\par
(Note that another way to solve this problem is to pre-select a set of
colors and then use
{\it ppmquant}{\rm 's}
{\bf -map}
option to separately quantize each pixmap to that set.)
\shead{SEE ALSO}
ppmquant(1), ppm(5)
\shead{BUGS}
It's a csh script.
Csh scripts are not portable to System V.
Scripts in general are not portable to non-Unix environments.
\shead{AUTHOR}
Copyright (C) 1991 by Jef Poskanzer.
% Permission to use, copy, modify, and distribute this software and its
% documentation for any purpose and without fee is hereby granted, provided
% that the above copyright notice appear in all copies and that both that
% copyright notice and this permission notice appear in supporting
% documentation.  This software is provided "as is" without express or
% implied warranty.
%
% end of input file: ppmquantall.1
%--------------------------------------------------

% -*-LaTeX-*-
% Converted automatically from troff to LaTeX
% by tr2latex ($Revision: 2.2 $$Date: 1992/04/27 15:13:26 $ by C. Engel)
% on Fri Oct  8 16:09:09 1993
% tr2latex was written by Kamal Al-Yahya at Stanford University
% (Kamal%Hanauma@SU-SCORE.ARPA)
% and substantially enhanced by Christian Engel at RWTH Aachen
% (krischan@informatik.rwth-aachen.de).
%
% troff input file: ppmqvga.1

\newpage
%--------------------------------------------------
% start of input file: ppmqvga.1
%
\phead{PPMQVGA}{1}{local}{}{}

'% Heading: name(sect)    center (paren)    name(sect)
\shead{NAME}
ppmqvga - 8 plane quantization
'% name [, name] ... \- brief description on a single line of \s-1INPUT\s+1
\shead{SYNOPSIS}
ppmqvga [ options ] [ input file ]
'% foo [ options ] files1 [ file2 [ ... ] ]
\shead{DESCRIPTION}
{\bf ppmqvga}
quantizes PPM files to 8 planes, with optional Floyd-Steinberg dithering.
Input is a PPM file from the file named, or standard input of no file is
provided.
%.SS Options
{\bf -d}
dither. Apply Floyd-Steinberg dithering to the data

{\bf -q}
quiet. Produces no progress reporting, and no terminal output unless
and error occurs.

{\bf -v}
verbose. Produces additional output describing the number of colors found,
and some information on the resulting mapping. May be repeated to generate
loads of internal table output, but generally only useful once.
\shead{EXAMPLES}
ppmqvga -d my\_image.ppm $|$ ppmtogif $>$my\_image.gif

tgatoppm zombie.tga $|$ ppmqvga $|$ ppmtotif $>$ zombie.tif
\shead{SEE ALSO}
ppmquant
\shead{DIAGNOSTICS}
Error messages if problems, various levels of optional progress reporting.
\shead{LIMITATIONS}
none known.
\shead{AUTHOR}
Original by Lyle Rains (lrains@netcom.com) as ppmq256 and ppmq256fs
combined, documented, and enhanced by Bill Davidsen (davidsen@crd.ge.com)
\shead{Copyright}
Copyright 1991,1992 by Bill Davidsen, all rights reserved.
The program and documentation may be freely distributed by anyone in source
or binary format. Please clearly note any changes.
%
% end of input file: ppmqvga.1
%--------------------------------------------------
 
% -*-LaTeX-*-
% Converted automatically from troff to LaTeX
% by tr2latex ($Revision: 2.2 $$Date: 1992/04/27 15:13:26 $ by C. Engel)
% on Fri Oct  8 16:08:49 1993
% tr2latex was written by Kamal Al-Yahya at Stanford University
% (Kamal%Hanauma@SU-SCORE.ARPA)
% and substantially enhanced by Christian Engel at RWTH Aachen
% (krischan@informatik.rwth-aachen.de).
%
% troff input file: ppmrelief.1

\newpage
%--------------------------------------------------
% start of input file: ppmrelief.1
%
\phead{ppmrelief}{1}{11 January 1991}{}{}

%.IX ppmrelief
\shead{NAME}
ppmrelief - run a Laplacian relief filter on a portable pixmap
\shead{SYNOPSIS}
{\bf ppmrelief}
{\rm [}{\it ppmfile}{\rm ]}
\shead{DESCRIPTION}
Reads a portable pixmap as input.
Does a Laplacian relief filter, and writes a portable pixmap as output.
%.IX "Laplacian relief"
\par
The Laplacian relief filter is described in ``Beyond Photography'' by Holzmann,
equation 3.19.
It's a sort of edge-detection.
%.IX "edge detection"
\shead{SEE ALSO}
pgmbentley(1), pgmoil(1), ppm(5)
\shead{AUTHOR}
\copyright 1990 by Wilson Bent (whb@hoh-2.att.com).
% Permission to use, copy, modify, and distribute this software and its
% documentation for any purpose and without fee is hereby granted, provided
% that the above copyright notice appear in all copies and that both that
% copyright notice and this permission notice appear in supporting
% documentation.  This software is provided "as is" without express or
% implied warranty.
%
% end of input file: ppmrelief.1
%--------------------------------------------------
 
% -*-LaTeX-*-
% Converted automatically from troff to LaTeX
% by tr2latex ($Revision: 2.2 $$Date: 1992/04/27 15:13:26 $ by C. Engel)
% on Mon Nov 29 13:24:20 1993
% tr2latex was written by Kamal Al-Yahya at Stanford University
% (Kamal%Hanauma@SU-SCORE.ARPA)
% and substantially enhanced by Christian Engel at RWTH Aachen
% (krischan@informatik.rwth-aachen.de).
%
% troff input file: ppmshift.1

\newpage
%--------------------------------------------------
% start of input file: ppmshift.1
%
\phead{ppmshift}{1}{16 November 1993}{}{}

%.IX ppmshift
\shead{NAME}
ppmshift - shift lines of a portable pixmap left or right by a random amount
\shead{SYNOPSIS}
ppmshift 
{\it shift}
{\rm [}{\it ppmfile}{\rm ]}
\shead{DESCRIPTION}
Reads a portable pixmap as input. Shifts every row of image data to the
left or right by a certain amount. The 'shift' parameter determines by how 
many pixels a row is to be shifted at most.
\par
Another one of those effects I intended to use for MPEG tests.
Unfortunately, this program will not help me here - it creates too random
patterns to be used for animations. Still, it might give interesting
results on still images.
\shead{EXAMPLE}
Check this out: Save your favourite model's picture from something like
alt.binaries.pictures.supermodels (ok, or from any other picture source),
convert it to ppm, and process it e.g. like this, assuming the picture is 
800x600 pixels:
\nofill\ind{1\parindent}\tt{}  \# take the upper half, and leave it like it is
  pnmcut 0 0 800 300 cs.ppm $>$upper.ppm

  \# take the lower half, flip it upside down, dim it and distort it a little
  pnmcut 0 300 800 300 cs.ppm $|$ pnmflip -tb $|$ ppmdim 0.7 $|$ 
     ppmshift 10 $>$lower.ppm

  \# and concatenate the two pieces
  pnmcat -tb upper.ppm lower.ppm $>$newpic.ppm
\fill 
The resulting picture looks like the image being reflected on a water 
surface with slight ripples.
\shead{SEE ALSO}
ppm(5), pnmcut(1), pnmflip(1), ppmdim(1), pnmcat(1)
\shead{AUTHOR}
\copyright 1993 by Frank Neumann
% Permission to use, copy, modify, and distribute this software and its
% documentation for any purpose and without fee is hereby granted, provided
% that the above copyright notice appear in all copies and that both that
% copyright notice and this permission notice appear in supporting
% documentation.  This software is provided "as is" without express or
% implied warranty.
%
% end of input file: ppmshift.1
%--------------------------------------------------

% -*-LaTeX-*-
% Converted automatically from troff to LaTeX
% by tr2latex ($Revision: 2.2 $$Date: 1992/04/27 15:13:26 $ by C. Engel)
% on Mon Nov 29 13:24:10 1993
% tr2latex was written by Kamal Al-Yahya at Stanford University
% (Kamal%Hanauma@SU-SCORE.ARPA)
% and substantially enhanced by Christian Engel at RWTH Aachen
% (krischan@informatik.rwth-aachen.de).
%
% troff input file: ppmspread.1

\newpage
%--------------------------------------------------
% start of input file: ppmspread.1
%
\phead{ppmspread}{1}{16 November 1993}{}{}

%.IX ppmspread
\shead{NAME}
ppmspread - displace a portable pixmap's pixels by a random amount
\shead{SYNOPSIS}
ppmspread
{\it amount}
{\rm [}{\it ppmfile}{\rm ]}
\shead{DESCRIPTION}
Reads a portable pixmap as input. Moves every pixel around a bit
relative to its original position. amount determines by how many
pixels a pixel is to be moved around at most.
\par
Pictures processed with this filter will seem to be somewhat 
dissolved or unfocussed (although they appear more coarse than
images processed by something like
{\it pnmconvol}
).
\shead{SEE ALSO}
ppm(5), pnmconvol(1)
\shead{AUTHOR}
\copyright 1993 by Frank Neumann
% Permission to use, copy, modify, and distribute this software and its
% documentation for any purpose and without fee is hereby granted, provided
% that the above copyright notice appear in all copies and that both that
% copyright notice and this permission notice appear in supporting
% documentation.  This software is provided "as is" without express or
% implied warranty
%
% end of input file: ppmspread.1
%--------------------------------------------------

% -*-LaTeX-*-
% Converted automatically from troff to LaTeX
% by tr2latex ($Revision: 2.2 $$Date: 1992/04/27 15:13:26 $ by C. Engel)
% on Fri Oct  8 16:08:50 1993
% tr2latex was written by Kamal Al-Yahya at Stanford University
% (Kamal%Hanauma@SU-SCORE.ARPA)
% and substantially enhanced by Christian Engel at RWTH Aachen
% (krischan@informatik.rwth-aachen.de).
%
% troff input file: ppmtoacad.1

\newpage
%--------------------------------------------------
% start of input file: ppmtoacad.1
%
\phead{ppmtoacad}{1}{10 October 1991}{}{}

%.IX ppmtoacad
%.IX AutoCAD
\shead{NAME}
ppmtoacad - convert portable pixmap to AutoCAD database or slide
\shead{SYNOPSIS}
\raggedright
{\bf ppmtoacad}
'in 15n
{\rm [}{\bf -dxb}{\rm ]}
{\rm [}{\bf -poly}{\rm ]}
{\rm [}{\bf -background}
{\it colour}{\rm ]}
{\rm [}{\bf -white}{\rm ]}
{\rm [}{\bf -aspect}
{\it ratio}{\rm ]}
{\rm [}{\bf -8}{\rm ]}
{\rm [}{\it ppmfile}{\rm ]}
\ind{-7.5em}
%.ad
\shead{DESCRIPTION}
Reads a portable pixmap as input.  Produces an AutoCAD{l}ide file or
binary database import (.dxb) file as output.
If no
{\it ppmfile}
is specified, input is read from standard input.
\shead{OPTIONS}
\begin{TPlist}{{\bf -dxb}}
\item[{{\bf -dxb}}]
An AutoCAD binary database import (.dxb) file is written.  This file
is read with the DXBIN command and, once loaded, becomes part of
the AutoCAD geometrical database and can be viewed and edited like
any other object.  Each sequence of identical pixels becomes a
separate object in the database; this can result in very large AutoCAD
drawing files.  However, if you want to trace over a bitmap, it lets
you zoom and pan around the bitmap as you wish.
\item[{{\bf -poly}}]
If the
{\bf -dxb}
option is not specified, the output of
{\bf ppmtoacad}
is an AutoCAD slide file.  Normally each row of pixels is
represented by an AutoCAD line entity.  If
{\bf -poly}
is selected, the pixels are rendered as filled polygons.  If the
slide is viewed on a display with higher resolution than the source
pixmap, this will cause the pixels to expand instead of appearing as
discrete lines against the screen background colour.  Regrettably,
this representation yields slide files which occupy more disc space
and take longer to display.
\item[{{\bf -background}{\it \ colour}
}]
Most AutoCAD display drivers can be configured to use any available
colour as the screen background.  Some users perfer a black screen
background, others white, while splinter groups advocate burnt ocher,
tawny puce, and shocking grey.  Discarding pixels whose closest
AutoCAD colour representation is equal to the background colour can
substantially reduce the size of the AutoCAD database or slide file
needed to represent a bitmap.  If no
{\bf -background}
colour is specified, the screen background colour is assumed to be
black.  Any AutoCAD colour number may be specified as the screen
background; colour numbers are assumed to specify the hues defined
in the standard AutoCAD 256 colour palette.
\item[{{\bf -white}}]
Since many AutoCAD users choose a white screen background, this option
is provided as a short-cut.  Specifying
{\bf -white}
is identical in effect to
{\bf -background\ 7}{\rm .}
\item[{{\bf -aspect}{\it \ ratio}
}]
If the source pixmap had non-square pixels, the ratio of the pixel
width to pixel height should be specified as
{\it ratio}{\rm .}
The resulting slide or .dxb file will be corrected so that pixels on
the AutoCAD screen will be square.  For example, to correct an image made
for a 320x200 VGA/MCGA screen, specify
{\bf -aspect\ 0.8333}{\rm .}
\item[{{\bf -8}}]
Restricts the colours in the output file to the 8 RGB shades.
\end{TPlist}

\par
All flags can be abbreviated to their shortest unique prefix.
\shead{BUGS}
AutoCAD has a fixed palette of 256 colours, distributed along the hue,
lightness, and saturation axes.  Pixmaps which contain many
nearly-identical colours, or colours not closely approximated by
AutoCAD's palette, may be poorly rendered.
\par
{\bf ppmtoacad}
works best if the system displaying its output supports the full 256
colour AutoCAD palette.  Monochrome, 8 colour, and 16 colour
configurations will produce less than optimal results.
\par
When creating a .dxb file or a slide file with the
{\bf -poly}
option,
{\bf ppmtoacad}
finds both vertical and horizontal runs of identical pixels and
consolidates them into rectangular regions to reduce the size of the
output file.  This is effective for images with large areas of
constant colour but it's no substitute for true raster to vector
conversion.  In particular, thin diagonal lines are not optimised at
all by this process.
\par
Output files can be huge.
\shead{SEE ALSO}
AutoCAD Reference Manual:
{\it Slide File Format}
and
{\it Binary\ Drawing\ Interchange\ (DXB)\ Files}{\rm ,}
{\bf ppm}{\rm (5)}
\shead{AUTHOR}
\ind{1\parindent}{\nofill
    John Walker
    Autodesk SA
    Avenue des Champs-Montants 14b
    CH-2074 MARIN
    Suisse/Schweiz/Svizzera/Svizra/Switzerland
\fill}
\begin{TPlist}{Usenet:}
\item[{Usenet:}]
kelvin@Autodesk.com
\item[{Fax:}]
038/33 88 15
\item[{Voice:}]
038/33 76 33
\end{TPlist}

\par
Permission to use, copy, modify, and distribute this software and its
documentation for any purpose and without fee is hereby granted,
without any conditions or restrictions.  This software is provided ``as
is'' without express or implied warranty.
\par
AutoCAD and Autodesk are registered trademarks of Autodesk, Inc.
%
% end of input file: ppmtoacad.1
%--------------------------------------------------
 
% -*-LaTeX-*-
% Converted automatically from troff to LaTeX
% by tr2latex ($Revision: 2.2 $$Date: 1992/04/27 15:13:26 $ by C. Engel)
% on Fri Oct  8 16:09:09 1993
% tr2latex was written by Kamal Al-Yahya at Stanford University
% (Kamal%Hanauma@SU-SCORE.ARPA)
% and substantially enhanced by Christian Engel at RWTH Aachen
% (krischan@informatik.rwth-aachen.de).
%
% troff input file: ppmtobmp.1

\newpage
%--------------------------------------------------
% start of input file: ppmtobmp.1
%
\phead{ppmtobmp}{1}{26 Oct 1992}{}{}

%.IX ppmtobmp
\shead{NAME}
ppmtobmp -- convert a portable pixmap into a BMP file
\shead{SYNOPSIS}
{\bf ppmtobmp}
{\rm [}{\it --windows}{\rm ]}
{\rm [}{\it --os2}{\rm ]}
{\rm [}{\it ppmfile}{\rm ]}
\shead{DESCRIPTION}
Reads a portable pixmap as input.
Produces a Microsoft Windows or OS/2 BMP file as output.
%.IX BMP
\shead{OPTIONS}
\begin{TPlist}{{\bf --windows}}
\item[{{\bf --windows}}]
Tells the program to produce a Microsoft Windows BMP file.
\item[{{\bf --os2}}]
Tells the program to produce an OS/2 BMP file.
(This is the default.)
\end{TPlist}

\par
All flags can be abbreviated to their shortest unique prefix.
\shead{SEE ALSO}
bmptoppm(1),
ppm(5)
\shead{AUTHOR}
\copyright 1992 by David W. Sanderson.
% Permission to use, copy, modify, and distribute this software and
% its documentation for any purpose and without fee is hereby granted,
% provided that the above copyright notice appear in all copies and
% that both that copyright notice and this permission notice appear in
% supporting documentation.  This software is provided "as is" without
% express or implied warranty.
%
% end of input file: ppmtobmp.1
%--------------------------------------------------
 
% -*-LaTeX-*-
% Converted automatically from troff to LaTeX
% by tr2latex ($Revision: 2.2 $$Date: 1992/04/27 15:13:26 $ by C. Engel)
% on Fri Oct  8 16:08:50 1993
% tr2latex was written by Kamal Al-Yahya at Stanford University
% (Kamal%Hanauma@SU-SCORE.ARPA)
% and substantially enhanced by Christian Engel at RWTH Aachen
% (krischan@informatik.rwth-aachen.de).
%
% troff input file: ppmtogif.1

\newpage
%--------------------------------------------------
% start of input file: ppmtogif.1
%
\phead{ppmtogif}{1}{30 June 1993}{}{}

%.IX ppmtogif
\shead{NAME}
ppmtogif - convert a portable pixmap into a GIF file
\shead{SYNOPSIS}
{\bf ppmtogif}
{\rm [}{\bf -interlace}{\rm ]}
{\rm [}{\bf -sort}{\rm ]}
{\rm [}{\bf -map}
{\it mapfile ]}
{\rm [}{\it ppmfile}{\rm ]}
\shead{DESCRIPTION}
Reads a portable pixmap as input.
Produces a GIF file as output.
%.IX GIF
\shead{OPTIONS}
\begin{TPlist}{{\bf -interlace}}
\item[{{\bf -interlace}}]
Tells the program to produce an interlaced GIF file.
\item[{{\bf -sort}}]
Produces a GIF file with a sorted color map.
\item[{{\bf -map}}]
{\it mapfile}

Uses the colors found in the
{\it mapfile}
to create the colormap in the GIF file, instead of the colors from
{\it ppmfile.}
The
{\it mapfile}
can be any
{\it ppm}
file; all that matters is the colors in it. If the colors in
{\it ppmfile}
do not match those in
{\it mapfile}
, they are matched to a ``best match''. A (much) better result can be obtained by
using the following filter in advance:

{\it ppmquant}
-floyd -map
{\it mapfile}
\end{TPlist}

\par
All flags can be abbreviated to their shortest unique prefix.
\shead{SEE ALSO}
giftoppm(1), ppmquant(1), ppm(5)
\shead{AUTHOR}
Based on GIFENCOD by David Rowley (mgardi@watdcsu.waterloo.edu).
Lempel-Ziv compression based on ``compress''.

\copyright 1989 by Jef Poskanzer.
% Permission to use, copy, modify, and distribute this software and its
% documentation for any purpose and without fee is hereby granted, provided
% that the above copyright notice appear in all copies and that both that
% copyright notice and this permission notice appear in supporting
% documentation.  This software is provided "as is" without express or
% implied warranty.
%
% The Graphics Interchange Format(c) is the Copyright property of
% CompuServe Incorporated.  GIF(sm) is a Service Mark property of
% CompuServe Incorporated.
%
% end of input file: ppmtogif.1
%--------------------------------------------------
 
% -*-LaTeX-*-
% Converted automatically from troff to LaTeX
% by tr2latex ($Revision: 2.2 $$Date: 1992/04/27 15:13:26 $ by C. Engel)
% on Fri Oct  8 16:08:51 1993
% tr2latex was written by Kamal Al-Yahya at Stanford University
% (Kamal%Hanauma@SU-SCORE.ARPA)
% and substantially enhanced by Christian Engel at RWTH Aachen
% (krischan@informatik.rwth-aachen.de).
%
% troff input file: ppmtoicr.1

\newpage
%--------------------------------------------------
% start of input file: ppmtoicr.1
%
\phead{ppmtoicr}{1}{30 July 1990}{}{}

%.IX ppmtoicr
\shead{NAME}
ppmtoicr - convert a portable pixmap into NCSA ICR format 
\shead{SYNOPSIS}
{\bf ppmtoicr}
{\rm [}{\bf -windowname}
{\it name}{\rm ]}
{\rm [}{\bf -expand}
{\it expand}{\rm ]}
{\rm [}{\bf -display}
{\it display}{\rm ]}
{\rm [}{\bf -rle}{\rm ]}
{\rm [}{\it ppmfile}{\rm ]}
\shead{DESCRIPTION}
Reads a portable pixmap file as input.
Produces an NCSA Telnet Interactive Color Raster graphic file as output.
%.IX "NCSA ICR"
If
{\it ppmfile}
is not supplied, 
{\it ppmtoicr}
will read from standard input.
\par
Interactive Color Raster (ICR) is a protocol for displaying raster
graphics on workstation screens. The protocol is implemented in NCSA
Telnet for the Macintosh version 2.3.
%.IX Macintosh
The ICR protocol shares
characteristics of the Tektronix graphics terminal emulation protocol.
%.IX Tektronix
For example, escape sequences are used to control the display.
\par
{\it ppmtoicr}
will output the appropriate sequences to create a window of the
dimensions of the input pixmap,
create a colormap of up to 256
colors on the display, then load the picture data into the window.
\par
Note that there is no icrtoppm tool - this transformation is one way.
\shead{OPTIONS}
\begin{TPlist}{{\bf -windowname}{\it name}
}
\item[{{\bf -windowname}{\it name}
}]
Output will be displayed in
{\it name}
(Default is to use
{\it ppmfile}
or ``untitled'' if standard input is read.)
\item[{{\bf -expand}{\it expand}
}]
Output will be expanded on display by factor 
{\it expand}
(For example, a value of 2 will cause four pixels to be displayed for
every input pixel.)
\item[{{\bf -display}{\it display}
}]
Output will be displayed on screen numbered 
{\it display}
\item[{{\bf -rle}}]
Use run-length encoded format for display. (This will nearly always
result in a quicker display, but may skew the colormap.)
\end{TPlist}

\shead{EXAMPLES}
To display a
{\it ppm}
file using the protocol:
\nofill
    ppmtoicr ppmfile
\fill
This will create a window named 
{\it ppmfile}
on the display with the correct dimensions for
{\it ppmfile,}
create and download a colormap of up
to 256 colors, and download the picture into the window. The same effect
may be achieved by the following sequence:
\nofill
    ppmtoicr ppmfile $>$ filename
    cat filename
\fill
To display a GIF 
file using the protocol in a window titled after the input file, zoom
the displayed image by a factor of 2, and
run-length encode the data:
\nofill
    giftoppm giffile $|$ ppmtoicr -w giffile -r -e 2
\fill
\shead{BUGS}
\par
The protocol uses frequent 
{\it fflush}
calls to speed up display. If the
output is saved to a file for later display via
{\it cat,}
drawing will be
much slower. In either case, increasing the Blocksize limit on the
display will speed up transmission substantially.
\shead{SEE ALSO}
{\bf ppm(5)}
\par\noindent
{\it NCSA\ Telnet\ for\ the\ Macintosh}{\rm ,}
University of Illinois at Urbana-Champaign (1989)
\shead{AUTHOR}
\copyright 1990 by Kanthan Pillay (svpillay@Princeton.EDU),
Princeton University Computing and Information Technology.
% Permission to use, copy, modify, and distribute this software and its
% documentation for any purpose and without fee is hereby granted, provided
% that the above copyright notice appear in all copies and that both that
% copyright notice and this permission notice appear in supporting
% documentation.  This software is provided "as is" without express or
% implied warranty.
%
% end of input file: ppmtoicr.1
%--------------------------------------------------
 
% -*-LaTeX-*-
% Converted automatically from troff to LaTeX
% by tr2latex ($Revision: 2.2 $$Date: 1992/04/27 15:13:26 $ by C. Engel)
% on Fri Oct  8 16:08:52 1993
% tr2latex was written by Kamal Al-Yahya at Stanford University
% (Kamal%Hanauma@SU-SCORE.ARPA)
% and substantially enhanced by Christian Engel at RWTH Aachen
% (krischan@informatik.rwth-aachen.de).
%
% troff input file: ppmtoilbm.1

\newpage
%--------------------------------------------------
% start of input file: ppmtoilbm.1
%
\phead{ppmtoilbm}{1}{29 August 1993}{}{}

%.IX ppmtoilbm
\shead{NAME}
ppmtoilbm - convert a portable pixmap into an ILBM file
\shead{SYNOPSIS}
{\bf ppmtoilbm}
{\rm [}{\bf -maxplanes}{\rm $|$}{\bf -mp}
{\it N}{\rm ]}
{\rm [}{\bf -fixplanes}{\rm $|$}{\bf -fp}
{\it N}{\rm ]}
{\rm [}{\bf -ham6}{\rm $|$}{\bf -ham8}{\rm ]}
{\rm [}{\bf -dcbits}{\rm $|$}{\bf -dcplanes}{\rm r}{\bf g}{\rm b}{\bf ]}
{\rm [}{\bf -normal}{\rm $|$}{\bf -hamif}{\rm $|$}{\bf -hamforce}{\rm $|$}{\bf -24if}{\rm $|$}{\bf -24force}{\rm $|$}
{\rm -dcif}{\bf $|$}{\rm -dcforce}{\bf $|$}{\rm -cmaponly}{\bf ]}
{\rm [}{\bf -ecs}{\rm $|$}{\bf -aga}{\rm ]}
{\rm [}{\bf -map}{\rm ppmfile}{\bf ]}
{\rm [}{\bf ppmfile}{\rm ]}
\shead{DESCRIPTION}
Reads a portable pixmap as input.
%.IX "ILBM"
Produces an ILBM file as output.
Supported ILBM types are:
\begin{TPlist}{Normal ILBMs with 1-16 planes.}
\item[{Normal ILBMs with 1-16 planes.}]
\item[{Amiga Hold-and-Modify (HAM) with 3-16 planes.}]
%.IX "Amiga"
%.IX "HAM"
\item[{24 bit.}]
\item[{Color map (BMHD + CMAP chunk only, nPlanes = 0).}]
\item[{Unofficial direct color.}]
1-16 planes for each color component.
\item[{Chunks written:}]
BMHD, CMAP, CAMG (only for HAM), BODY (not for colormap files)
unofficial DCOL chunk for direct color ILBM
\item[{Chunks ignored:}]
GRAB, DEST, SPRT, CRNG, CCRT, CLUT, DPPV, DRNG, EPSF
\item[{Other chunks (ignored but displayed in verbose mode):}]
NAME, AUTH, (c), ANNO, DPI
\end{TPlist}

\shead{OPTIONS}
Options marked with (*) can be prefixed with a ``no'',
{\it e.g.}, ``-nohamif''. All options can be abbreviated to
their shortest unique prefix.
\begin{TPlist}{{\bf -maxplanes $|$ -mp n}}
\item[{{\bf -maxplanes $|$ -mp n}}]
(default 5, minimum 1, maximum 16)
Maximum planes to write in a normal ILBM.  If the pixmap
does not fit into $<$n$>$ planes, ppmtoilbm writes a HAM file
(if -hamif is used), a 24bit file (if -24if is used) or a
direct color file (if -dcif is used) or aborts with an error.
\item[{{\bf -fixplanes $|$ -fp n}}]
(min 1, max 16)
If a normal ILBM is written, it will have exactly $<$n$>$ planes.
\item[{{\bf -hambits $|$ -hamplanes n}}]
(default 6, min 3, max 16)
Select number of planes for HAM picture.  The current Amiga
hardware supports 6 and 8 planes, so for now you should
only use this values.
\item[{{\bf -normal (default)}}]
Turns off -hamif/-24if/-dcif, -hamforce/-24force/-dcforce and
-cmaponly.
\item[{{\bf -hamif (*)}}]
\item[{{\bf -24if (*)}}]
\item[{{\bf -dcif (*)}}]
Write a HAM/24bit/direct color file if the pixmap does not
fit into $<$maxplanes$>$ planes.
\item[{{\bf -hamforce (*)}}]
\item[{{\bf -24force (*)}}]
\item[{{\bf -dcforce (*)}}]
Write a HAM/24bit/direct color file.
\item[{{\bf -dcbits $|$ -dcplanes r g b}}]
(default 5, min 1, max 16).
Select number of bits for red, green \& blue in a direct
color ILBM.
\item[{{\bf -ecs (default)}}]
Shortcut for: -hamplanes 6 -maxplanes 5
\item[{{\bf -aga}}]
\item[{{\bf Shortcut for: -hamplanes 8 -maxplanes 8}}]
\item[{{\bf -ham6}}]
\item[{{\bf Shortcut for: -hamplanes 6 -hamforce}}]
\item[{{\bf -ham8}}]
Shortcut for: -hamplanes 8 -hamforce
\item[{{\bf -map ppmfile}}]
Write a normal ILBM using the colors in $<$ppmfile$>$ as the
colormap. The colormap file also determines the number of
planes, a -maxplanes or -fixplanes option is ignored.
\item[{{\bf -cmaponly}}]
Write a colormap file: only BMHD and CMAP chunks, no BODY
chunk, nPlanes = 0.
\end{TPlist}

\shead{BUGS}
Needs a real colormap selection algorithm for HAM pictures,
instead of using a grayscale colormap.
\shead{REFERENCES}
Amiga ROM Kernel Reference Manual - Devices (3rd Ed.)
Addison Wesley, ISBN 0--201--56775--X
\shead{SEE ALSO}
ppm(5), ilbmtoppm(1)
\shead{AUTHORS}
\copyright 1989 by Jef Poskanzer.
\nwl
Modified August 1993 by Ingo Wilken
(Ingo.Wilken@informatik.uni-oldenburg.de).
% Permission to use, copy, modify, and distribute this software and its
% documentation for any purpose and without fee is hereby granted, provided
% that the above copyright notice appear in all copies and that both that
% copyright notice and this permission notice appear in supporting
% documentation.  This software is provided "as is" without express or
% implied warranty.
%
% end of input file: ppmtoilbm.1
%--------------------------------------------------
 
% -*-LaTeX-*-
% Converted automatically from troff to LaTeX
% by tr2latex ($Revision: 2.2 $$Date: 1992/04/27 15:13:26 $ by C. Engel)
% on Tue Nov 30 08:30:37 1993
% tr2latex was written by Kamal Al-Yahya at Stanford University
% (Kamal%Hanauma@SU-SCORE.ARPA)
% and substantially enhanced by Christian Engel at RWTH Aachen
% (krischan@informatik.rwth-aachen.de).
%
% troff input file: ppmtomitsu.1

\newpage
%--------------------------------------------------
% start of input file: ppmtomitsu.1
%
\phead{ppmtomitsu}{1}{29 Jan 1992}{}{}

%.IX ppmtomitsu
\shead{NAME}
ppmtomitsu - convert a portable pixmap to a Mitsubishi S340-10 file
\shead{SYNOPSIS}
{\bf ppmtomitsu}
{\rm [-sharpness}
{\it val}{\rm ]}
{\rm [}{\bf -enlarge}
{\it val}{\rm ]}
{\rm [}{\bf -media}
{\it string}{\rm ]}
{\rm [}{\bf -copy}
{\it val}{\rm ]}
{\rm [}{\bf -dpi300}{\rm ]}
{\rm [}{\bf -tiny}{\rm ]}
{\rm [}{\it ppmfile}{\rm ]}
\shead{DESCRIPTION}
Reads a portable pixmap as input and converts it into a format suitable
to be printed by a Mitsubishi S340-10 printer, or any other Mitsubishi
color sublimation printer.
\par
The Mitsubishi S340-10 Color Sublimation printer supports 24bit color.
Images of the available sizes take so long to transfer that there is a
fast method, employing a lookuptable, that ppmtomitsu will use if there
is a maximum of 256 colors in the pixmap.
ppmtomitsu will try to position your image to the center of the paper,
and will rotate your image for you if xsize is larger than ysize.
If your image is larger than the media allows, ppmtomitsu will quit
with an error message. (We decided that the media were too expensive
to have careless users produce misprints.) 
Once data transmission has started, the job can't be stopped in a
sane way without resetting the printer.
The printer understands putting together images in the printers memory;
ppmtomitsu doesn't utilize this as pnmcat etc provide the same functionality
and let you view the result on-screen, too.
The S340-10 is the lowest common denominator printer; for higher
resolution printers there's the dpi300 option. The other printers also
support higher values for enlarge eg, but I don't think that's essential
enough to warrant a change in the program.
\begin{TPlist}{{\bf -sharpness}{\it \ 1-4}
}
\item[{{\bf -sharpness}{\it \ 1-4}
}]
 'sharpness' designation. Default is to use the current sharpness.
\item[{{\bf -enlarge}{\it \ 1-3}
}]
Enlarge by a factor; Default is 1 (no enlarge)
\item[{{\bf -media}{\it \ A,\ A4,\ AS,\ A4S}
}]
Designate the media you're using. Default is 1184 x 1350, which will fit
on any media. A is 1216 x 1350, A4 is 1184 x 1452, AS is 1216 x 1650 and 
A4S is 1184 x 1754. A warning: If you specify a different media than the
printer currently has, the printer will wait until you put in the correct
media or switch it off.
\item[{{\bf -copy}{\it \ 1-9}
}]
The number of copies to produce. Default is 1.
\item[{{\bf -dpi300}
}]
Double the number of allowed pixels for a S3600-30 Printer in S340-10
compatibility mode. (The S3600-30 has 300 dpi).
\item[{{\bf -tiny}
}]
Memory-safing, but always slow. The printer will get the data line-by-line
in 24bit. It's probably a good idea to use this if your machine starts
paging a lot without this option.
\end{TPlist}

\shead{REFERENCES}
Mitsubishi Sublimation Full Color Printer S340-10 Specifications of
Parallel Interface LSP-F0232F
\shead{SEE ALSO}
ppmquant(1), pnmscale(1), ppm(5)
\shead{BUGS}
We didn't find any - yet. (Besides, they're called features anyway :-)
If you should find one, my email-adress is below.
\shead{AUTHOR}
\copyright 1992, 93 by S.Petra Zeidler, MPIfR Bonn, Germany.
(spz@specklec.mpifr-bonn.mpg.de)
% Permission to use, copy, modify, and distribute this software and its
% documentation for any purpose and without fee is hereby granted, provided
% that the above copyright notice appear in all copies and that both that
% copyright notice and this permission notice appear in supporting
% documentation.  This software is provided "as is" without express or
% implied warranty.
%
% end of input file: ppmtomitsu.1
%--------------------------------------------------

% -*-LaTeX-*-
% Converted automatically from troff to LaTeX
% by tr2latex ($Revision: 2.2 $$Date: 1992/04/27 15:13:26 $ by C. Engel)
% on Fri Oct  8 16:09:10 1993
% tr2latex was written by Kamal Al-Yahya at Stanford University
% (Kamal%Hanauma@SU-SCORE.ARPA)
% and substantially enhanced by Christian Engel at RWTH Aachen
% (krischan@informatik.rwth-aachen.de).
%
% troff input file: ppmtomap.1

\newpage
%--------------------------------------------------
% start of input file: ppmtomap.1
%
\phead{ppmtomap}{1}{11 August 1993}{}{}

%.IX ppmtomap
\shead{NAME}
ppmtomap - extract all colors from a portable pixmap
\shead{SYNOPSIS}
{\bf ppmtomap}
{\rm [}{\bf -sort}{\rm ]}
{\rm [}{\bf -square}{\rm ]}
{\rm [}{\it ppmfile}{\rm ]}
\shead{DESCRIPTION}
Reads a portable pixmap as input.
Produces a portable pixmap as output, representing a color map of the
input file. All N different colors found are put in an Nx1 portable
pixmap.
This color map file can be used as a mapfile for
{\it ppmquant}
or
{\it ppmtogif.}
\shead{OPTIONS}
\begin{TPlist}{{\bf -sort}}
\item[{{\bf -sort}}]
Produces a portable pixmap with the colors in some sorted order.
\item[{{\bf -square}}]
Produces a (more or less) square output file, instead of putting all
colors on the top row.
\end{TPlist}

\par
All flags can be abbreviated to their shortest unique prefix.
\shead{WARNING}
If you want to use the output file as a mapfile for
{\it ppmtogif,}
you first have to do a
{\it ppmquant 256,}
since
{\it ppmtomap}
is not limited to 256 colors (but to 65536).
\shead{SEE ALSO}
ppmtogif(1), ppmquant(1), ppm(5)
\shead{AUTHOR}
Marcel Wijkstra (wijkstra@fwi.uva.nl).

\copyright 1989 by Jef Poskanzer.
% Permission to use, copy, modify, and distribute this software and its
% documentation for any purpose and without fee is hereby granted, provided
% that the above copyright notice appear in all copies and that both that
% copyright notice and this permission notice appear in supporting
% documentation.  This software is provided "as is" without express or
% implied warranty.
%
% The Graphics Interchange Format(c) is the Copyright property of
% CompuServe Incorporated.  GIF(sm) is a Service Mark property of
% CompuServe Incorporated.
%
% end of input file: ppmtomap.1
%--------------------------------------------------
 
% -*-LaTeX-*-
% Converted automatically from troff to LaTeX
% by tr2latex ($Revision: 2.2 $$Date: 1992/04/27 15:13:26 $ by C. Engel)
% on Fri Oct  8 16:08:53 1993
% tr2latex was written by Kamal Al-Yahya at Stanford University
% (Kamal%Hanauma@SU-SCORE.ARPA)
% and substantially enhanced by Christian Engel at RWTH Aachen
% (krischan@informatik.rwth-aachen.de).
%
% troff input file: ppmtopcx.1

\newpage
%--------------------------------------------------
% start of input file: ppmtopcx.1
%
\phead{ppmtopcx}{1}{09 April 1990}{}{}

%.IX ppmtopcx
\shead{NAME}
ppmtopcx - convert a portable pixmap into a PCX file
\shead{SYNOPSIS}
{\bf ppmtopcx}
{\rm [}{\it ppmfile}{\rm ]}
\shead{DESCRIPTION}
Reads a portable pixmap as input.
Produces a PCX file as output.
%.IX PCX
\shead{SEE ALSO}
pcxtoppm(1), ppm(5)
\shead{AUTHOR}
\copyright 1990 by Michael Davidson.
% Permission to use, copy, modify, and distribute this software and its
% documentation for any purpose and without fee is hereby granted, provided
% that the above copyright notice appear in all copies and that both that
% copyright notice and this permission notice appear in supporting
% documentation.  This software is provided "as is" without express or
% implied warranty.
%
% end of input file: ppmtopcx.1
%--------------------------------------------------
 
% -*-LaTeX-*-
% Converted automatically from troff to LaTeX
% by tr2latex ($Revision: 2.2 $$Date: 1992/04/27 15:13:26 $ by C. Engel)
% on Fri Oct  8 16:08:53 1993
% tr2latex was written by Kamal Al-Yahya at Stanford University
% (Kamal%Hanauma@SU-SCORE.ARPA)
% and substantially enhanced by Christian Engel at RWTH Aachen
% (krischan@informatik.rwth-aachen.de).
%
% troff input file: ppmtopgm.1

\newpage
%--------------------------------------------------
% start of input file: ppmtopgm.1
%
\phead{ppmtopgm}{1}{23 December 1988}{}{}

%.IX ppmtopgm
\shead{NAME}
ppmtopgm - convert a portable pixmap into a portable graymap
\shead{SYNOPSIS}
{\bf ppmtopgm}
{\rm [}{\it ppmfile}{\rm ]}
\shead{DESCRIPTION}
Reads a portable pixmap as input.
Produces a portable graymap as output.
The quantization formula used is .299 r + .587 g + .114 b.
\par
Note that although there is a
{\it pgmtoppm}
program, it is not necessary
for simple conversions from
{\it pgm}
to
{\it ppm}{\rm ,}
because any ppm program can
read
{\it pgm}
(and
{\it pbm}
) files automagically.
{\it pgmtoppm}
is for colorizing a
{\it pgm}
file.  Also, see
{\it ppmtorgb3}
%.IX ppmtorgb3
for a different way of converting color to gray.
\shead{QUOTE}
\nofill
Cold-hearted orb that rules the night
Removes the colors from our sight
Red is gray, and yellow white
But we decide which is right
And which is a quantization error.
\fill
\shead{SEE ALSO}
pgmtoppm(1), ppmtorgb3(1), rgb3toppm(1), ppm(5), pgm(5)
\shead{AUTHOR}
\copyright 1989 by Jef Poskanzer.
% Permission to use, copy, modify, and distribute this software and its
% documentation for any purpose and without fee is hereby granted, provided
% that the above copyright notice appear in all copies and that both that
% copyright notice and this permission notice appear in supporting
% documentation.  This software is provided "as is" without express or
% implied warranty.
%
% end of input file: ppmtopgm.1
%--------------------------------------------------
 
% -*-LaTeX-*-
% Converted automatically from troff to LaTeX
% by tr2latex ($Revision: 2.2 $$Date: 1992/04/27 15:13:26 $ by C. Engel)
% on Fri Oct  8 16:08:54 1993
% tr2latex was written by Kamal Al-Yahya at Stanford University
% (Kamal%Hanauma@SU-SCORE.ARPA)
% and substantially enhanced by Christian Engel at RWTH Aachen
% (krischan@informatik.rwth-aachen.de).
%
% troff input file: ppmtopi1.1

\newpage
%--------------------------------------------------
% start of input file: ppmtopi1.1
%
\phead{ppmtopi1}{1}{19 July 1990}{}{}

%.IX ppmtopi1
\shead{NAME}
ppmtopi1 - convert a portable pixmap into an Atari Degas .pi1 file
\shead{SYNOPSIS}
{\bf ppmtopi1}
{\rm [}{\it ppmfile}{\rm ]}
\shead{DESCRIPTION}
Reads a portable pixmap as input.
Produces an Atari Degas .pi1 file as output.
%.IX Atari
%.IX "Degas .pi1"
\shead{SEE ALSO}
pi1toppm(1), ppm(5), pbmtopi3(1), pi3topbm(1)
\shead{AUTHOR}
\copyright 1991 by Steve Belczyk (seb3@gte.com) and Jef Poskanzer.
% Permission to use, copy, modify and distribute this software and its
% documentation for any purpose and without fee is hereby granted, provided
% that the above copyright notice appear in all copies and that both that
% copyright notice and this permission notice appear in supporting
% documentation.  This software is provided "as is" without express or
% implied warranty.
%
% end of input file: ppmtopi1.1
%--------------------------------------------------
 
% -*-LaTeX-*-
% Converted automatically from troff to LaTeX
% by tr2latex ($Revision: 2.2 $$Date: 1992/04/27 15:13:26 $ by C. Engel)
% on Fri Oct  8 16:08:54 1993
% tr2latex was written by Kamal Al-Yahya at Stanford University
% (Kamal%Hanauma@SU-SCORE.ARPA)
% and substantially enhanced by Christian Engel at RWTH Aachen
% (krischan@informatik.rwth-aachen.de).
%
% troff input file: ppmtopict.1

\newpage
%--------------------------------------------------
% start of input file: ppmtopict.1
%
\phead{ppmtopict}{1}{15 April 1990}{}{}

%.IX ppmtopict
\shead{NAME}
ppmtopict - convert a portable pixmap into a Macintosh PICT file
\shead{SYNOPSIS}
{\bf ppmtopict}
{\rm [}{\it ppmfile}{\rm ]}
\shead{DESCRIPTION}
Reads a portable pixmap as input.
Produces a Macintosh PICT file as output.
%.IX PICT
%.IX Macintosh
\par
The generated file is only the data fork of a picture.
You will need a program such as
{\it mcvert}
to generate a Macbinary or a BinHex file that contains the necessary
information to identify the file as a PICT file to MacOS.
\par
Even though PICT supports 2 and 4 bits per pixel,
{\it ppmtopict}
always generates an 8 bits per pixel file.
\shead{BUGS}
The picture size field is only correct if the output is to a file
since writing into this field requires seeking backwards on a file.
However the PICT documentation seems to suggest that this field is
not critical anyway since it is only the lower 16 bits of the picture size.
\shead{SEE ALSO}
picttoppm(1), ppm(5), mcvert(1)
\shead{AUTHOR}
\copyright 1990 by Ken Yap (ken@cs.rocester.edu).
% Permission to use, copy, modify, and distribute this software and its
% documentation for any purpose and without fee is hereby granted, provided
% that the above copyright notice appear in all copies and that both that
% copyright notice and this permission notice appear in supporting
% documentation.  This software is provided "as is" without express or
% implied warranty.
%
% end of input file: ppmtopict.1
%--------------------------------------------------
 
% -*-LaTeX-*-
% Converted automatically from troff to LaTeX
% by tr2latex ($Revision: 2.2 $$Date: 1992/04/27 15:13:26 $ by C. Engel)
% on Fri Oct  8 16:08:55 1993
% tr2latex was written by Kamal Al-Yahya at Stanford University
% (Kamal%Hanauma@SU-SCORE.ARPA)
% and substantially enhanced by Christian Engel at RWTH Aachen
% (krischan@informatik.rwth-aachen.de).
%
% troff input file: ppmtopj.1

\newpage
%--------------------------------------------------
% start of input file: ppmtopj.1
%
\phead{ppmtopj}{1}{13 July 1991}{}{}

%.IX ppmtopj
\shead{NAME}
ppmtopj - convert a portable pixmap to an HP PaintJet file
\shead{SYNOPSIS}
{\bf ppmtopj}
{\rm [-gamma}
{\it val}{\rm ]}
{\rm [}{\bf -xpos}
{\it val}{\rm ]}
{\rm [}{\bf -ypos}
{\it val}{\rm ]}
{\rm [}{\bf -back}
{\bf dark}{\rm $|$}{\bf lite}{\rm ]}
{\rm [}{\bf -rle}{\rm ]}
{\rm [}{\bf -center}{\rm ]}
{\rm [}{\bf -render}
{\bf none$|$snap$|$bw$|$dither$|$diffuse$|$monodither$|$monodiffuse$|$clusterdither$|$monoclusterdither}{\rm ]}
{\rm [}{\it ppmfile}{\rm ]}
\shead{DESCRIPTION}
Reads a portable pixmap as input and converts it into a format suitable
to be printed by an HP PaintJet printer.
\par
For best results, the input file should be in 8-color RGB form;
{\it i.e.}, it should have only
the 8 binary combinations of full-on and full-off primaries.
You could get this by sending the input file through
{\it ppmquant -map}
with a map file such as:
\nofill
    P3
    8 1
    255
    0 0 0      255 0 0    0 255 0    0 0 255
    255 255 0  255 0 255  0 255 255  255 255 255
\fill
Or else you could use use
{\it ppmdither -red 2 -green 2 -blue 2.}
\shead{OPTIONS}
\begin{TPlist}{{\bf -rle}}
\item[{{\bf -rle}}]
Run length encode the image.
(This can result in larger images)
\item[{{\bf -back}}]
Enhance the foreground by indicating if the background is light or
dark compated to the foreground.
\item[{{\bf -render}{\it \ alg}
}]
Use an internal rendering algorithm (default dither).
\item[{{\bf -gamma}{\it \ int}
}]
Gamma correct the image using the integet parameter as a gamma (default 0).
\item[{{\bf -center}}]
Center the image to an 8.5 by 11 page
\item[{{\bf -xpos}{\it \ pos}
}]
Move by pos pixels in the x direction.
\item[{{\bf -ypos}{\it \ pos}
}]
Move by pos pixels in the y direction.
\end{TPlist}

\shead{REFERENCES}
HP PaintJet XL Color Graphics Printer User's Guide
\shead{SEE ALSO}
pnmdepth(1), ppmquant(1), ppmdither(1), ppm(5)
\shead{BUGS}
Most of the options have not been tested because of the price of the paper.
\shead{AUTHOR}
\copyright 1991 by Christos Zoulas.
% Permission to use, copy, modify, and distribute this software and its
% documentation for any purpose and without fee is hereby granted, provided
% that the above copyright notice appear in all copies and that both that
% copyright notice and this permission notice appear in supporting
% documentation.  This software is provided "as is" without express or
% implied warranty.
%
% end of input file: ppmtopj.1
%--------------------------------------------------
 
% -*-LaTeX-*-
% Converted automatically from troff to LaTeX
% by tr2latex ($Revision: 2.2 $$Date: 1992/04/27 15:13:26 $ by C. Engel)
% on Fri Oct  8 16:09:10 1993
% tr2latex was written by Kamal Al-Yahya at Stanford University
% (Kamal%Hanauma@SU-SCORE.ARPA)
% and substantially enhanced by Christian Engel at RWTH Aachen
% (krischan@informatik.rwth-aachen.de).
%
% troff input file: ppmtopjxl.1

\newpage
%--------------------------------------------------
% start of input file: ppmtopjxl.1
%
\phead{ppmtopjxl}{1}{14 March 1991}{}{}

\shead{NAME}
ppmtopjxl - convert a portable pixmap into an HP PaintJet XL PCL file
\shead{SYNOPSIS}
ppmtopjxl [-nopack] [-gamma
{\it $<$n$>$}
] [-presentation] [-dark] [-diffuse] [-cluster] [-dither] [-xshift
{\it $<$s$>$}
] [-yshift
{\it $<$s$>$}
] [-xshift
{\it $<$s$>$}
] [-yshift
{\it $<$s$>$}
] [-xsize$|$-width$|$-xscale
{\it $<$s$>$}
] [-ysize$|$-height$|$-yscale
{\it $<$s$>$}
] [ppmfile]

\shead{DESCRIPTION}
Reads a portable pixmap as input.
Produces a PCL file suitable for printing on an HP PaintJet XL printer as
output.
\par
The generated file is not suitable for printing on a normal PrintJet printer.
The
{\bf --nopack}
option generates a file which does not use the normal TIFF 4.0 compression
method. This file might be printable on a normal PaintJet printer (not an XL).
\par
The
{\bf --gamma}
option sets the gamma correction for the image. The useful range for the
PaintJet XL is approximately 0.6 to 1.5.
\par
The rendering algorithm used for images can be altered with the
{\bf -dither,}
{\bf -cluster,}
and
{\bf -diffuse}
options. These options select ordered dithering, clustered ordered dithering,
or error diffusion respectively.
The
{\bf --dark}
option can be used to enhance images with a dark background when they are
reduced in size.
The
{\bf --presentation}
option turns on presentation mode, in which two passes are made over the paper
to increase ink density. This should be used only for images where quality is
critical.

\par
The image can be resized by setting the 
{\bf --xsize}
and 
{\bf --ysize}
options. The parameter to either of these options is interpreted as the
number of dots to set the width or height to, but an optional dimension of
`%
\bf pt%
\rm ' (points), `%
\bf dp%
\rm ' (decipoints), `%
\bf in%
\rm ' (inches), or
`%
\bf cm%
\rm ' (centimetres) may be appended.
If only one dimension is specified, the other will be scaled appropriately.

The options
{\bf --width}
and
{\bf --height}
are synonyms of
{\bf --xsize}
and
{\bf --ysize.}

The
{\bf --xscale}
and
{\bf --yscale}
options can alternatively be used to scale the image by a simple factor.

\par
The image can be shifted on the page by using the
{\bf --xshift}
and
{\bf --yshift}
options. These move the image the specified dimensions right and down.

\shead{SEE ALSO}
ppm(5)
\shead{AUTHOR}
Angus Duggan
%
% end of input file: ppmtopjxl.1
%--------------------------------------------------
 
% -*-LaTeX-*-
% Converted automatically from troff to LaTeX
% by tr2latex ($Revision: 2.2 $$Date: 1992/04/27 15:13:26 $ by C. Engel)
% on Fri Oct  8 16:08:56 1993
% tr2latex was written by Kamal Al-Yahya at Stanford University
% (Kamal%Hanauma@SU-SCORE.ARPA)
% and substantially enhanced by Christian Engel at RWTH Aachen
% (krischan@informatik.rwth-aachen.de).
%
% troff input file: ppmtopuzz.1

\newpage
%--------------------------------------------------
% start of input file: ppmtopuzz.1
%
\phead{ppmtopuzz}{1}{22 August 1990}{}{}

%.IX ppmtopuzz
\shead{NAME}
ppmtopuzz - convert a portable pixmap into an X11 ``puzzle'' file
\shead{SYNOPSIS}
{\bf ppmtopuzz}
{\rm [}{\it ppmfile}{\rm ]}
\shead{DESCRIPTION}
Reads a portable pixmap as input.
Produces an X11 ``puzzle'' file as output.
A ``puzzle'' file is for use with the
{\it puzzle}
%.IX puzzle
%.IX "X window system"
program included with the X11 distribution -
{\it puzzle}{\rm 's}
{\bf -picture}
flag lets you specify an image file.
\shead{SEE ALSO}
ppm(5), puzzle(1)
\shead{AUTHOR}
\copyright 1991 by Jef Poskanzer.
% Permission to use, copy, modify and distribute this software and its
% documentation for any purpose and without fee is hereby granted, provided
% that the above copyright notice appear in all copies and that both that
% copyright notice and this permission notice appear in supporting
% documentation.  This software is provided "as is" without express or
% implied warranty.
%
% end of input file: ppmtopuzz.1
%--------------------------------------------------
 
% -*-LaTeX-*-
% Converted automatically from troff to LaTeX
% by tr2latex ($Revision: 2.2 $$Date: 1992/04/27 15:13:26 $ by C. Engel)
% on Fri Oct  8 16:08:57 1993
% tr2latex was written by Kamal Al-Yahya at Stanford University
% (Kamal%Hanauma@SU-SCORE.ARPA)
% and substantially enhanced by Christian Engel at RWTH Aachen
% (krischan@informatik.rwth-aachen.de).
%
% troff input file: ppmtorgb3.1

\newpage
%--------------------------------------------------
% start of input file: ppmtorgb3.1
%
\phead{ppmtorgb3}{1}{10 January 1991}{}{}

%.IX ppmtorgb3
\shead{NAME}
ppmtorgb3 - separate a portable pixmap into three portable graymaps
\shead{SYNOPSIS}
{\bf ppmtorgb3}
{\rm [}{\it ppmfile}{\rm ]}
\shead{DESCRIPTION}
Reads a portable pixmap as input.
Writes three portable graymaps as output, one each for red, green, and blue.
%.IX "color separation"
\par
The output filenames are constructed by taking the input filename,
stripping off any extension, and appending ``.red'', ``.grn'', and
``.blu''.
For example, separating lenna.ppm would result in lenna.red, lenna.grn,
and lenna.blu.
If the input comes from stdin, the names are noname.red, noname.grn,
and noname.blu.
\shead{SEE ALSO}
rgb3toppm(1), ppmtopgm(1), pgmtoppm(1), ppm(5), pgm(5)
\shead{AUTHOR}
\copyright 1991 by Jef Poskanzer.
% Permission to use, copy, modify, and distribute this software and its
% documentation for any purpose and without fee is hereby granted, provided
% that the above copyright notice appear in all copies and that both that
% copyright notice and this permission notice appear in supporting
% documentation.  This software is provided "as is" without express or
% implied warranty.
%
% end of input file: ppmtorgb3.1
%--------------------------------------------------
 
% -*-LaTeX-*-
% Converted automatically from troff to LaTeX
% by tr2latex ($Revision: 2.2 $$Date: 1992/04/27 15:13:26 $ by C. Engel)
% on Fri Oct  8 16:08:57 1993
% tr2latex was written by Kamal Al-Yahya at Stanford University
% (Kamal%Hanauma@SU-SCORE.ARPA)
% and substantially enhanced by Christian Engel at RWTH Aachen
% (krischan@informatik.rwth-aachen.de).
%
% troff input file: ppmtosixel.1

\newpage
%--------------------------------------------------
% start of input file: ppmtosixel.1
%
\phead{ppmtosixel}{1}{26 April 1991}{}{}

%.IX ppmtosixel
\shead{NAME}
ppmtosixel - convert a portable pixmap into DEC sixel format
\shead{SYNOPSIS}
{\bf ppmtosixel}
{\rm [}{\bf -raw}{\rm ]}
{\rm [}{\bf -margin}{\rm ]}
{\rm [}{\it ppmfile}{\rm ]}
\shead{DESCRIPTION}
Reads a portable pixmap as input.
Produces sixel commands (SIX) as output.
The output is formatted for color printing, {\it e.g.}, for a DEC LJ250 color
inkjet printer. 
\par
If RGB values from the PPM file do not have maxval=100,
the RGB values are rescaled.
A printer control header and a color assignment table begin the SIX file.
Image data is written in a compressed format by default.
A printer control footer ends the image file.
\shead{OPTIONS}
\begin{TPlist}{{\bf -raw}}
\item[{{\bf -raw}}]
If specified, each pixel will be explicitly described in the image file.
If
{\bf -raw}
is not specified, output will default to compressed format in which
identical adjacent pixels are replaced by ``repeat pixel'' commands.
A raw file is often an order of magnitude larger than a compressed
file and prints much slower.
\item[{{\bf -margin}}]
If
{\bf -margin}
is not specified, the image will be start at the left margin
(of the window, paper, or whatever).
If
{\bf -margin}
is specified, a 1.5 inch left margin will offset the image.
\end{TPlist}

\shead{PRINTING}
Generally, sixel files must reach the printer unfiltered.
Use the lpr -x option or cat filename $>$ /dev/tty0?.
\shead{BUGS}
Upon rescaling, truncation of the least significant bits of RGB values
may result in poor color conversion.
If the original PPM maxval was greater than 100, rescaling also
reduces the image depth.
While the actual RGB values from the ppm file are more or less
retained, the color palette of the LJ250 may not match the colors
on your screen.
This seems to be a printer limitation.
\shead{SEE ALSO}
ppm(5)
\shead{AUTHOR}
\copyright 1991 by Rick Vinci.
% Permission to use, copy, modify, and distribute this software and its
% documentation for any purpose and without fee is hereby granted, provided
% that the above copyright notice appear in all copies and that both that
% copyright notice and this permission notice appear in supporting
% documentation.  This software is provided "as is" without express or
% implied warranty.
%
% end of input file: ppmtosixel.1
%--------------------------------------------------
 
% -*-LaTeX-*-
% Converted automatically from troff to LaTeX
% by tr2latex ($Revision: 2.2 $$Date: 1992/04/27 15:13:26 $ by C. Engel)
% on Fri Oct  8 16:08:58 1993
% tr2latex was written by Kamal Al-Yahya at Stanford University
% (Kamal%Hanauma@SU-SCORE.ARPA)
% and substantially enhanced by Christian Engel at RWTH Aachen
% (krischan@informatik.rwth-aachen.de).
%
% troff input file: ppmtotga.1

\newpage
%--------------------------------------------------
% start of input file: ppmtotga.1
%
\phead{ppmtotga}{1}{28 October 1991}{}{}

%.IX ppmtotga
\shead{NAME}
ppmtotga - convert portable pixmap into a TrueVision Targa file
\shead{SYNOPSIS}
{\bf ppmtotga}
{\rm [}{\bf -mono$|$-cmap$|$-rgb}{\rm ]}
{\rm [}{\bf -norle}{\rm ]}
{\rm [}{\it ppmfile}{\rm ]}
\shead{DESCRIPTION}
Reads a portable pixmap as input.
Produces a TrueVision Targa file as output.
\shead{OPTIONS}
\begin{TPlist}{{\bf -mono}}
\item[{{\bf -mono}}]
Forces Targa file to be of type 8 bit monochrome.  Input must be a portable
bitmap or a portable graymap.
\item[{{\bf -cmap}}]
Forces Targa file to be of type 24 bit colormapped.  Input must be a portable
bitmap, a portable graymap or a portable pixmap containing no more than
256 distinct colors.
\item[{{\bf -rgb}}]
Forces Targa file to be of type 24 bit unmapped color.
\item[{{\bf -norle}}]
Disables run-length encoding, in case you have a Targa reader which
can't read run-length encoded files.
\end{TPlist}

\par
All flags can be abbreviated to their shortest unique prefix.  If no
file type is specified the most highly constained compatible type is
used, where monochrome is more constained than colormapped which is in
turn more constained than unmapped.
\shead{BUGS}
Does not support all possible Targa file types.
Should really be in PNM, not PPM.
\shead{SEE ALSO}
tgatoppm(1), ppm(5)
\shead{AUTHOR}
\copyright 1989, 1991 by Mark Shand and Jef Poskanzer.
% Permission to use, copy, modify, and distribute this software and its
% documentation for any purpose and without fee is hereby granted, provided
% that the above copyright notice appear in all copies and that both that
% copyright notice and this permission notice appear in supporting
% documentation.  This software is provided "as is" without express or
% implied warranty.
%
% end of input file: ppmtotga.1
%--------------------------------------------------
 
% -*-LaTeX-*-
% Converted automatically from troff to LaTeX
% by tr2latex ($Revision: 2.2 $$Date: 1992/04/27 15:13:26 $ by C. Engel)
% on Fri Oct  8 16:08:59 1993
% tr2latex was written by Kamal Al-Yahya at Stanford University
% (Kamal%Hanauma@SU-SCORE.ARPA)
% and substantially enhanced by Christian Engel at RWTH Aachen
% (krischan@informatik.rwth-aachen.de).
%
% troff input file: ppmtouil.1

\newpage
%--------------------------------------------------
% start of input file: ppmtouil.1
%
\phead{ppmtouil}{1}{31 August 1990}{}{}

%.IX ppmtouil
\shead{NAME}
ppmtouil - convert a portable pixmap into a Motif UIL icon file
\shead{SYNOPSIS}
{\bf ppmtouil}
{\rm [}{\bf -name}
{\it uilname}{\rm ]}
{\rm [}{\it ppmfile}{\rm ]}
\shead{DESCRIPTION}
Reads a portable pixmap as input.
Produces a Motif UIL icon file as output.
%.IX Motif
%.IX UIL
%.IX "X window system"
\par
If the program was compiled with an rgb database specified, and
a RGB value from the ppm input matches a RGB value from the database,
then the corresponding color name mnemonic is printed in the UIL's colormap.
If no rgb database was compiled in, or if the RGB values don't match,
then the color
will be printed with the \#RGB, \#RRGGBB, \#RRRGGGBBB, or \#RRRRGGGGBBBB
hexadecimal format.
\shead{OPTIONS}
\begin{TPlist}{{\bf -name}}
\item[{{\bf -name}}]
Allows you to specify the prefix string which is printed
in the resulting UIL output.  If not specified, will default to the
filename (without extension) of the ppmfile argument.
If
{\bf -name}
is not specified and no ppmfile
is specified ({\it i.e.}, piped input), the prefix string will default to
the string ``noname''.
\end{TPlist}

\par
All flags can be abbreviated to their shortest unique prefix.
\shead{SEE ALSO}
ppm(5)
\shead{AUTHOR}
Converted by Jef Poskanzer from ppmtoxpm.c, which is
\copyright 1990 by Mark W. Snitily.
% Permission to use, copy, modify, and distribute this software and its
% documentation for any purpose and without fee is hereby granted, provided
% that the above copyright notice appear in all copies and that both that
% copyright notice and this permission notice appear in supporting
% documentation.  This software is provided "as is" without express or
% implied warranty.
%
% end of input file: ppmtouil.1
%--------------------------------------------------
 
% -*-LaTeX-*-
% Converted automatically from troff to LaTeX
% by tr2latex ($Revision: 2.2 $$Date: 1992/04/27 15:13:26 $ by C. Engel)
% on Fri Oct  8 16:08:59 1993
% tr2latex was written by Kamal Al-Yahya at Stanford University
% (Kamal%Hanauma@SU-SCORE.ARPA)
% and substantially enhanced by Christian Engel at RWTH Aachen
% (krischan@informatik.rwth-aachen.de).
%
% troff input file: ppmtoxpm.1

\newpage
%--------------------------------------------------
% start of input file: ppmtoxpm.1
%
\phead{ppmtoxpm}{1}{Tue Apr 9 1991}{}{}

\shead{NAME}
ppmtoxpm - convert a portable pixmap into an X11 pixmap
\shead{SYNOPSIS}
ppmtoxpm [-name $<$xpmname$>$] [-rgb $<$rgb-textfile$>$] [$<$ppmfile$>$]
\shead{DESCRIPTION}
Reads a portable pixmap as input.
Produces X11 pixmap  (version 3) as output which
can be loaded directly by the XPM library.
\par
The {\bf -nameoption} allows one to specify the prefix string which is printed
in the resulting XPM output.  If not specified, will default to the
filename (without extension) of the $<$ppmfile$>$ argument.
If {\bf -nameis} not specified and $<$ppmfile$>$
is not specified ({\it i.e.}, piped input), the prefix string will default to
the string ``noname''.
\par
The {\bf -rgboption} allows one to specify an X11 rgb text file for the
lookup of color name mnemonics.  This rgb text file is typically the
/usr/lib/X11/rgb.txt of the MIT X11 distribution, but any file using the
same format may be used.  When specified and
a RGB value from the ppm input matches a RGB value from the $<$rgb-textfile$>$,
then the corresponding color name mnemonic is printed in the XPM's colormap.
If {\bf -rgbis} not specified, or if the RGB values don't match, then the color
will be printed with the \#RGB, \#RRGGBB, \#RRRGGGBBB, or \#RRRRGGGGBBBB
hexadecimal format.
\par
All flags can be abbreviated to their shortest unique prefix.
\par
For example, to convert the file ``dot'' (found in /usr/include/X11/bitmaps),
from xbm to xpm one could specify
\begin{IPlist}
\IPitem{{}}
xbmtopbm dot $|$ ppmtoxpm -name dot
\end{IPlist}

\par
or, with a rgb text file (in the local directory)
\begin{IPlist}
\IPitem{{}}
xbmtopbm dot $|$ ppmtoxpm -name dot -rgb rgb.txt
\end{IPlist}

\shead{BUGS}
An option to match the closest (rather than exact) color name mnemonic
from the rgb text would be a desirable enhancement.
\par
Truncation of the least significant bits of a RGB value may result in
nonexact matches when performing color name mnemonic lookups.
\shead{SEE ALSO}
ppm(5)
\nwl
XPM Manual by Arnaud Le Hors (lehors@mirsa.inria.fr).
\shead{AUTHOR}
\copyright 1990 by Mark W. Snitily.

Permission to use, copy, modify, and distribute this software and its
documentation for any purpose and without fee is hereby granted, provided
that the above copyright notice appear in all copies and that both that
copyright notice and this permission notice appear in supporting
documentation.  This software is provided "as is" without express or
implied warranty.

This tool was developed for Schlumberger Technologies, ATE Division, and
with their permission is being made available to the public with the above
copyright notice and permission notice.

Upgraded to XPM2 by
   Paul Breslaw, Mecasoft SA, Zurich, Switzerland (paul@mecazh.uu.ch)
   Thu Nov  8 16:01:17 1990

Upgraded to XPM version 3 by
   Arnaud Le Hors (lehors@mirsa.inria.fr)
   Tue Apr 9 1991


%
% end of input file: ppmtoxpm.1
%--------------------------------------------------
 
% -*-LaTeX-*-
% Converted automatically from troff to LaTeX
% by tr2latex ($Revision: 2.2 $$Date: 1992/04/27 15:13:26 $ by C. Engel)
% on Fri Oct  8 16:09:00 1993
% tr2latex was written by Kamal Al-Yahya at Stanford University
% (Kamal%Hanauma@SU-SCORE.ARPA)
% and substantially enhanced by Christian Engel at RWTH Aachen
% (krischan@informatik.rwth-aachen.de).
%
% troff input file: ppmtoyuv.1

\newpage
%--------------------------------------------------
% start of input file: ppmtoyuv.1
%
\phead{ppmtoyuv}{1}{25 March 91}{}{}

%.IX ppmtoyuv
\shead{NAME}
ppmtoyuv - convert a portable pixmap into an Abekas YUV file
\shead{SYNOPSIS}
{\bf ppmtoyuv}
{\rm [}{\it ppmfile}{\rm ]}
\shead{DESCRIPTION}
Reads a portable pixmap as input.
Produces an Abekas YUV file as output.
%.IX Abekas
\shead{SEE ALSO}
yuvtoppm(1), ppm(5)
\shead{AUTHOR}
Marc Boucher (marc@PostImage.COM),
based on Example Conversion Program, A60/A64 Digital Video Interface
Manual, page 69.
\par
\copyright 1991 by DHD PostImage Inc.
\par
\copyright 1987 by Abekas Video Systems Inc.
% Permission to use, copy, modify, and distribute this software and its
% documentation for any purpose and without fee is hereby granted, provided
% that the above copyright notice appear in all copies and that both that
% copyright notice and this permission notice appear in supporting
% documentation.  This software is provided "as is" without express or
% implied warranty.
%
% end of input file: ppmtoyuv.1
%--------------------------------------------------
 
% -*-LaTeX-*-
% Converted automatically from troff to LaTeX
% by tr2latex ($Revision: 2.2 $$Date: 1992/04/27 15:13:26 $ by C. Engel)
% on Fri Oct  8 16:09:11 1993
% tr2latex was written by Kamal Al-Yahya at Stanford University
% (Kamal%Hanauma@SU-SCORE.ARPA)
% and substantially enhanced by Christian Engel at RWTH Aachen
% (krischan@informatik.rwth-aachen.de).
%
% troff input file: ppmtoyuvsplit.1

\newpage
%--------------------------------------------------
% start of input file: ppmtoyuvsplit.1
%
\phead{ppmtoyuvsplit}{1}{9 September 1993}{}{}

%.IX ppmtoyuvsplit
\shead{NAME}
ppmtoyuvsplit - convert a portable pixmap into 3 subsampled raw YUV files
\shead{SYNOPSIS}
{\bf ppmtoyuvsplit}
{\it basename}
{\rm [}{\it ppmfile}{\rm ]}
\shead{DESCRIPTION}
Reads a portable pixmap as input.
Produces 3 raw files basename.Y, basename.U and basename.V as output.
These files are the subsampled raw YUV representation of the input
pixmap, as required by the Stanford MPEG codec. The subsampling is done
by arithmetic mean of 4 pixels colors into one. The YUV values are scaled
according to CCIR.601, as assumed by MPEG.
\shead{SEE ALSO}
mpeg(1), ppm(5)
\shead{AUTHOR}
\copyright 1993 by Andre Beck. (Andre\_Beck@IRS.Inf.TU-Dresden.de).
\par
Based on ppmtoyuv.c.
% Permission to use, copy, modify, and distribute this software and its
% documentation for any purpose and without fee is hereby granted, provided
% that the above copyright notice appear in all copies and that both that
% copyright notice and this permission notice appear in supporting
% documentation.  This software is provided "as is" without express or
% implied warranty.
%
% end of input file: ppmtoyuvsplit.1
%--------------------------------------------------
 
% -*-LaTeX-*-
% Converted automatically from troff to LaTeX
% by tr2latex ($Revision: 2.2 $$Date: 1992/04/27 15:13:26 $ by C. Engel)
% on Fri Oct  8 16:08:33 1993
% tr2latex was written by Kamal Al-Yahya at Stanford University
% (Kamal%Hanauma@SU-SCORE.ARPA)
% and substantially enhanced by Christian Engel at RWTH Aachen
% (krischan@informatik.rwth-aachen.de).
%
% troff input file: psidtopgm.1

\newpage
%--------------------------------------------------
% start of input file: psidtopgm.1
%
\phead{psidtopgm}{1}{02 August 89}{}{}

%.IX psidtopgm
\shead{NAME}
psidtopgm - convert PostScript ``image'' data into a portable graymap
\shead{SYNOPSIS}
{\bf psidtopgm}
{\it width height bits/sample}
{\rm [}{\it imagedata}{\rm ]}
\shead{DESCRIPTION}
Reads the ``image'' data from a PostScript file as input.
%.IX PostScript
Produces a portable graymap as output.
\par
This is a very simple and limited program, and is here only because
so many people have asked for it.
To use it you have to
{\bf manually}
extract the readhexstring data portion from your PostScript file, and then
give the width, height, and bits/sample on the command line.
Before you attempt this, you should
{\bf at least}
read the description of the ``image'' operator in the PostScript Language
Reference Manual.
\par
It would probably not be too hard to write a script that uses this filter
to read a specific variety of PostScript image, but the variation is too
great to make a general-purpose reader.
Unless, of course, you want to write a full-fledged PostScript interpreter...
\shead{SEE ALSO}
pnmtops(1), pgm(5)
\shead{AUTHOR}
\copyright 1989 by Jef Poskanzer.
% Permission to use, copy, modify, and distribute this software and its
% documentation for any purpose and without fee is hereby granted, provided
% that the above copyright notice appear in all copies and that both that
% copyright notice and this permission notice appear in supporting
% documentation.  This software is provided "as is" without express or
% implied warranty.
%
% end of input file: psidtopgm.1
%--------------------------------------------------
 
% -*-LaTeX-*-
% Converted automatically from troff to LaTeX
% by tr2latex ($Revision: 2.2 $$Date: 1992/04/27 15:13:26 $ by C. Engel)
% on Fri Oct  8 16:09:00 1993
% tr2latex was written by Kamal Al-Yahya at Stanford University
% (Kamal%Hanauma@SU-SCORE.ARPA)
% and substantially enhanced by Christian Engel at RWTH Aachen
% (krischan@informatik.rwth-aachen.de).
%
% troff input file: qrttoppm.1

\newpage
%--------------------------------------------------
% start of input file: qrttoppm.1
%
\phead{qrttoppm}{1}{25 August 1989}{}{}

%.IX qrttoppm
\shead{NAME}
qrttoppm - convert output from the QRT ray tracer into a portable pixmap
\shead{SYNOPSIS}
{\bf qrttoppm}
{\rm [}{\it qrtfile}{\rm ]}
\shead{DESCRIPTION}
Reads a QRT file as input.
%.IX "QRT raytracer"
Produces a portable pixmap as output.
\shead{SEE ALSO}
ppm(5)
\shead{AUTHOR}
\copyright 1989 by Jef Poskanzer.
% Permission to use, copy, modify, and distribute this software and its
% documentation for any purpose and without fee is hereby granted, provided
% that the above copyright notice appear in all copies and that both that
% copyright notice and this permission notice appear in supporting
% documentation.  This software is provided "as is" without express or
% implied warranty.
%
% end of input file: qrttoppm.1
%--------------------------------------------------
 
% -*-LaTeX-*-
% Converted automatically from troff to LaTeX
% by tr2latex ($Revision: 2.2 $$Date: 1992/04/27 15:13:26 $ by C. Engel)
% on Fri Oct  8 16:09:28 1993
% tr2latex was written by Kamal Al-Yahya at Stanford University
% (Kamal%Hanauma@SU-SCORE.ARPA)
% and substantially enhanced by Christian Engel at RWTH Aachen
% (krischan@informatik.rwth-aachen.de).
%
% troff input file: rasttopnm.1

\newpage
%--------------------------------------------------
% start of input file: rasttopnm.1
%
\phead{rasttopnm}{1}{13 January 1991}{}{}

%.IX rasttopnm
\shead{NAME}
rasttopnm - convert a Sun rasterfile into a portable anymap
\shead{SYNOPSIS}
{\bf rasttopnm}
{\rm [}{\it rastfile}{\rm ]}
\shead{DESCRIPTION}
Reads a Sun rasterfile as input.
%.IX Sun
%.IX rasterfile
Produces a portable anymap as output.
The type of the output file depends on the input file - if it's
black \& white, a
{\it pbm}
file is written, else if it's grayscale a
{\it pgm}
file, else a
{\it ppm}
file.  The program tells you which type it is writing.
\shead{SEE ALSO}
pnmtorast(1), pnm(5)
\shead{AUTHOR}
\copyright 1989, 1991 by Jef Poskanzer.
% Permission to use, copy, modify, and distribute this software and its
% documentation for any purpose and without fee is hereby granted, provided
% that the above copyright notice appear in all copies and that both that
% copyright notice and this permission notice appear in supporting
% documentation.  This software is provided "as is" without express or
% implied warranty.
%
% end of input file: rasttopnm.1
%--------------------------------------------------
 
% -*-LaTeX-*-
% Converted automatically from troff to LaTeX
% by tr2latex ($Revision: 2.2 $$Date: 1992/04/27 15:13:26 $ by C. Engel)
% on Fri Oct  8 16:08:34 1993
% tr2latex was written by Kamal Al-Yahya at Stanford University
% (Kamal%Hanauma@SU-SCORE.ARPA)
% and substantially enhanced by Christian Engel at RWTH Aachen
% (krischan@informatik.rwth-aachen.de).
%
% troff input file: rawtopgm.1

\newpage
%--------------------------------------------------
% start of input file: rawtopgm.1
%
\phead{rawtopgm}{1}{15 June 1993}{}{}

%.IX rawtopgm
\shead{NAME}
rawtopgm - convert raw grayscale bytes into a portable graymap
\shead{SYNOPSIS}
{\bf rawtopgm}
{\rm [}{\bf -headerskip}
{\it N}{\rm ]}
{\rm [}{\bf -rowskip}
{\it N}{\rm ]}
{\rm [}{\bf -tb}{\rm $|$}{\bf -topbottom}{\rm ]}
{\rm [}{\it width}
{\it height}{\rm ]}
{\rm [}{\it imagedata}{\rm ]}
\shead{DESCRIPTION}
Reads raw grayscale bytes as input.
%.IX "raw grayscale"
Produces a portable graymap as output.
The input file is just grayscale bytes.
If you don't specify the width and height on the command line,
the program will check the size of the image and try to make
a quadratic image of it. It is an error to supply a non
quadratic image without specifying width and height.
The maxval is assumed to be 255.
\shead{OPTIONS}
\begin{TPlist}{{\bf -headerskip}}
\item[{{\bf -headerskip}}]
If the file has a header, you can use this flag to
skip over it.
\item[{{\bf -rowskip}}]
If there is padding at the ends of the rows, you can skip it with this flag.
Note that rowskip can be a real number.
Amazingly, I once had an image with 0.376 bytes of padding per row.
This turned out to be due to a file-transfer problem, but I was still
able to read the image.
\item[{{\bf -tb -topbottom}}]
Flips the image upside down.
The first pixel in a pgm file is in the lower left corner of the image.
For conversion from images with the first pixel in the upper left corner
({\it e.g.}, the Molecular Dynamics and Leica confocal formats) this flips the
image right.
This is equivalent to
{\bf rawtopgm\ [file]\ $|$\ pnmflip\ -tb .}
\end{TPlist}

\shead{BUGS}
If you don't specify the image width and height, the program will
try to read the entire image to a memory buffer. If you get a
message that states that you are out of memory, try to specify the width
and height on the command line. Also, the -tb option consumes much
memory.
\shead{SEE ALSO}
pgm(5), rawtoppm(1), pnmflip(1)
\shead{AUTHORS}
\copyright 1989 by Jef Poskanzer.
\nwl
Modified June 1993 by Oliver Trepte (oliver@fysik4.kth.se).
% Permission to use, copy, modify, and distribute this software and its
% documentation for any purpose and without fee is hereby granted, provided
% that the above copyright notice appear in all copies and that both that
% copyright notice and this permission notice appear in supporting
% documentation.  This software is provided "as is" without express or
% implied warranty.
%
% end of input file: rawtopgm.1
%--------------------------------------------------
 
% -*-LaTeX-*-
% Converted automatically from troff to LaTeX
% by tr2latex ($Revision: 2.2 $$Date: 1992/04/27 15:13:26 $ by C. Engel)
% on Fri Oct  8 16:09:01 1993
% tr2latex was written by Kamal Al-Yahya at Stanford University
% (Kamal%Hanauma@SU-SCORE.ARPA)
% and substantially enhanced by Christian Engel at RWTH Aachen
% (krischan@informatik.rwth-aachen.de).
%
% troff input file: rawtoppm.1

\newpage
%--------------------------------------------------
% start of input file: rawtoppm.1
%
\phead{rawtoppm}{1}{06 February 1991}{}{}

%.IX rawtoppm
\shead{NAME}
rawtoppm - convert raw RGB bytes into a portable pixmap
\shead{SYNOPSIS}
{\bf rawtoppm}
{\rm [}{\bf -headerskip}
{\it N}{\rm ]}
{\rm [}{\bf -rowskip}
{\it N}{\rm ]}
{\rm [}{\bf -rgb}{\rm $|$}{\bf -rbg}{\rm $|$}{\bf -grb}
{\rm $|$}{\bf -gbr}{\rm $|$}{\bf -brg}{\rm $|$}{\bf -bgr}
{\rm ]}
{\rm [}{\bf -interpixel}{\rm $|$}{\bf -interrow}{\rm ]}
{\it width height}
{\rm [}{\it imagedata}{\rm ]}
\shead{DESCRIPTION}
Reads raw RGB bytes as input.
%.IX "raw RGB"
Produces a portable pixmap as output.
The input file is just RGB bytes.
You have to specify the width and height on the command line,
since the program obviously can't get them from the file.
The maxval is assumed to be 255.
If the resulting image is upside down, run it through
{\bf pnmflip\ -tb .}
%.IX pnmflip
\shead{OPTIONS}
\begin{TPlist}{{\bf -headerskip}}
\item[{{\bf -headerskip}}]
If the file has a header, you can use this flag to
skip over it.
\item[{{\bf -rowskip}}]
If there is padding at the ends of the rows, you can skip it with this flag.
\item[{{\bf -rgb -rbg -grb -gbr -brg -bgr}}]
These flags let you specify alternate color orders.  The default is
{\bf -rgb}{\rm .}
\item[{{\bf -interpixel -interrow}}]
These flags let you specify how the colors are interleaved.
The default is
{\bf -interpixel}{\rm ,}
meaning interleaved by pixel.
A byte of red, a byte of green, and a byte
of blue, or whatever color order you specified.
{\bf -interrow}
means interleaved by row - a row of red, a row of green, a row of blue,
assuming standard rgb color order.
An
{\bf -interplane}
flag  - all the red pixels, then all the green, then all the blue - would
be an obvious extension, but is not implemented.
You could get the same effect by splitting the file into three parts
(perhaps using
{\it dd}{\rm ),}
turning each part into a PGM file with rawtopgm, and then combining them
with rgb3toppm.
\end{TPlist}

\shead{SEE ALSO}
ppm(5), rawtopgm(1), rgb3toppm(1), pnmflip(1)
\shead{AUTHOR}
\copyright 1991 by Jef Poskanzer.
% Permission to use, copy, modify, and distribute this software and its
% documentation for any purpose and without fee is hereby granted, provided
% that the above copyright notice appear in all copies and that both that
% copyright notice and this permission notice appear in supporting
% documentation.  This software is provided "as is" without express or
% implied warranty.
%
% end of input file: rawtoppm.1
%--------------------------------------------------
 
% -*-LaTeX-*-
% Converted automatically from troff to LaTeX
% by tr2latex ($Revision: 2.2 $$Date: 1992/04/27 15:13:26 $ by C. Engel)
% on Fri Oct  8 16:09:02 1993
% tr2latex was written by Kamal Al-Yahya at Stanford University
% (Kamal%Hanauma@SU-SCORE.ARPA)
% and substantially enhanced by Christian Engel at RWTH Aachen
% (krischan@informatik.rwth-aachen.de).
%
% troff input file: rgb3toppm.1

\newpage
%--------------------------------------------------
% start of input file: rgb3toppm.1
%
\phead{rgb3toppm}{1}{15 February 1990}{}{}

%.IX rgb3toppm
\shead{NAME}
rgb3toppm - combine three portable graymaps into one portable pixmap
\shead{SYNOPSIS}
{\bf rgb3toppm}
{\it redpgmfile greenpgmfile bluepgmfile}
\shead{DESCRIPTION}
Reads three portable graymaps as input.
Combines them and produces one portable pixmap as output.
\shead{SEE ALSO}
ppmtorgb3(1), pgmtoppm(1), ppmtopgm(1), ppm(5), pgm(5)
\shead{AUTHOR}
\copyright 1991 by Jef Poskanzer.
% Permission to use, copy, modify, and distribute this software and its
% documentation for any purpose and without fee is hereby granted, provided
% that the above copyright notice appear in all copies and that both that
% copyright notice and this permission notice appear in supporting
% documentation.  This software is provided "as is" without express or
% implied warranty.
%
% end of input file: rgb3toppm.1
%--------------------------------------------------
 
% -*-LaTeX-*-
% Converted automatically from troff to LaTeX
% by tr2latex ($Revision: 2.2 $$Date: 1992/04/27 15:13:26 $ by C. Engel)
% on Fri Oct  8 16:09:32 1993
% tr2latex was written by Kamal Al-Yahya at Stanford University
% (Kamal%Hanauma@SU-SCORE.ARPA)
% and substantially enhanced by Christian Engel at RWTH Aachen
% (krischan@informatik.rwth-aachen.de).
%
% troff input file: sirtopnm.1

\newpage
%--------------------------------------------------
% start of input file: sirtopnm.1
%
\phead{sirtopnm}{1}{20 March 1991}{}{}

\shead{NAME}
sirtopnm - convert a Solitaire file into a portable anymap
\shead{SYNOPSIS}
{\bf sirtopnm}
{\rm [}{\it sirfile}{\rm ]}
\shead{DESCRIPTION}
Reads a Solitaire Image Recorder file as input.
Produces a portable anymap as output.
The type of the output file depends on the input file - if it's
an MGI TYPE 17 file, a
{\it pgm}
file is written. If it's an MGI TYPE 11 file, a
{\it ppm}
file is written.  The program tells you which type it is writing.
\shead{BUGS}

\shead{SEE ALSO}
pnmtosir(1), pnm(5)
\shead{AUTHOR}
\copyright 1991 by Marvin Landis.
% Permission to use, copy, modify, and distribute this software and its
% documentation for any purpose and without fee is hereby granted, provided
% that the above copyright notice appear in all copies and that both that
% copyright notice and this permission notice appear in supporting
% documentation.  This software is provided "as is" without express or
% implied warranty.

%
% end of input file: sirtopnm.1
%--------------------------------------------------
 
% -*-LaTeX-*-
% Converted automatically from troff to LaTeX
% by tr2latex ($Revision: 2.2 $$Date: 1992/04/27 15:13:26 $ by C. Engel)
% on Fri Oct  8 16:09:02 1993
% tr2latex was written by Kamal Al-Yahya at Stanford University
% (Kamal%Hanauma@SU-SCORE.ARPA)
% and substantially enhanced by Christian Engel at RWTH Aachen
% (krischan@informatik.rwth-aachen.de).
%
% troff input file: sldtoppm.1

\newpage
%--------------------------------------------------
% start of input file: sldtoppm.1
%
\phead{sldtoppm}{1}{10 October 1991}{}{}

%.IX sldtoppm
%.IX AutoCAD
\shead{NAME}
sldtoppm - convert an AutoCAD slide file into a portable pixmap
\shead{SYNOPSIS}
\raggedright
{\bf sldtoppm}
'in 14n
{\rm [}{\bf -adjust}{\rm ]}
{\rm [}{\bf -dir}{\rm ]}
{\rm [}{\bf -height}{\rm $|$}{\bf -ysize}
{\it s}{\rm ]}
{\rm [}{\bf -info}{\rm ]}
{\rm [}{\bf -lib}{\rm $|$}{\bf -Lib}
{\it name}{\rm ]}
{\rm [}{\bf -scale}
{\it s}{\rm ]}
{\rm [}{\bf -verbose}{\rm ]}
{\rm [}{\bf -width}{\rm $|$}{\bf -xsize}
{\it s}{\rm ]}
{\rm [}{\it slidefile}{\rm ]}
\ind{-7.0em}
%.ad
\shead{DESCRIPTION}
Reads an AutoCAD{l}ide file and outputs a portable pixmap.
If no
{\it slidefile}
is specified, input is read from standard input.
The ppmdraw library is used to convert the vector and polygon
information in the slide file to a pixmap; see the file ppmdraw.h for
details on this package.
\shead{OPTIONS}
\begin{TPlist}{{\bf -adjust}}
\item[{{\bf -adjust}}]
If the display on which the slide file was created had non-square
pixels, when the slide is processed with
{\bf sldtoppm}
and the
{\bf -adjust}
option is not present, the following warning will appear:
\ind{-6.8in}
Warning - pixels on source screen were non-square.
\nwl
Specifying
{\bf -adjust}
will correct image width to compensate.
\ind{-7.0in}
Specifying the
{\bf -adjust}
option causes
{\bf sldtoppm}
to scale the width of the image so that pixels in the resulting
portable pixmap are square (and hence circles appear as true circles,
not ellipses).  The scaling is performed in the vector domain, before
scan converting the objects.  The results are, therefore, superior in
appearance to what you'd obtain were you to perform the equivalent
scaling with
{\bf pnmscale}
after the bitmap had been created.
\item[{{\bf -dir}}]
The input is assumed to be an AutoCAD slide library file.  A directory
listing each slide in the library is printed on standard error.
\item[{{\bf -height}{\it \ size}
}]
Scales the image in the vector domain so it is
{\it size}
pixels in height.  If no
{\bf -width}
or
{\bf -xsize}
option is specified, the width will be adjusted to preserve the
pixel aspect ratio. 
\item[{{\bf -info}}]
Dump the slide file header on standard error, displaying the original
screen size and aspect ratio among other information.
\item[{{\bf -lib}{\it \ name}
}]
Extracts the slide with the given
{\it name}
from the slide library given as input.  The specified
{\it name}
is converted to upper case.
\item[{{\bf -Lib}{\it \ name}
}]
Extracts the slide with the given
{\it name}
from the slide library given as input.  The
{\it name}
is used exactly as specified; it is not converted to upper case.
\item[{{\bf -scale}{\it \ s}
}]
Scales the image by factor
{\it s}{\rm ,}
which may be any floating point value greater than zero.  Scaling is
done after aspect ratio adjustment, if any.  Since scaling is
performed in the vector domain, before rasterisation, the results look
much better than running the output of
{\bf sldtoppm}
through
{\bf pnmscale}{\rm .}
\item[{{\bf -verbose}}]
Dumps the slide file header and lists every vector and polygon in the
file on standard error.
\item[{{\bf -width}{\it \ size}
}]
Scales the image in the vector domain so it is
{\it size}
pixels wide.  If no
{\bf -height}
or
{\bf -ysize}
option is specified, the height will be adjusted to preserve the
pixel aspect ratio. 
\item[{{\bf -xsize}{\it \ size}
}]
Scales the image in the vector domain so it is
{\it size}
pixels wide.  If no
{\bf -height}
or
{\bf -ysize}
option is specified, the height will be adjusted to preserve the
pixel aspect ratio. 
\item[{{\bf -ysize}{\it \ size}
}]
Scales the image in the vector domain so it is
{\it size}
pixels in height.  If no
{\bf -width}
or
{\bf -xsize}
option is specified, the width will be adjusted to preserve the
pixel aspect ratio. 
\end{TPlist}

\par
All flags can be abbreviated to their shortest unique prefix.
\shead{BUGS}
Only Level 2 slides are converted.  Level 1 format has been obsolete
since the advent of AutoCAD Release 9 in 1987, and was not portable
across machine architectures.
\par
Slide library items with names containing 8 bit (such as ISO) or 16
bit (Kanji, for example) characters may not be found when chosen with the
{\bf -lib}
option unless
{\bf sldtoppm}
has been built with character set conversion functions appropriate to
the locale.  You can always retrieve slides from libraries regardless
of the character set by using the
{\bf -Lib}
option and specifying the precise name of library member.  Use the
{\bf -dir}
option to list the slides in a library if you're unsure of the
exact name.
\shead{SEE ALSO}
AutoCAD Reference Manual:
{\it Slide\ File\ Format}{\rm ,}
{\bf pnmscale}{\rm (1),}
{\bf ppm}{\rm (5)}
\shead{AUTHOR}
\ind{1\parindent}{\nofill
    John Walker
    Autodesk SA
    Avenue des Champs-Montants 14b
    CH-2074 MARIN
    Suisse/Schweiz/Svizzera/Svizra/Switzerland
\fill}
\begin{TPlist}{Usenet:}
\item[{Usenet:}]
kelvin@Autodesk.com
\item[{Fax:}]
038/33 88 15
\item[{Voice:}]
038/33 76 33
\end{TPlist}

\par
Permission to use, copy, modify, and distribute this software and its
documentation for any purpose and without fee is hereby granted,
without any conditions or restrictions.  This software is provided ``as
is'' without express or implied warranty.
\par
AutoCAD and Autodesk are registered trademarks of Autodesk, Inc.
%
% end of input file: sldtoppm.1
%--------------------------------------------------
 
% -*-LaTeX-*-
% Converted automatically from troff to LaTeX
% by tr2latex ($Revision: 2.2 $$Date: 1992/04/27 15:13:26 $ by C. Engel)
% on Fri Oct  8 16:09:03 1993
% tr2latex was written by Kamal Al-Yahya at Stanford University
% (Kamal%Hanauma@SU-SCORE.ARPA)
% and substantially enhanced by Christian Engel at RWTH Aachen
% (krischan@informatik.rwth-aachen.de).
%
% troff input file: spctoppm.1

\newpage
%--------------------------------------------------
% start of input file: spctoppm.1
%
\phead{spctoppm}{1}{19 July 1990}{}{}

%.IX spctoppm
\shead{NAME}
spctoppm - convert an Atari compressed Spectrum file into a portable pixmap
\shead{SYNOPSIS}
{\bf spctoppm}
{\rm [}{\it spcfile}{\rm ]}
\shead{DESCRIPTION}
Reads an Atari compressed Spectrum file as input.
%.IX Atari
%.IX Spectrum
Produces a portable pixmap as output.
\shead{SEE ALSO}
sputoppm(1), ppm(5)
\shead{AUTHOR}
\copyright 1991 by Steve Belczyk (seb3@gte.com) and Jef Poskanzer.
% Permission to use, copy, modify and distribute this software and its
% documentation for any purpose and without fee is hereby granted, provided
% that the above copyright notice appear in all copies and that both that
% copyright notice and this permission notice appear in supporting
% documentation.  This software is provided "as is" without express or
% implied warranty.
%
% end of input file: spctoppm.1
%--------------------------------------------------
 
% -*-LaTeX-*-
% Converted automatically from troff to LaTeX
% by tr2latex ($Revision: 2.2 $$Date: 1992/04/27 15:13:26 $ by C. Engel)
% on Fri Oct  8 16:08:37 1993
% tr2latex was written by Kamal Al-Yahya at Stanford University
% (Kamal%Hanauma@SU-SCORE.ARPA)
% and substantially enhanced by Christian Engel at RWTH Aachen
% (krischan@informatik.rwth-aachen.de).
%
% troff input file: spottopgm.1

\newpage
%--------------------------------------------------
% start of input file: spottopgm.1
%
\phead{spottopgm}{1L}{}{}{}

\shead{NAME}
spottopgm -- convert SPOT satellite images to Portable Greymap format
\shead{SYNTAX}
spottopgm [--1$|$2$|$3] [Firstcol Firstline Lastcol Lastline] inputfile
\shead{OPTIONS}
\begin{TPlist}{{\bf --1$|$2$|$3}}
\item[{{\bf --1$|$2$|$3}}]
Extract the given colour from the SPOT image. The colours are infra-red,
visible light and ultra-violet, although I don't know which corresponds
to which number. If the image is in colour, this will be announced on
standard error. The default colour is 1.
\item[{{\bf Firstcol Firstline Lastcol Lastline}}]
Extract the specified rectangle from the SPOT image. Most SPOT images are
3000 lines long and 3000 or more columns wide. Unfortunately the SPOT format
only gives the width and not the length. The width is printed on standard
error. The default rectangle is the width of the input image by 3000 lines.
\end{TPlist}

\shead{DESCRIPTION}
{\it Spottopgm}
converts the named
{\bf inputfile}
into Portable Greymap format, defaulting to the first color and the whole
SPOT image unless specified by the options.
\shead{INSTALLATION}
You
{\bf must}
edit the source program and either define BIGENDIAN or LITTLEENDIAN,
and fix the typedefs for uint32t, uint16t and uint8t appropriately.
\shead{BUGS}
Currently
{\it spottopgm}
doesn't determine the length of the input file; this would involve two
passes over the input file. It defaults to 3000 lines instead.
\par
{\it Spottopgm}
could extract a three-color image (ppm), but I didn't feel like making the
program more complicated than it is now. Besides, there is no one-to-one
correspondence between red, green, blue and infra-red, visible and
ultra-violet.
\par
I've only had a limited number of SPOT images to play with, and therefore
wouldn't guarantee that this will work on any other images.
\shead{AUTHOR}
Warren Toomey  wkt@csadfa.cs.adfa.oz.au
\shead{SEE ALSO}
The rest of the Pbmplus suite.
%
% end of input file: spottopgm.1
%--------------------------------------------------
 
% -*-LaTeX-*-
% Converted automatically from troff to LaTeX
% by tr2latex ($Revision: 2.2 $$Date: 1992/04/27 15:13:26 $ by C. Engel)
% on Fri Oct  8 16:09:03 1993
% tr2latex was written by Kamal Al-Yahya at Stanford University
% (Kamal%Hanauma@SU-SCORE.ARPA)
% and substantially enhanced by Christian Engel at RWTH Aachen
% (krischan@informatik.rwth-aachen.de).
%
% troff input file: sputoppm.1

\newpage
%--------------------------------------------------
% start of input file: sputoppm.1
%
\phead{sputoppm}{1}{19 July 1990}{}{}

%.IX sputoppm
\shead{NAME}
sputoppm - convert an Atari uncompressed Spectrum file into a portable pixmap
\shead{SYNOPSIS}
{\bf sputoppm}
{\rm [}{\it spufile}{\rm ]}
\shead{DESCRIPTION}
Reads an Atari uncompressed Spectrum file as input.
%.IX Atari
%.IX Spectrum
Produces a portable pixmap as output.
\shead{SEE ALSO}
spctoppm(1), ppm(5)
\shead{AUTHOR}
\copyright 1991 by Steve Belczyk (seb3@gte.com) and Jef Poskanzer.
% Permission to use, copy, modify and distribute this software and its
% documentation for any purpose and without fee is hereby granted, provided
% that the above copyright notice appear in all copies and that both that
% copyright notice and this permission notice appear in supporting
% documentation.  This software is provided "as is" without express or
% implied warranty.
%
% end of input file: sputoppm.1
%--------------------------------------------------
 
% -*-LaTeX-*-
% Converted automatically from troff to LaTeX
% by tr2latex ($Revision: 2.2 $$Date: 1992/04/27 15:13:26 $ by C. Engel)
% on Fri Oct  8 16:09:04 1993
% tr2latex was written by Kamal Al-Yahya at Stanford University
% (Kamal%Hanauma@SU-SCORE.ARPA)
% and substantially enhanced by Christian Engel at RWTH Aachen
% (krischan@informatik.rwth-aachen.de).
%
% troff input file: tgatoppm.1

\newpage
%--------------------------------------------------
% start of input file: tgatoppm.1
%
\phead{tgatoppm}{1}{26 August 1989}{}{}

%.IX tgatoppm
\shead{NAME}
tgatoppm - convert TrueVision Targa file into a portable pixmap
\shead{SYNOPSIS}
{\bf tgatoppm}
{\rm [}{\bf -debug}{\rm ]}
{\rm [}{\it tgafile}{\rm ]}
\shead{DESCRIPTION}
Reads a TrueVision Targa file as input.
%.IX TrueVision
%.IX Targa
Produces a portable pixmap as output.
\shead{OPTIONS}
\begin{TPlist}{{\bf -debug}}
\item[{{\bf -debug}}]
Causes the header information to be dumped to stderr.
\end{TPlist}

\par
All flags can be abbreviated to their shortest unique prefix.
%.BUGS
Should really be in PNM, not PPM.
\shead{SEE ALSO}
ppmtotga(1), ppm(5)
\shead{AUTHOR}
Partially based on tga2rast, version 1.0, by Ian J. MacPhedran.

\copyright 1989 by Jef Poskanzer.
% Permission to use, copy, modify, and distribute this software and its
% documentation for any purpose and without fee is hereby granted, provided
% that the above copyright notice appear in all copies and that both that
% copyright notice and this permission notice appear in supporting
% documentation.  This software is provided "as is" without express or
% implied warranty.
%
% end of input file: tgatoppm.1
%--------------------------------------------------
 
% -*-LaTeX-*-
% Converted automatically from troff to LaTeX
% by tr2latex ($Revision: 2.2 $$Date: 1992/04/27 15:13:26 $ by C. Engel)
% on Fri Oct  8 16:09:36 1993
% tr2latex was written by Kamal Al-Yahya at Stanford University
% (Kamal%Hanauma@SU-SCORE.ARPA)
% and substantially enhanced by Christian Engel at RWTH Aachen
% (krischan@informatik.rwth-aachen.de).
%
% troff input file: tifftopnm.1

\newpage
%--------------------------------------------------
% start of input file: tifftopnm.1
%
\phead{tifftopnm}{1}{13 January 1991}{}{}

%.IX tifftopnm
\shead{NAME}
tifftopnm - convert a TIFF file into a portable anymap
\shead{SYNOPSIS}
{\bf tifftopnm}
{\rm [}{\bf -headerdump}{\rm ]}
{\rm tifffile}
\shead{DESCRIPTION}
Reads a TIFF file as input.
%.IX TIFF
Produces a portable anymap as output.
The type of the output file depends on the input file - if it's
black \& white, a
{\it pbm}
file is written, else if it's grayscale a
{\it pgm}
file, else a
{\it ppm}
file.  The program tells you which type it is writing.
\shead{OPTIONS}
\begin{TPlist}{{\bf -headerdump}}
\item[{{\bf -headerdump}}]
Dump TIFF file information to stderr.  This information may be useful 
in debugging TIFF file conversion problems.  
\end{TPlist}

\par
All flags can be abbreviated to their shortest unique prefix.
\shead{SEE ALSO}
pnmtotiff(1), pnm(5)
\shead{BUGS}
This program is not self-contained.  To use it you must fetch the
TIFF Software package listed in the OTHER.SYSTEMS file and configure
PBMPLUS to use libtiff.  See PBMPLUS's Makefile for details on this
configuration.
\shead{AUTHOR}
Derived by Jef Poskanzer from tif2ras.c, which is
\copyright 1990 by Sun Microsystems, Inc.\hfil\break
Author: Patrick J. Naughton (naughton@wind.sun.com).
% Permission to use, copy, modify, and distribute this software and its
% documentation for any purpose and without fee is hereby granted,
% provided that the above copyright notice appear in all copies and that
% both that copyright notice and this permission notice appear in
% supporting documentation.
% 
% This file is provided AS IS with no warranties of any kind.  The author
% shall have no liability with respect to the infringement of copyrights,
% trade secrets or any patents by this file or any part thereof.  In no
% event will the author be liable for any lost revenue or profits or
% other special, indirect and consequential damages.
%
% end of input file: tifftopnm.1
%--------------------------------------------------
 
% -*-LaTeX-*-
% Converted automatically from troff to LaTeX
% by tr2latex ($Revision: 2.2 $$Date: 1992/04/27 15:13:26 $ by C. Engel)
% on Fri Oct  8 16:08:14 1993
% tr2latex was written by Kamal Al-Yahya at Stanford University
% (Kamal%Hanauma@SU-SCORE.ARPA)
% and substantially enhanced by Christian Engel at RWTH Aachen
% (krischan@informatik.rwth-aachen.de).
%
% troff input file: xbmtopbm.1

\newpage
%--------------------------------------------------
% start of input file: xbmtopbm.1
%
\phead{xbmtopbm}{1}{31 August 1988}{}{}

%.IX xbmtopbm
\shead{NAME}
xbmtopbm - convert an X11 or X10 bitmap into a portable bitmap
\shead{SYNOPSIS}
{\bf xbmtopbm}
{\rm [}{\it bitmapfile}{\rm ]}
\shead{DESCRIPTION}
Reads an X11 or X10 bitmap as input.
Produces a portable bitmap as output.
%.IX "X bitmap"
%.IX "X window system"
\shead{SEE ALSO}
pbmtoxbm(1), pbmtox10bm(1), pbm(5)
\shead{AUTHOR}
\copyright 1988 by Jef Poskanzer.
% Permission to use, copy, modify, and distribute this software and its
% documentation for any purpose and without fee is hereby granted, provided
% that the above copyright notice appear in all copies and that both that
% copyright notice and this permission notice appear in supporting
% documentation.  This software is provided "as is" without express or
% implied warranty.
%
% end of input file: xbmtopbm.1
%--------------------------------------------------
 
% -*-LaTeX-*-
% Converted automatically from troff to LaTeX
% by tr2latex ($Revision: 2.2 $$Date: 1992/04/27 15:13:26 $ by C. Engel)
% on Fri Oct  8 16:09:05 1993
% tr2latex was written by Kamal Al-Yahya at Stanford University
% (Kamal%Hanauma@SU-SCORE.ARPA)
% and substantially enhanced by Christian Engel at RWTH Aachen
% (krischan@informatik.rwth-aachen.de).
%
% troff input file: ximtoppm.1

\newpage
%--------------------------------------------------
% start of input file: ximtoppm.1
%
\phead{ximtoppm}{1}{25 March 1990}{}{}

%.IX ximtoppm
\shead{NAME}
ximtoppm - convert an Xim file into a portable pixmap
\shead{SYNOPSIS}
{\bf ximtoppm}
{\rm [}{\it ximfile}{\rm ]}
\shead{DESCRIPTION}
Reads an Xim file as input.
%.IX Xim
%.IX "X window system"
Produces a portable pixmap as output.
The Xim toolkit is included in the contrib tree of the X.V11R4 release.
\shead{SEE ALSO}
ppm(5)
\shead{AUTHOR}
\copyright 1991 by Jef Poskanzer.
% Permission to use, copy, modify, and distribute this software and its
% documentation for any purpose and without fee is hereby granted, provided
% that the above copyright notice appear in all copies and that both that
% copyright notice and this permission notice appear in supporting
% documentation.  This software is provided "as is" without express or
% implied warranty.
%
% end of input file: ximtoppm.1
%--------------------------------------------------
 
% -*-LaTeX-*-
% Converted automatically from troff to LaTeX
% by tr2latex ($Revision: 2.2 $$Date: 1992/04/27 15:13:26 $ by C. Engel)
% on Fri Oct  8 16:09:05 1993
% tr2latex was written by Kamal Al-Yahya at Stanford University
% (Kamal%Hanauma@SU-SCORE.ARPA)
% and substantially enhanced by Christian Engel at RWTH Aachen
% (krischan@informatik.rwth-aachen.de).
%
% troff input file: xpmtoppm.1

\newpage
%--------------------------------------------------
% start of input file: xpmtoppm.1
%
\phead{xpmtoppm}{1}{16 August 1990}{}{}

\shead{NAME}
xpmtoppm - convert an X11 pixmap into a portable pixmap
\shead{SYNOPSIS}
{\bf xpmtoppm}
{\rm [}{\it xpmfile}{\rm ]}
\shead{DESCRIPTION}
Reads an X11 pixmap (XPM version 1 or 3) as input.
Produces a portable pixmap as output.
\shead{KNOWN BUGS}
The support to XPM version 3 is limited. Comments can only be single lines
and there must be for every pixel a default colorname for a color type visual.
\shead{SEE ALSO}
ppmtoxpm(1), ppm(5)
\nwl
XPM Manual by Arnaud Le Hors (lehors@mirsa.inria.fr).
\shead{AUTHOR}
\copyright 1991 by Jef Poskanzer.
Upgraded to support XPM version 3 by Arnaud Le Hors
(lehors@mirsa.inria.fr) Tue Apr 9 1991.
% Permission to use, copy, modify, and distribute this software and its
% documentation for any purpose and without fee is hereby granted, provided
% that the above copyright notice appear in all copies and that both that
% copyright notice and this permission notice appear in supporting
% documentation.  This software is provided "as is" without express or
% implied warranty.
%
% end of input file: xpmtoppm.1
%--------------------------------------------------
 
% -*-LaTeX-*-
% Converted automatically from troff to LaTeX
% by tr2latex ($Revision: 2.2 $$Date: 1992/04/27 15:13:26 $ by C. Engel)
% on Mon Feb  7 08:46:26 1994
% tr2latex was written by Kamal Al-Yahya at Stanford University
% (Kamal%Hanauma@SU-SCORE.ARPA)
% and substantially enhanced by Christian Engel at RWTH Aachen
% (krischan@informatik.rwth-aachen.de).
%
% troff input file: xvminitoppm.1

\newpage
%--------------------------------------------------
% start of input file: xvminitoppm.1
%
\phead{xvminitoppm}{1}{14 December 1993}{}{}

\shead{NAME}
xvminitoppm - convert a XV ``thumbnail'' picture to PPM
\shead{SYNOPSIS}
{\bf xvminitoppm}
{\rm [}{\it xvminipic}{\rm ]}
\shead{DESCRIPTION}
Reads a XV ``thumbnail'' picture (a miniature picture generated by
the ``VisualSchnauzer'' browser) as input.
Produces a portable pixmap as output.
\shead{SEE ALSO}
ppm(5), xv(1)
\shead{AUTHOR}
Copyright (C) 1993 by Ingo Wilken
% Permission to use, copy, modify, and distribute this software and its
% documentation for any purpose and without fee is hereby granted, provided
% that the above copyright notice appear in all copies and that both that
% copyright notice and this permission notice appear in supporting
% documentation.  This software is provided "as is" without express or
% implied warranty.
%
% end of input file: xvminitoppm.1
%--------------------------------------------------

% -*-LaTeX-*-
% Converted automatically from troff to LaTeX
% by tr2latex ($Revision: 2.2 $$Date: 1992/04/27 15:13:26 $ by C. Engel)
% on Fri Oct  8 16:09:28 1993
% tr2latex was written by Kamal Al-Yahya at Stanford University
% (Kamal%Hanauma@SU-SCORE.ARPA)
% and substantially enhanced by Christian Engel at RWTH Aachen
% (krischan@informatik.rwth-aachen.de).
%
% troff input file: xwdtopnm.1

\newpage
%--------------------------------------------------
% start of input file: xwdtopnm.1
%
\phead{xwdtopnm}{1}{11 January 1991}{}{}

%.IX xwdtopnm
\shead{NAME}
xwdtopnm - convert a X11 or X10 window dump file into a portable anymap
\shead{SYNOPSIS}
{\bf xwdtopnm}
{\rm [}{\it xwdfile}{\rm ]}
\shead{DESCRIPTION}
Reads a X11 or X10 window dump file as input.
%.IX XWD
%.IX "X window system"
Produces a portable anymap as output.
The type of the output file depends on the input file - if it's
black \& white, a
{\it pbm}
file is written, else if it's grayscale a
{\it pgm}
file, else a
{\it ppm}
file.  The program tells you which type it is writing.
\par
Using this program, you can convert anything on an X workstation's screen
into an anymap.
Just display whatever you're interested in, do an xwd, run it through
xwdtopnm, and then use pnmcut to select the part you want.
\shead{BUGS}
I haven't tested this tool with very many configurations, so there are
probably bugs.
Please let me know if you find any.
\shead{SEE ALSO}
pnmtoxwd(1), pnm(5), xwd(1)
\shead{AUTHOR}
\copyright 1989, 1991 by Jef Poskanzer.
% Permission to use, copy, modify, and distribute this software and its
% documentation for any purpose and without fee is hereby granted, provided
% that the above copyright notice appear in all copies and that both that
% copyright notice and this permission notice appear in supporting
% documentation.  This software is provided "as is" without express or
% implied warranty.
%
% end of input file: xwdtopnm.1
%--------------------------------------------------
 
% -*-LaTeX-*-
% Converted automatically from troff to LaTeX
% by tr2latex ($Revision: 2.2 $$Date: 1992/04/27 15:13:26 $ by C. Engel)
% on Fri Oct  8 16:08:14 1993
% tr2latex was written by Kamal Al-Yahya at Stanford University
% (Kamal%Hanauma@SU-SCORE.ARPA)
% and substantially enhanced by Christian Engel at RWTH Aachen
% (krischan@informatik.rwth-aachen.de).
%
% troff input file: ybmtopbm.1

\newpage
%--------------------------------------------------
% start of input file: ybmtopbm.1
%
\phead{ybmtopbm}{1}{06 March 1990}{}{}

%.IX ybmtopbm
\shead{NAME}
ybmtopbm - convert a Bennet Yee ``face'' file into a portable bitmap
\shead{SYNOPSIS}
{\bf ybmtopbm}
{\rm [}{\it facefile}{\rm ]}
\shead{DESCRIPTION}
Reads a file acceptable to the
{\it face}
and
{\it xbm}
programs by Bennet Yee (bsy+@cs.cmu.edu).
%.IX face
Writes a portable bitmap as output.
\shead{SEE ALSO}
pbmtoybm(1), pbm(5), face(1), face(5), xbm(1)
\shead{AUTHOR}
\copyright 1991 by Jamie Zawinski and Jef Poskanzer.
% Permission to use, copy, modify, and distribute this software and its
% documentation for any purpose and without fee is hereby granted, provided
% that the above copyright notice appear in all copies and that both that
% copyright notice and this permission notice appear in supporting
% documentation.  This software is provided "as is" without express or
% implied warranty.
%
% end of input file: ybmtopbm.1
%--------------------------------------------------
 
% -*-LaTeX-*-
% Converted automatically from troff to LaTeX
% by tr2latex ($Revision: 2.2 $$Date: 1992/04/27 15:13:26 $ by C. Engel)
% on Fri Oct  8 16:09:12 1993
% tr2latex was written by Kamal Al-Yahya at Stanford University
% (Kamal%Hanauma@SU-SCORE.ARPA)
% and substantially enhanced by Christian Engel at RWTH Aachen
% (krischan@informatik.rwth-aachen.de).
%
% troff input file: yuvsplittoppm.1

\newpage
%--------------------------------------------------
% start of input file: yuvsplittoppm.1
%
\phead{yuvsplittoppm}{1}{26 August 93}{}{}

%.IX yuvsplittoppm
\shead{NAME}
yuvplittoppm - convert a Y- an U- and a V-file into a portable pixmap.
\shead{SYNOPSIS}
{\bf yuvsplittoppm}
{\it basename width height}
[-ccir601]
\shead{DESCRIPTION}
Reads three files, containing the YUV components, as input.
These files are
{\it basename}
Y,
{\it basename}
U
and
{\it basename}
.V .
Produces a portable pixmap on stdout.

Since the YUV files are raw files, the dimensions
{\it width}
and
{\it height}
must be specified on the command line.
\shead{OPTIONS}
\begin{TPlist}{{\bf -ccir601}}
\item[{{\bf -ccir601}}]
Assumes that the YUV triplets are scaled into the smaller range of the
CCIR 601 (MPEG) standard. Else, the JFIF (JPEG) standard is assumed.
\end{TPlist}

\shead{SEE ALSO}
ppmtoyuvsplit(1), yuvtoppm(1), ppm(5)
\shead{AUTHOR}
Marcel Wijkstra (wijkstra@fwi.uva.nl), based on {\it ppmtoyuvsplit.}
%
% end of input file: yuvsplittoppm.1
%--------------------------------------------------
 
% -*-LaTeX-*-
% Converted automatically from troff to LaTeX
% by tr2latex ($Revision: 2.2 $$Date: 1992/04/27 15:13:26 $ by C. Engel)
% on Fri Oct  8 16:09:06 1993
% tr2latex was written by Kamal Al-Yahya at Stanford University
% (Kamal%Hanauma@SU-SCORE.ARPA)
% and substantially enhanced by Christian Engel at RWTH Aachen
% (krischan@informatik.rwth-aachen.de).
%
% troff input file: yuvtoppm.1

\newpage
%--------------------------------------------------
% start of input file: yuvtoppm.1
%
\phead{yuvtoppm}{1}{25 March 91}{}{}

%.IX yuvtoppm
\shead{NAME}
yuvtoppm - convert Abekas YUV bytes into a portable pixmap
\shead{SYNOPSIS}
{\bf yuvtoppm}
{\it width height}
{\rm [}{\it imagedata}{\rm ]}
\shead{DESCRIPTION}
Reads raw Abekas YUV bytes as input.
%.IX Abekas
Produces a portable pixmap as output.
The input file is just YUV bytes.
You have to specify the width and height on the command line,
since the program obviously can't get them from the file.
The maxval is assumed to be 255.
\shead{SEE ALSO}
ppmtoyuv(1), ppm(5)
\shead{AUTHOR}
Marc Boucher (marc@PostImage.COM),
based on Example Conversion Program, A60/A64 Digital Video Interface
Manual, page 69.
\par
\copyright 1991 by DHD PostImage Inc.
\par
\copyright 1987 by Abekas Video Systems Inc.
% Permission to use, copy, modify, and distribute this software and its
% documentation for any purpose and without fee is hereby granted, provided
% that the above copyright notice appear in all copies and that both that
% copyright notice and this permission notice appear in supporting
% documentation.  This software is provided "as is" without express or
% implied warranty.
%
% end of input file: yuvtoppm.1
%--------------------------------------------------
 
% -*-LaTeX-*-
% Converted automatically from troff to LaTeX
% by tr2latex ($Revision: 2.2 $$Date: 1992/04/27 15:13:26 $ by C. Engel)
% on Fri Oct  8 16:09:34 1993
% tr2latex was written by Kamal Al-Yahya at Stanford University
% (Kamal%Hanauma@SU-SCORE.ARPA)
% and substantially enhanced by Christian Engel at RWTH Aachen
% (krischan@informatik.rwth-aachen.de).
%
% troff input file: zeisstopnm.1

\newpage
%--------------------------------------------------
% start of input file: zeisstopnm.1
%
\phead{zeisstopnm}{1}{15 June 1993}{}{}

%.IX zeisstopnm
\shead{NAME}
zeisstopnm - convert a Zeiss confocal file into a portable anymap
\shead{SYNOPSIS}
{\bf zeisstopnm}
{\rm [}{\it -pgm}
$|$
{\it -ppm}{\rm ]}
{\rm [}{\it zeissfile}{\rm ]}
\shead{DESCRIPTION}
Reads a Zeiss confocal file as input.
Produces a portable anymap as output.
The type of the output file depends on the input file -
if it's grayscale a
{\it pgm}
file, else a
{\it ppm}
file will be produced.
The program tells you which type it is writing.
\shead{OPTIONS}
\begin{TPlist}{{\bf -pgm}}
\item[{{\bf -pgm}}]
Force the output to be a
{\it pgm}
file.
\item[{{\bf -ppm}}]
Force the output to be a
{\it ppm}
file.
\end{TPlist}

\shead{SEE ALSO}
pnm(5)
\shead{AUTHOR}
\copyright 1993 by Oliver Trepte (oliver@fysik4.kth.se).
% Permission to use, copy, modify, and distribute this software and its
% documentation for any purpose and without fee is hereby granted, provided
% that the above copyright notice appear in all copies and that both that
% copyright notice and this permission notice appear in supporting
% documentation.  This software is provided "as is" without express or
% implied warranty.
%
% end of input file: zeisstopnm.1
%--------------------------------------------------
\end{document}
